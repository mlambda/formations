
\begin{frame}{Description}
  Cette formation présente les fondamentaux du Deep Learning appliqués à la vision par ordinateur ainsi que les principales techniques utilisées dans l'industrie.
  Les travaux pratiques s'appuieront sur des données réelles et présenteront des modèles récents.
  Certains points aborderont des sujets de recherche récents.
\end{frame}

\fmg{
  \begin{frame}{Description}
    \V{"img/deep-meme" | image("tw")}
  \end{frame}
}

\begin{frame}{Prérequis}
  \begin{itemize}
    \item Connaissances de base dans un langage de programmation
    \item Avoir des notions d'algèbre linéaire (calcul matriciel, dérivées, etc.) et de statistiques est un plus
    \item Avoir un \textbf{compte Google} afin de pouvoir faire les TPs dans \bluelink{https://colab.research.google.com/notebooks/welcome.ipynb}{Google Colaboratory}
  \end{itemize}
\end{frame}

\begin{frame}{Objectifs pédagogiques}
  \begin{itemize}
    \item Prétraiter des images
    \item Construire des réseaux de convolution traitant des images
    \item Identifier les mécanismes de regularisation
    \item Détecter des objets dans des images
    \item Classer des images
    \item Générer des images
    \item Extraire des résultats actionnables
  \end{itemize}
\end{frame}
