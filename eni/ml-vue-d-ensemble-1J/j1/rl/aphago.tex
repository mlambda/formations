\begin{frame}
  \frametitle{AlphaGo}
  AlphaGo un bon exemple de développement industriel en deep learning.
  \begin{itemize}
  \item \textbf{deep artificial neural networks}
  \item \textbf{Monte Carlo tree search}
  \item \textbf{Reinforcement Learning}
  \item \textbf{Très gros moyens !}    
  \end{itemize}
\end{frame}

\begin{frame}
  \frametitle{Go}
  \begin{minipage}[c]{0.6\linewidth}
    \imgtw[1]{go-regles}
  \end{minipage}\hfill
  \begin{minipage}[c]{0.33\linewidth}
    \begin{center}
      \underline{Go Vs Echecs} \\
      \textbf{Le nombre de coups} \\
      $\approx$ 200 contre $\approx$ 35 \\
      \textbf{Le nombre de tours} \\
      $\approx$ 300 contre $\approx$ 40 \\
      $\Rightarrow$ $\approx 10^{120}\lll \;\approx 10^{690}$
    \end{center}
  \end{minipage}\hfill
\end{frame}

\begin{frame}
  \frametitle{AlphaGo}
  \begin{minipage}[c]{0.50\linewidth}
    \begin{itemize}
    \item $softmax(convolutions^{13}(19*19*48{bits}))$
    \item 48bits : input sur la position, la configuration , ...
    \item Appris sur données supervisées
    \end{itemize}
  \end{minipage}\hfill
  \begin{minipage}[c]{0.49\linewidth}
  \imgth[0.8]{alphago-value}
  \end{minipage}\hfill
\end{frame}

\begin{frame}
  \frametitle{AlphaGo}
  \begin{minipage}[c]{0.50\linewidth}
    \imgth[0.8]{alphago-policy}
  \end{minipage}\hfill
  \begin{minipage}[c]{0.49\linewidth}
    \begin{itemize}
    \item Appris sur du self-play
    \item A l'aide d'un MCTS qui apprend à générer des parties efficace en fonction de la value function (fixée) et policy function (que l'on est en train d'apprendre)
    \end{itemize}
  \end{minipage}\hfill
\end{frame}

\begin{frame}
  \frametitle{AlphaGo}
  Monte-Carlo tree search
  \imgtw[0.8]{alphago-mcts}
  \begin{itemize}
  \item Selection par maximum de (Q-value + u(P)) où P est une proba à priori pour chaque coup.
  \item Expansion en fonction de la policy qui produit des probabilités qui sont alors stockés pour chaque action
  \end{itemize}
\end{frame}

\begin{frame}
  \frametitle{AlphaGo}
  2 phases pour apprendre :
  \begin{description}
  \item[Value network  :]
  \item 3 semaines x (50 x TPU) 
  \item[1 million de parties APV-MCTS :]
  \item 1 journée x (5000 x TPU)
  \item $\;$
  \item[50 TPU $\approx$ 2000 TFLOPS]
  \item core i7 : $\approx$ 0.07 TFLOPS (donc il en faudrait $\approx$ 30 000)
  \end{description}
\end{frame}

\begin{frame}
  \frametitle{Go}
  \begin{minipage}[c]{0.7\linewidth}
    \imgtw[1]{go-winrate-dan}
  \end{minipage}\hfill
  \begin{minipage}[c]{0.26\linewidth}
    \begin{adjustbox}{scale=0.8}
      \begin{tabular}{| r | l |}
        \hline
        EGF & Classement \\ \hline
        2940 & 9 dan pro \\ \hline
        2820 & 8 dan amateur \\ \hline
        2700 & 7 dan amateur \\ \hline
        2600 & 6 dan amateur \\ \hline
        2500 & 5 dan \\ \hline
        2400 & 4 dan \\ \hline
        2300 & 3 dan \\ \hline
        2200 & 2 dan \\ \hline
        2100 & 1 dan \\ \hline
        2000 & 1 kyu \\ \hline
        1900 & 2 kyu \\ \hline
        1800 & 3 kyu \\ \hline
        1500 & 6 kyu \\ \hline
        1000 & 11 kyu \\ \hline
        500 & 16 kyu \\ \hline
        100 & 20 kyu \\ \hline
      \end{tabular}
    \end{adjustbox}
  \end{minipage}\hfill
\end{frame}

\begin{frame}
  \frametitle{AlphaGo}
  Plusieurs versions :
  \begin{itemize}
  \item \textbf{AlphaGo Vs Fan Hui 5 à 0} Octobre 2015
  \item \textbf{AlphaGo vs Lee Sedol 4 à 1} Mars 2016 (Prédiction de Rémi Coulom en 2016 : Résultat pas avant 15 ans)
  \item \textbf{AlphaGo-zero} considéré comme ayant au moins 500 EGF au dessus de AlphaGo. 2017
  \end{itemize}
\end{frame}

\begin{frame}
  \frametitle{AlphaGo}
  \begin{center}
    AlphaGo vs Lee Sedol (ou Fan Hui)
  \end{center}
  \huge
  1 202 processeurs et 176 TPU
\end{frame}

\begin{frame}
  \frametitle{AlphaGo Zero}
  \begin{itemize}
  \item \textbf{Plus simple : 4 TPU}
  \item \textbf{Plus fort : bat alphago 100:0}
  \item \textbf{Plus général : self-play uniquement}
  \end{itemize}  
  \imgtw[1]{alphagozero-lerningcurve}
\end{frame}

\begin{frame}
  \frametitle{Alpha Zero}
  \begin{itemize}
  \item \textbf{Encore plus général}
  \item \textbf{Bat tous les bot d'échec, go et shogi}
  \end{itemize}  
\end{frame}

\begin{frame}
  \frametitle{Alpha Zero}
  Relativisons
  \begin{itemize}
  \item Un cerveau humain consomme $\approx$ 20 Watts
  \item AlphaGo $\approx$ 440 000 Watts (440 grille-pains)
  \item Un tableur gagnerait n'importe quelle compétition de calcul mental
  \end{itemize}
\end{frame}

