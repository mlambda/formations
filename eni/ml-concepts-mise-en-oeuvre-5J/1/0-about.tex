\documentclass{formation}
\title{À propos de cette formation}
\subtitle{Module 0}

\begin{document}

\maketitle

\begin{frame}
  \frametitle{Votre formateur}
  \begin{description}
  \item[Nom] Giraud François-Marie
  \item[Courriel] giraud.francois@gmail.com
  \item[Activité] Consultant/Formateur en IA
  \item[Spécialité] Machine Learning
  \item[Parcours] Master IAD de l'UPMC, Ingénieur de recherche
  \end{description}
\end{frame}

\begin{frame}
  \frametitle{Votre formation — description}
  Cette formation présente les \textbf{fondamentaux du machine
    learning} ainsi que les principales \textbf{techniques utilisées
    dans l’industrie}.
\end{frame}

\begin{frame}
  \frametitle{Votre formation — profil des stagiaires}
  Développeurs, ingénieurs informatiques désireux d’utiliser les
  techniques d’apprentissage automatique pour exploiter les données à
  leur disposition. \\
  \newline
  Bon niveau général en informatique, à l’aise en programmation. \\
  \begin{center}
    \green{Avez-vous un compte google ?}\\
    $\;$\\
    \green{Connaissez-vous le Python ?}
  \end{center}
\end{frame}

\begin{frame}
  \frametitle{Votre formation — objectifs à atteindre}

  \begin{itemize}
  \item poser un problème de machine learning
  \item prétraiter des données
  \item construire des modèles d'apprentissage pour des données
    annotées comme non-annotées
  \item gérer les apprentissages de vos modèles
  \item extraire des résultats actionnables
  \end{itemize}
\end{frame}

\begin{frame}
  \frametitle{Votre formation — programme}

  \begin{itemize}

  \item Introduction au machine learning
  \item Notions de maths
  \item Fondamentaux

  \item Régressions linéaire et logistique
  \item Machine à Vecteurs de Support (SVM)
  \item Arbres de décision

  \item Réduction de dimensionnalité et clustering
  \item Détection d'anomalies

  \item Réseaux de neurones
  \item Embeddings

  \item Système de recommandation
  \end{itemize}
\end{frame}

\begin{frame}
  \frametitle{Votre formation — ressources}
  Les supports utilisées vous seront remis à chaque début de cours.
\end{frame}

\begin{frame}
  \frametitle{Tour de table — présentez-vous}
  \begin{itemize}
  \item Votre nom
  \item Votre métier
  \item Votre société client si appliquable
  \item Vos compétences dans les domaines liés à cette formation
  \item Vos objectifs et vos attentes vis-à-vis de cette formation
  \end{itemize}
\end{frame}

\begin{frame}
  \frametitle{Horaires}
  \begin{figure}
    \centering
    \begin{tabular}{ccc}
      \toprule
               & matin            & après-midi        \\ \midrule
      lundi    & 9h00-12h00       & 14h00-17h30       \\
      mardi    & 9h00-12h00       & 14h00-17h30       \\
      mercredi & 9h00-12h00       & 14h00-17h30       \\
      jeudi    & 9h00-12h00       & 14h00-17h30       \\
      vendredi & 9h00-12h00       & 14h00-17h30       \\
      \bottomrule
    \end{tabular}
  \end{figure}
\end{frame}

\begin{frame}
  \frametitle{Pauses}
  Une pause de 10 minutes pour couper chaque demi-journée.
\end{frame}

\end{document}
%%% Local Variables:
%%% mode: latex
%%% TeX-master: t
%%% End:
