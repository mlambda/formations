\begin{frame}{Outils}
    Plusieurs outils sont disponibles pour explorer des données. On
    utilise principalement des plots pour :
    \begin{itemize}
    \item se renseigner sur une distribution
    \item se renseigner sur la corrélation de deux distributions
    \item visualiser des corrélations linéaires
    \end{itemize}
    Les outils suivants sont sauf mention contraire présents dans
    \bluelink{https://seaborn.pydata.org/}{\texttt{seaborn}}.
  \end{frame}
  
  \begin{frame}{Outils — count plot}
    \V{["img/countplot-nzmog", "th", 0.7] | image}
  \end{frame}
  
  \begin{frame}{Outils — Histogramme}
    \V{["plt/histogram", "th", 0.7] | image}
  \end{frame}
  
  \begin{frame}{Outils — Diagramme quantile-quantile}
    \V{["plt/qq-plot", "th", 0.5] | image}
    Attention, pas \bluelink{https://seaborn.pydata.org/}{\texttt{seaborn}}
    mais
    \bluelink{http://www.statsmodels.org/stable/index.html}{\texttt{statsmodel}}
    ou
    \bluelink{https://docs.scipy.org/doc/scipy/reference/stats.html}{\texttt{scipy.stats}}.
  \end{frame}
  
  \begin{frame}{Outils — Nuage de points}
    \V{["plt/scatter-plot", "th", 0.7] | image}
  \end{frame}
  
  \begin{frame}{Outils — Diagramme en bâtons}
    \V{["plt/bar-plot", "th", 0.7] | image}
  \end{frame}
  
  \begin{frame}{Outils — Diagramme en violons}
    \V{["plt/violin-plot", "th", 0.7] | image}
  \end{frame}
  
  \begin{frame}{Outils — pair plot}
    \V{["img/pairplot-nzmog", "th", 0.7] | image}
  \end{frame}
  
  \begin{frame}{Outils — Matrice de corrélation}
    \V{["plt/corrmat", "th", 0.7] | image}
  \end{frame}