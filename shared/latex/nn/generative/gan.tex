\begin{frame}{Principe}
  \begin{itemize}[<+->]
    \item Utiliser le pouvoir d'un réseau pour en entraîner un autre, génératif
    \item Modéliser l'apprentissage comme un jeu
    \item Le premier réseau génère des exemples depuis du bruit
    \item Le deuxième distingue entre les générations du premier et les vrais exemples
  \end{itemize}
\end{frame}

\begin{frame}{Architecture}
  \V{"tikz/neural-networks/gan/gan" | image("tw", 1)}
\end{frame}

\begin{frame}{Discriminateur}
  Entraîné pour maximiser la probabilité d'assigner $1$ aux vrais exemples et $0$ aux faux~:

  \[
    \max_D
      \left\{
        \underbrace{\green{y \log D(\vect{x})}}_{\mathclap{\text{vrai exemple}}}
        +
        \underbrace{\red{(1 - y) \log (1 - D(\vect{x}))}}_{\mathclap{\text{faux exemple}}}
      \right\}
  \]

  \begin{itemize}[<+(1)->]
    \item Pour un vrai exemple ($y=1$), 
          le terme à maximiser est $\green{\log D(\vect{x})}$
          $\Rightarrow$ $D(\vect{x})$ doit tendre vers $1$
    \item Pour un faux exemple ($y=0$), le terme à maximiser est $\red{\log (1 - D(\vect{x}))}$

    $\Rightarrow$ $D(\vect{x})$ doit tendre vers $0$
  \end{itemize}
\end{frame}

\begin{frame}{Générateur}
  Entraîné pour minimiser la probabilité que $D$ assigne $0$ à de faux exemples~:

  \[
    \min_G
      \left\{
        (1 - y) \log (1 - D(G(\vect{z})))
      \right\}
  \]
  Le terme à minimiser est $\log (1 - D(G(\vect{z})))$

  $\Rightarrow$ $D(G(\vect{z}))$ doit tendre vers $1$
\end{frame}

\begin{frame}{Fonction d'entraînement complète}
  Les deux équations précédentes, 
  combinées sur tous les exemples (vrais et faux)~:

  \[
    \max_D \min_G \left\{
      \underbrace{\mathbb{E}_{\vect{x} \sim \text{Data}} \log D(\vect{x})}_{\mathclap{\text{Seulement $D$}}}
      + \underbrace{\mathbb{E}_{\vect{z} \sim \text{Noise}} \log (1 - D(G(\vect{z})))}_{\mathclap{\text{$D$ \& $G$}}}
    \right\}
  \]

  $\Rightarrow$ Description du jeu minimax entre $G$ \& $D$.
\end{frame}

\begin{frame}{Algorithme d'apprentissage}
  Pour chaque itération d'apprentissage~:

  \begin{enumerate}[<+(1)->]
    \item Échantillonner le vecteur de bruit $\vect{Z}$
    \item Échantillonner un batch de vrais exemples $\vect{X}$
    \item Mettre à jour les paramètres de $D$ ($\theta_D$) 
          par montée de gradient~:
    \[
      \theta_D \leftarrow \theta_D + \alpha \nabla_{\theta_D} \frac{1}{m} \sum_{i=1}^m \left[
        \log D\left(\vect{X}_i\right) + \log\left(1 - D\left(G\left(\vect{Z}_i\right)\right)\right)
      \right]
    \]
    \item Mettre à jour les paramètres de $G$ ($\theta_G$) 
          par descente de gradient~:
    \[
      \theta_G \leftarrow \theta_G - \alpha \nabla_{\theta_G} \frac{1}{m} \sum_{i=1}^m \left[
        \log\left(1 - D\left(G\left(\vect{Z}_i\right)\right)\right)
      \right]
    \]
  \end{enumerate}
  \onslide<+(1)->{Les étapes 3. et 4. peuvent être répétées plusieurs fois.}
\end{frame}

\begin{frame}{Stabilité de l'apprentissage}
  Entraîner un GAN est difficile~:
  \begin{itemize}[<+(1)->]
    \item Le discriminateur peut dominer
    \item Le générateur peut boucler sur certaines générations (mode collapse)
    \item Il peut être nécessaire d'entraîner un réseau plus que l'autre
    \item La distribution du bruit est importante
    \item Le pas d'apprentissage est important
    \item …
  \end{itemize}

  \onslide<+(1)->{Conseils utiles~: \url{https://github.com/soumith/ganhacks}}
\end{frame}
