\begin{frame}{Classification}
    \V{["img/d2l/lenet", "tw", 1] | image}
\end{frame}

\begin{frame}{Segmentation}
    \V{["tikz/neural-networks/cnn/segmentation", "tw", 1.0] | image}
\end{frame}

\begin{frame}{Inpainting}
    \V{["tikz/neural-networks/cnn/inpainting", "tw", 1.0] | image}
\end{frame}

\begin{frame}{Super résolution}
    \V{["tikz/neural-networks/cnn/super-resolution", "tw", 1.0] | image}
\end{frame}

\begin{frame}{Détection d'anomalie}
    \V{["tikz/neural-networks/cnn/anomaly-detection", "tw", 1.0] | image}
\end{frame}

\begin{frame}{Détection d'anomalie : Une méthode moderne}
    Vitjan Zavrtanik , Matej Kristan et Danijel Skcaj : Reconstruction by inpainting for visual anomaly detection.
    \alert{TODO}
\end{frame}

\begin{frame}{Détection d'anomalie : Une méthode moderne}
    \V{["tikz/neural-networks/cnn/inpainting-anomaly-detect-step1-mask", "th", 0.7] | image}
\end{frame}

\begin{frame}{Détection d'anomalie : Une méthode moderne}
    \V{["tikz/neural-networks/cnn/inpainting-anomaly-detect-step2-autoencoder-predict", "tw", 1.0] | image}
\end{frame}

\begin{frame}{Détection d'anomalie : Une méthode moderne}
    \V{["tikz/neural-networks/cnn/inpainting-anomaly-detect-step3-assemble", "th", 0.7] | image}
\end{frame}
  
\begin{frame}{Détection d'anomalie : Une méthode moderne}
    \alert{TODO}
    comparaison entre l'original et l'assemblage
\end{frame}

