\begin{frame}{Problème des MLPs pour les images}
  Supposons que nous voulons détecter un objet dans une image. Un bon modèle devrait~:
  \begin{itemize}[<+(1)->]
    \item Trouver l'objet où qu'il soit dans l'image
    \item Utiliser les pixels locaux autour de l'objet pour prendre sa décision
  \end{itemize}

  \onslide<+(1)->{Les MLPs standards ne satisfont pas ces critères.}

  \onslide<+(1)->{$\Rightarrow$ Les réseaux convolutifs, si.}
\end{frame}

\begin{frame}{Convolution}
  Bloc clef~: l'opération de corrélation croisée (appelée par erreur convolution).

  Elle incorpore la \textbf{localité} et l'\textbf{invariance à la translation}.

  \V{["img/d2l/correlation", "tw", 0.8] | image}
\end{frame}

\begin{frame}{Convolution sur plusieurs canaux}
  La slide précédente est incomplète~: une image a souvent plusieurs canaux (RGB) --- les représentations intermédiaires d'un CNN aussi.

  Un noyau de convolution a plusieurs canaux. Un par canal d'entrée:

  \V{["img/d2l/conv-multi-in", "th", 0.3] | image}

  Habituellement la dimension de profondeur est omise quand on décrit une convolution (convolution $2 \times 2$ au lieu de $2 \times 2 \times 2$).
\end{frame}

\begin{frame}{Remplissage}
  Le remplissage (\textit{padding} en anglais) contrôle le changement de taille dû à l'opérateur de corrélation croisée en ajoutant des $0$s:

  \V{["img/d2l/conv-pad", "th", 0.4] | image}
\end{frame}

\begin{frame}{Pas de convolution}
  Les pas de convolution (\textit{strides} en anglais) permettent de réduire les dimensions spatiales dans un CNN~:

  \V{["img/d2l/conv-stride", "th", 0.4] | image}

  Ici, on est passé de $5 \times 5$ à $2 \times 2$ grâce aux pas de convolution $(3, 2)$.
\end{frame}

\begin{frame}{Agrégation}
  \textit{Pooling} en anglais. L'autre mécanisme pour réduire les dimensions spatiales~:

  \V{["tikz/neural-networks/cnn/maxpool", "tw", 0.8] | image}

  Une agrégation $n \times m$ est souvent utilisée avec des pas $(n, m)$.
\end{frame}

\begin{frame}{Combinaison des blocs de base en un réseau convolutif}
  Yann LeCun ---~maintenant lauréat du prix Turing~--- a proposé la première combinaison de ces blocs simples en un réseau complet, LeNet~:

  \V{["img/d2l/lenet", "tw", 1] | image}
\end{frame}

\begin{frame}{Schéma de LeNet}
  \V{["img/d2l/lenet-vert", "th", 0.7] | image}
\end{frame}

\begin{frame}{Entraînement}
  Par descente de gradient, comme les MLPs.

  Moins de paramètres, plus d'opérations~: cible parfaite pour les GPUs.

  \V{["img/gpus", "th", 0.4] | image}
\end{frame}

\begin{frame}{Lien avec le cortex visuel}
  \begin{minipage}[l]{0.50\linewidth}
    \V{["img/visual-cortex", "tw", 1.0] | image}
  \end{minipage}\hfill
  \begin{minipage}[l]{0.49\linewidth}
    Couches successives:
    \begin{description}[<+(1)->]
    \item[V1] Orientation, lignes
    \item[V2] Formes, tailles, couleurs
    \item[V3] Motricité
    \item[V4] Reconnaissance d'objets
    \item[V5] Suivi d'objets
    \end{description}
  \end{minipage}\hfill
\end{frame}

\begin{frame}{Démonstration}
  \bluelink{http://www.cs.cmu.edu/~aharley/nn_vis/cnn/3d.html}{Convolutional neural networks}
\end{frame}
