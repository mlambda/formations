\begin{frame}{Détection d'anomalie : Une méthode moderne}
    [2021] \textit{Vitjan Zavrtanik , Matej Kristan et Danijel Skcaj : \\Reconstruction by inpainting for visual anomaly detection.} 
\end{frame}

\begin{frame}{Etape 1 : Masquage}
    \V{["tikz/neural-networks/cnn/inpainting-anomaly-detect-step1-mask", "th", 0.7] | image}
\end{frame}

\begin{frame}{Etape 2 : Prédiction des parties manquantes}
    \V{["tikz/neural-networks/cnn/inpainting-anomaly-detect-step2-autoencoder-predict", "tw", 1.0] | image}
\end{frame}

\begin{frame}{Etape 3 : Assemblage des parties prédites}
    \V{["tikz/neural-networks/cnn/inpainting-anomaly-detect-step3-assemble", "th", 0.7] | image}
\end{frame}
  
\begin{frame}{Etape 4 : Comparaison à l'original}
    \V{["tikz/neural-networks/cnn/inpainting-anomaly-detect-step4-compare-input-output", "th", 0.8] | image}
\end{frame}
  
\begin{frame}{Détection d'anomalie : Une méthode moderne}
    \begin{center}
        \bluelink{https://colab.research.google.com/drive/1QbpY8vfXb-yJQ3N-Amf-2pANyxlEYH_-\#forceEdit=true&sandboxMode=true}{Démo \og Reconstruction by inpainting for visual anomaly detection \fg}
    \end{center}
\end{frame}

