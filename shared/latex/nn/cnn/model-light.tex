\begin{frame}{Images et réseaux de neurones}
  Supposons que nous voulons détecter un objet dans une image. Un bon modèle devrait~:
  \begin{itemize}
    \item Trouver l'objet où qu'il soit dans l'image
    \item Utiliser les pixels locaux autour de l'objet pour prendre sa décision
  \end{itemize}

  Les réseaux de neurones standards ne satisfont pas ces critères.

  $\Rightarrow$ Les réseaux convolutifs, si.
\end{frame}

\begin{frame}{Convolution}
  Bloc clef~: l'opération de corrélation croisée (appelée par erreur convolution).

  Elle incorpore la \textbf{localité} et l'\textbf{invariance à la translation}.

  \V{"img/d2l/correlation" | image("tw", 0.8)}
\end{frame}

\begin{frame}{Agrégation}
  \textit{Pooling} en anglais. L'autre mécanisme pour réduire les dimensions spatiales~:

  \V{"tikz/neural-networks/cnn/maxpool" | image("tw", 0.8)}

  Une agrégation $n \times m$ est souvent utilisée avec des pas $(n, m)$.
\end{frame}

\begin{frame}{Combinaison des blocs de base en un réseau convolutif}
  Yann LeCun ---~maintenant lauréat du prix Turing~--- a proposé la première combinaison de ces blocs simples en un réseau complet, LeNet~:

  \V{"img/d2l/lenet" | image("tw", 1)}
\end{frame}

\begin{frame}{Schéma de LeNet}
  \V{"img/d2l/lenet-vert" | image("th", 0.7)}
\end{frame}

\begin{frame}{Entraînement}
  Par descente de gradient, comme les MLPs.

  Moins de paramètres, plus d'opérations~: cible parfaite pour les GPUs.

  \V{"img/gpus" | image("th", 0.4)}
\end{frame}

\begin{frame}{Démonstration}
  \bluelink{https://adamharley.com/nn_vis/cnn/3d.html}{Convolutional neural networks}
\end{frame}

\begin{frame}{Lien avec le cortex visuel}
  \begin{minipage}[l]{0.50\linewidth}
    \V{"img/visual-cortex" | image("tw", 1.0)}
  \end{minipage}\hfill
  \begin{minipage}[l]{0.49\linewidth}
    Couches successives:
    \begin{description}[<+(1)->]
    \item[V1] Orientation, lignes
    \item[V2] Formes, tailles, couleurs
    \item[V3] Motricité
    \item[V4] Reconnaissance d'objets
    \item[V5] Suivi d'objets
    \end{description}
  \end{minipage}\hfill
\end{frame}
