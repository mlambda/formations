\begin{frame}
  \frametitle{Introduction}
  Filtrage de l'information : \\
  \newline
  Produire une information personnalisé pour un ``client''
\end{frame}

\begin{frame}
  \frametitle{Exemples}
  \begin{minipage}[l]{0.33\linewidth}
    \V{["img/logos/google-ads", "tw", 1] | image}
  \end{minipage}\hfill
  \begin{minipage}[l]{0.33\linewidth}
    \V{["img/logos/amazon", "tw", 1] | image}
  \end{minipage}\hfill
  \begin{minipage}[l]{0.33\linewidth}
    \V{["img/logos/netflix", "tw", 1] | image}
  \end{minipage}\hfill
\end{frame}

\begin{frame}
  \frametitle{Intérêt Mutuel}
  \begin{itemize}[<+->]
  \item Facilitation des usages pour le ``client''
  \item Des bénéfices accrus pour le ``vendeur''
  \end{itemize}
\end{frame}

\begin{frame}
  \frametitle{Types de recommandation}
  3 types de recommandation :
  \begin{itemize}[<+->]
  \item centrée sur l'utilisateur (user-based)
  \item centrée sur le contenu (item-based)
  \item colaborative (ou sociale)
  \end{itemize}
\end{frame}

\begin{frame}
  \frametitle{Collecte des données}
  2 types de collecte des données : 
  \begin{itemize}[<+->]
  \item active
  \item passive
  \end{itemize}
\end{frame}
