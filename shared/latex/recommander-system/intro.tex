\begin{frame}
  \frametitle{Introduction}
  Filtrage de l'information : \\
  \newline
  Produire une information personnalisée pour un \og client\fg
\end{frame}

\begin{frame}
  \frametitle{Exemples}
  \begin{minipage}[l]{0.33\linewidth}
    \V{["img/logos/google-ads", "tw", 1] | image}
  \end{minipage}\hfill
  \begin{minipage}[l]{0.33\linewidth}
    \V{["img/logos/amazon", "tw", 1] | image}
  \end{minipage}\hfill
  \begin{minipage}[l]{0.33\linewidth}
    \V{["img/logos/netflix", "tw", 1] | image}
  \end{minipage}\hfill
\end{frame}

\begin{frame}
  \frametitle{Intérêt Mutuel}
  \begin{itemize}
  \item Facilitation des usages pour le \og client\fg
  \item Des bénéfices accrus pour le \og vendeur\fg
  \end{itemize}
\end{frame}

\begin{frame}
  \frametitle{Types de recommandation}
  2 types de recommandation :
  \begin{itemize}
  \item Basée sur le contenu (content-based)
  \item Basée sur l'utilisateur (filtrage collaboratif)
  \end{itemize}
\end{frame}

\begin{frame}
  \frametitle{Collecte des données}
  2 types de collecte des données : 
  \begin{itemize}
  \item active
  \item passive
  \end{itemize}
\end{frame}

\begin{frame}
  \frametitle{Données}
  Données explicites et implicites:
  \begin{itemize}
    \item Explicite : donnée de qualité mais contraignante pour l'utilisateur (donc potentiellement biaisé)\\
    \item Implicite : donnée objective mais pauvre du point de vue de la qualité  
  \end{itemize}
\end{frame}

% \begin{frame}
%   \frametitle{Contrainte}
%   Grosse volumétrie de données $\Rightarrow$ algorithmes simples et rapides 
% \end{frame}

% \begin{frame}
%   \frametitle{Évaluation}
%   On utilise des techniques d'évaluation du machine learning impliquant des bases d'apprentissage et de test.
%   \begin{center}
%     $RMSE = \sqrt{\frac{\sum_{u,i}(u(i)_{pred}-u(i))^2}{\#Q}}$
%   \end{center}
%   où $\#Q$ est le nombre de prédictions que l'on fait dans la base.
% \end{frame}