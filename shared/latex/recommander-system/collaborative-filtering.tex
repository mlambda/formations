\begin{frame}{Introduction}
  La différence entre le content-based filtering et le filtrage collaborative se situe sur l'objet sur lequel on mesure la similarité.\\[2ex]

  Le \alert{filtrage collaborative} mesure la similarité entre les utilisateurs plutôt qu'entre les contenus à recommander.
\end{frame}
\begin{frame}{Introduction}
  \V{["tikz/recommander-system/collaborative-filtering", "tw", 1] | image}
\end{frame}

\begin{frame}{Données}
  Exemple de la base de données :
  \begin{center}
  \begin{tabular}{|l|c|c|c|c|}
    \hline
     & film 1 & film 2 & film 3 & film 4 \\
    \hline
    u1 & ? & 2 & 5 & ? \\
    \hline
    u2 & 5 & 1 & 4 & 3 \\
    \hline
    u3 & ? & ? & 1 & 5 \\
    \hline
    u4 & 3 & 3 & ? & 4 \\
    \hline
  \end{tabular}
  \end{center}
\end{frame}

\begin{frame}
  \frametitle{Principe}
  Principe : Une bonne recommandation est faite à partir d'individus similaires\\
  \newline
  Condition : Avoir un échantillon important d'individus représentatifs
  \V{["img/logos/amazon", "tw", 0.6] | image}
\end{frame}

\begin{frame}{Score par similarité}
  Anne veut voir un film :
  \begin{enumerate}
    \item On peut trouver les utilisateurs qui ont les mêmes préférences qu'Anne (Voir p.\ref{collabf:sim_user})
    \item On regarde les films préféres de ces utilisateurs (Voir p.\ref{collabf:select_users}) 
    \item On conseille le \og \textit{meilleur}\fg{} film pour ce groupe (Voir p.\ref{collabf:rating_sim})  
  \end{enumerate}

\end{frame}

\begin{frame}{Score par similarité -- Similarite}
  \label{collabf:sim_user}
  On définie une métrique de similarité $sim(u,u')$ entre les utilisateurs comme pour le content-based filtering (p.\ref{contentbasedf:metrics}) 

  \alert{Similarité de corrélation (Pearson)}:\\
    $$cor(a,b)=\frac{ \sum_{i \in I} (a_i - \bar{a})((b_i - \bar{b})}{\sqrt{\sum_{i \in I} (a_i - \bar{a})^2*\sum_{i \in I} (b_i - \bar{b})^2}}$$ \\
  \alert{Similarité cosinus} :\\
    $$cos(a,b)=\frac{a\cdot b^T}{||a||*||b||}$$
  
\end{frame}

\begin{frame}{Score par similarité -- Sélection des utilisateurs}
  \label{collabf:select_users}
  On choisit un groupe $U$ de $N$ utilisateurs avec la similarité la plus grande avec notre utilisateur cible $u$.
\end{frame}

\newcommand{\colorline}[1]{%
  \tikz
    \draw[#1 ,very thick] (0,0) -- (0.2,0) ;%
}
\begin{frame}{Score par similarité -- Ratings}
  \label{collabf:rating_sim}
  On estime le rating $r_{uj}$ d'un film $j$ non vu par l'utilisateur $u$ en fonction des autres utilisateurs $U$ :
  $$
    r_{uj} = {\color{red}\bar{r}_u} + \frac{\sum_{u' \in U} {\color{orange} sim(u,u')} {\color{blue}(r_{u'j} - \bar{r}_{u'})}}{\sum_{u' \in U} sim(u,u')}
  $$
  \begin{minipage}[t]{0.49\linewidth}
    \begin{itemize}
      \item[\colorline{red}] Moyenne des notes données par $u$
      \item[\colorline{orange}] Similarité entre $u$ et les autres utilisateurs utilisée comme pondération
    \end{itemize}
  \end{minipage}
  \begin{minipage}[t]{0.49\linewidth}
    \begin{itemize}
      \item[\colorline{blue}] Écart sur tous les films à la note moyenne donnée par $u'$. Certains utilisateurs notent plus sévèrement que d'autres
    \end{itemize}
  \end{minipage}
  
\end{frame}

\begin{frame}{Score par factorisation de matrice}
  On repart des données avec une approche différente
  \begin{center}
    \begin{tabular}{|l|c|c|c|c|}
      \hline
      & film1 & film2 & film3 & film4 \\
      \hline
      u1 & ? & 2 & 5 & ? \\
      \hline
      u2 & 5 & 1 & 4 & 3 \\
    \hline
    u3 & ? & ? & 1 & 5 \\
    \hline
    u4 & 3 & 3 & ? & 4 \\
    \hline
    \end{tabular}\\
    \end{center}
\end{frame}

\begin{frame}{Factorisation de matrice -- Concept}
  Pour prédire, on va supposer que les utilisateurs et les éléments sont résumables sur un \alert{ensemble de caractéristiques} et que chaque film possède ou non ces caractéristiques.
  \vfill
  \textbf{SAUF} qu'il s'agira de caractéristique purement mathématiques contenues dans des \alert{vecteur d'embeddings}.
\end{frame}

\begin{frame}{Factorisation de matrice -- Concept}
  \V{["img/collab-filt-matrix-factorization", "tw", 0.9]| image}
\end{frame}

\begin{frame}{Factorisation de matrice -- Méthodes}
  Trois différentes approches pour trouver les vecteurs d'embeddings~:
  \begin{itemize}
    \item SVD 
    \item Moindre carrés
    \item Regression par descente de gradient
  \end{itemize}
\end{frame}

\begin{frame}
  \frametitle{SVD}
  Utiilisation de SVD pour réduire le nombre de dimensions du problème.
  $M = U^p*V^p*I^p$,  où : \\
  $M$ est la matrice user/item de $\mathbf{R}^{n*m}$ \\
  $U^p$ la matrice des vecteurs de base orthonormée de $\mathbf{R}^{n}$ (input) \\
  $V^p$ la matrice diagonale des valeurs singulières de $U$ \\
  $I^p$ la matrice des vecteurs de base orthonormée de $\mathbf{R}^{m}$ (output) \\
  M = 
  \begin{tabular}{|c|c|c|}
    \hline
    2 & 5 & ? \\
    \hline
    1 & 4 & 3 \\
    \hline
    ? & 1 & 5 \\
    \hline
    3 & ? & 4 \\
    \hline
  \end{tabular}
  =
  \begin{center}
  \begin{tabular}{|c|c|}
    \hline
    $U^p_{(1,1)}$ & $U^p_{(1,2)}$  \\
    \hline
    $U^p_{(2,1)}$ & $U^p_{(2,2)}$  \\
    \hline
    $U^p_{(3,1)}$ & $U^p_{(3,2)}$  \\
    \hline
    $U^p_{(4,1)}$ & $U^p_{(4,2)}$  \\
    \hline
  \end{tabular}
  *
  \begin{tabular}{|c|c|}
    \hline
    $V^p_{(1)}$ & 0  \\
    \hline
    0 & $V^p_{(2)}$  \\
    \hline
  \end{tabular}
  *
  \begin{tabular}{|c|c|c|}
    \hline
    $I^p_{(1,1)}$ & $I^p_{(1,2)}$ & $I^p_{(1,3)}$  \\
    \hline
    $I^p_{(2,1)}$ & $I^p_{(2,2)}$ & $I^p_{(2,3)}$  \\
    \hline
  \end{tabular}
  \end{center}
\end{frame}

\begin{frame}
  \frametitle{SVD}
  Avantages : \\
  On choisit le nombre de valeurs singulières de notre transformation (PCA). \\
  Filtrage collaboratif user-based $\Rightarrow U^p$\\
  Filtrage collaboratif item-based $\Rightarrow I^p$
\end{frame}

\begin{frame}
  \frametitle{SVD}
  La SVD est extrèmement coûteuse sur des grandes matrices :
  \begin{center}
    $O(min(m*n^2,m^2*n))$
  \end{center}
  Heureusement des algorithmes approchés existent. Ces approches tirent parti du fait que les matrices sont généralement sparses et que l'on ne s'intéresse qu'a un nombre limité de valeurs singulières. \\
  Par exemple, on peut calculer les 20 premières valeurs singulières d'une matrice 100k x 100k avec 1M valeurs non-nulles en moins d'une seconde (redsvd).
\end{frame}

\begin{frame}{Recherche des embeddings par regression}
  Utilisation d'Embeddings sur les utilisateurs et les items afin de pouvoir exprimer la matrice des ratings ainsi : \\
  $M_{i,j} \approx U^{embed}*I^{embed}$   
\end{frame}

\begin{frame}{Recherche des embeddings par regression}
  Trois fonctions objectif possibles :
  \begin{itemize}
    \item Ignorer les valeurs non observées
    \item Considérer les valeurs non observées comme valeurs nulles
    \item Considérer les valeurs non observées comme un constant
  \end{itemize} 
\end{frame}

\begin{frame}{Recherche des embeddings par regression}
  Ignorer les valeurs non observées :
  \begin{center}
    \begin{tabular}{|l|c|c|c|c|}
      \hline
      & film1 & film2 & film3 & film4 \\
      \hline
      u1 & ? & \cellcolor{NavyBlue!30}2 & \cellcolor{NavyBlue!30}5 & ? \\
      \hline
      u2 & \cellcolor{NavyBlue!30}5 & \cellcolor{NavyBlue!30}1 & \cellcolor{NavyBlue!30}4 & \cellcolor{NavyBlue!30}3 \\
      \hline
      u3 & ? & ? & \cellcolor{NavyBlue!30}1 & \cellcolor{NavyBlue!30}5 \\
      \hline
      u4 & \cellcolor{NavyBlue!30}3 & \cellcolor{NavyBlue!30}3 & ? & \cellcolor{NavyBlue!30}4 \\
      \hline
    \end{tabular}\\
  \end{center}

  \alert{Fonction de coût}:

  $$\sum_{(i,j) \in obs}(M_{ij} - U_i\cdot V_j)^2$$
\end{frame}

\begin{frame}{Recherche des embeddings par regression}
  Remplacer les valeurs non observées par des valeurs nulles:
  \begin{center}
    \begin{tabular}{|l|c|c|c|c|}
      \hline
      & film1 & film2 & film3 & film4 \\
      \hline
      u1 & \cellcolor{NavyBlue!30}? & \cellcolor{NavyBlue!30}2 & \cellcolor{NavyBlue!30}5 & \cellcolor{NavyBlue!30}? \\
      \hline
      u2 & \cellcolor{NavyBlue!30}5 & \cellcolor{NavyBlue!30}1 & \cellcolor{NavyBlue!30}4 & \cellcolor{NavyBlue!30}3 \\
      \hline
      u3 & \cellcolor{NavyBlue!30}? & \cellcolor{NavyBlue!30}? & \cellcolor{NavyBlue!30}1 & \cellcolor{NavyBlue!30}5 \\
      \hline
      u4 & \cellcolor{NavyBlue!30}3 & \cellcolor{NavyBlue!30}3 & \cellcolor{NavyBlue!30}? & \cellcolor{NavyBlue!30}4 \\
    \hline
    \end{tabular}\\
    \end{center}

  \alert{Fonction de coût}:

  $$\sum_{(i,j)}(M_{ij} - U_i\cdot V_j)^2$$
\end{frame}

\begin{frame}{Recherche des embeddings par regression}
  Remplacer les valeurs non observées par des constantes:
  \begin{center}
    \begin{tabular}{|l|c|c|c|c|}
      \hline
      & film1 & film2 & film3 & film4 \\
      \hline
      u1 & \cellcolor{Auburn!30}? & \cellcolor{NavyBlue!30}2 & \cellcolor{NavyBlue!30}5 & \cellcolor{Auburn!30}? \\
      \hline
      u2 & \cellcolor{NavyBlue!30}5 & \cellcolor{NavyBlue!30}1 & \cellcolor{NavyBlue!30}4 & \cellcolor{NavyBlue!30}3 \\
      \hline
      u3 & \cellcolor{Auburn!30}? & \cellcolor{Auburn!30}? & \cellcolor{NavyBlue!30}1 & \cellcolor{NavyBlue!30}5 \\
      \hline
      u4 & \cellcolor{NavyBlue!30}3 & \cellcolor{NavyBlue!30}3 & \cellcolor{Auburn!30}? & \cellcolor{NavyBlue!30}4 \\
    \hline
    \end{tabular}\\
    \end{center}

  \alert{Fonction de coût}:

  $${\color{NavyBlue!50}\sum_{(i,j) \in obs}(M_{ij} - U_i\cdot V_j)^2} + {\color{Auburn!50}\sum_{(i,j) \notin obs}(0 - U_i\cdot V_j)^2} $$
\end{frame}

