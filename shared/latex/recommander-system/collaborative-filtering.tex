\begin{frame}{Introduction}
  La différence entre le content-based filtering et le filtrage collaborative se situe sur l'objet sur lequel on mesure la similarité.\\[2ex]

  Le \alert{filtrage collaborative} mesure la similarité entre les utilisateurs plutôt qu'entre les contenus à recommander.
\end{frame}
\begin{frame}{Introduction}
  \V{["tikz/recommander-system/collaborative-filtering", "tw", 1] | image}
\end{frame}

\begin{frame}{Données}
  Exemple de la base de données :
  \begin{center}
  \begin{tabular}{|l|c|c|c|c|}
    \hline
     & film 1 & film 2 & film 3 & film 4 \\
    \hline
    u1 & ? & 2 & 5 & ? \\
    \hline
    u2 & 5 & 1 & 4 & 3 \\
    \hline
    u3 & ? & ? & 1 & 5 \\
    \hline
    u4 & 3 & 3 & ? & 4 \\
    \hline
  \end{tabular}
  \end{center}
\end{frame}

\begin{frame}
  \frametitle{Principe}
  Principe : Une bonne recommandation est faite à partir d'individus similaires\\
  \newline
  Condition : Avoir un échantillon important d'individus représentatifs
  \V{["img/logos/amazon", "tw", 0.6] | image}
\end{frame}

\begin{frame}
  \frametitle{Principe}
  \begin{center}
    \begin{tabular}{|l|c|c|c|c|}
      \hline
      & film1 & film2 & film3 & film4 \\
      \hline
      u1 & ? & 2 & 5 & ? \\
      \hline
      u2 & 5 & 1 & 4 & 3 \\
    \hline
    u3 & ? & ? & 1 & 5 \\
    \hline
    u4 & 3 & 3 & ? & 4 \\
    \hline
    \end{tabular}\\
    \end{center}
  Comment prédire les notes manquantes ? (et donc faire une recommandation)
\end{frame}

\begin{frame}{Prédiction}
  Pour prédire, on va supposer que les utilisateurs et les éléments sont résumables sur un ensemble de caractéristiques contenu dans un \alert{vecteur d'embeddings}.
\end{frame}

\begin{frame}
  \frametitle{user-based}
  Similarité de corrélation (Pearson):\\
  \begin{center}
    $cor(a,b)=\frac{ \sum_{i \in I} (a_i - \bar{a})((b_i - \bar{b})}{\sqrt{\sum_{i \in I} (a_i - \bar{a})^2*\sum_{i \in I} (b_i - \bar{b})^2}}$ \\
  \end{center}
  Similarité cosinus :\\
  \begin{center}
    $cos(a,b)=\frac{a*b^t}{|a|*|b|}$
  \end{center}
  Et plein d'autres dans la littérature...
\end{frame}

\begin{frame}
  \frametitle{user-based}
  Agrégation des notes des K utilisateurs retenus:
  \begin{itemize}
  \item Moyenne
  \item Moyenne pondéré par la similarité
  \item ...
  \end{itemize}
\end{frame}

\begin{frame}
  \frametitle{user-based}
  \begin{itemize}[<+->]
  \item Des calculs simples mais il faut parcourir toute la base dès qu'on a une nouvelle entrée ou modification, heureusement des heuristiques existent (\href{https://en.wikipedia.org/wiki/Locality-sensitive_hashing}{Locality-sensitive hashing})
  \item ``Cold start problem''
  \end{itemize}
\end{frame}
