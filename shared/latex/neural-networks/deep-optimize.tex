\begin{frame}{Disparition du gradient (ou son explosion)}
  \V{["vanishing-gradient3", "tw", 1.0] | image}
\end{frame}

\begin{frame}{Disparition des gradients}
  Vanishing Gradient $\Rightarrow$ utilisation de ReLu plutôt que les sigmoïdes
  \vfill
  \V{["activations/relu", "tw", 1] | image}
\end{frame}

\begin{frame}{Explosion des gradients}
  Exploding Gradient $\Rightarrow$ Gradient clipping
  \V{["gradient-clipping", "tw", 0.9] | image}
  Dans le cadre des réseaux réccurents $\Rightarrow$ initialisation avec des matrices orthogonales
\end{frame}

\begin{frame}{Algorithme d'optimisation}
  Optimisateur plus rapide que SGD
  \begin{itemize}[<+(1)->]
    \item Adaptative Gradient Algorithm (AdaGrad)
    \item Root Mean Square Propagation (RMSProp)
    \item Adaptative Moment Estimation (Adam)    $\Leftarrow$
    \item ... AdaBound (2019) ?
  \end{itemize}
\end{frame}

\begin{frame}{Adaptative Gradient Algorithm (AdaGrad)}
  Calcul d'un learning rate adapté à chaque itération, pour chaque paramètre du modèle.
\end{frame}

\begin{frame}{Root Mean Square Propagation (RMSProp)}
  Utilisation d'inertie dans le gradient appliqué :
  $$
    Grad_t = Grad_t + Grad_{t-1}
  $$
\end{frame}
