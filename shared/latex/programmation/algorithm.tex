\begin{frame}{Algorithme}
    Un algorithme est une suite finie et non ambiguë d'instructions et d’opérations permettant de résoudre une classe de problèmes.
    \V{["img/al-khwarizmi-timbre", "th", 0.4] | image}    
\end{frame}

\begin{frame}{Algorithme}
    Un algorithme est défini par :
    \begin{itemize}
        \item Ses \textbf{entrées} (sous formes de \textbf{variables})
        \item Ses \textbf{sorties} (sous formes de \textbf{variables})
        \item Les étapes nécéssaires pour transformer les entrées en sorties
    \end{itemize}
    On peut alors en déduire son rendement
\end{frame}

\begin{frame}{Example d'algorithme en français}
    \begin{description}
        \item [Entrée] X : Une liste de nombres
        \item [Sorties] Y : Le nombre maximum dans la liste 
        \item [instructions]
        \begin{itemize}
            \item Index = 0
            \item $Y \leftarrow X[0]$ 
            \item  Comparez le nombre maximum supposé avec le nombre suivant de la liste.
            \item  Si le nombre suivant est supérieur au nombre maximum supposé, remplacez le nombre maximum supposé par le nombre suivant.
            \item  Répétez les étapes 2 et 3 jusqu'à ce que tous les nombres de la liste aient été comparés.
            \item  Renvoyez le nombre maximum supposé final comme le nombre maximum dans la liste.
        \end{itemize}
    \end{description}
\end{frame}



