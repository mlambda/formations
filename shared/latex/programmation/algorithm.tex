\begin{frame}{Définition}
    Un algorithme est une suite finie et non ambiguë d'instructions et d’opérations permettant de résoudre une classe de problèmes.
    \V{["img/al-khwarizmi-timbre", "th", 0.4] | image}    
\end{frame}

\begin{frame}{Caractéristiques}
    Un algorithme est défini par :
    \begin{itemize}
        \item Ses \textbf{entrées} (sous formes de \textbf{variables})
        \item Ses \textbf{sorties} (sous formes de \textbf{variables})
        \item Les étapes nécéssaires pour transformer les entrées en sorties
    \end{itemize}
    On peut alors en déduire son rendement
\end{frame}

\begin{frame}{Quelques instructions en français}
    
    \begin{description}[<+->]
        \item [\blue{ACCES}] $X[5]\;\;\;\;\;$ \green{"On accède au 5ème élément de la liste $X$"}
        \item []
        \item [\blue{AFFECTATION}] $Y \leftarrow X\;$  \green{"On affecte à $Y$ la valeur contenue dans $X$"}
        \item []
        \item [\blue{TEST}] 
        \begin{description}
            \item [SI] $X == Y$
            \item [ALORS] $X \leftarrow X + 10$
            \item [SINON] $Y \leftarrow X - 12$
        \end{description}
        \item []
        \item [\blue{BOUCLE}] 
        \begin{description}
            \item [TANT QUE] $X > Y$
            \item [ALORS] $X \leftarrow X - 1$
        \end{description}
    \end{description}

    
\end{frame}

\begin{frame}{Example d'algorithme en français}
    \begin{description}
        \item [Entrée] X : Une liste de 10 nombres
        \item [Sorties] Y : Le nombre maximum dans la liste 
        \item [instructions]
        \begin{itemize}
            \item $\text{IndexMax} \leftarrow 9$
            \item $Y \leftarrow X[0]$ 
            \item TANT QUE $\text{Index}$ < $\text{IndexMax}$
            \begin{itemize}
                \item SI $Y < X[\text{Index}]$ ALORS $Y \leftarrow X[\text{Index}]$ 
                \item $\text{Index} \leftarrow \text{Index} + 1$
            \end{itemize}
            \item RENVOYER Y
        \end{itemize}
    \end{description}
\end{frame}



