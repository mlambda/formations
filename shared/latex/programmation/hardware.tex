\begin{frame}{Mémoire}
    Composant capable de stocker de l'information. \\
    \newline
    Différentes \textbf{bandes de mémoire} :
    \begin{description}
        \item[Cache] Mémoire interne du processeur
        \item[RAM] Mémoire vive qui sera utilisée par les programmes
        \item[Disque] Mémoire de stockage
        \item[autres supports] disquette, clé usb, etc ...
    \end{description}
\end{frame}

\begin{frame}{Processeur}
    Composant capable d'effectuer des opérations arithmétiques et logiques :
    \begin{itemize}
        \item Gestion de \textbf{la bande de mémoire}
        \begin{itemize}
            \item déplacer la tête de lecture
            \item lire/écrire
            \item ...
        \end{itemize}
        \item Opération sur des \textbf{variables} stockées en \textbf{Cache} :
        \begin{itemize}
            \item ET
            \item OU
            \item XOR
            \item ...
        \end{itemize}
    \end{itemize}
    Ces opérations sont utilisées dans un ordre précis, décrit par un programme.
\end{frame}

\begin{frame}{Synthèse des différentes mémoires}
    \V{"img/memory-hierarchy" | image("th", 0.5)}
    Utile uniquement à garder en tête quand, on cherche à optimiser un programme !
\end{frame}
