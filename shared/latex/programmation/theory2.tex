
\begin{frame}{Le codage des entiers}
    \textbf{Exemple 1 :}
    \begin{center}
        La table de codage : \\
        $\;$\\
        \begin{tabular}{|*{2}{l||r|r|r|r|r|r|r|r|}}
            \hline
                table de codage & 128 & 64 & 32 & 16 & 8 & 4 & 2 & 1 \\
            \hline
                un code         & 0   & 0  & 0  & 0  & 0 & 0 & 0 & 1 \\
            \hline
        \end{tabular} \\
        $\;$\\
        Le nombre en question : \textbf{1} \\
    \end{center}
    \textbf{Exemple 2 :}
    \begin{center}
        La table de codage : \\
        $\;$\\
        \begin{tabular}{|*{2}{l||r|r|r|r|r|r|r|r|}}
            \hline
                table de codage & 128 & 64 & 32 & 16 & 8 & 4 & 2 & 1 \\
            \hline
                un code         & 0   & 0  & 0  & 0  & 0 & 0 & 0 & 0 \\
            \hline
        \end{tabular} \\
        $\;$\\
        Le nombre en question : \textbf{0} \\
    \end{center}
\end{frame}

\begin{frame}{Le codage des entiers}
    \textbf{Exemple 2 :}
    \begin{center}
        La table de codage : \\
        $\;$\\
        \begin{tabular}{|*{2}{l||r|r|r|r|r|r|r|r|}}
            \hline
                table de codage & 128 & 64 & 32 & 16 & 8 & 4 & 2 & 1 \\
            \hline
                un code         & 0   & 0  & 0  & 0  & 0 & 1 & 0 & 1 \\
            \hline
        \end{tabular} \\
        $\;$\\
        Le nombre en question : \textbf{5} \\
    \end{center}
    \textbf{Exemple 3 :}
    \begin{center}
        La table de codage : \\
        $\;$\\
        \begin{tabular}{|*{2}{l||r|r|r|r|r|r|r|r|}}
            \hline
                table de codage & 128 & 64 & 32 & 16 & 8 & 4 & 2 & 1 \\
            \hline
                un code         & 0   & 0  & 0  & 1  & 0 & 0 & 1 & 0 \\
            \hline
        \end{tabular} \\
        $\;$\\
        Le nombre en question : \textbf{18} \\
    \end{center}
\end{frame}


\begin{frame}{Le codage des entiers}
    \textbf{Exemple 4 :}
    \begin{center}
        La table de codage : \\
        $\;$\\
        \begin{tabular}{|*{2}{l||r|r|r|r|r|r|r|r|}}
            \hline
                table de codage & 128 & 64 & 32 & 16 & 8 & 4 & 2 & 1 \\
            \hline
                un code         & 0   & 0  & 0  & 0  & 1 & 1 & 1 & 1 \\
            \hline
        \end{tabular} \\
        $\;$\\
        Le nombre en question : \textbf{?} \\
    \end{center}
    \textbf{Exemple 5 :}
    \begin{center}
        La table de codage : \\
        $\;$\\
        \begin{tabular}{|*{2}{l||r|r|r|r|r|r|r|r|}}
            \hline
                table de codage & 128 & 64 & 32 & 16 & 8 & 4 & 2 & 1 \\
            \hline
                un code         & 1   & 1  & 1  & 1  & 1 & 1 & 1 & 1 \\
            \hline
        \end{tabular} \\
        $\;$\\
        Le nombre en question : \textbf{?} \\
    \end{center}
\end{frame}

\begin{frame}{Le codage des entiers}
    \textbf{Exemple 4 :}
    \begin{center}
        La table de codage : \\
        $\;$\\
        \begin{tabular}{|*{2}{l||r|r|r|r|r|r|r|r|}}
            \hline
                table de codage & 128 & 64 & 32 & 16 & 8 & 4 & 2 & 1 \\
            \hline
                un code         & 0   & 0  & 0  & 0  & 1 & 1 & 1 & 1 \\
            \hline
        \end{tabular} \\
        $\;$\\
        Le nombre en question : \textbf{15} \\
    \end{center}
    \textbf{Exemple 5 :}
    \begin{center}
        La table de codage : \\
        $\;$\\
        \begin{tabular}{|*{2}{l||r|r|r|r|r|r|r|r|}}
            \hline
                table de codage & 128 & 64 & 32 & 16 & 8 & 4 & 2 & 1 \\
            \hline
                un code         & 1   & 1  & 1  & 1  & 1 & 1 & 1 & 1 \\
            \hline
        \end{tabular} \\
        $\;$\\
        Le nombre en question : \textbf{255} \\
    \end{center}
\end{frame}

\begin{frame}{Le codage d'informations complexes}
    On organise ces variables en groupes, très souvent hommogènes :
    \begin{description}[<+->]
        \item [Les COLLECTIONS] des ensembles d'éléments uniques, sans ordre et très pratiques pour les opération d'union ou intersection.
        \item [Les LISTES] un vecteur de nombres, une chaine de caractère, une matrice de pixels, ...
        \item [ETC ...]
    \end{description}
\end{frame}

\begin{frame}{L'indexation de Liste}
    \V{["tikz/math/tensor1D", "tw", 1.0] | image}
\end{frame}

\begin{frame}{L'indexation de Matrice}
    \V{["tikz/math/tensor2D", "tw", 1.0] | image}
\end{frame}

\begin{frame}{L'indexation de Tenseur 3D}
    \V{["tikz/math/tensor3D", "tw", 1.0] | image}
\end{frame}
