\begin{frame}{Définition}
    On résoud des \textbf{problèmes} avec des algorithmes
    \V{"img/al-khwarizmi-timbre" | image("th", 0.4)}    
\end{frame}

\begin{frame}{Machine de Turing}
    \V{"img/turing-machine" | image("th", 0.7)}
\end{frame}

\begin{frame}{Le codage de l'information}
    On formalise les problèmes à l'aide de variables définies par :
    \begin{description}[<+->]
        \item [Le TYPE] nombre entier, nombre réel, caractère, ...
        \item [La TAILLE] en bits : 8 bits, 32 bits, 64 bits, ...
        \item [Le CODAGE] 
        \begin{itemize}
            \item signé ou non-signé pour les nombres
            \item ASCII, UTF8 pour les caractères
            \item ...
        \end{itemize}
    \end{description}
\end{frame}

\begin{frame}{Le codage d'informations complexes}
    On organise ces variables en groupes, très souvent hommogènes :
    \begin{description}[<+->]
        \item [Les LISTES] un vecteur de nombres, une chaine de caractère, une matrice de pixels, ...
        \item [Les COLLECTIONS] des ensembles d'éléments uniques, sans ordre et très pratiques pour les opération d'union ou intersection.
        \item [ETC ...]
    \end{description}
\end{frame}
