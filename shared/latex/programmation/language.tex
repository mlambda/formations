\begin{frame}{Langages de programmation}
    \begin{itemize}
        \item Un alphabet
        \item Un vocabulaire
        \item Des règles de grammaire
        \item environnement de traduction vers le "langage" machine : \textbf{le compilateur}
    \end{itemize}
\end{frame}

\begin{frame}{Les paradigmes de programmation}
    \begin{description}
        \item [Procédural] Au plus proche des instructions du processeur \\(C, Fortran, Cobol, ...)
        \item [Déclaratif]
        \begin{itemize}
            \item Fonctionnel : résolution de lambda-calcul par la récursivité (Haskell, Ocaml, ...)
            \item Logique : résolution de prédicats par des règles déductives (Prolog, ...)
        \end{itemize}
        \item [Orienté Objet] Introduction de la notion d'Objet et d'héritage \\(Java, Swift, ...)
        \item [Visuel] Le program est défini par un schéma \\(Delphi, Smalltalk, ...)
        \item [Basé web] Le programme est envoyé à travers le web pour être exécuté chez le client (PHP, Javascript, ...)
        \item [ETC ...]
    \end{description}
\end{frame}
