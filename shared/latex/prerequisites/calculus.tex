\begin{frame}{Utilité}
  Souvent besoin de minimiser une fonction en machine learning.
\end{frame}

\begin{frame}{Idée clef}

  Décider d'un $x$ de départ puis suivre la pente jusqu'au minimum.

  \onslide<2->{Pente = dérivée}

  \onslide<3->{→ Modifier itérativement $x$ par un pas vers l'opposé
    de la dérivée.}
\end{frame}

\begin{frame}{Pente positive}

  \V{"plt/slope-positive" | image("tw", 0.5)}

  Opposé de la pente = $-2$. Avec un pas de $0,1$, on passe de $1$ à $0,8$.
\end{frame}

\begin{frame}{Pente négative}

  \V{"plt/slope-negative" | image("tw", 0.5)}

  Opposé de la pente = $2$. Avec un pas de $0,1$, on passe de $-1$ à $-0,8$.
\end{frame}

\begin{frame}{Exemple en 2 dimensions}
  Dérivée → gradient
  \V{"plt/gradient-descent" | image("th", 0.65)}
\end{frame}
