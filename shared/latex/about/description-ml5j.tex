\begin{frame}
  \frametitle{description}
  Cette formation présente les \textbf{fondamentaux du machine
    learning} ainsi que les principales \textbf{techniques utilisées
    dans l’industrie}.
\end{frame}

\begin{frame}
  \frametitle{profil des stagiaires}
  Développeurs, ingénieurs informatiques désireux d’utiliser les
  techniques d’apprentissage automatique pour exploiter les données à
  leur disposition. \\
  \newline
  Bon niveau général en informatique, à l’aise en programmation. \\
  \begin{center}
    \green{Avez-vous un compte google ?}\\
    $\;$\\
    \green{Connaissez-vous le Python ?}
  \end{center}
\end{frame}

\begin{frame}
  \frametitle{objectifs à atteindre}

  \begin{itemize}
  \item poser un problème de machine learning
  \item prétraiter des données
  \item construire des modèles d'apprentissage pour des données
    annotées comme non-annotées
  \item gérer les apprentissages de vos modèles
  \item extraire des résultats actionnables
  \end{itemize}
\end{frame}

\begin{frame}
  \frametitle{programme}

  \begin{itemize}
  \item Introduction au machine learning
  \item Notions de maths
  \item Fondamentaux
  \item Régressions linéaire et logistique
  \item Machine à Vecteurs de Support (SVM)
  \item Arbres de décision
  \item Réduction de dimensionnalité et clustering
  \item Détection d'anomalies
  \item Réseaux de neurones
  \item Embeddings
  \item Système de recommandation
  \end{itemize}
\end{frame}
