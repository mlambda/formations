
\begin{frame}{Description}
  Ce séminaire présente les principales approches et outils de l'intelligence dans la résolution de problèmes, avec de nombreux exemples.
\end{frame}

\begin{frame}{Prérequis}
  Bonne connaissance en gestion d'un projet numérique.
  Expérience requise.
\end{frame}

\begin{frame}{Objectifs pédagogiques}
  \begin{itemize}
  \item Comprendre réellement ce que sont les outils Machine et Deep Learning, leurs potentiels et leurs limites
  \item Avoir une vision à date de l'état de l'art de ces domaines
  \item Connaître et comprendre les applications de ces domaines à différents domaines de l'industrie
  \item Maîtriser les méthodologies et connaître les outils propres aux projets d'intelligence artificielle
  \end{itemize}
\end{frame}

\begin{frame}{Programme}
  \begin{itemize}
  \item Qu'est-ce que l'Intelligence Artificielle
  \item Réseaux de neurones et Deep Learning
  \item Applications du Deep Learning
  \item Quels problèmes peut-on adresser avec le Machine/Deep Learning
  \item Génération d'un Dataset
  \item Recherche de la solution optimale
  \item Les outils
  \end{itemize}
\end{frame}
