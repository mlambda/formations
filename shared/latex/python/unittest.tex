\begin{frame}{Introduction}
  Buts~:

  \begin{itemize}[<+->]
    \item Disposer d'un cadre logiciel de test similaire à JUnit en Java dans la bibliothèque standard
    \item Gérer le cycle de vie en 4 phases d'un test
  \end{itemize}
\end{frame}

\begin{frame}{Principes}
  \begin{itemize}[<+->]
    \item Définition de scénarios de tests en créant des classes héritant de \texttt{unittest.TestCase}
    \item Chaque scénario comporte plusieurs tests
    \item Chaque test est défini avec des méthodes \texttt{assert…} spécifiques
    \item Préparation et nettoyage au niveau de la fonction avec les méthodes \texttt{setUp} et \texttt{tearDown}
    \item Préparation et nettoyage au niveau de la classe avec les méthodes \texttt{setUpClass} et \texttt{tearDownClass}
  \end{itemize}
\end{frame}

\begin{frame}{Méthodes de test}
  \bluelink{https://docs.python.org/fr/3/library/unittest.html\#test-cases}{Un grand nombre de méthodes de test sont disponibles.}
\end{frame}

\begin{frame}{Ordre d'exécution}

  \begin{enumerate}[<+->]
    \item \texttt{setUpClass}
    \item \texttt{setUp}
    \item La fonction de test
    \item \texttt{tearDown}
    \item \texttt{tearDownClass}
  \end{enumerate}

  \onslide<+->{Avec 2--4 répétés autant de fois qu'il y a de tests avant l'exécution de 5.}
\end{frame}

\begin{frame}{Exemple d'utilisation}
  \mintedpycode{python/unittest/simple_usage}
\end{frame}

\begin{frame}{Appel de la suite de test}
  Pour faire tourner la suite de tests, on utilise le plus souvent~:

  \mintedcustomcode{python/unittest/run}{console}

  Il est aussi possible de spécifier un fichier, une classe ou une fonction à exécuter.
\end{frame}

\begin{frame}{Module de mocking}
  \texttt{unittest} dispose d'un module de mocking très intéressant~: \texttt{unittest.mock}.

  Deux points importants~:
  \begin{itemize}[<+->]
    \item Le décorateur / gestionnaire de contexte \texttt{patch}, qui permet de remplacer temporairement un objet
    \item La classe \texttt{Mock}, qui permet de remplacer des objets intestables
  \end{itemize}
  \onslide<+->{Il est possible de tester quelles méthodes ont été appelées, et comment, sur un objet \texttt{Mock}.}
\end{frame}
