\begin{frame}{Introduction}
  But~:

  \begin{itemize}[<+->]
    \item Personnaliser la création de classes
    \item Donner beaucoup (sûrement trop) de flexibilité
  \end{itemize}
\end{frame}

\begin{frame}{Mécanisme}
  La méta-classe fonctionne comme une classe pour un objet --- elle gouverne de la même manière le comportement de la classe~:
  \begin{itemize}[<+->]
    \item la fonction \texttt{\_\_new\_\_} de la méta-classe est appelée pour créer l'objet représentant la classe
    \item cet objet est initialisé avec la fonction \texttt{\_\_init\_\_} de la méta-classe
    \item sa fonction \texttt{\_\_call\_\_} est appelée à chaque création d'un objet
  \end{itemize}

  \onslide<+->{La méta-classe par défaut est \texttt{type}}
\end{frame}

\begin{frame}{Quelle méthode surcharger ?}
  \begin{description}[<+->]
    \item[\texttt{\_\_new\_\_}] Pour modifier le processus de création de classe
    \item[\texttt{\_\_init\_\_}] Pour modifier la classe une fois qu'elle est créée
    \item[\texttt{\_\_call\_\_}] Pour modifier le processus de création d'objets
  \end{description}
\end{frame}

\begin{frame}{Possibilités}
  \begin{itemize}[<+->]
    \item Modifier les méthodes de la classe
    \item Modifier les parents de la classe
    \item Rajouter des attributs
    \item Personnaliser la création d'objets (retourner toujours la même instance par exemple)
    \item …
  \end{itemize}
\end{frame}

\begin{frame}{Conflits de méta-classes}
  Si deux méta-classes peuvent être appliquées pour créer une classe, un conflit apparait.

  Certains modules de la bibliothèque standard utilisent des méta-classes, cela limite leur utilisabilité.
\end{frame}

\begin{frame}{Utilité}
  Les méta-classes sont ce que Tim Peters (très grand contributeur de Python et inventeur du Timsort) appelle «~une solution en recherche de problème~».

  Il dit aussi~:

  Metaclasses are deeper magic than 99\% of users should ever worry about. If you wonder whether you need them, you don't (the people who actually need them know with certainty that they need them, and don't need an explanation about why).
\end{frame}

\begin{frame}{Alternatives}
  Les méta-classes étant assez complexes, il est préférable d'utiliser~:

  \begin{itemize}[<+->]
    \item Les décorateurs de classes
    \item La méthode \texttt{\_\_init\_subclass\_\_}, introduite dans Python 3.6
    \item L'héritage classique
  \end{itemize}
\end{frame}
