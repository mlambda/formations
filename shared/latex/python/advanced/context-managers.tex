\begin{frame}{Introduction}
  Buts~:
  \begin{itemize}[<+->]
    \item Introduire un mécanisme sûr de contexte pour un bloc d'instructions
    \item Éviter les erreurs courantes de gestion de ressources
  \end{itemize}
\end{frame}

\begin{frame}{Syntaxe}
  Instructions placées dans un bloc \texttt{with}. Celui-ci introduit un ou plusieurs contextes actifs.  
\end{frame}

\begin{frame}{Exemple --- Gestion de fichiers}
  \mintedpycode{python/context-managers/file}
\end{frame}

\begin{frame}{Exemple --- Contextes multiples}
  \mintedpycode{python/context-managers/multiple}
\end{frame}

\begin{frame}{Création d'un gestionnaire de contexte}
  Implémentation d'une classe avec des méthodes \texttt{\_\_enter\_\_} et \texttt{\_\_exit\_\_}.

  \mintedpycode{python/context-managers/creation-manual}
\end{frame}

\begin{frame}{Recommandé~: utilisation de contextlib}
  Façon beaucoup plus naturelle de définir un contexte~:

  \mintedpycode{python/context-managers/creation-contextlib}
\end{frame}

\begin{frame}{Contexte qui retourne une valeur}
  Pour retourner une valeur~:

  \mintedpycode{python/context-managers/value}
\end{frame}

\begin{frame}{Contextes déjà implémentés}
  Beaucoup d'objets possèdent habituellement les méthodes être utilisés en tant que contextes~:
  \begin{itemize}[<+->]
    \item Les fichiers
    \item Les sessions HTTP
    \item Les connexions aux bases de données
    \item …
  \end{itemize}
\end{frame}

\begin{frame}{Contextes apportés par \texttt{contextlib}}
  \texttt{contextlib} propose aussi quelques contextes très intéressants~:
  \begin{description}[<+->]
    \item[\texttt{closing}] Ferme ce qu'on lui donne en argument à la fin du bloc
    \item[\texttt{suppress}] Supprime les exceptions du bloc
    \item[\texttt{redirect\_stdout}] Redirige la sortie standard
  \end{description}
\end{frame}

\begin{frame}{Utilisation d'un contexte en tant que décorateur}
  En héritant de \texttt{contextlib.ContextDecorator} quand on définit un contexte, il peut aussi être utilisé comme décorateur.

  Les contextes définis avec \texttt{contextlib.contextmanager} ont aussi cette propriété.
\end{frame}
