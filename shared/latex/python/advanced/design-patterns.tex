\begin{frame}{Introduction}
  Les patrons de conceptions ou (\textit{design patterns}) sont des solutions reconnues comme bonnes à des problèmes courants.
\end{frame}

\begin{frame}{Origine}
  Livre \textit{Design Patterns – Elements of Reusable Object-Oriented Software} du «~Gang of Four~» (Erich Gamma, Richard Helm, Ralph Johnson et John Vlissides) paru en 1994.
\end{frame}

\begin{frame}{En Python}
  Les patrons nécessaires en Python sont moins nombreux qu'en Java ou d'autres langages similaires.
  On en retrouve tout de même plusieurs très régulièrement tels que les patrons~:

  \begin{itemize}
    \item composite (même nom en anglais)
    \item monteur (\textit{builder})
    \item adaptateur (\textit{adapter})
    \item fabrique (\textit{factory})
    \item itérateur (\textit{iterator})
    \item observateur (\textit{observer})
  \end{itemize}
\end{frame}

\begin{frame}{Formalisme}
  Dans leur ouvrage, le GoF présente chaque patron sous forme d'un diagramme UML.
  C'est aussi comme cela qu'ils sont le plus souvent présentés aujourd'hui.

  Ce n'est souvent pas adapté pour le langage Python, qui ne possède pas toutes les caractéristiques du Java (par exemple la visibilité).
\end{frame}

\begin{frame}{Démonstration}
  Étude de quelques patrons sur le site \bluelink{https://refactoring.guru/design-patterns/python}{refactoring.guru}.
\end{frame}
