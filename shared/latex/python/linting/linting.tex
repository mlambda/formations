\begin{frame}{Introduction}
  Buts~:
  \begin{itemize}[<+->]
    \item Respect des normes dans une équipe
    \item Minimiser les pertes de temps en revue de code
    \item Produire un logiciel plus maintenable
  \end{itemize}
\end{frame}

\begin{frame}{Possibilités}
  Les points majeurs que l'on vérifie habituellement~:

  \begin{itemize}[<+->]
    \item Types
    \item Éléments de code non utilisés
    \item Documentation non présente ou incomplète
    \item Motifs qui sont souvent des bugs
    \item Style du code
  \end{itemize}
\end{frame}

\begin{frame}{Outils}
  L'organisation PyCQA maintient plusieurs outils de vérification, parmi eux~:

  \begin{itemize}[<+->]
    \item \texttt{flake8}
    \item \texttt{pylint}
    \item \texttt{RedBaron}
  \end{itemize}

  \onslide<+->{On utilise aussi beaucoup \texttt{mypy} pour vérifier les types.}
\end{frame}

\begin{frame}{Typage}
  Sûrement le point le plus important à vérifier~:

  \begin{itemize}[<+->]
    \item Permet de modifier aggressivement le logiciel
    \item Résout une grande partie des bugs avant l'intégration à la base de code
    \item Documente le code
    \item Permet des générations automatiques (CLI avec Typer par exemple)
  \end{itemize}

  \onslide<+->{Le passage à une base de code typée peut être progressif.}
\end{frame}

\begin{frame}{Notations de typage}
  On type une expression en précédant l'annotation de type de \texttt{:}.

  Pour les retours de fonctions, on utilise \texttt{->}.

  \mintedpycode{python/linting/mypy}

  Des types additionnels sont présents dans le module \texttt{typing}.
\end{frame}

\begin{frame}{Vérifications générales}
  L'outil incontournable est \texttt{flake8}. En fonction des extensions installées, il vérifiera~:

  \begin{itemize}[<+->]
    \item La syntaxe
    \item Les éléments non utilisés
    \item Les imports
    \item La complexité du code
    \item …
  \end{itemize}
\end{frame}

\begin{frame}{Extensions de \texttt{flake8}}
  Quelques extensions très pratiques~:

  \begin{itemize}
    \item \texttt{flake8-bugbear}
    \item \texttt{flake8-docstrings}
    \item \texttt{flake8-import-order}
    \item \texttt{flake8-black}
    \item \texttt{flakehell}
  \end{itemize}

  Retrouvez une liste plus complète dans ce \bluelink{https://github.com/DmytroLitvinov/awesome-flake8-extensions}{dépôt \texttt{git}}.
\end{frame}

\begin{frame}{Configuration de \texttt{flake8}}
  \texttt{flake8} se configure dans le fichier \texttt{.flake8} à la racine de votre projet.

  \mintedcustomcode{python/linting/flake8}{ini}
\end{frame}

\begin{frame}{Vérification du style}
  Pour le style, je vous recommande fortement l'utilisation de \texttt{black}~:

  \begin{itemize}[<+->]
    \item Formatage automatique (transformation en AST puis en Python)
    \item Possibilité de vérifier l'application sur la base de code
    \item Style homogène
    \item Plus de problèmes de style en revue de code !
  \end{itemize}
\end{frame}

\begin{frame}{Vérification de la documentation}
  Utilisation de \texttt{pylint} avec seulement ces options~:

  \begin{itemize}[<+->]
    \item \texttt{missing-param-doc}
    \item \texttt{differing-param-doc}
    \item \texttt{differing-type-doc}
    \item \texttt{missing-return-doc}
  \end{itemize}

  \onslide<+->{Ainsi que \texttt{flake8-docstrings} pour le style des docstrings.}
\end{frame}

\begin{frame}{Préconisations d'utilisation}
  Recommandé~: regrouper les vérifications dans une seule commande (avec un Makefile par exemple).

  \begin{itemize}[<+->]
    \item Faire tourner ces vérifications avant chaque PR (hook git possible)
    \item Les intégrer à la CI
    \item \alert{Une PR ne doit pas être revue tant que ces soucis ne sont pas réglés}
  \end{itemize}
\end{frame}

\begin{frame}{Paramétrage}
  \begin{itemize}[<+->]
    \item Définir en équipe les points à vérifier
    \item Analyser régulièrement le coût/bénéfice de chaque élément vérifié
    \item Rajouter ou enlever des vérifications en fonction de ces analyses
  \end{itemize}
\end{frame}

\begin{frame}{Reproductibilité}
  Il faut gérer les outils de vérification comme des dépendances~:

  \begin{itemize}[<+->]
    \item Chaque collègue doit avoir les mêmes résultats étant donné une base de code
    \item Les outils doivent être faciles à installer, y compris sur la CI
  \end{itemize}

  \onslide<+->{Attention toutefois à ne pas installer ces dépendances en prod.}
\end{frame}
