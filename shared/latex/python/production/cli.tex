\begin{frame}{Introduction}
  Buts~:
  \begin{itemize}[<+->]
    \item Créer des outils facilement manipulables
    \item Pouvoir scripter à partir de vos programmes Python
  \end{itemize}
\end{frame}

\begin{frame}{Mécanismes}
  \begin{itemize}[<+->]
    \item \texttt{sys.argv} contient les arguments passés à votre programme
    \item \texttt{argparse}, \texttt{click} ou encore \texttt{Typer} parsent ce tableau pour vous
  \end{itemize}  
\end{frame}

\begin{frame}{\texttt{argparse}}
  Pour définir une ILC avec \texttt{argparse}, il faut~:
  \begin{itemize}[<+->]
    \item Créer un parser
    \item Appeler ce parser depuis le point d'entrée du programme
    \item Traiter les arguments parsés
  \end{itemize}
\end{frame}

\begin{frame}{Création d'un parser}
  Création d'un \texttt{argparse.ArgumentParser} puis ajout d'arguments avec la méthode \texttt{add\_argument}~:
  \mintedpycode{python/cli/creation}
\end{frame}

\begin{frame}{Ajouter un argument}
  On peut spécifier à \texttt{add\_argument} de nombreux points. Parmi eux~:

  \begin{itemize}[<+->]
    \item Une option courte (\texttt{"-d"}) et longue (\texttt{"--directory"})
    \item Une chaîne d'aide (\texttt{help="Aide de l'argument."})
    \item Un type (\texttt{type=int})
    \item Une action, pour créer des flags booléens par exemple
    \item Une valeur par défaut
  \end{itemize}

  \onslide<+->{La \bluelink{https://docs.python.org/fr/3/library/argparse.html\#the-add-argument-method}{documentation officielle} est très complète.}
\end{frame}

\begin{frame}{Commandes multiples}
  Possibilité de définir plusieurs commandes : création des sous-parsers que l'on traite ensuite comme le parser principal~:

  \mintedpycode{python/cli/subparsers}
\end{frame}

\begin{frame}{Utilisation}
  Appel de la méthode \texttt{parse\_args} puis traitement des arguments recupérés.

  \mintedpycode{python/cli/usage}
\end{frame}

\begin{frame}{Utilisation avancée qui couple la CLI à la fonction appelée}
  \mintedpycode{python/cli/usage-coupled}

  \begin{itemize}
    \item \texttt{vars} transforme l'objet \texttt{args} en dictionnaire
    \item \texttt{**} «~éclate~» le dictionnaire en mots-clefs et valeurs
    \item suppression préalable des arguments non-utilisés avec \texttt{delattr}
  \end{itemize}
\end{frame}

\begin{frame}{Quelques bonus apportés par \texttt{argparse}}
  \begin{itemize}[<+->]
    \item Aide générée automatiquement
    \item Messages d'erreurs informatifs quand les arguments sont incorrects
  \end{itemize}
\end{frame}
