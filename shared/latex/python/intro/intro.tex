\begin{frame}{Historique}
  \begin{description}
  \item[1989] Création du langage par Guido Van Russum
  \item[2001] Lancement de la Python Software Foundation
  \item[2001] Passage en GPL
  \item[2009] Python 3
  \end{description}
  \V{"img/logos/python" | image("tw", 0.8)}
\end{frame}

\begin{frame}{Caractéristiques}
  Python est :
  \begin{description}[<+(1)->]
  \item[Interprété] Et compilé à la volée, modules en C
  \item[Orienté objet] (mais pas que)
  \item[Portable] Compatible  avec toutes les plateformes actuelles
  \item[Flexible] Couteau suisse, de l'admin système au webdev
  \item[Populaire] Top 5 des langages les plus utilisés depuis des années
  \end{description}
\end{frame}

\begin{frame}{Points forts/faibles}
  \begin{minipage}[l]{0.49\linewidth}
    Atouts
    \begin{itemize}
    \item Stable
    \item Multi-plateforme
    \item Facile à apprendre
    \item Grande communauté (le plus utilisé depuis 2019)
    \item « Batteries included » : un besoin, un module
    \end{itemize}
  \end{minipage}\hfill
  \begin{minipage}[l]{0.49\linewidth}
    Inconvénients
    \begin{itemize}
    \item Non-compilé
      \begin{itemize}
      \item Plus lent qu'un langage bas-niveau
      \item Optimiser une opération $\Rightarrow$ pas facile à apprendre
      \end{itemize}
    \end{itemize}
  \end{minipage}\hfill
\end{frame}

\begin{frame}{Plates-formes}
  Différents interpréteurs~:
  \begin{itemize}
  \item CPython/Pypy $\Rightarrow$ C/C++
  \item Jython $\Rightarrow$ JVM
  \item IronPython $\Rightarrow$ .Net
  \end{itemize}
\end{frame}

\begin{frame}{Domaines}
  Domaines d'applications :
  \begin{itemize}
  \item Web (Django ,Flask, ...)
  \item Sciences (Data mining, Machine learning, Physique, ...)
  \item OS (Linux, Raspberry, Script administration système, ...)
  \item Éducation (Initiation à la programmation)
  \item CAO 3D (FreeCAD, pythonCAD, ...)
  \item Multimédia (Kodi, ...)
  \end{itemize}
\end{frame}

\begin{frame}{Environnements de travail}
  Les trois principaux IDEs Python sont~:

  \begin{description}[<+(1)->]
    \item[Visual Studio Code] IDE gratuit proposé par Microsoft
    \item[PyCharm] IDE en double version gratuite/professionnelle proposé par JetBrains
    \item[Spyder] IDE open source et gratuit orienté data science
  \end{description}

  \onslide<+(1)->{
    Ils intègrent tous les outils modernes de développement.

    Cependant vous pouvez bien sûr utiliser votre éditeur de texte favori (emacs, vim, atom, sublime text, …).
  }
\end{frame}

\begin{frame}{Versions 2 et 3 de Python}
  \begin{description}[<+->]
    \item[Version 2] Plus supportée depuis le premier janvier 2020. Toujours présente dans les bases legacy.
    \item[Version 3] Version actuelle de Python. Tous les nouveaux développements doivent l'utiliser.
  \end{description}
\end{frame}

\begin{frame}{Ressources}
  \begin{itemize}
    \item \bluelink{https://docs.python.org/fr/3/}{Documentation officielle Python}
    \item \bluelink{https://www.reddit.com/r/learnpython/}{Reddit (forum) d'apprentissage de Python}
    \item \bluelink{https://github.com/nzmognzmp/langage-python/releases/tag/1.0.1}{Support de cours, télécharger le pdf}
    \item \bluelink{http://pythontutor.com/}{Python Tutor}
  \end{itemize}
\end{frame}

\begin{frame}{Installation}
  Consulter le très bon \bluelink{https://realpython.com/installing-python/}{billet de blog} Real Python sur le sujet.
\end{frame}
