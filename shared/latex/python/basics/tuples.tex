\begin{frame}{Caractéristiques de \texttt{tuple}}
  Les n-uplets (\emph{tuples} en anglais), sont très similaires aux listes, mais sont~:

  \begin{itemize}[<+(1)->]
    \item Immuables
    \item Adaptés aux éléments hétérogènes
    \item Hashables si leurs éléments sont immuables
    \item Très utiles pour retourner plusieurs éléments dans une fonction
  \end{itemize}
\end{frame}

\begin{frame}{Utilisation d'un \texttt{tuple}}
  \mintedpycode{python/basics/tuples/usage}
\end{frame}

\begin{frame}{Définitions multiples avec un \texttt{tuple}}
  \mintedcustom{python/basics/tuples/multiple-definitions}{py}{text}
\end{frame}

\begin{frame}{Retours multiples avec un \texttt{tuple}}
  \mintedcustom{python/basics/tuples/multiple-returns}{py}{text}
\end{frame}

\begin{frame}{Échange de variables avec un \texttt{tuple}}
  \mintedcustom{python/basics/tuples/simple-swap}{py}{text}

  \onslide<+(1)->{
    Utilisé pour implémenter le tri à bulles~:

    \mintedpycode{python/basics/tuples/bubble-sort}
  }
\end{frame}
