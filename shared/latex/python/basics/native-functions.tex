\begin{frame}{Introduction}
  Certaines fonctions ne nécessitent pas d'import. Elles sont toujours disponibles.

  Leur liste complète est disponible dans la \bluelink{https://docs.python.org/fr/3/library/functions.html}{documentation Python}.
\end{frame}

\begin{frame}{Création et conversion de types natifs}
  \texttt{int}, \texttt{str}, \texttt{float}, …
\end{frame}

\begin{frame}{Différents mécanismes d'itération}
  \texttt{all}, \texttt{any}, \texttt{range}, \texttt{sum}, …
\end{frame}

\begin{frame}{Manipulation d'objets}
  \texttt{setattr}, \texttt{delattr}, …
\end{frame}

\begin{frame}{Test d'appartenance à une classe}
  \texttt{isinstance}, \texttt{issubclass}, \texttt{type}
\end{frame}

\begin{frame}{Opérations de base}
  \texttt{divmod}, \texttt{pow}
\end{frame}

\begin{frame}{Méthode d'apprentissage}
  Les fonctions natives, comme le reste de la bibliothèque standard, s'apprennent au fil des projets~:

  \begin{itemize}[<+(1)->]
    \item En code review, soyez curieu⋅x⋅se
    \item Consultez la documentation régulièrement
  \end{itemize}
\end{frame}
