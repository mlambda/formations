\begin{frame}{Introduction}
  Certaines fonctions ne nécessitent pas d'import. Elles sont toujours disponibles.

  Leur liste complète est disponible dans la \bluelink{https://docs.python.org/fr/3/library/functions.html}{documentation Python}.
\end{frame}

\begin{frame}{Création et conversion de types natifs}
  \pycon{python/basics/native-functions/basic-types}
\end{frame}

\begin{frame}{Mécanismes d'itération}
  \pycon{python/basics/native-functions/iteration}
\end{frame}

\begin{frame}{Mécanismes d'agrégation}
  \pycon{python/basics/native-functions/aggregation}
\end{frame}

\begin{frame}{Manipulation d'objets}
  \pycon{python/basics/native-functions/objects}
\end{frame}

\begin{frame}{Test d'appartenance à une classe}
  \pycon{python/basics/native-functions/type-checking}
\end{frame}

\begin{frame}{Opérations de base}
  \pycon{python/basics/native-functions/basic-operations}
\end{frame}

\begin{frame}{Méthode d'apprentissage}
  Les fonctions natives, comme le reste de la bibliothèque standard, s'apprennent au fil des projets~:

  \begin{itemize}[<+(1)->]
    \item En code review, soyez curieu⋅x⋅se
    \item Consultez la documentation régulièrement
  \end{itemize}
\end{frame}
