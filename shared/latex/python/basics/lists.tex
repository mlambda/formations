\begin{frame}{Caractéristiques de \texttt{list}}
  \begin{itemize}[<+->]
    \item Collection la plus utilisée en Python
    \item Peut contenir des éléments de types différents (\emph{très déconseillé})
    \item A de bonnes performances pour un usage standard
    \item Plus proche d'un tableau que d'une liste chaînée (accès $\mathcal{O}(1)$ par exemple)
    \item Pas adaptée au calcul scientifique (pour ça, utiliser \texttt{numpy})
  \end{itemize}
\end{frame}

\begin{frame}{Manipulation basique d'une liste}
  \mintedpycode{python/basics/lists/usage}
\end{frame}

\begin{frame}{Tranche de liste}
  Une tranche de liste sélectionne une sous-partie de la liste~:
  \mintedpycode{python/basics/lists/slice}
\end{frame}

\begin{frame}{Fonction \texttt{range}}
  La fonction \texttt{range} permet de créer des suites d'indices~:

  \mintedpycode{python/basics/lists/range}

  Par défaut elle renvoie des objets de type \texttt{range}.
\end{frame}
