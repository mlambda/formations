\begin{frame}{Introduction}
  Python utilise principalement trois structures de contrôle de flot~: \texttt{if}, \texttt{while} et \texttt{for}.

  Donc pas de \texttt{switch} comme dans d'autres langages.

  Nous  verrons plus tard qu'il existe aussi une structure \texttt{try}/\texttt{except} pour la gestion d'erreurs.

  Il existe aussi une instruction \texttt{match} depuis Python 3.10 mais ce n'est pas encore une version stable.
\end{frame}

\begin{frame}{Structure conditionnelle}
  Utilisation de \texttt{if}, \texttt{elif} et \texttt{else}~:
  \mintedpycode{python/basics/control-structures/if}
\end{frame}

\begin{frame}{Structure de boucle «~Pour dans~»}
  Utilisation de \texttt{for}~:
  \mintedpycode{python/basics/control-structures/for}
\end{frame}

\begin{frame}{Structure de boucle «~tant que~»}
  \mintedpycode{python/basics/control-structures/while}
\end{frame}

\begin{frame}{\texttt{break} \& \texttt{continue}}
  \begin{description}[<+->]
    \item[\texttt{break}] Utilisé pour sortir de la boucle la plus intérieure
    \item[\texttt{continue}] Utilisé pour passer directement à l'itération suivante de la boucle la plus intérieure
  \end{description}

  \onslide<+->{\mintedcustom{python/basics/control-structures/break-continue}{py}{text}}
\end{frame}

\begin{frame}{Clause \texttt{else} pour les boucles}
  La clause \texttt{else} est exécutée si on sort normalement de la boucle (sans \texttt{break})~:

  \mintedcustom{python/basics/control-structures/else}{py}{text}
\end{frame}
