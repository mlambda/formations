\begin{frame}{Introduction}
  Python utilise principalement trois structures de contrôle de flot~: \texttt{if}, \texttt{while} et \texttt{for}.

  Donc pas de \texttt{switch} comme dans d'autres langages.

  Nous  verrons plus tard qu'il existe aussi une structure \texttt{try}/\texttt{except} pour la gestion d'erreurs.

  Il existe aussi une instruction \texttt{match} depuis Python 3.10 mais ce n'est pas encore une version stable.
\end{frame}

\begin{frame}{Structure conditionnelle}
  Utilisation de \texttt{if}, \texttt{elif} et \texttt{else}~:
  \pycon{python/basics/control-structures/if}
\end{frame}

\begin{frame}{Structure de boucle «~Pour dans~» ⋅ Avec des indices}
  La fonction \texttt{range} permet de créer des suites d'indices~:

  \mintedpycode{python/basics/control-structures/range}

  Par défaut elle renvoie des objets de type \texttt{range}, plus efficaces que des listes.

  Utilisation dans une boucle~:

  \pycon{python/basics/control-structures/for-indices}
\end{frame}

\begin{frame}{Structure de boucle «~Pour dans~» ⋅ Avec une liste}
  \pycon{python/basics/control-structures/for-list}
\end{frame}

\begin{frame}{Structure de boucle «~Pour dans~» ⋅ Avec un dictionnaire}
  \pycon{python/basics/control-structures/for-dict}
\end{frame}

\begin{frame}{Structure de boucle «~Pour dans~» ⋅ Avec plusieurs itérables}
  Utilisation de la fonction \texttt{zip}~:

  \pycon{python/basics/control-structures/for-zip}

  Fonctionne avec un nombre arbitraire d'itérables.
\end{frame}

\begin{frame}{Structure de boucle «~tant que~»}
  \pycon{python/basics/control-structures/while}
\end{frame}

\begin{frame}{\texttt{break} \& \texttt{continue}}
  \begin{description}
    \item[\texttt{break}] Utilisé pour sortir de la boucle la plus intérieure
    \item[\texttt{continue}] Utilisé pour passer directement à l'itération suivante de la boucle la plus intérieure
  \end{description}

  \pycon{python/basics/control-structures/break-continue}
\end{frame}

\begin{frame}{Clause \texttt{else} pour les boucles}
  La clause \texttt{else} est exécutée si on sort de la boucle sans \texttt{break}~:

  \pycon{python/basics/control-structures/else}
\end{frame}
