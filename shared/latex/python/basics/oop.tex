\begin{frame}{Introduction}
  La programmation orientée objet est un style de programmation ayant pour buts~:

  \begin{itemize}
    \item l'encapsulation
    \item la modularité
    \item la réutilisabilité du code
  \end{itemize}
\end{frame}

\begin{frame}{Idée clef}
  Définition d'\alert{objets} qui regroupent~:

  \begin{itemize}
    \item des données
    \item des méthodes pour traiter ces données
  \end{itemize}
\end{frame}

\begin{frame}{Définition d'objets}
  Pour définir des objets, on écrit une classe, celle-ci est une «~recette~» pour obtenir un objet (qu'on appelle aussi instance)~:

  \mintedpycode{python/basics/oop/basic-class}

  \begin{itemize}
    \item \texttt{Point} est le nom de classe
    \item \texttt{\_\_init\_\_} est une méthode spéciale pour initialiser notre objet
    \item \texttt{self.x} \& \texttt{self.y} sont des attributs d'instance
    \item \texttt{distance} est une méthode arbitraire que l'on a définie
  \end{itemize}
\end{frame}

\begin{frame}{Utilisation d'objets}
  Une fois la classe définie, on peut l'utiliser comme tout autre objet Python~:

  \pycon{python/basics/oop/basic-class-usage}
\end{frame}

\begin{frame}{Différence entre attributs d'instance et de classe}
  \pycon{python/basics/oop/class-instance-attributes}

  Si un attribut d'instance a le même nom qu'un attribut de classe, il est prioritaire (\texttt{objet.nom} référera à l'attribut d'instance).
\end{frame}

\begin{frame}{Quelques points importants}
  \begin{itemize}
    \item On nomme le premier argument \texttt{self} par convention
    \item Pas de notion de méthodes ou d'attributs privés en Python
    \item Pour \alert{dire} qu'un un attribut ou une méthode sont privés, on ajoute un \texttt{\_} devant le nom (\texttt{self.\_x})
  \end{itemize}
\end{frame}

\begin{frame}{Polymorphisme}
  Adaptation du comportement de fonctions ou opérateurs au type des arguments qu'on lui passe.

  Par exemple, l'opérateur \texttt{+} en Python peut être~:

  \begin{itemize}
    \item La concaténation de chaîne de caractères
    \item La concaténation de listes
    \item L'addition d'entiers
    \item …
  \end{itemize}

  $\Rightarrow$ Une méthode différente est appelée dans chaque cas~!

  \begin{alertblock}{Attention}
    On parle de polymorphisme aussi pour des fonctions simples ou des méthodes de classes, pas seulement pour des opérateurs.
  \end{alertblock}
\end{frame}

\begin{frame}{Exemple de polymorphisme par surcharge d'opérateur}
  \pycon{python/basics/oop/polymorphism-addition}
\end{frame}

\begin{frame}{Héritage}
  Pour définir une classe, on peut partir d'une classe existante plutôt que depuis 0. C'est la notion d'héritage~:

  \pycon{python/basics/oop/inheritance}
\end{frame}

\begin{frame}{Héritage multiple}
  On peut partir de plusieurs classes pour définir une nouvelle classe.

  \begin{itemize}
    \item L'ordre dans lequel les classes parentes seront sollicitées pour trouver une méthode appelée avec \texttt{super()} est appelé MRO (\textit{Method Resolution Order})
    \item On peut accéder au MRO d'une classe par son attribut \texttt{\_\_mro\_\_}
  \end{itemize}

  \begin{alertblock}{Attention}
    Utiliser l'héritage multiple est courant en Python mais rend le code plus complexe à bien des égards.
  \end{alertblock}
\end{frame}
