\begin{frame}{Introduction}
  Pour intéragir avec des bases de données, une librairie~: SQLAlchemy.

  Polyvalente, simple pour les cas simples, configurable pour les cas complexes.
\end{frame}

\begin{frame}{Connexion à une base de données}
  Avec l'URL de la base de données (ici par exemple pour une BDD de débug)~:

  \mintedpycode{python/sys-admin/db-connect}
\end{frame}

\begin{frame}{Lecture de données}
  \mintedpycode{python/sys-admin/db-read}
\end{frame}

\begin{frame}{Écriture de données}
  \mintedpycode{python/sys-admin/db-write}
\end{frame}

\begin{frame}{Pour aller plus loin}
  Pour les scripts complexes, SQLAlchemy propose aussi un ORM~:

  \begin{itemize}[<+->]
    \item Correspondance entre les objets Pythons et les objets en base
    \item Construction facilitée de requêtes
    \item Typage fort
  \end{itemize}
\end{frame}

\begin{frame}{Compatibilité}
  SQLAlchemy est compatible avec toutes les grandes BDDs.

  Seulement un sous-ensemble de fonctionnalités est exposé pour certaines BDDs.
\end{frame}
