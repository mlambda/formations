\begin{frame}{Historique}
  \begin{description}
  \item[1989] Création du langage par Guido Van Russum
  \item[2001] Lancement de la Python Software Foundation
  \item[2001] Passage en GPL
  \item[2009] Python 3
  \end{description}
  \V{["python_logo", "tw", 0.8] | image}
\end{frame}

\begin{frame}{Caractéristiques}
  Python est :
  \begin{description}[<+(1)->]
  \item[Interprété] Et compilé à la volée, modules en C
  \item[Orienté objet] (mais pas que)
  \item[Portable] Compatible  avec toutes les plateformes actuelles
  \item[Flexible] Couteau suisse, de l'admin système au webdev
  \item[Populaire] Top 5 des langages les plus utilisés depuis des années
  \end{description}
\end{frame}

\begin{frame}{Points forts/faibles}
  \begin{minipage}[l]{0.49\linewidth}
    Atouts
    \begin{itemize}
    \item Stable
    \item multi-plateforme
    \item Facile à apprendre
    \item Grande communauté (le plus utilisé depuis 2019)
    \item un besoin, un module
    \end{itemize}
  \end{minipage}\hfill
  \begin{minipage}[l]{0.49\linewidth}
    Inconvénients
    \begin{itemize}
    \item Non-compilé
      \begin{itemize}
      \item Plus lent qu'un langage bas-niveau
      \item Optimiser une opération $\Rightarrow$ pas facile à apprendre
      \end{itemize}
    \end{itemize}
  \end{minipage}\hfill
\end{frame}

\begin{frame}{Plateformes}
  Différents interpréteurs :
  \begin{itemize}
  \item Python/CPython $\Rightarrow$ C
  \item Jython $\Rightarrow$ Java
  \item IronPython $\Rightarrow$ .Net
  \end{itemize}
\end{frame}

\begin{frame}{Domaines}
  Domaines d'applications :
  \begin{itemize}
  \item Web (Django ,Flask, ...)
  \item Sciences (Data mining, Machine learning, Physique, ...)
  \item OS (Linux, Raspberry, Script administration système, ...)
  \item Éducation (Initiation à la programmation)
  \item CAO 3D (FreeCAD, pythonCAD, ...)
  \item Multimédia (Kodi, ...)
  \end{itemize}
\end{frame}

\begin{frame}{Syntaxe}
  Utilisation de l'indentation pour délimiter les blocs :
  \mintedpy{python-blocs}
\end{frame}
