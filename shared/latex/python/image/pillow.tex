\begin{frame}{Introduction}
    \begin{itemize}
        \item \alert{Pillow} (PIL) est une bibliothèque Python populaire pour le traitement d'images.
        \item Elle offre une large gamme de fonctionnalités pour manipuler et éditer des
              images.
        \item Compatibilité avec de nombreux formats de fichiers image.
    \end{itemize}
\end{frame}

\begin{frame}{Fonctionnalités principales}
    \begin{enumerate}
        \item Ouverture et enregistrement d'images dans divers formats (JPEG, PNG, GIF,
              etc.).
        \item Manipulation d'images, y compris le redimensionnement, la rotation et la
              recadrage.
        \item Filtrage d'images avec des filtres prédéfinis ou personnalisés.
        \item Ajout de texte et de dessins sur les images.
        \item Traitement avancé des couleurs, y compris la conversion de modes de couleurs.
    \end{enumerate}
\end{frame}

\begin{frame}{Ouvrir une image}

    \mintedcustomcode{python/image/pillow-open-1}{pycon}

    Ou en utilisant un gestionnaire de contexte

    \mintedcustomcode{python/image/pillow-open-2}{pycon}
    
\end{frame}

\begin{frame}{Ouvrir une image -- Précision gestionnaire de contexte}
    \begin{center}
        \alert{ATTENTION}
    \end{center}
    
    Le gestionnaire de contexte de pillow ne charge pas l'image en mémoire tant que la fonction \texttt{Image.load} n'a pas été appelée. 
    
    Celle-ci est automatiquement appelée dès lors qu'une méthode nécessite d'accéder à la valeur d'un pixel.
    
    \mintedcustomcode{python/image/pillow-open-context-error}{pycon}
\end{frame}

\begin{frame}{Exemple d'utilisation}

    \begin{block}{Redimensionner une image}
        \texttt{new\_size = (800, 600) \\
            resized\_image = image.resize(new\_size)}
    \end{block}

    \begin{block}{Enregistrer une image}
        \texttt{resized\_image.save('nouvelle\_image.jpg')}
    \end{block}
\end{frame}

\begin{frame}{Exemple d'utilisation (suite)}
    \begin{block}{Appliquer un filtre}
        \texttt{from PIL import ImageFilter \\
            blurred\_image = image.filter(ImageFilter.BLUR)}
    \end{block}

    \begin{block}{Ajouter du texte}
        \texttt{from PIL import ImageDraw \\
            draw = ImageDraw.Draw(image) \\
            draw.text((10, 10), "Hello, Pillow!", fill=(255, 255, 255))}
    \end{block}
\end{frame}

\begin{frame}{Changement de Mode de Couleurs}
    \begin{itemize}
        \item Pillow permet de changer le mode de couleurs d'une image.
        \item Les modes courants incluent RGB, CMYK, L (niveaux de gris), etc.
        \item Exemple : \texttt{image = image.convert('CMYK')}
    \end{itemize}
\end{frame}

\begin{frame}{Récupération des Layers}
    \begin{itemize}
        \item Certaines images peuvent contenir plusieurs layers ou canaux (par exemple, une
              image RGBA).
        \item Vous pouvez extraire et manipuler ces layers individuellement.
        \item Exemple : \texttt{r, g, b, a = image.split()}
    \end{itemize}
\end{frame}

\begin{frame}{Conclusion}
    \begin{itemize}
        \item Pillow est une bibliothèque puissante et polyvalente pour le traitement
              d'images en Python.
        \item Elle facilite la manipulation, la création et la modification d'images.
        \item Compatible avec de nombreux formats de fichiers image.
        \item Documentation complète et communauté active.
    \end{itemize}
\end{frame}