\begin{frame}{Introduction}
  Contrôler son environnement logiciel pour~:
  \begin{itemize}[<+->]
    \item Rendre son environnement de développement reproductible
    \item Contrôler les dépendances pour la prod
    \item Déployer son environnement facilement sur différents clouds
  \end{itemize}
\end{frame}

\begin{frame}{Isolation de l'environnement --- \texttt{venv}}
  \texttt{venv} permet d'isoler un environnement Python~:
  \begin{itemize}[<+->]
    \item N'intéragit pas avec l'environnement système
    \item Permet la cohabitation de plusieurs environnements incompatibles
    \item Plus rapide et natif que les solutions basées sur les conteneurs
    \item Copie d'une distribution Python de base + customisation
  \end{itemize}
\end{frame}

\begin{frame}{Utilisation de \texttt{venv}}
  \mintedcustomcode{python/environment/venv}{console}
\end{frame}

\begin{frame}{Contrôle des dépendances}
  But~: décrire précisément quelles libraires (et leur version) sont utilisées.

  Plusieurs bons outils~:

  \begin{description}[<+->]
    \item[\texttt{pip} + \texttt{setup.py}] Outil traditionnel, définit un paquet Python
    \item[\texttt{pip} + \texttt{requirements.txt}] Liste simple de dépendances
    \item[\texttt{pip} + \texttt{pip-compile}] Lister les dépendances puis les geler pour la stabilité
    \item[\texttt{Pipenv}] Définition moderne d'une \textbf{application} Python (pas librairie)
    \item[\texttt{poetry}] Définition moderne d'application ou librairie
  \end{description}
\end{frame}

\begin{frame}{Utilisation de \texttt{poetry}}
  \begin{itemize}[<+->]
    \item Définition d'un fichier de configuration (\texttt{poetry init} peut aider)
    \item Gestion des dépendances avec \texttt{poetry add} et \texttt{poetry remove}
    \item Mise à jour des dépendances avec \texttt{poetry update}
    \item Installation de dépendances définies avec \texttt{poetry install}
    \item Gestion de la version avec \texttt{poetry version}
    \item Publication du paquet avec \texttt{poetry publish}
  \end{itemize}
\end{frame}

\begin{frame}{Faciliter le déploiement}
  Quelques plate-formes minoritaires acceptent les paquets Python pour le déploiement.

  Le plus souvent, une étape intermédiaire est nécessaire avant le déploiement, la transformation en conteneur~:

  \begin{itemize}[<+->]
    \item Permet de contrôler les librairies natives en plus des librairies Python
    \item Plus grande robustesse si le logiciel s'exécute sur plusieurs OS
    \item Plus lourd à mettre en place que \texttt{venv}
    \item Plus adapté pour la prod (déploiement facile par Kubernetes)
  \end{itemize}

  \onslide<+->{$\Rightarrow$ Dockerisation du paquet Python à l'aide d'un fichier \texttt{Dockerfile}}
\end{frame}

\begin{frame}{Exemple de \texttt{Dockerfile}}
  \mintedcustomcode{python/environment/Dockerfile}{docker}
\end{frame}