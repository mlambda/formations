\begin{frame}{Introduction}
  Pour intéragir avec des bases de données, une librairie~: SQLAlchemy.

  Polyvalente, simple pour les cas simples, configurable pour les cas complexes.
\end{frame}

\begin{frame}{Connexion à une base de données}
  Avec l'URL de la base de données (ici par exemple pour une BDD de débug)~:

  \mintedpycode{python/databases/connect}
\end{frame}

\begin{frame}{Lecture de données}
  \mintedpycode{python/databases/read}
\end{frame}

\begin{frame}{Écriture de données}
  \mintedpycode{python/databases/write}
\end{frame}

\begin{frame}{ORM}
  En plus d'être un moteur SQL, SQLAlchemy propose un ORM~:

  \begin{itemize}
    \item Modèle décrit par des classes Python
    \item Objets synchronisés entre la base de données et le code Python
  \end{itemize}
\end{frame}

\begin{frame}{Création de tables}
  \mintedpycode{python/databases/tables}
\end{frame}

\begin{frame}{Insertion avec l'ORM}
  \mintedpycode{python/databases/orm-write}
\end{frame}

\begin{frame}{Sélection avec l'ORM}
  \mintedpycode{python/databases/orm-select}
\end{frame}

\begin{frame}{Compatibilité}
  SQLAlchemy est compatible avec toutes les grandes BDDs.

  Seulement un sous-ensemble de fonctionnalités est exposé pour certaines BDDs.
\end{frame}

\begin{frame}{Pour aller plus loin}
  \bluelink{https://docs.sqlalchemy.org/en/14/tutorial/index.html}{Cours SQLAlchemy}
\end{frame}
