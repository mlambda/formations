\begin{frame}{CSV}
  \begin{itemize}
    \item Enregistrements séparés par des sauts de ligne
    \item Valeurs séparées par des virgules
    \item Plusieurs variantes possibles pour autoriser les virgules et sauts de ligne dans les valeurs
    \item Lisible par l'humain
    \item Pratique pour une petite expérience
    \item Mieux vaut utiliser d'autres formats en production
  \end{itemize}
\end{frame}

\begin{frame}{JSON}
  \begin{itemize}
    \item Format très utilisé sur le web car excellent support en javascript
    \item Support de colonnes complexes
    \item Pas de schéma (disponible avec des extensions)
    \item Combinaison d'objets simples~: dictionnaires, listes, valeurs de base
    \item Variante JSONL adaptée aux données volumineuses
    \item Lisible par l'humain
    \item Mieux vaut utiliser un autre format pour maximiser les performances
  \end{itemize}
\end{frame}

\begin{frame}{Parquet}
  \begin{itemize}
    \item Format orienté colonne généraliste très utilisé en production
    \item Support de colonnes complexes
    \item Évolutif
    \item Séparable
    \item Compressé
    \item Pas lisible par l'humain
  \end{itemize}

  $\Rightarrow$ Format à utiliser par défaut en PySpark
\end{frame}

\begin{frame}{HDF5}
  \begin{itemize}
    \item Format de type archive (conteneur de fichiers)
    \item Deux types d'éléments~: tableaux multidimensionnels et groupes
    \item Base de nombreuses surcouches (NetCDF, CGNS, …)
  \end{itemize}
\end{frame}

\begin{frame}{Et d'autres…}
  Chaque domaine a ses formats spécifiques. Entre deux formats, on peut choisir en se basant sur quelques critères

  \begin{itemize}
    \item Orientation colonne (plus rapide pour de nombreuses utilisations)
    \item Évolutivité
    \item Compressibilité
    \item Séparabilité
    \item Lisibilité par l'humain
    \item Qualité des bibliothèques de manipulation
  \end{itemize}
\end{frame}
