\begin{frame}{Introduction}
  \texttt{pandas} permet de facilement manipuler des données.

  Avec, on peut~:

  \begin{itemize}
    \item Charger beaucoup de formats de données dans une structure facilement manipulable
    \item Filtrer, grouper, séparer, réagencer, combiner des données
    \item Résumer, agréger, observer
    \item Gérer des valeurs manquantes
  \end{itemize}
\end{frame}

\begin{frame}{Structures de données}
  La librairie est articulée autour de deux structures de données~:

  \begin{description}
    \item[\texttt{Series}] Série indexée de valeurs
    \item[\texttt{DataFrame}] Plusieurs séries de valeurs avec le même index. Proche de l'idée d'une feuille Excel
  \end{description}

  Une \texttt{Series} n'a qu'une dimension. C'est l'équivalent d'une seule colonne Excel.

  Une \texttt{DataFrame} a deux dimensions. C'est l'équivalent d'une feuille Excel. Elle est constituée de plusieurs \texttt{Series}, chacune étant une colonne.
\end{frame}

\begin{frame}{Import de données dans une \texttt{DataFrame}}
  
\end{frame}

\begin{frame}{Création de \texttt{DataFrame}}
  
\end{frame}

\begin{frame}{Description}
  
\end{frame}

\begin{frame}{Opérations}
  
\end{frame}

\begin{frame}{Indiçage}
  
\end{frame}

\begin{frame}{Valeurs manquantes}
  
\end{frame}

\begin{frame}{Regroupements}
  
\end{frame}

\begin{frame}{Intéropérabilité avec \texttt{numpy}}
  
\end{frame}

\begin{frame}{Documentation}
  \begin{itemize}
    \item \bluelink{https://pandas.pydata.org/docs/user_guide/index.html}{Documentation officielle}
    \item \bluelink{https://pandas.pydata.org/Pandas_Cheat_Sheet.pdf}{Feuille d'aide}
  \end{itemize}
\end{frame}
