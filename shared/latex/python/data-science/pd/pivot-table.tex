\begin{frame}{tables pivots}
  Les tables pivots permettent de synthétiser des informations contenues dans notre \texttt{pd.DataFrame} \\
  Ex: une valeure moyenne par catégories, une sorte de groupby avec un aggrégateur.
\end{frame}

\begin{frame}{Pivots}
  \mintedcustomcode{python/data-science/pandas/pivot}{pycon}
\end{frame}

\begin{frame}{Pivots}
  \mintedcustomcode{python/data-science/pandas/pivot2}{pycon}
\end{frame}

\begin{frame}{Pivots}
  Pas de couple index-colonne identique !
  \mintedcustomcode{python/data-science/pandas/pivot5}{pycon}
\end{frame}

\begin{frame}{Pivots}
  \mintedcustomcode{python/data-science/pandas/pivot3}{pycon}
\end{frame}

\begin{frame}{Pivots}
  \mintedcustomcode{python/data-science/pandas/pivot4}{pycon}
\end{frame}


\begin{frame}{tables pivots}
  \mintedcustomcode{python/data-science/pandas/pivot-table}{pycon}
\end{frame}

\begin{frame}{tables pivots}
  \mintedcustomcode{python/data-science/pandas/pivot-table2}{pycon}
\end{frame}

\begin{frame}{tables pivots}
  \mintedcustomcode{python/data-science/pandas/pivot-table3}{pycon}
\end{frame}

\begin{frame}{Matrices de contingence}
  Tables qui résument l'information en comptant le nombre d'occurence de chaque clé
  du premier argument en regroupant les résultats suivant les clés du deuxième argument. \\
  \mintedcustomcode{python/data-science/pandas/cross-table}{pycon}
\end{frame}
