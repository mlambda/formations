\begin{frame}{Split-Apply-Combine}
  \begin{description}
    \item[Séparer]   en groupes suivant un ou des critères
    \item[Appliquer] une fonction à chaque groupe
    \item[Combiner]  les résultats dans un \texttt{pd.Series} ou \texttt{pd.DataFrame}
  \end{description}
\end{frame}

\begin{frame}{Design}
  La création du groupby vérifie uniquement la validité du mapping. \\
  Le traitement de l'opération n'a lieu que lorsque c'est nécessaire. \\
  \mintedcustomcode{python/data-science/pandas/groupby-type}{pycon}
\end{frame}

\begin{frame}{Indiçage simple}
  \mintedcustomcode{python/data-science/pandas/groupby}{pycon}
\end{frame}

\begin{frame}{Indiçage simple avec une fonction}
  Utilisation d'une fonction prenant un \texttt{index} du \texttt{pd.DataFrame} en entrée et qui renvoie un booléen~:
  \mintedcustomcode{python/data-science/pandas/groupby-function}{pycon}
\end{frame}

\begin{frame}{Indiçage simple et plusieurs critères}
  \mintedcustomcode{python/data-science/pandas/groupby-multi-criteria}{pycon}
\end{frame}

\begin{frame}{Indiçage hiérarchique}
  \mintedcustomcode{python/data-science/pandas/groupby-multi-index}{pycon}
\end{frame}

\begin{frame}{Indiçage hiérarchique}
  Utilisation des deux variables catégorielles comme index de nos échantillons.

  \mintedcustomcode{python/data-science/pandas/groupby-multi-index2}{pycon}
\end{frame}

\begin{frame}{Indiçage hiérarchique}
  Utilisation du mot clé \texttt{level} à la place de \texttt{by} (mot clé correspondant au premier argument).

  \mintedcustomcode{python/data-science/pandas/groupby-multi-index3}{pycon}
\end{frame}

\begin{frame}{Indiçage hiérarchique}
  \mintedcustomcode{python/data-science/pandas/groupby-multi-index4}{pycon}
\end{frame}

\begin{frame}{Split-Apply-Combine}
  \begin{description}
    \item[Séparer]   en groupes suivant un ou des critères
    \item[\underline{Appliquer}] une fonction à chaque groupe
    \item[Combiner]  les résultats dans un \texttt{pd.Series} ou \texttt{pd.DataFrame}
  \end{description}
\end{frame}

\begin{frame}{Split-\texttt{Apply}-Combine}
  On considère 3 types de fonctions pour l'étape \texttt{Apply}
  \begin{description}[r,labelwidth=\widthof{   Transformation :}]
    \item[Agrégation :]   somme, moyenne, décomptes, ... 
    \item[Transformation :] valeurs manquantes, normalisation, ... 
    \item[Filtrage :]  groupes trop petits ou avec une mauvaise moyenne ... 
  \end{description}
\end{frame}

\begin{frame}{Aggrégation}
  \mintedcustomcode{python/data-science/pandas/groupby-aggregate}{pycon}
\end{frame}

\begin{frame}{Aggrégation}
  \mintedcustomcode{python/data-science/pandas/groupby-aggregate2}{pycon}
\end{frame}

\begin{frame}{Aggrégation}
  \bluelink{https://pandas.pydata.org/docs/user_guide/groupby.html\#built-in-aggregation-methods}{built-in-aggregation-methods}
\end{frame}

\begin{frame}{Transformation}
  Une transformation est un GroupBy dont le résultat à le même index que celui sur lequel on effectue l'opération GroupBy. \\
  Généralement utilisé pour ajouter une nouvelle colonne.
  \mintedcustomcode{python/data-science/pandas/groupby-transform}{pycon}
\end{frame}

\begin{frame}{Transformation}
  La fonction \texttt{transform} est l'équivalent de \texttt{aggregate} et peut utiliser
  \bluelink{https://pandas.pydata.org/docs/user_guide/groupby.html\#built-in-transformation-methods}{les methodes de transformation fournies par l'API}. \\
  On peut aussi utiliser les fonctions d'aggrégation de l'API, auquel cas la fonction sera broadcastée à travers le groupe. 
  \mintedcustomcode{python/data-science/pandas/groupby-transform2}{pycon}
\end{frame}

\begin{frame}{Apply}
  \mintedcustomcode{python/data-science/pandas/groupby-apply}{pycon}
  Moins efficace que groups.agg("mean") !
\end{frame}

\begin{frame}{Apply}
  \mintedcustomcode{python/data-science/pandas/groupby-apply2}{pycon}
  Plus rapide avec \texttt{transform} !
\end{frame}

\begin{frame}{Apply}
  On peut souvent optimiser ce calcul en passant par les built-in-methods mais c'est souvent plus compliqué à formuler \\
  \mintedcustomcode{python/data-science/pandas/groupby-apply3}{pycon}
\end{frame}

\begin{frame}{\texttt{apply} et \texttt{transform}}
  La fonction en argument :
  \begin{description}
    \item[input]  \begin{itemize}
                    \item \texttt{apply} passe toutes les colonnes de chaque groupes dans un \texttt{pd.DataFrame}.
                    \item \texttt{transform} passe chaque colonne de chaque groupe dans une \texttt{pd.Series}.
                  \end{itemize}
    \item[output]  \begin{itemize}
                    \item \texttt{apply} peut renvoyer un scalaire, \texttt{pd.Series}, \texttt{pd.DataFrame}, numpy array ou même une liste python.
                    \item \texttt{transform} doit renvoyer une \texttt{pd.Series} (ou au moins une Array 1D) de la même taille que le groupe.
                  \end{itemize}
  \end{description}
\end{frame}

\begin{frame}{Filtrage}
  \mintedcustomcode{python/data-science/pandas/groupby-filter}{pycon}
  \bluelink{https://pandas.pydata.org/docs/user_guide/groupby.html\#built-in-filtrations}{built-in-filtrations}
\end{frame}
