\begin{frame}{Introduction}
  \texttt{numpy} est une librairie centrée sur un concept~: le tableau multidimensionnel.

  La manipulation de tableaux \texttt{numpy} est très rapide~: on constate des améliorations de ×100 à ×500 par rapport à des listes Python.
\end{frame}

\begin{frame}{Import de \texttt{numpy}}
  La manière standard d'importer \texttt{numpy} est de renommer l'import \texttt{np}~:

  \mintedcustomcode{python/data-science/import}{py}
\end{frame}

\begin{frame}{Création de tableau ⋅ Tableaux constants}
  \mintedcustomcode{python/data-science/creation-scratch}{pycon}

  Des versions \texttt{\_like} existent aussi pour \texttt{np.empty} et \texttt{np.zeros}.
\end{frame}

\begin{frame}{Création de tableau ⋅ Depuis des données existantes}
  La méthode \texttt{np.array} convertit des itérables en tableau~:

  \mintedcustomcode{python/data-science/creation-existing}{pycon}
\end{frame}

\begin{frame}{Création de tableau ⋅ Suite de valeurs}
  \mintedcustomcode{python/data-science/creation-interval}{pycon}
\end{frame}

\begin{frame}{Création de tableau ⋅ Génération aléatoire}
  \mintedcustomcode{python/data-science/creation-random}{pycon}
\end{frame}

\begin{frame}{Étude d'un tableau existant}
  \mintedcustomcode{python/data-science/shape}{pycon}
\end{frame}

\begin{frame}{Opérations sur un tableau}
  \mintedcustomcode{python/data-science/operations}{pycon}
\end{frame}

\begin{frame}{Opérations sur des axes spécifiques}
  \mintedcustomcode{python/data-science/reducing}{pycon}
\end{frame}

\begin{frame}{Manipulation des axes}
  \mintedcustomcode{python/data-science/axes}{pycon}
\end{frame}

\begin{frame}{Redimensionnement}
  \mintedcustomcode{python/data-science/reshape}{pycon}
\end{frame}

\begin{frame}{Indiçage dans un tableau ⋅ Indiçage direct}
  Indiçages positif et négatif comme pour les listes.

  On sépare les indices de chaque dimension par une virgule.

  \mintedcustomcode{python/data-science/indexing-simple}{pycon}
\end{frame}

\begin{frame}{Indiçage dans un tableau ⋅ Tranches}
  \mintedcustomcode{python/data-science/indexing-slicing}{pycon}
\end{frame}

\begin{frame}{Indiçage dans un tableau ⋅ Raccourci}
  \mintedcustomcode{python/data-science/indexing-shortcuts}{pycon}
\end{frame}

% TODO Broadcast, map et compagnie