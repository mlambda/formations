\begin{frame}{Introduction}

    \texttt{matplotlib} est la librairie graphique de référence pour la visualisation 2D (la seule ?).
    
    Un très grand nombre de librairies gravitent autour de matplotlib pour répondre à différents besoins et différents domaines.
\end{frame}

\begin{frame}{Un premier plot avec \texttt{plot(x, y)}}
    \begin{minipage}[t]{0.49\linewidth}
        \mintedcustomcode{python/data-science/matplotlib/simple}{py}
    \end{minipage}
    \begin{minipage}[t]{0.49\linewidth}
        \vfill
        \V{"plt/mpl/simple" | image("tw", 1)}  
    \end{minipage}
\end{frame}

\begin{frame}{Différents types de plot : \texttt{scatter(x, y)}}
    \begin{minipage}[t]{0.49\linewidth}
        \mintedcustomcode{python/data-science/matplotlib/scatter}{py}
    \end{minipage}
    \begin{minipage}[t]{0.49\linewidth}
        \vfill
        \V{"plt/mpl/scatter" | image("tw", 1)} 
    \end{minipage}
\end{frame}

\begin{frame}{Différents types de plot : \texttt{fill\_between(x, y, y1)}}
    \begin{minipage}[t]{0.49\linewidth}
        \mintedcustomcode{python/data-science/matplotlib/fill-between}{py}
    \end{minipage}
    \begin{minipage}[t]{0.49\linewidth}
        \vfill
        \V{"plt/mpl/fill-between" | image("tw", 1)} 
    \end{minipage}
\end{frame}

\begin{frame}{Différents types de plot : \texttt{bar(x, y)}}
    \begin{minipage}[t]{0.49\linewidth}
        \mintedcustomcode{python/data-science/matplotlib/bar}{py}
    \end{minipage}
    \begin{minipage}[t]{0.49\linewidth}
        \vfill
        \V{"plt/mpl/bar" | image("tw", 1)} 
    \end{minipage}
\end{frame}

\begin{frame}{Différents types de plot : \texttt{imshow(Z)}}
    \begin{minipage}[t]{0.49\linewidth}
        \mintedcustomcode{python/data-science/matplotlib/imshow}{py}

    \end{minipage}
    \begin{minipage}[t]{0.49\linewidth}
        \vfill
        \V{"plt/mpl/imshow" | image("tw", 1)} 
    \end{minipage}
\end{frame}

\begin{frame}{Différents types de plot : \texttt{imshow(im)}}
    \alert{\texttt{imshow}} est aussi souvent utilisé pour afficher des images au format \texttt{RGB} ou en noir et blanc. 
    
    Dans le cas du noir et blanc, il faut ajouter \texttt{cmap='gray'} ou \texttt{cmap='gray\_r'}.
\end{frame}

\begin{frame}{Différents types de plot : \texttt{imshow(im)}}
    \begin{minipage}[t]{0.49\linewidth}
        \mintedcustomcode{python/data-science/matplotlib/imshow2}{py}

    \end{minipage}
    \begin{minipage}[t]{0.49\linewidth}
        \vfill
        \V{"plt/mpl/imshow2" | image("tw", 1)} 
    \end{minipage}
\end{frame}

\begin{frame}{Différents types de plot : \texttt{contour(X, Y, Z)}}
    \begin{minipage}[t]{0.49\linewidth}
        \mintedcustomcode{python/data-science/matplotlib/contour}{py}
    \end{minipage}
    \begin{minipage}[t]{0.49\linewidth}
        \vfill
        \V{"plt/mpl/contour" | image("tw", 1)} 
    \end{minipage}
\end{frame}

\begin{frame}{Différents types de plot : \texttt{hist(x)}}
    \begin{minipage}[t]{0.49\linewidth}
        \mintedcustomcode{python/data-science/matplotlib/hist}{py}
    \end{minipage}
    \begin{minipage}[t]{0.49\linewidth}
        \vfill
        \V{"plt/mpl/hist" | image("tw", 1)} 
    \end{minipage}
\end{frame}

\begin{frame}{Différents types de plot : \texttt{boxplot(X)}}
    \begin{minipage}[t]{0.49\linewidth}
        \mintedcustomcode{python/data-science/matplotlib/boxplot}{py}
    \end{minipage}
    \begin{minipage}[t]{0.49\linewidth} 
        \vfill
        \V{"plt/mpl/boxplot" | image("tw", 1)}
    \end{minipage}
\end{frame}

\begin{frame}{Une librairie simple \visible<2>{?}}
    Ces premiers exemples en quelques lignes laissent supposer une librairie facile d'accès et simple d'utilisation pour la création de visualisation.

    \visible<2>{\centering\Large\alert{\textbf{Il n'en est rien.}}}
\end{frame}

\begin{frame}{Une librairie riche et complexe}
    \begin{itemize}
        \positem Matplotlib est un libraire hautement configurable. C'est ce qui permet à beaucoup d'autres librairies de s'appuyer sur elle.
        \negitem Mais cela vient nécessairement avec un coût pour réussir à faire des schémas esthétiques
        \negitem Et malheureusement, la documentation est améliorable
    \end{itemize}
    
\end{frame}

\begin{frame}{Anatomie d'une figure}
    \V{"plt/mpl/anatomy" | image("th", 0.75)}
\end{frame}

\begin{frame}{Anatomie d'une figure}
    \begin{minipage}{0.49\linewidth}
        \begin{description}
          \item[Figure] Structure de plus haut niveau qui contiendra tous les éléments à afficher
          \item[Axes] Élément contenant un graphique. Un ou plusieurs \texttt{axes} par \texttt{figure}
          \item[Spine] Cadre d'un \texttt{axe}
          \item[Legend] Légende associée à un \texttt{axe}
          \item[Title] Titre associé à un \texttt{axe}
      \end{description}
    \end{minipage}
    \begin{minipage}{0.49\linewidth}
        \V{"plt/mpl/anatomy" | image("tw", 1)}
    \end{minipage}
\end{frame}

\begin{frame}{Anatomie d'une figure}
    \begin{minipage}{0.49\linewidth}
        \begin{description}
            \item[Suptitle] Titre associé à une \texttt{figure}
            \item[Label] Descripteur d'\texttt{axe} associé à l'axe \texttt{x} et \texttt{y}
            \item[Tick] Marqueur sur les spines pour représenter l'échelle des valeurs
            \item[Grid] Grille affichée sur la visualisation
        \end{description}
    \end{minipage}
    \begin{minipage}{0.49\linewidth}
        \V{"plt/mpl/anatomy" | image("tw", 1)}
    \end{minipage}
\end{frame}

\begin{frame}{Paramétrisation}
    Il existe une grande quantité d'éléments pouvant être modifiés pour changer la visualisation.
\end{frame}

\begin{frame}{Taille et résolution}
    \begin{minipage}[t]{0.49\linewidth}
        \mintedcustomcode{python/data-science/matplotlib/classique-start}{py}
    \end{minipage}
    \begin{minipage}[t]{0.49\linewidth}
        \V{"plt/mpl/classique-start" | image("tw", 0.65)} 
    \end{minipage}
\end{frame} 

\begin{frame}{Titre et légendes}
    \begin{minipage}{0.49\linewidth}
        \mintedcustomcode{python/data-science/matplotlib/legend}{py}
    \end{minipage}
    \begin{minipage}{0.49\linewidth}
        \V{"plt/mpl/legend" | image("tw", 1)} 
    \end{minipage}
\end{frame}

\begin{frame}{Changer les limites}
    \begin{minipage}{0.49\linewidth}
        \mintedcustomcode{python/data-science/matplotlib/limits}{py}
    \end{minipage}
    \begin{minipage}{0.49\linewidth}
        \V{"plt/mpl/limits" | image("tw", 1)} 
    \end{minipage}
\end{frame}

\begin{frame}{Modifier les spines}
    \begin{minipage}{0.52\linewidth}
        \mintedcustomcode{python/data-science/matplotlib/spines}{py}
    \end{minipage}
    \begin{minipage}{0.46\linewidth}
        \V{"plt/mpl/spines" | image("tw", 1)}
    \end{minipage}
\end{frame}

\begin{frame}{Ajouter les labels}
    \begin{minipage}[c]{0.49\linewidth}
        \mintedcustomcode{python/data-science/matplotlib/labels}{py}
    \end{minipage}
    \begin{minipage}[c]{0.49\linewidth}
        \V{"plt/mpl/labels" | image("tw", 1)} 
    \end{minipage}
\end{frame}

\begin{frame}{Modifier les ticks}
    \begin{minipage}[c]{0.49\linewidth}
        \mintedcustomcode{python/data-science/matplotlib/ticks}{py}
    \end{minipage}
    \begin{minipage}[c]{0.49\linewidth}
        \V{"plt/mpl/ticks" | image("tw", 1)} 
    \end{minipage}
\end{frame}

\begin{frame}{Plots multiples et projection}
    \V{"plt/mpl/scales" | image("tw", 0.6)} 
\end{frame}

\begin{frame}{Sous-plots}
    \mintedcustomcode{python/data-science/matplotlib/subplots1}{py}
\end{frame}

\begin{frame}{Sous-plots}
    \mintedcustomcode{python/data-science/matplotlib/subplots2}{py}
\end{frame}

\begin{frame}{Sauvegarder une visualisation}
    \mintedcustomcode{python/data-science/matplotlib/save}{py}
\end{frame}

\begin{frame}{Documentation}
  \begin{itemize}
    \item \bluelink{https://matplotlib.org/stable/users/index}{Documentation officielle}
    \item Feuilles d'aides~:
      \begin{itemize}
        \item \bluelink{https://matplotlib.org/cheatsheets/cheatsheets.pdf}{Principale}
        \item \bluelink{https://matplotlib.org/cheatsheets/handout-beginner.pdf}{Supplémentaire, niveau débutant}
        \item \bluelink{https://matplotlib.org/cheatsheets/handout-intermediate.pdf}{Supplémentaire, niveau intermédiaire}
        \item \bluelink{https://matplotlib.org/cheatsheets/handout-tips.pdf}{Supplémentaire, quelques astuces}
      \end{itemize}
    \item \bluelink{https://github.com/rougier/scientific-visualization-book}{Livre niveau intermédiaire/avancé sur \texttt{matplotlib}}
    \item \bluelink{https://pyviz.org/}{Pyviz~: liste des librairies dépendantes de \texttt{matplotlib}}
  \end{itemize}
\end{frame}
