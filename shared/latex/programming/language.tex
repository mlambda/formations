\begin{frame}{Définition}
    \begin{itemize}
        \item Un alphabet
        \item Un vocabulaire
        \item Des règles de grammaire
        \item Environnement de traduction vers le « langage » machine~:
    \end{itemize}
    \begin{center}
        \textbf{le compilateur}
    \end{center}
\end{frame}

\begin{frame}{Les paradigmes de programmation}
    \begin{description}
        \item[Procédural] Au plus proche des instructions du processeur 
                        \\(C, Fortran, Cobol, …)
        \item[Déclaratif]
        \begin{description}
            \item[Fonctionnel] Résolution de lambda-calcul par la récursivité (Haskell, Ocaml, ...)
            \item[Logique] Résolution de prédicats par des règles déductives (Prolog, ...)
        \end{description}
        \item[Orienté Objet] Introduction de la notion d'objet et d'héritage
                            \\(Java, Swift, …)
        \item[Visuel] Le programme est défini par un schéma 
                    \\(Delphi, Smalltalk, …)
        \item[Basé web] Le programme est envoyé à travers le web pour être exécuté chez le client (Javascript, …)
        \item[…]
    \end{description}
\end{frame}

\begin{frame}{Un code source}
    \mintedpycode{programming/program-exemple}
\end{frame}
