\begin{frame}{Définition}
    Un algorithme est une suite finie et non ambiguë d'instructions et d’opérations permettant de résoudre une classe de problèmes.
    \V{"img/al-khwarizmi-timbre" | image("th", 0.4)}    
\end{frame}

\begin{frame}{Caractéristiques}
    Un algorithme est défini par :
    \begin{itemize}
        \item Ses \textbf{entrées} (sous formes de \textbf{variables})
        \item Ses \textbf{sorties} (sous formes de \textbf{variables})
        \item Les étapes (\textbf{instructions}) nécéssaires pour transformer les entrées en sorties
    \end{itemize}
\end{frame}

\begin{frame}{Exemple}
    Une recette de gâteau :
    \begin{itemize}
        \item Ses \textbf{entrées} une liste d'ingrédients et d'outils
        \item Ses \textbf{sorties} un gâteau
        \item Les \textbf{instructions} :
        \begin{itemize}
            \item prendre un récipient
            \item mélanger les ingrédients
            \item ...
            \item cuire la pâte
            \item ...
        \end{itemize}    
    \end{itemize}
\end{frame}

\begin{frame}{Quelques instructions en français}
    
    \begin{description}
        \item [\blue{Accès}] $X[5]\;\;\;\;\;$ \green{"On accède au 5ème élément de la liste $X$"}
        \item []
        \item [\blue{Assignation}] $Y \leftarrow X\;$  \green{"On assigne à $Y$ la valeur contenue dans $X$"}
        \item []
        \item [\blue{Test}] 
        \begin{description}
            \item [SI] $X == Y$
            \item [ALORS] $X \leftarrow X + 10$
            \item [SINON] $Y \leftarrow X - 12$
        \end{description}
        \item []
        \item [\blue{Boucle}] 
        \begin{description}
            \item [TANT QUE] $X > Y$
            \item [ALORS] $X \leftarrow X - 1$
        \end{description}
    \end{description}

    
\end{frame}
