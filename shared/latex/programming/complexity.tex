\begin{frame}{Définition}
    Soit un algorithme que l'on veut executer sur N éléments, \\
    on appelle \textbf{complexité algorithmique} l'ordre de grandeur du nombre d'instructions qu'il faut executer pour terminer l'algorithme.
\end{frame}

\begin{frame}{Exercice 1}
    \begin{tabular}{|r|r|r|r|r|r|r|r|}
        \hline
            15 & 22 & 56 & 17 & 19 & 34 & 12 & 11 \\
        \hline
    \end{tabular} \\
    \newline
    Combien d'opérations (ici, des comparaisons) faut-il effectuer 
    pour retrouver la valeur minimum dans un tableau de N éléments ?
\end{frame}

\begin{frame}{Solution}
    Notre algorithme qui prend un tableau de N éléments comparables et 
    renvoi la valeur minimum du tableau à une complexité de N, notée \textbf{$O(N)$}. \\
\end{frame}

\begin{frame}{Exercice 2}
    \begin{tabular}{|r|r|r|r|r|r|r|r|}
        \hline
            15 & 22 & 56 & 17 & 19 & 34 & 12 & 11 \\
        \hline
    \end{tabular} \\
    \newline
    Pour trier les éléments dans un tableau, on peut utiliser l'algorithme précédent.\\
    Pour un tableau de taille N, combien de fois devrais-je appeller l'algorithme précédent ?\\
    Quelle serait alors la complexité de notre algorithme de tri ?
\end{frame}

\begin{frame}{Solution}
    Notre algorithme qui prend un tableau de N éléments comparables et 
    renvoi un tableau trié nécéssite N appels à l'algorithme de récupération du minimum pour un total de \\
    \newline
    $(1+2+...+N)=\frac{N(N+1)}{2}$ opérations, notée \textbf{$O(N^2)$}. \\
\end{frame}

\begin{frame}{Complexité complexe}
    En réalité, il faut prendre en compte différentes complexités :
    \begin{description}
        \item[\textbf{Cas Optimal :}]         Dans le meilleur des cas
        \item[\textbf{Cas moyen :}]           Nombre moyen d'opérations
        \item[\textbf{Pire des cas :}]        Ce que l'on ne peut pas éviter d'arriver
        \item[\textbf{Stabilité :}]           des résultats souvent moyens ou pas ?
        \item[\textbf{Spatiale :}]            La taille en mémoire nécessaire pour executer l'algorithme
    \end{description}
    \bluelink{https://fr.wikipedia.org/wiki/Algorithme_de_tri\#Comparaison_des_algorithmes}{wikipédia Algorithme de tri Comparaison des algorithmes}
\end{frame}

\begin{frame}{}
    Tous l'art de faire des programmes efficaces repose sur le choix des \textbf{structures de données adaptées} et d'un respect de la \textbf{complexité algorithmique} dans les procédures que l'on met en place.
\end{frame}
