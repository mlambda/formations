\begin{frame}
  \frametitle{Collecte}
  Des sources variées :
  \begin{itemize}
  \item Wikipedia
  \item Articles de journaux
  \item Littérature
  \item User Generated Content
    \begin{itemize}
    \item Blogs
    \item Commentaires
    \item Réseaux sociaux
    \end{itemize}
  \end{itemize}
  Une source $\Rightarrow$ un «~web scrapper~»
\end{frame}

\begin{frame}{Tokenisation}
  Séparer une chaîne de caractères en token n'est pas trivial :

  \begin{center}
    Le Dr. Pond èleve des poules. L'éleveur les sur-exploite.
  \end{center}

  (en phrases ou en mots)
\end{frame}

\begin{frame}{Étiquetage morpho-syntaxique}
  \imgtw[0.8]{pos-tagging}
\end{frame}

\begin{frame}{Reconnaissance d'entités nommées}
  \imgtw[0.8]{named-entity-recognition}
\end{frame}

\begin{frame}{Analyse syntaxique}
  \imgtw[0.8]{parse-tree}
\end{frame}

\begin{frame}{Lemmatisation}
  Exprimer les mots sous leur forme canonique : \\

  \begin{tabular}{ll}
    \textbf{Mot} & \textbf{Lemme} \\
    jouant & jouer \\
    ont été jouées & jouer \\
    étoiles & étoile \\
    claires & clair \\
    noire & noir
  \end{tabular}
\end{frame}

\begin{frame}
  \frametitle{Outlis --- WordNet}
  \imgth[0.7]{wordnet-1}  
  Projet sur le français : WOLF (Wordnet Libre du Français) 
\end{frame}

\begin{frame}
  \frametitle{Outils --- DBpedia}
  \imgtw[0.9]{dbpedia}
\end{frame}

\fmg{\begin{frame}
  \frametitle{Utilisation actuelle}
  \imgtw[0.9]{chatbot}
\end{frame}}

\fmg{\begin{frame}
  \frametitle{Maintenabilité}
  \imgtw[0.9]{usine-a-gaz}
\end{frame}}

