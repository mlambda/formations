\begin{frame}
  \frametitle{TF-IDF}
  Un document = un vecteur de la taille d'un dictionnaire. (donc à dimension fixe) \\
  \begin{center}
    $\boxed{w_{ij} = tf_{ij}\log{\frac{N}{df_i}}}$
  \end{center}
  où :

  \begin{description}[<+->]
    \item[$tf_{ij}$] Nombre d'occurrences du mot $i$ dans le document $j$
    \item[$df_i$] Nombre de documents contenant le mot $i$
    \item[$N$] Nombre total de documents
  \end{description}
  \onslide<+->{$\Rightarrow$ Produit scalaire, SVM, arbres, réseaux de neurones, …}
\end{frame}

\begin{frame}{TF-IDF --- Intérêt de $df_i$}
  Loi de Zipf, justifiant l'utilisation du terme $df_i$
  \imgtw[0.9]{zipf-law}
\end{frame}

\begin{frame}{TF-IDF}
  Utilisation de $n$-grammes de mots ou de caractères.
  \begin{center}
  Exemple~: «~Le chien mange de la viande~»
  \end{center}

  Bigrammes de mots~:

  \begin{itemize}
  \item le-chien, chien-mange, mange-de, de-la, la-viande
  \end{itemize}

  Trigrammes de caractères~:

  \begin{itemize}
  \item le\_, e\_c, \_ch, chi, hie, ien, en\_, n\_m, \_ma, man, …
  \end{itemize}
\end{frame}

\begin{frame}{Analyse sémantique latente}
  Latent Semantic Analysis (LSA). Famille algorithmique des modèles thématiques (topic models).

  $\approx$ PCA sur la matrice des documents, de telle sorte qu'on obtient des relations entre les mots.

  Les directions principales de projection nous donnent des «~concepts~» généraux, ou topics, liés au langage.

  Par exemple~: un axe correspond au champ lexical du sport, un autre à celui de l'économie, etc.
\end{frame}

\begin{frame}
  \frametitle{Word Embeddings} 
  mot = indice dans un dictionnaire (dimension > 30000)
  \newline
  mot = vecteur ``sémantique'' (dimension < 1000) 
  \begin{itemize}
  \item word2vec
  \item CBOW/Skip-Gram
  \item GloVe
  \item Thought vector (pour des phrases ou même des documents entiers)
  \item ...
  \end{itemize}
\end{frame}

\begin{frame}
  \frametitle{Word Embeddings} 
  \imgtw{word2vec}
\end{frame}

\begin{frame}
  \frametitle{Word Embeddings} 
  \imgtw[0.6]{word2vec-analogy}
  \begin{center}
    \bluelink{https://projector.tensorflow.org/}{Visualisation de l'espace word2vec} \\
    \bluelink{http://vectors.nlpl.eu/repository/}{Word Embeddings à télécharger}
  \end{center}
\end{frame}

