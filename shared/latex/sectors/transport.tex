\begin{frame}{Voiture autonome}
  \V{["img/voiture-autonome-conducteur", "tw", 0.8] | image}
\end{frame}

\begin{frame}{Voiture autonome ⋅ Caractéristiques}
  \begin{description}
    \item[Atouts] Peu de degrés de liberté
    \item[Inconvéniants] Hétérogénéité de l'environnement
  \end{description}
  \V{["img/voiture-autonome-navya", "th", 0.5] | image}
\end{frame}

\begin{frame}{Voiture autonome ⋅ Hiérarchie}
  Plusieurs niveaux d'autonomie (\textit{Society of Automotive Engineers})~:
  \begin{description}
    \item[1] Aide à la conduite (régulateur de vitesse, freinage d'urgence, …)
    \item[2] Automatisation partielle (régulateur d'embouteillage, assistant de parking, …)
    \item[3] Automatisation situationnelle avec surveillance humaine (pilote d'autoroute, parking automatique, …)
    \item[4] Automatisation situationnelle sans surveillance humaine
    \item[5] Autonomie complète
  \end{description}
\end{frame}

\begin{frame}{Voiture autonome ⋅ Capteurs}
  \V{["img/voiture-autonome", "tw", 0.8] | image}
\end{frame}

\begin{frame}{Voiture autonome ⋅ 2004 --- DARPA Grand Challenge}
  150 miles ($\approx$ 240 km) dans le désert du Nevada. Aucun gagnant~!

  5/23 arrivent au bout en 2005.

  \V{["img/darpa-2005", "th", 0.5] | image}
\end{frame}

\begin{frame}{Voiture autonome ⋅ 2007 --- DARPA Urban Challenge}
  60 miles ($\approx$ 100 km) en milieu urbain + respect du code de la route (sans gps).

  6/35 franchissent la ligne d'arrivée.

  \V{["img/darpa-2007", "th", 0.5] | image}
\end{frame}

\begin{frame}{Voiture autonome ⋅ 2010 --- Google Car}
  Projet visant une voiture autonome de niveau 5.

  \V{["img/google-car", "th", 0.6] | image}
\end{frame}

\begin{frame}{Voiture autonome ⋅ 2015 --- Tesla Autopilot}
  Uniquement avec caméra, pas de LIDAR.

  \V{["img/tesla-autopilot", "tw", 0.6] | image}
\end{frame}

\begin{frame}{Voiture autonome ⋅ Fin 2018 --- Waymo One déployé à Phoenix}
  \V{["img/waymo", "th", 0.6] | image}
\end{frame}

\begin{frame}{Voiture autonome ⋅ Accidents}
  \begin{columns}
    \begin{column}{0.6\textwidth}
      \begin{itemize}
        \item 6 accident mortels implicant des voitures autonomes (5 Tesla, 1 Uber).
        \item Au 7 mai 2016 ---~2ème accident~--- Autopilot avait parcouru 210 millions de kilomètres.
        \item 1 seul (Uber) ou le pilote automatique est mis en cause.
      \end{itemize}
    \end{column}
    \begin{column}{0.4\textwidth}
      \V{["img/uber-crash", "tw", 1] | image}
    \end{column}
  \end{columns}
\end{frame}

\begin{frame}{Voiture autonome ⋅ Acteurs}
  \V{["img/voiture-autonome-acteurs", "tw", 1] | image}
\end{frame}

\begin{frame}
  \frametitle{Voiture autonome ⋅ Impact}
  \V{["img/voiture-autonome-changements", "tw", 1] | image}
\end{frame}

\begin{frame}{Automobile --- pas seulement la voiture autonome}
  L'IA pour les véhicules, pas uniquement de la conduite automatisée :
  \begin{itemize}
  \item Optimisation des trajets (Uber, Watson on Wheels, Waze)
  \item Maintenance préventive par analyse des vibrations
  \end{itemize}
\end{frame}

\begin{frame}{Transport ferroviaire}
  \V{["img/railway", "th", 0.7] | image}
\end{frame}

\begin{frame}{Transport ferroviaire}
  \begin{itemize}
  \item Prévision de retard et replanification
  \item Conduite
  \item Maintenance préventive par classification d'images (rails, caténaires, isolateurs, materiel roulant, végétation ...)
  \item Relation client
  \end{itemize}
  \V{["img/rail-detect-defauts", "tw", 0.8] | image}
\end{frame}

\begin{frame}
  \frametitle{Aéronautique}
  \begin{minipage}[c]{0.49\linewidth}
    \begin{itemize}
    \item Production des pièces
    \item Contrôle qualité des pièces
    \item Maintenance préventive
    \item Pilote automatique
    \item Prédiction de retard
    \item Optimisation des flux
    \item Relation Client
    \item Gestion du personnel
    \end{itemize}
  \end{minipage}\hfill
  \begin{minipage}[c]{0.49\linewidth}
    \V{["img/avion-intelligent", "tw", 0.8] | image}
  \end{minipage}\hfill
\end{frame}

