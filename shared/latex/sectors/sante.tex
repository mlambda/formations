\begin{frame}{Médecine et machine learning}
  3 sous-domaines :
  \begin{itemize}
  \item Diagnostiques
  \item Thérapies
  \item Gestion du suivi
  \end{itemize}
\end{frame}

\begin{frame}{Médecine et machine learning}
  Cabinet Frost \& Sullivan :
  \begin{itemize}
    \item \$90 milliards pour le marché de la santé numérique en 2019
    \item \$234,5 milliards d’ici 2023
  \end{itemize}
\end{frame}

\begin{frame}{Diagnostique}
  Des données :
  \begin{itemize}
    \item Images (multispectrales)
    \item ECG
    \item Données génomiques 
    \item Données phénotypiques
  \end{itemize}
\end{frame}

\begin{frame}{Diagnostique}
  Nouveaux usages :
  \begin{itemize}
    \item Smartphones
    \item Bracelets connectés
    \item …
  \end{itemize}
\end{frame}


\begin{frame}{Diagnostique}
  Des pathologies :
  \begin{itemize}
    \item du coeur
    \item neurologiques
    \item ophtalimiques
    \item Rhumatologiques
    \item Cancers
    \item ...
  \end{itemize}
\end{frame}

\begin{frame}{Diagnostique}
  Ophtalmologie, analyse d'image
  \newline
  \newline
  \begin{minipage}[c]{0.65\linewidth}
    \begin{itemize}
    \item Microrétinopathies diabétiques (IDx, AIvision.health, AiScreenings, Eyenuk)
    \item Glaucome (Watson)
    \item Dégénérescence maculaire (DeepMind)
    \item Décollement de rétine (DeepMind)
    \item ...
    \item DeepMind $\approx$50 pathologies : 94\% de précision
    \end{itemize}
  \end{minipage}\hfill
  \begin{minipage}[c]{0.34\linewidth}
    \V{"img/diagramme-oeil" | image("tw", 1)}
  \end{minipage}\hfill
\end{frame}

\begin{frame}{Diagnostique}
  \begin{itemize}
  \item oncologie
  \item cerveau (IRM par Qynapse)
  \item poumon (radiographie par Enlitic, Riverain Technologies ou Infervision)
  \item foie (scanner + IRM par Guerbet + Watson)
  \item sein (mamographies par Volpara Solution, QViewMedical ou Therapixel)
  \item peau (Meilleurs résultats de l'IA face à des spécialistes)
  \item biopsie (exploitation d'analyses de cellules par WebMicroscope)
  \item En particulier pour la prostate par KeeLab ou la vessie par VitaDX
  \end{itemize}
\end{frame}

\begin{frame}{Diagnostique}
  Toutes les pathologies diagnosticables sont potentiellement apprenables
    \begin{itemize}
    \item Cardiologie (échographie, radiographie, IRM, ECG)
    \item Squelette (Ostéoporose, compression de vertèbre)
    \item Système nerveux (lésions du cerveau, sclérose en plaque)
    \end{itemize}
\end{frame}

\begin{frame}{Diagnostique}
  Génomique :
  \newline
  \begin{minipage}[c]{0.49\linewidth}
    \begin{itemize}
    \item Énorme problème de dimensionnalité
    \item Cycle de vie des gènes
    \item Analyse de corrélation entre génome et phénotype
    \end{itemize}
  \end{minipage}\hfill
  \begin{minipage}[c]{0.49\linewidth}
    \V{"img/adn-data" | image("tw", 1)}
  \end{minipage}\hfill
\end{frame}

\begin{frame}{Thérapie}
  Exploitation de gros volumes de données (Génomes, publications)
  \newline
  \newline
  \begin{minipage}[c]{0.49\linewidth}
    \V{"img/adn-big-data" | image("tw", 0.8)}
  \end{minipage}\hfill
  \begin{minipage}[c]{0.49\linewidth}
    \V{"img/big-data-documents" | image("tw", 0.8)}
  \end{minipage}\hfill
\end{frame}


\begin{frame}{Thérapie}
  Découverte de molécules (ReLeaSe, ...) 
  \V{"img/molecule-discovery" | image("th", 0.65)}
\end{frame}

\begin{frame}{Thérapie}
  Modélisation de molécules (DeepMind)
  \V{"img/alphafold" | image("tw", 1)}
\end{frame}

\begin{frame}{Thérapie}
  Simulation biologiques de l'effet de médicaments 
  \begin{itemize}
    \item IKTOS
    \item Atomwise
    \item \dots
  \end{itemize}
\end{frame}

\begin{frame}{Thérapie}
  \begin{minipage}{0.49\linewidth}
    Robo-chirurgien Smart Tissue Autonomous Robot (STAR)
    \begin{itemize}
      \item Sutures automatiques en 2014
      \item Appendicetomie autonome en 2016
      \item Meilleure précision que les chirurgiens dans de \bluelink{https://www.science.org/doi/pdf/10.1126/scirobotics.abj2908}{+ en + de domaines}
      \end{itemize}
  \end{minipage}
  \begin{minipage}{0.49\linewidth}
    \V{"img/robot-chirurgien" | image("th", 0.7)}
  \end{minipage}
\end{frame}

\begin{frame}{Thérapie}
  Lunettes pour aveugles (Panda guide, OrCam Technologies)
  \V{"img/lunettes-pour-aveugle" | image("tw", 0.8)}
\end{frame}
