\begin{frame}{Deployment Strategies}
  Several criteria allow for choosing a deployment strategy:

  \begin{itemize}
    \item Service traffic
    \item Audience (public, internal, other services, ...)
    \item Deployment frequency
    \item ...
  \end{itemize}

  Key concepts:
  \begin{itemize}
    \item Progressive deployment
    \item Rollback (ability to return to the previous state of the system)
  \end{itemize}
\end{frame}

\begin{frame}{Shadow Deployment}
  \begin{itemize}
    \item Deployment of the model alongside the existing system
    \item Model outputs are not used by the application
    \item Analysis of model outputs and decision to continue deployment or not
  \end{itemize}
\end{frame}

\begin{frame}{Blue/Green Deployment}
  \begin{itemize}
    \item Deployment of the new model (green) alongside the existing system (blue)
    \item When tests are successful on the green system, traffic is redirected there
    \item Allows for rollback if issues arise
  \end{itemize}
\end{frame}

\begin{frame}{Canary Deployment}
  \begin{itemize}
    \item Similar to blue/green, two parallel systems
    \item Gradual ramp-up of the green system
  \end{itemize}
\end{frame}

\begin{frame}{Implementation}
  The two most popular options:
  \begin{itemize}
    \item Kubernetes + Istio
    \item Internal tooling
  \end{itemize}
\end{frame}

\begin{frame}{CI/CD}
  The launch of these deployments can be done via git tags and operations in CI/CD.

  This is the \bluelink{https://www.atlassian.com/git/tutorials/gitops}{GitOps} approach, with the \bluelink{https://dvc.org/doc/gto}{GTO} project from DVC, for example.
\end{frame}
