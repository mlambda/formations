\begin{frame}{Data Organization}
  Several options for:
  \begin{itemize}
    \item Structuring (schema, description, ...)
    \item Scaling (SQL, NoSQL, Distributed file system, ...)
  \end{itemize}
\end{frame}

\begin{frame}{\textit{Feature Store}}
  In an MLOps context, often an intermediate step between the original source and processing:
  
  \begin{itemize}
    \item Allows storage optimized for ML
    \item Avoids recomputing the same features
    \item Enables discoverability
  \end{itemize}
  
  See \bluelink{https://docs.feast.dev/}{Feast} for example.
\end{frame}

\begin{frame}{Data Acquisition}
  \begin{itemize}
    \item Aim for a short first iteration to get feedback for subsequent phases
    \item List potential sources and their time/money budget
    \item Potentially have the ML team do initial annotation
    \item Define competent profiles
  \end{itemize}
\end{frame}

\begin{frame}{Iterations on Data}
  \begin{itemize}
    \item Order of magnitude: not more than x10 at once
    \item Work jointly on source quality, processes, volume
    \item Very different in prototyping and production phases:
      \begin{description}
        \item[Prototyping] Gathering enough data to decide go/no-go
        \item[Production] Deep work as the main vector to reach performance ceiling
      \end{description}
  \end{itemize}
\end{frame}

\begin{frame}{Metadata}
  Because data is the heart of an ML system:
  \begin{itemize}
    \item Preservation of their provenance
    \item Tracing of their transformations
    \item Close notion, reproducibility of data acquisition and transformations:
      \begin{itemize}
        \item Increasingly important for regulation
        \item Essential for debugging
      \end{itemize}
  \end{itemize}
\end{frame}

\begin{frame}{Metadata — Consequences}
  Metadata preservation requires:
  
  \begin{itemize}
    \item Strict definition of data acquisition processes
    \item Strict definition of transformations
    \item Implementation of dedicated systems
  \end{itemize}
  
  → Forces many architecture decisions
\end{frame}

\begin{frame}{Processing Pipelines}
  \begin{itemize}
    \item Responding to reproducibility and process automation issues
    \item As often in computing: directed acyclic graphs of operations
    \item Several solutions exist depending on needs
  \end{itemize}
\end{frame}

\begin{frame}{Desirable Characteristics}
  Strengths of a processing pipeline
  \begin{itemize}
    \item Adaptation to batch (development) as well as real-time (production)
    \item Faithfulness of development/production processing
    \item Scalability
    \item Deployment on various targets
    \item Ease of development
    \item Integration with the ecosystem
  \end{itemize}
\end{frame}

\begin{frame}{Main Processing Pipeline Solutions}
  Strong points of each solution:
  
  \begin{description}
    \item[\bluelink{https://www.tensorflow.org/tfx}{TFX}] Deployment options, dev/prod parity, performance
    \item[\bluelink{https://www.kubeflow.org/}{KubeFlow}] Kubernetes integration
    \item[\bluelink{https://dvc.org/}{DVC}] Reproducibility \& “low tech” solution
    \item[\bluelink{https://mlflow.org/}{MLFlow}] Flexible, easy to adopt, supported by clouds
    \item[\bluelink{https://www.pachyderm.com/}{Pachyderm}] Kubernetes integration
  \end{description}
\end{frame}
