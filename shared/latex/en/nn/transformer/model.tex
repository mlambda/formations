\begin{frame}{Modèle à attention}
  \begin{center}
    \bluelink{https://demo.allennlp.org/reading-comprehension}{Démo modèle à attention}
  \end{center}
\end{frame}

\notmog{
  \begin{frame}{Réseaux transformeurs}
    \V{["img/transformer-fun", "tw", 0.9] | image}
  \end{frame}
}

\begin{frame}{Vue générale}
  \V{["img/transformer-network-1", "tw", 0.9] | image}
\end{frame}

\begin{frame}{Architecture encodeur-décodeur}
  \V{["img/transformer-network-2", "tw", 0.9] | image}
\end{frame}

\begin{frame}{Architecture encodeur-décodeur --- Rappel}
  Rappelez-vous, les encodeurs-décodeurs :
  \V{["img/rnn-seq2seq", "tw", 0.9] | image}
\end{frame}

\begin{frame}{Architecture profonde}
  Couches «~simples~» empilées~:
  \V{["img/transformer-network-3", "tw", 0.9] | image}
\end{frame}

\begin{frame}{Architecture globale de l'encodeur et du décodeur}
  \V{["img/transformer-network-5", "tw", 0.9] | image}
\end{frame}

\begin{frame}{Architecture globale de l'encodeur}
  \V{["img/transformer-network-4", "tw", 0.9] | image}
\end{frame}

\begin{frame}{Forme des données en jeu dans l'encodeur}
  \V{["img/transformer-network-7", "tw", 0.9] | image}
\end{frame}

\begin{frame}{Forme des données en jeu dans l'encodeur}
  \V{["img/transformer-network-8", "tw", 0.9] | image}
\end{frame}

\begin{frame}{Self-Attention --- Éléments utilisés}
  \V{["img/transformer-network-9", "tw", 0.9] | image}
\end{frame}

\begin{frame}{Self-Attention --- Combinaison des éléments}
  \V{["img/transformer-network-13", "th", 0.9] | image}
\end{frame}

\begin{frame}{Self-Attention --- Obtention du résultat}
  \V{["img/transformer-network-14", "tw", 0.9] | image}
  Avec le softmax défini ainsi : $\sigma(z_j)=\frac {\mathrm{e}^{z_j}}{\sum _{k=1}^{K}\mathrm{e}^{z_{k}}}$
\end{frame}

\begin{frame}{Self-Attention --- Run sur 2 mots}
  \V{["img/transformer-network-10", "tw", 0.9] | image}
\end{frame}

\begin{frame}{Self-Attention --- Run sur 2 mots}
  \V{["img/transformer-network-11", "tw", 0.9] | image}
\end{frame}

\begin{frame}{Self-Attention --- Run sur 2 mots}
  \V{["img/transformer-network-12", "th", 0.9] | image}
\end{frame}

\begin{frame}{Self-Attention --- Visualisation}
  \V{["img/transformer-network-self-attention", "th", 0.9] | image}
\end{frame}

\begin{frame}{Attention multi-têtes --- Têtes d'attention}
  \V{["img/transformer-network-15", "tw", 0.9] | image}
\end{frame}

\begin{frame}{Attention multi-têtes --- Combinaison}
  \V{["img/transformer-network-16", "tw", 0.9] | image}
\end{frame}

\begin{frame}{Encodeur --- Résumé}
  Un modèle sans récurrence, uniquement des sommes pondérées.

  \V{["img/transformer-network-17", "tw", 0.9] | image}
\end{frame}

\begin{frame}{Encodeur --- Gestion de la position}
  Pour l'instant, modèle invariant à l'ordre.

  $\Rightarrow$ Nécessité d'encoder la position (encodage positionnel).
\end{frame}

\begin{frame}{Encodeur --- Encodage positionnel --- Fonction hardcodée}
  \V{["img/transformer-network-18", "tw", 0.9] | image}
  $\text{PE}_{\text{pos}, 2i} = \sin{(\text{pos} / 10000^{2i / d_{\text{model}}})}$ \\
  $\text{PE}_{\text{pos}, 2i + 1}= \cos{(\text{pos} / 10000^{2i / d_{\text{model}}})}$
\end{frame}

\begin{frame}{Encodeur --- Encodage positionnel --- Alternative}
  Il est aussi possible d'apprendre les embeddings de position, sans détérioration des performances.
\end{frame}

\begin{frame}{Encodeur --- Encodage positionnel --- Exemple}
  \V{["img/transformer-network-19", "tw", 0.9] | image}
\end{frame}

\begin{frame}{Encodeur --- Encodage positionnel --- Visualisation}
  \V{["img/positional-encoding", "tw", 0.9] | image}
\end{frame}

\begin{frame}{Décodeur --- Exemple de décodage}
  \V{["img/transformer-decoding-init-0", "tw", 0.9] | image}
\end{frame}

\begin{frame}{Décodeur --- Exemple de décodage}
  \V{["img/transformer-decoding-init-157", "tw", 0.9] | image}
\end{frame}

\begin{frame}{Décodeur --- Exemple de décodage}
  \V{["img/transformer-decoding-0", "tw", 0.9] | image}
\end{frame}

\begin{frame}{Décodeur --- Exemple de décodage}
  \V{["img/transformer-decoding-14", "tw", 0.9] | image}
\end{frame}

\begin{frame}{Décodeur --- Exemple de décodage}
  \V{["img/transformer-decoding-38", "tw", 0.9] | image}
\end{frame}

\begin{frame}{Décodeur --- Exemple de décodage}
  \V{["img/transformer-decoding-91", "tw", 0.9] | image}
\end{frame}

\begin{frame}{Décodeur --- Exemple de décodage}
  \V{["img/transformer-decoding-112", "tw", 0.9] | image}
\end{frame}

\begin{frame}{Décodeur --- Exemple de décodage}
  \V{["img/transformer-decoding-137", "tw", 0.9] | image}
\end{frame}

\begin{frame}{Décodeur --- Exemple de décodage}
  \V{["img/transformer-decoding-188", "tw", 0.9] | image}
\end{frame}

\begin{frame}{Décodeur --- Exemple de décodage}
  \V{["img/transformer-decoding-207", "tw", 0.9] | image}
\end{frame}

\begin{frame}{Décodeur --- Exemple de décodage}
  \V{["img/transformer-decoding-234", "tw", 0.9] | image}
\end{frame}

\begin{frame}{Décodeur --- Exemple de décodage}
  \V{["img/transformer-decoding-306", "tw", 0.9] | image}
\end{frame}

\begin{frame}{Décodeur --- Exemple de décodage}
  \V{["img/transformer-decoding-333", "tw", 0.9] | image}
\end{frame}

\begin{frame}{Transformer Network}
  Les illustrations sont tirées du billet de blog de Jay Alammar~:
  \url{http://jalammar.github.io/illustrated-transformer/}
\end{frame}
