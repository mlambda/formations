\begin{frame}{(Weak) relation with biology}
  \V{"img/biological-neuron" | image("tw", 1, "en")}
\end{frame}

\begin{frame}{Artificial neuron model}
  \V{"tikz/neural-networks/model-en" | image("tw", 0.9)}
\end{frame}

\begin{frame}{Combination to form a neural network}
  Combination of many neurons to form a neural network:
  \begin{description}
    \item[In parallel] Compute results independently in a same layer
    \item[In series] Take as inputs the output of the previous layer
  \end{description}
\end{frame}

\begin{frame}{Shallow network}
  \centering
  \V{"tikz/neural-networks/shallow-ff-en" | image("tw", 1)}
\end{frame}

\begin{frame}{One hidden layer network}
  \centering
  \V{"tikz/neural-networks/one-hidden-layer-ff-en" | image("tw", 1)}
\end{frame}

\begin{frame}{Deep network}
  \centering
  \V{"tikz/neural-networks/deep-ff-en" | image("tw", 1)}
\end{frame}

\begin{frame}{Activation functions--- ReLU}
  \begin{center}
    \V{"tikz/activations/relu" | image("th", 0.7, "en")}
  \end{center}

\end{frame}

\begin{frame}{A theoretically interesting model}
  \textbf{Kurt Hornik, 1991:} Universal approximation theorem
  \V{"tikz/activations/approximation" | image("tw", 1, "en")}

  \bluelink{https://youtu.be/5oEBiico-I4}{Visualization of the approximation process}
\end{frame}
