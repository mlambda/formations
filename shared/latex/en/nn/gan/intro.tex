\begin{frame}{Principle}
  \begin{itemize}[<+->]
    \item Leverage the discriminative power of NNs to train a generative model
    \item Model learning as a game between two networks
    \item The first network generates samples from random noise
    \item The second network tries to find out whether a sample is true or generated
  \end{itemize}
\end{frame}

\begin{frame}{Architecture}
  \V{"tikz/neural-networks/gan/gan-en" | image("tw", 1)}
\end{frame}

\begin{frame}{Discriminator}
  Trained to maximize the probability of assigning $1$ to true data and $0$ to fake data:

  \[
    \max_D
      \left\{
        \underbrace{\green{y \log D(\vect{x})}}_{\mathclap{\text{true data}}}
        +
        \underbrace{\red{(1 - y) \log (1 - D(\vect{x}))}}_{\mathclap{\text{fake data}}}
      \right\}
  \]

  \begin{itemize}[<+(1)->]
    \item For true data ($y=1$), the term to maximize is $\green{\log D(\vect{x})}$

    $\Rightarrow$ $D(\vect{x})$ should be close to $1$
    \item For fake data ($y=0$), the term to maximize is $\red{\log (1 - D(\vect{x}))}$

    $\Rightarrow$ $D(\vect{x})$ should be close to $0$
  \end{itemize}
\end{frame}

\begin{frame}{Generator}
  Trained to minimize the probability of $D$ assigning $0$ to fake data:

  \[
    \min_G
      \left\{
        (1 - y) \log (1 - D(G(\vect{z})))
      \right\}
  \]
  The term to minimize is $\log (1 - D(G(\vect{z})))$

  $\Rightarrow$ $D(G(\vect{z}))$ should be close to $1$
\end{frame}

\begin{frame}{Complete training function}
  Previous equations combined and averaged aver all training samples:

  \[
    \max_D \min_G \left\{
      \underbrace{\mathbb{E}_{\vect{x} \sim \text{Data}} \log D(\vect{x})}_{\mathclap{\text{Only $D$}}}
      + \underbrace{\mathbb{E}_{\vect{z} \sim \text{Noise}} \log (1 - D(G(\vect{z})))}_{\mathclap{\text{Both $D$ \& $G$}}}
    \right\}
  \]

  $\Rightarrow$ Description of the minimax game between $G$ \& $D$.
\end{frame}

\begin{frame}{Learning algorithm}
  For each training iteration:

  \begin{enumerate}[<+(1)->]
    \item Sample minimatch of noise $\vect{Z}$
    \item Sample minimatch of true examples $\vect{X}$
    \item Update the discriminator parameters ($\theta_D$) with gradient ascent:
    \[
      \theta_D \leftarrow \theta_D + \alpha \nabla_{\theta_D} \frac{1}{m} \sum_{i=1}^m \left[
        \log D\left(\vect{X}_i\right) + \log\left(1 - D\left(G\left(\vect{Z}_i\right)\right)\right)
      \right]
    \]
    \item Update the generator parameters ($\theta_G$) with gradient descent:
    \[
      \theta_G \leftarrow \theta_G - \alpha \nabla_{\theta_G} \frac{1}{m} \sum_{i=1}^m \left[
        \log\left(1 - D\left(G\left(\vect{Z}_i\right)\right)\right)
      \right]
    \]
  \end{enumerate}
  \onslide<+(1)->{Steps 3. and 4. can be repeated several times to stabilize training.}
\end{frame}

\begin{frame}{Training stability}
  Training a GAN is hard:

  \begin{itemize}[<+(1)->]
    \item The discriminator can take over
    \item The generator can loop on only some values (mode collapse)
    \item Depending on the dataset one network might need more updates than the other
    \item The noise distribution is important
    \item The learning rate is important
    \item …
  \end{itemize}

  \onslide<+(1)->{Useful tips: \url{https://github.com/soumith/ganhacks}}
\end{frame}
