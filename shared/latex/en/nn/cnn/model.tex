\begin{frame}{Problem with MLPs for images}
  Say you want to detect objects in an image. A good model should:
  \begin{itemize}
    \item Find the object regardless of its location
    \item Use local data around the object to produce a decision, not the opposite side of the image
  \end{itemize}

  Standard MLPs don't meet those criteria.

  $\Rightarrow$ Convolutional neural networks do.
\end{frame}

\begin{frame}{Convolution}
  Key building block: the cross-correlation operation (misnamed convolution).

  It embodies both \textbf{locality} and \textbf{translation invariance}.

  \V{"img/d2l/correlation" | image("tw", 0.8, "en")}
\end{frame}

\begin{frame}{Convolution on multiple channels}
  Previous slide was simplified: input and intermediate layers often have several channels (think RGB for an image).

  The convolution also has a channels and is applied to all channels:

  \V{"img/d2l/conv-multi-in" | image("th", 0.3, "en")}

  Usually the channel dimension is omitted when describing the convolution ($2 \times 2$ convolution instead of $2 \times 2 \times 2$).
\end{frame}

\begin{frame}{Convolution padding}
  Padding controls the change in size due to the convolution operation by adding $0$s:

  \V{"img/d2l/conv-pad" | image("th", 0.4, "en")}

  If we want a $3 \times 3$ output that matches our input, what can we do?
\end{frame}

\begin{frame}{Convolution stride}
  Convolution strides allow to reduce dimensionality in a convolutional network:

  \V{"img/d2l/conv-stride" | image("th", 0.4, "en")}

  Here the input went from $5 \times 5$ to $2 \times 2$ thanks to the $(3, 2)$ strides
\end{frame}

\begin{frame}{Pooling}
  Pooling is the other standard mechanism to reduce dimension:

  \V{"img/d2l/pooling" | image("tw", 0.8, "en")}

  $n \times m$ pooling is often used with strides $(n, m)$.
\end{frame}

\begin{frame}{Combination into a convolutional neural network}
  Yann LeCun ---~now Turing Award recipient~--- first combined those blocks into a successful network with LeNet:

  \V{"img/d2l/lenet" | image("tw", 1, "en")}
\end{frame}

\begin{frame}{Schematic view of LeNet}
  \V{"img/d2l/lenet-vert" | image("th", 0.7, "en")}
\end{frame}

\begin{frame}{Training}
  Done by gradient descent, just like MLPs.

  Less parameters but more operations: greatly benefits from GPUs.

  \V{"img/gpus" | image("th", 0.4, "en")}
\end{frame}

\begin{frame}{Visual cortex architecture}
  \begin{minipage}[l]{0.50\linewidth}
    \V{"img/visual-cortex" | image("tw", 1.0, "en")}
  \end{minipage}\hfill
  \begin{minipage}[l]{0.49\linewidth}
    Layered connections:
    \begin{description}
    \item[V1] lines orientation
    \item[V2] shapes, sizes, colors
    \item[V3] motricity
    \item[V4] object recognition
    \item[V5] object tracking
    \end{description}
  \end{minipage}\hfill
\end{frame}

\begin{frame}{Demonstration}
  \bluelink{https://adamharley.com/nn_vis/cnn/3d.html}{Convolutional neural networks}
\end{frame}
