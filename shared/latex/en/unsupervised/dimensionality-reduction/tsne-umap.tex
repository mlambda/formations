\begin{frame}{Projections non linéaires}
  Plus intéressant pour découvrir des patterns en haute dimension~:

  \begin{itemize}[<+->]
    \item t-SNE
    \item UMAP
  \end{itemize}
\end{frame}

\begin{frame}{Méthode}
  Algorithmes de disposition de graphe.

  Minimisation de la différence topologique entre deux graphes~:

  \begin{itemize}[<+->]
    \item Graphe construit en haute dimension
    \item Graphe construit dans la dimension de projection
  \end{itemize}
\end{frame}

\begin{frame}{Paramètres}
  Dans les deux algorithmes, on peut contrôler le trade-off structure local / structure globale~:

  \begin{itemize}[<+->]
    \item La perplexité de t-SNE $\approx$ nombre de voisins considérés comme connectés
    \item UMAP utilise explicitement un argument de nombre de voisins connectés
  \end{itemize}
\end{frame}

\begin{frame}{t-SNE --- Bonnes pratiques \& Analyse}
  \bluelink{https://distill.pub/2016/misread-tsne/}{Article t-SNE}
\end{frame}

\begin{frame}{UMAP --- Bonnes pratiques \& Analyse}
  \bluelink{https://pair-code.github.io/understanding-umap/}{Article UMAP}
\end{frame}
