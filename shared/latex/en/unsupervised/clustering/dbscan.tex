
\begin{frame}{DBSCAN --- Introduction}
  Density-Based Spatial Clustering of Applications with Noise…

  \begin{itemize}
    \item Handles noise (clusterless points)
    \item Works well in practice
    \item Does not require a predefined number of clusters
  \end{itemize}
\end{frame}

\begin{frame}{DBSCAN --- Process}
  As long as there are unlabeled points:

  \begin{enumerate}
    \item Define an $\epsilon$ distance and a minimum number of neighbors $v$
    \item Take a non-labeled point at random and look at its $epsilon$-neighborhood
    \item If there are $v$ neighbors, we create a new cluster
      \begin{itemize}
      \item[Iterate] Expand the cluster from near to near data points
      \end{itemize}
    \item Otherwise do nothing (Data point is noise)
    \end{enumerate}
\end{frame}

\begin{frame}{DBSCAN --- Demonstration}
  \V{["img/dbscan-1", "tw", 0.9] | image}
\end{frame}

\begin{frame}{DBSCAN --- Demonstration}
  \V{["img/dbscan-2", "tw", 0.9] | image}
\end{frame}

\begin{frame}{DBSCAN --- Demonstration}
  \V{["img/dbscan-3", "tw", 0.9] | image}
\end{frame}

\begin{frame}{DBSCAN --- Demonstration}
  \V{["img/dbscan-4", "tw", 0.9] | image}
\end{frame}

\begin{frame}{DBSCAN --- Demonstration}
  \V{["img/dbscan-5", "tw", 0.9] | image}
\end{frame}

\begin{frame}{DBSCAN --- Demonstration}
  \V{["img/dbscan-6", "tw", 0.9] | image}
\end{frame}

\begin{frame}{DBSCAN --- Demonstration}
  \V{["img/dbscan-7", "tw", 0.9] | image}
\end{frame}

\begin{frame}{DBSCAN --- Demonstration}
  \V{["img/dbscan-8", "tw", 0.9] | image}
\end{frame}

\begin{frame}{DBSCAN --- Demonstration}
  \V{["img/dbscan-9", "tw", 0.9] | image}
\end{frame}
