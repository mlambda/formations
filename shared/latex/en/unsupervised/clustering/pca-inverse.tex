\begin{frame}{Clustering par PCA}
  (Souvenez-vous)

  \begin{itemize}[<+->]
    \item Calcul de la matrice de covariance (resp corrélation)~:
      \[
      \frac{1}{N} \times \overline{X^T} \times \overline{X}
      \quad
      ( \frac{1}{N} \times \widetilde{X^T} \times \widetilde{X} )
      \]
    \item Calcul des vecteurs propres de cette matrice~: ce sont les combinaisons linéaires des nouvelles features
    \item Tri des vecteurs par valeur propre décroissante
    \item Réduction de dimension : on ne conserve que les $k$ premiers vecteurs propres
  \end{itemize}
\end{frame}

\begin{frame}{Clustering par PCA}
  Considérer les individus comme des features et les features comme des individus.
  
  Les vecteurs propres ayant une grande valeur propre peuvent être considérés comme des centre de cluster d'individus.
\end{frame}
