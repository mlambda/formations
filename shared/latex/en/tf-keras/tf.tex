\begin{frame}{Introduction}
  \V{["img/logos/tensorflow", "th", 0.7, "en"] | image}
\end{frame}

\begin{frame}{API structure}
  \centering
  \V{["tikz/tf-api-en", "tw", 1] | image}
\end{frame}

\begin{frame}{Two styles}
  \begin{description}
    \item[Graph mode] Definition of a computation graph for later execution
    \item[Eager mode] Definition of Python functions that directly operate on values
  \end{description}
\end{frame}

\begin{frame}{TensorFlow 1 vs TensorFlow 2}
  \centering
  Main differences

  \begin{minipage}[l]{0.5\linewidth}
    \centering
    \textbf{TF1}
    \begin{itemize}
      \item Graph mode by default
      \item Independent High level API Keras
    \end{itemize}
  \end{minipage}\hfill
  \begin{minipage}[l]{0.49\linewidth}
    \setcounter{beamerpauses}{1}
    \centering
    \textbf{TF2}
    \begin{itemize}
      \item Eager mode by default
      \item High level API Keras merged in TensorFlow 2
    \end{itemize}
  \end{minipage}\hfill
\end{frame}

\begin{frame}{Tensors --- Core objects in TensorFlow 2}
  Multidimensional arrays, used for:

  \begin{itemize}
    \item Model weights
    \item Data
  \end{itemize}

  \onslide<+->{\texttt{tf.Tensor} $\approx$ \texttt{numpy.ndarray} in the universe of TensorFlow 2.}
\end{frame}

\begin{frame}{Tensors --- Creation}
  \begin{itemize}
    \item Directly from \texttt{numpy.ndarray} or lists
    \item Randomly sampled
    \item Result of a call to a TensorFlow 2 function with \texttt{numpy.ndarray}
    \item \texttt{tf.data} pipeline
  \end{itemize}
\end{frame}

\begin{frame}{\texttt{tf.Variable}}
  Mutable wrapper over \texttt{tf.Tensor} for model weights.
\end{frame}

\begin{frame}{Computation graph}
  \V{["tikz/tf-graph", "th", 0.8, "en"] | image}
\end{frame}

\begin{frame}{Computation graph}
  Conversion from eager mode to graph mode by \texttt{tf.function}. Allows:

  \begin{itemize}
    \item Computation graph optimization
    \item Easy deployment to Android, iOS, TPU, GPU, …
    \item Running the model without a Python interpreter!
  \end{itemize}
\end{frame}

\begin{frame}{Automatic differentiation}
  Gradients automatic computation to ease the backpropagation implementation during gradient descent.

  \texttt{tf.GradientTape} records operations on variables: allows autodiff.
\end{frame}

\begin{frame}{Code organization}
  \texttt{tf.Module} groups code of a layer or logical unit.

  A model = a combination of \texttt{tf.Module}s.

  Main API:

  \begin{itemize}
    \item Access model variables with \texttt{model.variables}
    \item Access trainable model variables with \texttt{model.trainable\_variables}
    \item Save weights with \texttt{tf.train.Checkpoint}
    \item Load weights with \texttt{tf.train.Checkpoint.restore}
    \item Save full model with \texttt{tf.saved\_model.save}
    \item Load full model with \texttt{tf.saved\_model.load}
  \end{itemize}
\end{frame}