\begin{frame}{CNIL}
  \V{["img/logos/cnil", "tw", 0.8] | image}
\end{frame}

\begin{frame}{Comité éthique \& intelligence artificielle}
  Étude \& rapport par la CNIL début 2017. Recommandations~:
  \begin{itemize}[<+->]
    \item Former à l’éthique tous les acteurs-maillons de la « chaîne algorithmique » 
    \item Rendre les systèmes algorithmiques compréhensibles
    \item Designer les systèmes algorithmiques au service de la liberté humaine
    \item Constituer une plateforme nationale d’audit des algorithmes
    \item Encourager la recherche sur l’IA éthique
    \item Renforcer la fonction éthique au sein des entreprises
  \end{itemize}
\end{frame}

\begin{frame}{RGPD}
  Introduit en 2018 par le parlement européen.
  \V{["img/logos/rgpd", "tw", 0.8] | image}
\end{frame}

\begin{frame}{Loi bioéthique}
  À propos du «~traitement algorithmique de données massives~» pour des «~actes à visée préventive, diagnostique ou thérapeutique~»~: le professionnel de santé sera tenu d'informer le patient «~de cette utilisation et des modalités d’action de ce traitement~».
\end{frame}
