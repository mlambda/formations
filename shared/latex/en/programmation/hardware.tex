\begin{frame}{Memory}
    Component capable of storing information. \\
    \newline
    Different \textbf{memory tiers}:
    \begin{description}
        \item[Cache] Internal memory of the processor
        \item[RAM] Random Access Memory used by programs
        \item[Disk] Storage memory
        \item[Other media] floppy disk, USB drive, etc.
    \end{description}
\end{frame}

\begin{frame}{Processor}
    Component capable of performing arithmetic and logical operations:
    \begin{itemize}
        \item Management of \textbf{memory tiers}
        \begin{itemize}
            \item Moving the read head
            \item Read/write operations
            \item ...
        \end{itemize}
        \item Operations on \textbf{variables} stored in \textbf{Cache}:
        \begin{itemize}
            \item AND
            \item OR
            \item XOR
            \item ...
        \end{itemize}
    \end{itemize}
    These operations are performed in a specific order, as defined by a program.
\end{frame}

\begin{frame}{Summary of Different Memory Tiers}
    \V{"img/memory-hierarchy" | image("th", 0.5)}
    Useful to keep in mind when optimizing a program!
\end{frame}