\begin{frame}{Introduction}
  \V{["img/ml-craftmanship", "tw", 0.8] | image}
\end{frame}


\begin{frame}{Statistical biases}
  Be careful about the biases of your data!
  \begin{itemize}[<+->]
  \item Confounding variables (\emph{e.g.} \emph{obesity} in the correlation between \emph{eating meat} and \emph{colon cancer})
  \item Statistical biases
    \begin{itemize}
    \item selection, autoselection
    \item measure
    \item attrition
    \item …
    \end{itemize}
  \item Spuriously correlated variables 
  \end{itemize}
\end{frame}

\begin{frame}{Data quality}
  \begin{center}
    Trash in $\Rightarrow$ Trash out
  \end{center}

  Crucial since a ML program is defined \textbf{in large parts} by data.
\end{frame}

\begin{frame}{Data processing pipeline}
  \begin{itemize}
  \item Missing values handling
  \item Preprocessing (text, image)
  \item Standardization
  \item Transformation
  \end{itemize}
\end{frame}

\begin{frame}{Missing values}
  Detrimental to lots of models. Several options:
  \begin{itemize}[<+(1)->]
  \item Delete samples with missing values
  \item Delete features with missing values
  \item Replace values (imputation):
    \begin{itemize}
    \item by a constant
    \item by the mean of the feature
    \item by the prediction of a model based on the other features
    \end{itemize}
  \end{itemize}
\end{frame}

\begin{frame}{Preprocessing}
  \begin{itemize}[<+->]
  \item Tokenize, POS-tag text (\url{https://spacy.io/})
  \item Use a pretrained network to classify images
    (\url{https://keras.io/applications/})
  \item Apply a Fourier transform on sound data
  \item …
  \end{itemize}

  \onslide<+->{Very dependent on the application domain.}
\end{frame}

\begin{frame}{Standardization}
  Many models are more efficient with normally distributed variables and values around $[-5, 5]$ :
  \begin{itemize}
  \item Center on zero and divide by standard deviation
  \item Handle non-normality with a Box-Cox transformation or other corrections
  \end{itemize}
\end{frame}

\begin{frame}{Categorical variables}
  When you have categorical data and the model you want to use do not handle them, you can use:
  \begin{itemize}
  \item Label encoding if ordinal
  \item One-hot encoding if nominal
  \end{itemize}
\end{frame}

\begin{frame}{Label encoding}
  For ordinal variables:
  \begin{columns}
    \begin{column}{.5\tw}
      \begin{figure}
        \centering
        Ordinal\\[.5cm]
        \begin{tabular}{c}
          \toprule
          Temperature \\
          \midrule
          Cold \\
          Cold \\
          Lukewarm \\
          Warm \\
          Lukewarm \\
          \bottomrule
        \end{tabular}
      \end{figure}
    \end{column}
    \begin{column}{.5\tw}
      \begin{figure}
        \centering
        Label encoded\\[.5cm]
        \begin{tabular}{ccc}
          \toprule
          Temperature \\
          \midrule
          1 \\
          1 \\
          2 \\
          3 \\
          2 \\
          \bottomrule
        \end{tabular}
      \end{figure}
    \end{column}
  \end{columns}
\end{frame}

\begin{frame}{One-hot encoding}
  Replace a feature by $n$ features, $n$ being the number of categories.
  \begin{columns}
    \begin{column}{.5\tw}
      \begin{figure}
        \centering
        Nominal\\[.5cm]
        \begin{tabular}{c}
          \toprule
          Color \\
          \midrule
          Red \\
          Red \\
          Yellow \\
          Green \\
          Yellow \\
          \bottomrule
        \end{tabular}
      \end{figure}
    \end{column}
    \begin{column}{.5\tw}
      \begin{figure}
        \centering
        One-hot encoded\\[.5cm]
        \begin{tabular}{ccc}
          \toprule
          Red & Yellow & Green \\
          \midrule
          1 & 0 & 0 \\
          1 & 0 & 0 \\
          0 & 1 & 0 \\
          0 & 0 & 1 \\
          0 & 1 & 0 \\
          \bottomrule
        \end{tabular}
      \end{figure}
    \end{column}
  \end{columns}
\end{frame}

\begin{frame}{Data exploration}
  Goals:
  \begin{itemize}[<+(1)->]
  \item Notice missing preprocessing
  \item Understand the output variable: distribution, class balance, correlated features, …
  \item Detect correlations between input variables
  \item Estimate the necessary model complexity
  \end{itemize}

  \onslide<+(1)->{
    \begin{alertblock}{Warning}
      It is important not to explore the test data. It should be kept secret at all times.
    \end{alertblock}
  }
\end{frame}

\begin{frame}{Plotting}
  Several plots can be created to explore data depending on the features visualized. Main possibilities:

  \begin{itemize}
  \item Describe a distribution
  \item Describe the correlation of two distributions
  \item Visualize summaries for several variables at once
  \end{itemize}
  The plots shown in the coming slides are created with
  \bluelink{https://seaborn.pydata.org/}{\texttt{seaborn}}.
\end{frame}

\begin{frame}{Plotting — Count plot}
  \V{["img/countplot-nzmog", "th", 0.7, "en"] | image}
\end{frame}

\begin{frame}{Plotting — Histogram}
  \V{["plt/histplot-en", "th", 0.7] | image}
\end{frame}

\begin{frame}{Plotting — Q-Q plot}
  \V{["plt/qq-plot-en", "th", 0.5] | image}
  Not available in \bluelink{https://seaborn.pydata.org/}{\texttt{seaborn}}
  but in
  \bluelink{http://www.statsmodels.org/stable/index.html}{\texttt{statsmodel}}
  or
  \bluelink{https://docs.scipy.org/doc/scipy/reference/stats.html}{\texttt{scipy.stats}}.
\end{frame}

\begin{frame}{Plotting — Bar plot}
  \V{["img/barplot-nzmog", "th", 0.7, "en"] | image}
\end{frame}

\begin{frame}{Plotting — Scatter plot}
  \V{["img/scatter-nzmog", "th", 0.7, "en"] | image}
\end{frame}

\begin{frame}{Plotting — Violin plot}
  \V{["img/violinplot-nzmog", "th", 0.7, "en"] | image}
\end{frame}

\begin{frame}{Plotting — Pair plot}
  \V{["img/pairplot-nzmog", "th", 0.7, "en"] | image}
\end{frame}

\begin{frame}{Plotting — Correlation matrix}
  \V{["plt/corrmat-en", "th", 0.7] | image}
\end{frame}

\begin{frame}{Guidelines to explore a dataset}
  \begin{itemize}[<+->]
  \item Analyze the output variable(s) (countplot/distplot)
  \item Find the input variables with the highest correlation to the output variable(s)
  \item Analyze those variables
  \item Look for outliers to filter/investigate
  \end{itemize}
\end{frame}
