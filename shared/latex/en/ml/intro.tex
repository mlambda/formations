\begin{frame}{A vast field}
  \V{["plt/data-science-en", "th", 0.7] | image}
\end{frame}

\begin{frame}{Name hierarchy}
  \V{["img/ia-ml-deep", "th", 0.8] | image}
\end{frame}

\begin{frame}{Machine Learning}
  New way to approach \textbf{software development}.
  \vfill
  \begin{block}{Paradigm shift}
  Explicit programming $\rightarrow$ Implicit programming
  \end{block}
\end{frame}

\begin{frame}{Implicit programming}
  Implicitly define an algorithm \emph{based on data}.

  Usually: find a function that given $N$ data points (with $D$ dimensions) will compute $N$ outputs (with $K$ dimensions).
\end{frame}

\begin{frame}{Engineering}
  \V{["img/ml-craftmanship", "tw", 0.8] | image}
\end{frame}

\begin{frame}{Raw material: data}
  Each aspect of the problem is represented by one or several variables:
  \V{["tikz/variables-en", "tw", 0.8] | image}
\end{frame}

\begin{frame}{Learning paradigms}

  \textbf{Supervised}, \textbf{unsupervised} or \textbf{reinforcement} learning?
\end{frame}

\begin{frame}{Supervised learning}
  \textbf{Predict} a numerical value (\textbf{regression}) or a class (\textbf{classification}).
\end{frame}

\begin{frame}{Unsupervised learning}
  Discover \textbf{structure} in data:

  \begin{itemize}[<+(1)->]
    \item Anomaly detection
    \item Clustering
    \item Dimensionality reduction
  \end{itemize}
\end{frame}

\begin{frame}{Reinforcement learning}
  Learn an efficient \textbf{policy} in a \textbf{world} where \textbf{actions} are \textbf{rewarded} (possibly negatively)
  \vfill
  \begin{columns}[T]
    \begin{column}{0.5\textwidth}
      \V{["img/chess", "tw", 0.9, "en"] | image}
    \end{column}
    \begin{column}{0.5\textwidth}
      \V{["img/darpa-2007", "tw", 0.9, "en"] | image}
    \end{column}
  \end{columns}
\end{frame}

\begin{frame}{Example --- Univariate linear regression}
  Predict the value of a variable based on the value of another variable

  \V{["plt/reg-plot-en", "th", 0.65] | image}
\end{frame}

\begin{frame}{Example --- Multivariate linear regression}
  Predict the value of a variable based on the value of other variables

  \V{["plt/regression-hyperplan", "th", 0.65, "en"] | image}
\end{frame}

\begin{frame}{Example --- Classification with CNNs --- Model}
  \V{["img/cnn-schema", "tw", 1] | image}
\end{frame}

\begin{frame}{Example --- Classification with CNNs --- Data}
  \V{["img/caltech256", "tw", 0.8] | image}
\end{frame}

\begin{frame}{Example --- Reinforcement learning}
  \V{["img/car-dashboard", "tw", 0.8] | image}
\end{frame}

\begin{frame}{Domain topology}
  \V{["tikz/ml-illustration-en", "tw", 0.8] | image}
\end{frame}

\begin{frame}{Aspects of the ML domain}
  There are many ways to approach machine learning:
  \begin{itemize}[<+(1)->]
  \item Learning paradigms (supervised, unsupervised, reinforcement, online, …)
  \item Models (trees, grammars, automata, neural networks, …)
  \item Data (tabular, images, text, video, graphs, …)
  \item Techniques (statistical, symbolic, probabilistic, …)
  \item Constraints (realtime, embedded, big data, multilingual, …)
  \end{itemize}

  \onslide<+(1)->{→ Extremely vast field field.}
\end{frame}

\begin{frame}{Criteria to navigate the ML domain}
  \begin{itemize}[<+->]
  \item Quantity of available data
  \item Quality of the learning signal
  \item Difficulty of the problem to solve
  \item Interpretability needs
  \item Technical constraints
  \item … and others, dependending on your application domain
  \end{itemize}
\end{frame}

\begin{frame}{Takeaways}
  \begin{itemize}
  \item The machine learning field is vast
  \item There is probably a model/paradigm for your needs
  \item The key point is to correctly define the criteria to optimize
  \end{itemize}
\end{frame}

\begin{frame}{Discussion}
  \begin{itemize}
  \item To which data are you going to apply machine learning?
  \item To solve which problem?
  \item Will you require interpretability or will raw prediction performance be enough?
  \item Do you have technical constraints?
  \end{itemize}
\end{frame}
