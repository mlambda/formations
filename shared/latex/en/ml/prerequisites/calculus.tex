
\begin{frame}{Goal}
  Understand how to minimize a continuous function with calculus.
\end{frame}

\begin{frame}{Key idea}

  \begin{enumerate}
    \item Choose a starting point and \emph{follow the slope} of the function to get closer to a minimum
    \item Repeat
  \end{enumerate}

  Slope = derivative

  → To \emph{follow the slope}, slightly pull the current point towards the opposite of the derivative.
\end{frame}

\begin{frame}{Positive derivative}

  \V{["plt/slope-positive-en", "th", 0.5] | image}

  Opposite of the derivative = $-2$. With a factor of $0.1$, the current point goes from $1$ to $0.8$.
\end{frame}

\begin{frame}{Negative derivative}

  \V{["plt/slope-negative-en", "th", 0.5] | image}

  Opposite of the derivative = $2$. With a factor of $0.1$, the current point goes from $-1$ to $-0.8$.
\end{frame}

\begin{frame}{Multivariate case}
  Use of partial derivatives instead of derivatives. Same principle:
  \V{["plt/gradient-descent-en", "th", 0.65] | image}
\end{frame}
