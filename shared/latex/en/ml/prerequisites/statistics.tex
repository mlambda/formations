
\begin{frame}{Goal}
  \begin{itemize}
  \item Describe and understand data
  \item Correct it to ease future processing
  \end{itemize}
\end{frame}

\begin{frame}{Types of variables}
  \V{["tikz/variables-en", "tw", 1] | image}
\end{frame}

\begin{frame}{Hypotheses}

  Prerequisites for most of the statistical tools:
  \begin{itemize}
  \item All data samples must be generated from \textbf{the same distribution}
  \item All data samples must be independent from the others
  \end{itemize}
  \alert{Not always the case in practice, why?}
\end{frame}

\begin{frame}{Variance}
  Measures how far numbers are spread out from their average value:

  \[
    V(X) = \mathbb{E}\left[(X - \mathbb{E}[X])^2\right]
  \]

  To compute it:

  \[
    V(X) = \frac{1}{n}\sum_{i = 1}^{n}(x_i - \bar{x})^2
  \]
\end{frame}

\begin{frame}{Standard deviation}
  Square root of the variance:

  \[
    \sigma(X) = \sqrt{V(X)}
  \]

\end{frame}

\begin{frame}{Standard deviation — Rule of 68, 95 and 99.7}

  For gaussian distributions:

  \V{["img/68-95-99,7", "th", 0.7] | image}
\end{frame}

\begin{frame}{Quartiles, quantiles}

  Quartiles ($Q_1$, $Q_2$ et $Q_3$) divide data in 4
  intervals containing the same number of samples.

  More generally, $n$ intervals can be used (quantiles). Most used besides quartiles are deciles and percentiles.
\end{frame}

\begin{frame}{Boxplot}

  \V{["img/boxplot", "th", 0.8] | image}
\end{frame}

\begin{frame}{Covariance}
  Measures the joint variability of 2 random variables (here $X$ and $Y$):

  \[
    V(X) = \mathbb{E}\left[(X - \mathbb{E}[X])(X - \mathbb{E}[X])\right]
  \]
  \[
    \cov(X, Y) = \mathbb{E}\left[(X - \mathbb{E}[X])(Y - \mathbb{E}[Y])\right]
  \]

  To compute it:

  \[
    \cov(X, Y) = \frac{1}{n}\sum_{i = 1}^{n}(x_i - \bar{x})(y_i - \bar{y})
  \]
\end{frame}

\begin{frame}{Correlation}
  Covariance divided by the product of the standard deviations:

  \[
    \corr(X, Y) = \frac{\cov(X, Y)}{\sigma_X\sigma_Y}
  \]

  \alert{Interest ?}\visible<+(1)->{ Unitless.}
\end{frame}

\begin{frame}{Normality test}

  Test if a distribution is well modeled by a gaussian distribution:

  \begin{itemize}
  \item Skewness
  \item Kurtosis
  \item …
  \end{itemize}
\end{frame}

\begin{frame}{Skewness}

  \V{["img/skew", "tw", 1] | image}

  \[
    \asym(X) = \mathbb{E} \left[\left(\frac{X - \bar{X}}{\sigma}\right)^3\right]
  \]
\end{frame}

\begin{frame}{Kurtosis}

  \V{["img/kurtosis", "th", 0.5] | image}

  \[
    \kurt(X) = \mathbb{E} \left[\left(\frac{X - \mu}{\sigma}\right)^4\right]
  \]
\end{frame}

\begin{frame}{Correction of a non-normal distribution}
  Skewness and kurtosis can be corrected with a Box-Cox transformation (or a log transformation).
\end{frame}
