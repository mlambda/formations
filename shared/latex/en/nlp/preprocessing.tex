\begin{frame}{Collecte}
  Des sources variées :
  \begin{itemize}
  \item Wikipedia
  \item Articles de journaux
  \item Littérature
  \item User Generated Content
    \begin{itemize}
    \item Blogs
    \item Commentaires
    \item Réseaux sociaux
    \end{itemize}
  \end{itemize}
  Une source $\Rightarrow$ un «~web scraper~»
\end{frame}

\begin{frame}{Caractères accentués et spéciaux}
  \mintedcustom{demo-special-character}{py}{text}
\end{frame}

\begin{frame}{Tokenisation}
  Séparer une chaîne de caractères en tokens n'est pas trivial :

  \begin{center}
    Le Dr. Pond èleve des poules. L'éleveur les sur-exploite.
  \end{center}

  (en phrases ou en mots)
\end{frame}

\begin{frame}{Étiquetage morpho-syntaxique}
  \V{["img/pos-tagging", "tw", 0.8] | image}
\end{frame}

\begin{frame}{Reconnaissance d'entités nommées}
  \V{["img/named-entity-recognition", "tw", 0.8] | image}
\end{frame}

\begin{frame}{Analyse syntaxique}
  \V{["img/parse-tree", "tw", 0.8] | image}
\end{frame}

\begin{frame}{Lemmatisation}
  Exprimer les mots (ou groupes de mots) sous une forme canonique : \\

  \centering
  \begin{tabular}{cc}
    \textbf{Mot} & \textbf{Lemme} \\
    jouant & jouer \\
    ont été jouées & jouer \\
    étoiles & étoile \\
    claires & clair \\
    noire & noir
  \end{tabular}
\end{frame}

\begin{frame}{Outils --- WordNet}
  \V{["img/wordnet-1", "th", 0.7] | image}  
  Projet sur le français : WOLF (Wordnet Libre du Français) 
\end{frame}

\begin{frame}{Outils --- DBpedia}
  \V{["img/dbpedia", "tw", 0.9] | image}
\end{frame}

\notmog{
  \begin{frame}{Utilisation actuelle}
    \V{["img/chatbot", "tw", 0.9] | image}
  \end{frame}
}

\notmog{
  \begin{frame}{Maintenabilité}
    \V{["img/usine-a-gaz", "tw", 0.9] | image}
  \end{frame}
}