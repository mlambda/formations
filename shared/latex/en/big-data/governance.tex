\begin{frame}{Governance in big data}
  \begin{itemize}
    \item Establish company-wide (or department-wide) guidelines
    \item Ensure that those guidelines are applied
  \end{itemize}
\end{frame}

\begin{frame}{Big data governance aspects}
  \begin{description}
    \item[Data cataloging] Data discovery and understanding
    \item[Data quality] Data quality monitoring and fixing
    \item[Metadata] Structure, semantics, guidelines about the data
    \item[Data lineage] Data origin \& transformation tracking
    \item[Data security] Breach and unauthorized access prevention
  \end{description}
\end{frame}

\begin{frame}{Data governance tools example — Data model}
  Produce a model of all the data related to (a part of) the company. Reflexion organized in three tiers~:

  \begin{description}
    \item[Conceptual tier] Description of the semantics of modeled data (\textit{e.g.}, UML)
    \item[Logical tier] Logical organization of the data (\textit{e.g.}, database tables definition)
    \item[Physical tier] Pysical organization of the data (\textit{e.g.}, database configuration, sharding, etc.)
  \end{description}

  → Makes all subsequent data usage easier by providing a clear and optimized interface to the data.
\end{frame}

\begin{frame}{Data governance tools example — Data catalog}
  Centralized place to organize data~:

  \begin{itemize}
    \item Source of truth for all the data of (a part of) the company
    \item Includes metadata (including provenance \& lineage)
    \item Makes all the data discoverable
  \end{itemize}

  → Reduces duplication, greatly enhances data usage company-wide.
\end{frame}
