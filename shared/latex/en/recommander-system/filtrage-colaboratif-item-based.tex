\begin{frame}
  \frametitle{item-based}
  \begin{center}
    \begin{tabular}{|l|c|c|c|c|}
      \hline
      & item 1 & item 2 & item 3 & item 4 \\
      \hline
      u1 & 1 & 2 & 5 & ? \\
      \hline
      u2 & 5 & 1 & 4 & 3 \\
      \hline
      u3 & ? & ? & 1 & 5 \\
      \hline
      u4 & 3 & 3 & ? & 4 \\
      \hline
    \end{tabular}
  \end{center}
  Pour remplir les cellules manquantes, au lieu de traiter les lignes on traite les colonnes. \\
  Une première solution est de transposer la matrice et d'appliquer l'algorithme décrit pour le user-based.
\end{frame}

\begin{frame}
  \frametitle{item-based}
  Méthode couramment utilisée $\Rightarrow$ régression simplifié. \\
  \begin{center}
    \begin{tabular}{|l|c|c|c|c|}
      \hline
      & item 1 & item 2 & item 3 & item 4 \\
      \hline
      u1 & 1 & 2 & 5 & \red{?} \\
      \hline
      u2 & 5 & 1 & 4 & 3 \\
      \hline
      u3 & ? & ? & 1 & 5 \\
      \hline
      u4 & 3 & 3 & ? & 4 \\
      \hline
    \end{tabular}
  \end{center}
\end{frame}

\begin{frame}
  \frametitle{item-based}
  Sachant la colonne item1, on cherche uniquement une valeur de biais telle que :
  \begin{center}
    $item4 = item1 + b_1$
  \end{center}
  Pour ce faire, on prend le biais moyen entre ces deux colonnes :
  \begin{center}
    $b_1 = \frac{\sum_{i \in I} (item4_i - item1_i)}{\#I}$, dans notre exemple : $b_1 = \frac{(3-5)+(4-3)}{2} = -0.5$ \\
    $\;$\\
    \begin{tabular}{|l|c|c|c|c|}
      \hline
      & item 1 & item 2 & item 3 & item 4 \\
      \hline
      u1 & 1 & 2 & 5 & \red{?} \\
      \hline
      u2 & \blue{5} & 1 & 4 & \blue{3} \\
      \hline
      u3 & \blue{?} & ? & 1 & \blue{5} \\
      \hline
      u4 & \blue{3} & 3 & ? & \blue{4} \\
      \hline
    \end{tabular}
  \end{center}
\end{frame}

\begin{frame}
  \frametitle{item-based}
  On calcul alors les biais pour chaque colonne, nous donnant une prediction de la valeur u1(item4) sachant u1(item1), une autre sachant u1(item2) et une dernière sachant u1(item3). \\
  La prédiction finale pour u1(item4) est la moyenne pondérée de toutes ces prédictions. \\
  $b_1 = -0.5 \Rightarrow u1(item4)_1 = 0.5$ \\
  $b_2 = 1.5 \Rightarrow u1(item4)_2 = 2.5$ \\
  $b_3 = 1.5 \Rightarrow u1(item4)_3 = 2.5$ 
  \begin{center}
    $u1(item4) = \frac{\#I_1*0.5+\#I_2*2.5\#I_3*2.5}{\#I_1+\#I_2+\#I_3} = \frac{11}{6} \approx 1.83$ 

    \begin{tabular}{|l|c|c|c|c|}
      \hline
      & item 1 & item 2 & item 3 & item 4 \\
      \hline
      u1 & 1 & 2 & 5 & \textbf{\green{1.83}} \\
      \hline
      u2 & 5 & 1 & 4 & 3 \\
      \hline
      u3 & ? & ? & 1 & 5 \\
      \hline
      u4 & 3 & 3 & ? & 4 \\
      \hline
    \end{tabular}
  \end{center}
\end{frame}

\begin{frame}
  \frametitle{item-based}
  \begin{itemize}[<+->]
  \item Calculs simples à mettre en oeuvre et à tenir à jour \\
  \item Beaucoup de parmètres à maintenir quand on a beaucoup d'item : $\left(\frac{\#item(\#item-1)}{2}\right)$
  \item ``Cold start problem''
  \end{itemize}
\end{frame}

