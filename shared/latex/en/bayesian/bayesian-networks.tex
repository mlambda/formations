\begin{frame}{Modéles graphiques}
  Utilisation de graphes orientés pour représenter les dépendances entre les variables :
  \begin{itemize}
  \item Noeuds $\Leftrightarrow$ Variables
  \item Arcs $\Leftrightarrow$ Probabilités conditionnelles
  \end{itemize}
  \V{["tikz/graph/oriented", "th", 0.5] | image}
\end{frame}

\begin{frame}{Exemple de distributions}
  Graphe orienté acyclique
  \V{["tikz/bayesian/net", "th", 0.6] | image}
\end{frame}

\begin{frame}{Structure}
  Si possible, la structure est donnée par un expert.

  Dans le cas contraire, la recherche de structure est un problème NP-complet. Pour une centaine de variables dans le graphe, cela reste "calculable". Au delà, il faut utiliser des méthodes approchées
  \begin{itemize}
  \item Optimisation d'un score (ex : probabilité à priori des données d'apprentissage) à l'aide d'une stratégie de recherche
  \item Markov Chain Monte-Carlo
  \item ...
  \end{itemize}
\end{frame}

\begin{frame}{Paramètres}
  L'apprentissage des paramètres du modèle se fait à l'aide d'algorithmes de type Espérance-Maximisation.
\end{frame}

\begin{frame}{Utilisations}
  \begin{columns}
    \begin{column}{0.5\linewidth}
      \begin{itemize}
      \item Diagnostique médical
      \item Diagnostique machine
      \item Analyse de risque
      \item Prédiction supervisée
      \item Détection d'anomalie
      \item ...
      \end{itemize}
    \end{column}
    \begin{column}{0.5\linewidth}
      \V{["tikz/graph/oriented", "tw", 0.6] | image}
    \end{column}
  \end{columns}
\end{frame}
