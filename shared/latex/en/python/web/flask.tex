\begin{frame}{Introduction}
    \alert{\textbf{Flask}} is a micro web framework written in Python.\\
    
    \V{"img/logos/flask" | image("tw", 0.5, "en")}

    It has been used by \textit{Pinterest} and \textit{Linkedin}.

\end{frame}

\begin{frame}{Hello world app}

    \mintedpycode{python/web/flask-hello-world}
    \begin{itemize}
        \item \texttt{Flask(\_\_name\_\_)} creates the main application that will manages your website or API.
        \item \texttt{\@app.route("/")} defines an endpoint. The "/" indicates the root on your server.
    \end{itemize}
    

\end{frame}

\begin{frame}{Run your Flask application}
When your code of your application is done, you can launch it with the following command:

\texttt{\$ flask --app hello run}
\\
In this code, \texttt{hello} is the name of your python file or your python package.
\end{frame}

\begin{frame}{Flask project layout}
    \begin{minipage}{0.49\linewidth}
        \dirtree{%
        .1 project-name/.
        .2 package-name/.
        .3 \alert{\_\_init\_\_.py}.
        .3 blog.py.
        .3 \alert{templates/}.
        .4 index.html.
        .3 \alert{static/}.
        .4 style.css.
        .4 script.js.
        .2 tests/.
        .3 contest.py.
        .2 .venv/.
        .2 pyproject.toml.
        .2 MANIFEST.in.
        }
    \end{minipage}
    \begin{minipage}{0.49\linewidth}
        The classic layout should follow a specific structure, in particular, in the package folder.
        \begin{itemize}
            \item \alert{Python files} in the package contains endpoints and logic of your app
            \item \alert{\texttt{templates/}} is a folder containing html files which are used to generated custom pages using jinja2 templates
            \item \alert{\texttt{static/}} folder contains files that will not be modified
        \end{itemize}
    \end{minipage}

\end{frame}

\begin{frame}{Template and static files access and rendering}
    \begin{itemize}
        \item \texttt{url\_for} method gives access to any file stored in the static folder as an URL. See p.\ref{flask:jinja-template}
        \item \texttt{render\_template} method generates an html file from a jinja template. Data can be provided to customize the HTML. See p.\ref{flask:endpoint}
    \end{itemize}
    

\end{frame}

\begin{frame}{Jinja Template}\label{flask:jinja-template}

    \mintedcustomcode{html/flask/flask-base}{html}
    
\end{frame}

\begin{frame}{Jinja Template}
    A jinja template can be extended 
    \mintedcustomcode{html/flask/flask-login}{html}
\end{frame}


\begin{frame}{Endpoint}\label{flask:endpoint}

    \mintedpycode{python/web/flask-endpoint}

\end{frame}

\begin{frame}{Requests}
    Flask provided an object \texttt{Request} which captures and parse any kind of request passed to the server (\texttt{GET}, \texttt{POST}, etc.)

    \mintedpycode{python/web/flask-request}

    It provides different methods to access data in the request depending of the data type (\texttt{file}, \texttt{form}, \texttt{args}, etc.)

    For more details, see \bluelink{https://flask.palletsprojects.com/en/2.3.x/api/\#flask.Request}{\texttt{Request}} documentation 
    

\end{frame}