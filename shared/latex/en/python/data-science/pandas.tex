\begin{frame}{Introduction}
  \texttt{pandas} allows to easily manipulate data.

  Using it, it is possible to~:

  \begin{itemize}
    \item Load many data formats in an easy to use structure
    \item Filter, group, separate, rearrange, combine data
    \item Summarize, aggregate, observe
    \item Manage missing values
  \end{itemize}
\end{frame}

\begin{frame}{Data structures}
  The library is articulated around two data structures~:

  \begin{description}
    \item[\texttt{Series}] Indexed series of values
    \item[\texttt{DataFrame}] Several series of values with the same index
  \end{description}

  A \texttt{Series} has only one dimension, it represents several samples of the same variable.

  A \texttt{DataFrame} has two dimensions. It represents as many variables as it has columns. It consists of one \texttt{Series} per column.

  \begin{alertblock}{Learning tip}
    It is very important to learn to recognize when you are handling a \texttt{Series} or \texttt{DataFrame}. Pay extra attention to this!
  \end{alertblock}
\end{frame}

\begin{frame}{Index concept}
  The concept of index is \textbf{very} important in Pandas. It is the major point, with the columns heterogeneity, which distinguishes \texttt{pandas} from \texttt{numpy}.

  Index is modified with the associated values during all operations in \texttt{pandas}, it is \textbf{extremely} used in \texttt{pandas}.

  \mintedcustomcode{python/data-science/pandas/index}{pycon}
\end{frame}

\begin{frame}{Index concept ⋅ \texttt{DataFrame}}
  A \texttt{DataFrame} contains \textbf{two} indexes~:
  \begin{description}
    \item[\texttt{df.index}] The line index
    \item[\texttt{df.columns}] The column index
  \end{description}

  \mintedcustomcode{python/data-science/pandas/index-df}{pycon}

  \begin{alertblock}{Potential confusion}
    The name of the row index (\texttt{df.index}) is simitar to the index type : \texttt{Index}.
    The column index (\texttt{df.columns}) is however also an index (\texttt{Index} type).
  \end{alertblock}
\end{frame}

\begin{frame}{\texttt{pandas} inport}
  The standard way to import \texttt{pandas} is to rename the import \texttt{pd}~:
  \mintedcustomcode{python/data-science/pandas/import}{py}
\end{frame}

\begin{frame}{Import data into a \texttt{DataFrame}}
  \texttt{pandas} can deal with data from many formats with the corresponding \texttt{read\_*}~methods: \texttt{read\_csv}, \texttt{read\_excel}, \texttt{read\_json}, \texttt{read\_xml}, \texttt{read\_parquet}, ...

  \mintedcustomcode{python/data-science/pandas/read-csv}{pycon}

  Many arguments are available to control precisely the data parsing (here an example with \texttt{index\_col}).
\end{frame}

\begin{frame}{Create \texttt{DataFrame} ⋅ Default index}
  Create a \texttt{DataFrame} by specifying columns or lines :

  \mintedcustomcode{python/data-science/pandas/creation-dataframe-no-index}{pycon}
\end{frame}

\begin{frame}{\texttt{DataFrame} creation ⋅ Explicit index}
  To use an existing index or specify one different from the default numeric index, use the \texttt{index}~ argument:

  \mintedcustomcode{python/data-science/pandas/creation-dataframe-explicit-index}{pycon}
\end{frame}

\begin{frame}{\texttt{Series} creation}
  \mintedcustomcode{python/data-science/pandas/creation-series}{pycon}

  Index can be omitted if the default one is appropriate.
\end{frame}

\begin{frame}{Display}
  \mintedcustomcode{python/data-science/pandas/display}{pycon}

  You can also just use \texttt{print(df)} which combines \texttt{df.head()} \& \texttt{df.tail()}.
\end{frame}

\begin{frame}{Description ⋅ General}
  \mintedcustomcode{python/data-science/pandas/description}{pycon}
\end{frame}

\begin{frame}{Description ⋅ Numerical features}
  \mintedcustomcode{python/data-science/pandas/description-numerical}{pycon}

  If there are too many columns for the output to be readable, use \texttt{df.describe().transpose()}.
\end{frame}

\begin{frame}{Description ⋅ Value set and value count}
  It is possible to retrieve the set of all different values in a series with \texttt{unique()}.
  
  Or retrieve a count of each value with \texttt{value\_counts()}.
  \mintedcustomcode{python/data-science/pandas/value-counts}{pycon}
\end{frame}

\begin{frame}{Operations}
  Similarly to \texttt{numpy}~: operations are vectorized and very efficient compared to Python iterative operations.

  \texttt{pandas} has, in addition to operations similar to \texttt{numpy}, a rich library of functions to handle times and strings.

  \begin{alertblock}{Main difference}
    By default, operations on a \texttt{DataFrame} are applied \textbf{per column}. In \texttt{numpy}, they apply to all the elements.
  \end{alertblock}
\end{frame}

\begin{frame}{Operations ⋅ Examples}
  \mintedcustomcode{python/data-science/pandas/operations}{pycon}
\end{frame}

\begin{frame}{Indexing}
  Two modes of indexing exist~:
  \begin{description}
    \item[\texttt{loc}] Based on the index values
    \item[\texttt{iloc}] Based on the position (like \texttt{numpy})
  \end{description}
  $\Rightarrow$ Use the most readable / practical in a given situation.
\end{frame}

\begin{frame}{Indexing ⋅ Example \texttt{loc} \& \texttt{iloc}}
  \mintedcustomcode{python/data-science/pandas/loc-iloc}{pycon}
\end{frame}

\begin{frame}{Missing values}
  \mintedcustomcode{python/data-science/pandas/fillna}{pycon}
\end{frame}

\begin{frame}{Groupings}
  It is possible to work on data subsets with the function \texttt{df.groupby}~:
  \mintedcustomcode{python/data-science/pandas/groupby}{pycon}
\end{frame}

\begin{frame}{Interoperability with \texttt{numpy}}
  It is possible to pass \texttt{DataFrame}  to \texttt{numpy} functions which are applied \textbf{element by element}.

  A numpy array might be extract from a \texttt{Series} or \texttt{DataFrame} with the attribute \texttt{values}

  \mintedcustomcode{python/data-science/pandas/interoperability}{pycon}
\end{frame}

\begin{frame}{Documentation}
  \begin{itemize}
    \item \bluelink{https://pandas.pydata.org/docs/user_guide/index.html}{Official Documentation}
    \item \bluelink{https://pandas.pydata.org/Pandas_Cheat_Sheet.pdf}{Cheat Sheet}
  \end{itemize}
\end{frame}