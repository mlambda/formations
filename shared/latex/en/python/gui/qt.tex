\begin{frame}{Qt}
    \vfill
    \alert{Qt} is an cross-platform open-sourced library for GUI.
    \vfill
    It is developed in C++, but the Qt Company proposes langage bindings for Python, Javascript, C\# and Rust.
    \vfill
    \V{"img/logos/qt-black" | image("tw", 0.3, "en")}
    \vfill
\end{frame}

\begin{frame}{Structure}
    In most cases,  the objects used in the original C++ framework are the same in Python.
\end{frame}
\begin{frame}{Minimal application}
    A Qt interface requires one and only one \texttt{QApplication}. This object is responsible for the main loop.\\\\
    The minimal application will require to create a window with \texttt{QMainWindow}
\end{frame}

\begin{frame}{Minimal application : an example}
    \mintedpycode{python/gui/qt-main}
\end{frame}

\begin{frame}{QMainWindow}
    In most cases, \texttt{QMainWindow} is used by subclassing it to create an new class which will be your application's main window.
    \mintedpycode{python/gui/qt-main-bis}
\end{frame}

\begin{frame}{Widgets}
    Qt offers the usual widgets to build any user interface:
    \begin{itemize}
        \item \texttt{QLabel}
        \item \texttt{QPushButton}
        \item \texttt{QRadioButton}
        \item \texttt{QCheckBox}
        \item etc.
    \end{itemize}
\end{frame}

\begin{frame}{Widgets}
    Each widget offers numerous methods to control their appearance, behaviours or binding
    
    \mintedpycode{python/gui/qt-checkbox}
    
    Most settings for widget are specified using already defined keys like~\texttt{Qt.Checked}.
\end{frame}

\begin{frame}{Signals \& Events}
    Action on the interface are mostly captured using methods provided by the widgets themselves. For instance:
    \begin{itemize}
        \item \texttt{QButton} has an event attribut \texttt{clicked}
        \item \texttt{QLabel} has an event attribut \texttt{textChanged}
    \end{itemize} 
    
    These signals implement a \texttt{connect} method to bind them to a specific action that you implemented.
    
\end{frame}

\begin{frame}{Signals \& Events: an example}

    \mintedpycode{python/gui/qt-event}
    
\end{frame}

\begin{frame}{Layout}
    Qt proposes different layouts to control widgets' position.
    \begin{itemize}
        \item \texttt{QHBoxLayout} Each element is added horizontally from left to right
        \item \texttt{QVBoxLayout} Each element is element is added vertically from top to bottom
        \item \texttt{QGridLayout} Each element is added in a grid by specifying row and column numbers
        \item \texttt{QStackLayout} Each element is stack on the z axis. Can be used to implement an interface with tabs
    \end{itemize}
    \centering\alert{\textbf{Layouts can be nested}}
    
\end{frame}

\begin{frame}{Layout: an example}
    
    \mintedpycode{python/gui/qt-layout}   
    
\end{frame}