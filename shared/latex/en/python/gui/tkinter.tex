\begin{frame}{What is Tkinter?}
    \alert{Tkinter} is a standard library providing interface to the \alert{Tcl/Tk} GUI toolkit (see \bluelink{https://www.tcl.tk/software/tcltk/}{Tcl/Tk})

    \vfill

    Tcl/Tk has its own \bluelink{https://www.tcl.tk/about/language.html}{language},
    but python provides an interface to use it directly.
\end{frame}

\begin{frame}{Import}
    Two different main imports exists:
    \mintedpycode{python/gui/tkinter-import}

    \begin{itemize}
        \item \textbf{tkinter} module (\texttt{tk}) imbed all the class to construct any graphic interface. But with a specific appearance designed by Tcl/Tk
        \item \textbf{ttk} (\textit{themed tk}) module implements most of the classes present in the main module, but uses the general appearance of the OS
    \end{itemize}

\end{frame}

\begin{frame}{Root element}
    Any graphic interface with tkinter should start with a root object instance of \texttt{tk.Tk()}.

    \mintedpycode{python/gui/tkinter-root}

    This object will lauch the main loop with \texttt{root.mainloop()} and will be
    the ancestor (direct or not) for any future component of the interface.

\end{frame}

\begin{frame}{Frame}
    In most cases, it will be necessary to create a \texttt{Frame} element which will be your main window.

    \mintedpycode{python/gui/tkinter-frame}

    The parent of the frame is the root.
\end{frame}

\begin{frame}{Graphic Widgets}
    Usual graphic widgets are implemented:
    \begin{itemize}
        \item Button
        \item Label
        \item Radiobutton
        \item Scrollbar
        \item etc.
    \end{itemize}
    Most widget exits in a themed version in \texttt{ttk}.
    \vfill

    Each widget is created by specifing its parent and should specified how to place it with the \alert{Geometry Manager}.

\end{frame}

\begin{frame}{Geometry Manager}
    When an object/widget is created is should call a method to specify how to place it in the frame. Three different methods co-exists:
    \begin{itemize}
        \item \texttt{pack} place element one after the other (packing). Options can be set to pack in different direction
        \item \texttt{grid} place elements on a grid by specifing row and column numbers
        \item \texttt{place} place elements by specifing the pixel coordinates
    \end{itemize}
\end{frame}

\begin{frame}{Variables}
    Variables can be define to get or display values :
    \begin{itemize}
        \item \texttt{StringVar}
        \item \texttt{IntVar}
        \item \texttt{FloatVar}
        \item etc.
    \end{itemize}

    They can be used to capture a value in an \texttt{Entry} field or display in a \texttt{Label}.

\end{frame}

\begin{frame}{Binding mechanism}
    Tkinter provides a binding mechanism to user action (key pressed, mouse actions, etc.)

    \mintedpycode{python/gui/tkinter-bind}
    
    The brief summary of event codes can be find on this \bluelink{python/gui/tkinter-frame}{stackoverflow exchange}.

\end{frame}