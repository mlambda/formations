\begin{frame}{Introduction}
  La programmation réseau vise à échanger des données~:

  \begin{itemize}
    \item sur un même ordinateur (communication inter-processus)
    \item sur un réseau local
    \item sur internet
  \end{itemize}
\end{frame}

\begin{frame}{Concept clef}
  Le concept clé de la programmation réseau est la socket.

  La socket est l'interface d'envoi et de réception de données.
\end{frame}

\begin{frame}{Module de la bibliothèque standard}
  Le module qui les implémente en Python est le module \bluelink{https://docs.python.org/fr/3/library/socket.html}{\texttt{socket}}.
\end{frame}

\begin{frame}{Protocole de transfert}
  \begin{description}
    \item[UDP] Paramètre \texttt{socket.SOCK\_DGRAM} à la création. Pas de garantie de livraison des paquets, ordre des paquets non garanti
    \item[TCP] Paramètre \texttt{socket.SOCK\_STREAM} à la création. Garantie de livraison et d'ordre
  \end{description}

  TCP > UDP, sauf quand on préfère les gains potentiels de performance aux garanties de TCP.
\end{frame}

\begin{frame}{Principe de fonctionnement}
  Une socket expose principalement les méthodes suivantes~:

  \begin{description}
    \item[bind] Associe une socket avec une adresse
    \item[listen] Prépare la socket pour les connexions
    \item[accept] Initialise la connexion côté serveur
    \item[connect] Initialise la connexion côté client
    \item[send] Envoie des données
    \item[recv] Reçoit des données
    \item[close] Ferme la socket
  \end{description}

  \texttt{bind}, \texttt{listen} et \texttt{accept} sont spécifiques au serveur.

  \texttt{connect} est spécifique au client.
\end{frame}

\begin{frame}{Diagramme de séquence}
  \V{["tikz/python/sockets", "th", 0.75] | image}
\end{frame}

\begin{frame}{Exemple de serveur écho}
  \mintedpycode{python/network/echo-server}
\end{frame}

\begin{frame}{Exemple de client écho}
  \mintedpycode{python/network/echo-client}
\end{frame}

\begin{frame}{Limitations des serveurs sockets}
  \begin{itemize}
    \item Besoin de définir un protocol applicatif
    \item Gestion délicate de l'envoi et de la réception de messages pour des serveurs gérant plusieurs connexions
  \end{itemize}
\end{frame}

\begin{frame}{Pour aller plus loin}
  Un \bluelink{https://realpython.com/python-sockets/}{excellent tutoriel} est disponible sur Real Python. Il aborde~:

  \begin{itemize}
    \item Le module \texttt{select} pour mettre en place un serveur qui accepte plusieurs connexions
    \item La définition d'une classe pour mettre en place un protocol applicatif
  \end{itemize}

\end{frame}

\begin{frame}{Alternatives}
  Utiliser des surcouches, comme HTTP avec \bluelink{https://flask.palletsprojects.com/}{Flask}, ou une surcouche de HTTP comme \bluelink{https://grpc.io/docs/languages/python/}{gRPC}, un protocole de RPC avec sérialisation binaire extrêmement performant.
\end{frame}
