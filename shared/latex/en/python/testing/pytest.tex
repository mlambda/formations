\begin{frame}{Introduction}
  Goals:
  \begin{itemize}[<+->]
  \item Simplify the writing of tests compared to \texttt{unittest}
  \item Provide a modular mechanism for setup and cleanup
  \end{itemize}
  \end{frame}
  
  \begin{frame}{Usage}
  \texttt{pytest} uses simple \texttt{assert}s to create tests.
  
  \mintedpycode{python/testing/pytest/simple_usage}
  \end{frame}
  
  \begin{frame}{Fixture}
  A fixture defines a test environment:
  
  \begin{itemize}[<+->]
  \item Connection to a test database
  \item Creating a directory with files to be tested
  \item ...
  \end{itemize}
  \end{frame}
  
  \begin{frame}{Using a Fixture}
  \mintedpycode{python/testing/pytest/simple_fixture}
  \end{frame}
  
  \begin{frame}{Modularity of Fixtures}
  A fixture can use multiple other fixtures.
  
  A function can use multiple fixtures.
  \end{frame}
  
  \begin{frame}{Fixture Memoization}
  \texttt{pytest} defines several levels of fixture calculation:
  
  \begin{itemize}[<+->]
  \item Each function (by default)
  \item Each class
  \item Each module
  \item Each package
  \item For the entire test session
  \end{itemize}
  \end{frame}
  
  \begin{frame}{Fixtures Used by Multiple Modules}
  Fixtures defined in the \texttt{conftest.py} file can be used by all modules in the directory.
  \end{frame}
  
  \begin{frame}{Compatibility}
  \texttt{pytest} can run test suites defined with \texttt{unittest} or \texttt{nose}.
  \end{frame}
  
  \begin{frame}{Extensions}
  \bluelink{https://docs.pytest.org/en/latest/reference/plugin_list.html}{Many plugins available.}
  \end{frame}