\begin{frame}{Introduction}
  Buts~:
  \begin{itemize}[<+->]
    \item Simplifier l'écriture de tests par rapport à \texttt{unittest}
    \item Proposer un mécanisme modulaire de préparation et nettoyage
  \end{itemize}
\end{frame}

\begin{frame}{Utilisation}
  \texttt{pytest} utilise de simples \texttt{assert}s pour créer des tests.

  \mintedpycode{python/testing/pytest/simple_usage}
\end{frame}

\begin{frame}{Fixture}
  Une fixture définit un environnement de test~:

  \begin{itemize}[<+->]
    \item Connexion à une BDD de test
    \item Création d'un répertoire avec des fichiers à tester
    \item …
  \end{itemize}
\end{frame}

\begin{frame}{Utilisation d'une fixture}
  \mintedpycode{python/testing/pytest/simple_fixture}
\end{frame}

\begin{frame}{Modularité des fixtures}
  Une fixture peut utiliser plusieurs autres fixtures.

  Une fonction peut utiliser plusieurs fixtures.
\end{frame}

\begin{frame}{Mémoisation des fixtures}
  \texttt{pytest} définit plusieurs niveaux de calcul des fixtures~:

  \begin{itemize}[<+->]
    \item Chaque fonction (par défaut)
    \item Chaque classe
    \item Chaque module
    \item Chaque paquet
    \item Pour la session de test entière
  \end{itemize}
\end{frame}

\begin{frame}{Fixtures utilisées par plusieurs modules}
  Les fixtures présentes dans le fichier \texttt{conftest.py} sont utilisables par tous les modules du répertoire.
\end{frame}

\begin{frame}{Compatibilité}
  \texttt{pytest} peut faire tourner des suites de tests définies avec \texttt{unittest} ou \texttt{nose}.
\end{frame}

\begin{frame}{Extensions}
  \bluelink{https://docs.pytest.org/en/latest/reference/plugin_list.html}{Beaucoup de plugins disponibles.}
\end{frame}
