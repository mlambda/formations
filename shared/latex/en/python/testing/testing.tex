\begin{frame}{Introduction}
  Goals:
  \begin{itemize}[<+->]
  \item Ensure software quality
  \item Measure regressions with each update
  \item Refactor with confidence
  \item Document code with usage examples for each part of the codebase
  \item Document assumptions made about the code
  \end{itemize}
  \end{frame}
  
  \begin{frame}{Test-Driven Development (TDD)}
  Interesting development approach based on a cycle:
  
  \begin{enumerate}[<+->]
  \item Add a unit test that aims to ensure functional requirements are met
  \item Verify that the test fails
  \item Create the minimal code that solves the test
  \item Verify that all tests pass
  \end{enumerate}
  \end{frame}
  
  \begin{frame}{Different Types of Tests}
  Functional Tests:
  
  \begin{description}[<+->]
  \item[Unit] Verify a small unit of code (function or class)
  \item[Integration] Test the combination of medium-sized code units
  \item[Regression] Ensure that the level of quality does not decrease
  \item[Smoke] Verify critical (and basic) functionality
  \end{description}
  
  \onslide<+->{Non-functional Tests:}
  
  \begin{description}[<+->]
  \item[Stress] Measure the system's resistance under high load
  \item[Recovery] Measure the system's ability to recover
  \item[Security] Check for vulnerabilities
  \item[…]
  \end{description}
  \end{frame}
  
  \begin{frame}{Unit Tests}
  The core of testing: they verify that the basic building blocks work.
  
  \begin{itemize}[<+->]
  \item Usually multiple tests per \alert{public API function}
  \item Each test ensures that preconditions, postconditions, and invariants are met
  \end{itemize}
  \end{frame}
  
  \begin{frame}{Integration Tests}
  \begin{itemize}[<+->]
  \item Verify that multiple subsystems of the system work correctly together
  \item Very important for making changes to the system with confidence
  \end{itemize}
  \end{frame}
  
  \begin{frame}{End-to-End Tests}
  \begin{itemize}[<+->]
  \item Tests the entire system chain
  \item Replicates real-world use cases as closely as possible
  \end{itemize}
  \end{frame}
  
  \begin{frame}{Regression Tests}
  \begin{itemize}[<+->]
  \item Combination of unit and integration tests
  \item Also include tests for known bugs (1 bug fixed = 1 test)
  \item Verify that software quality is maintained after a modification
  \end{itemize}
  \end{frame}
  
  \begin{frame}{Some Testing Techniques}
  \begin{description}[<+->]
  \item[Mocking] Replace modules or functions with mocks
  \item[Generation] Provide many random inputs to a test to detect bugs
  \item[Mutation] Mutate the program, and the test suite should fail after the mutation
  \item[Contracts] Often based on generation, they specify invariants and pre/post conditions
  \end{description}
  \end{frame}
  
  \begin{frame}{Testing Techniques --- Mocking}
  \begin{itemize}[<+->]
  \item Replace a class, a method, or a module attribute with a fake object
  \item Allows testing of systems that are not accessible during tests
  \item Implemented by \texttt{unittest.mock} or \texttt{pytest.monkeypatching}
  \end{itemize}
  
  \onslide<+->{\mintedpycode{python/testing/testing/mocking}}
  \end{frame}
  
  \begin{frame}{Testing Techniques --- Generation}
  \begin{itemize}[<+->]
  \item Often coupled with property-based testing
  \item Focuses on discovering pre/post conditions and invariants
  \item Test case generation is automatic
  \item Reduces error cases to simple cases
  \end{itemize}
  \end{frame}
  
  \begin{frame}{Testing Techniques --- Mutation}
  \begin{itemize}[<+->]
  \item Assumption that a good test suite fails on different code
  \item Performs subtle mutations
  \item Counts failed and unfinished tests
  \end{itemize}
  \end{frame}
  
  \begin{frame}{Testing Techniques --- Contracts}
  \begin{itemize}[<+->]
  \item Extension of property-based testing
  \item Explicitly specifies expectations on functions and classes
  \end{itemize}
  
  \mintedpycode{python/testing/testing/contracts}
  \end{frame}
  
  \begin{frame}{Best Practices}
  \begin{itemize}[<+->]
  \item Follow the four-step structure: setup, execution, validation, cleanup
  \item Create independent tests
  \item Test only the public API, never implementation details
  \item Keep tests fast
  \item Use mutation testing or fuzzing
  \item Practice inversion of control
  \end{itemize}
  \end{frame}
  
  \begin{frame}{Inversion of Control --- Coupling}
  \mintedpycode{python/testing/testing/coupling}
  \end{frame}
  
  \begin{frame}{Inversion of Control --- Cohesion}
  Best practice: each part is testable and easily replaceable.
  
  The instantiation logic varies depending on the context of use.
  
  Allows dependency injection.
  
  \mintedpycode{python/testing/testing/cohesion}
  \end{frame}
  
  \begin{frame}{Setting Up in Python}
  \begin{description}[<+->]
  \item[\texttt{unittest}] Testing framework from the standard library
  \item[\texttt{pytest}] Most popular testing framework
  \item[\texttt{hypothesis}] Fuzzing, generation strategies
  \item[\texttt{mutmut}] Mutation testing
  \item[\texttt{deal}] Contracts
  \end{description}
  
  \onslide<+->{The first three libraries have many extensions.}
  \end{frame}