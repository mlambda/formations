\begin{frame}{Introduction}
  XML (eXtensible Markup Language) est un format à mi-chemin entre un format dédié aux machines et un format lisible par les humains.

  Il est très utilisé pour la configuration et l'échange de petits volumes de données.
\end{frame}

\begin{frame}{Exemple}
  \mintedcustomcode{python/xml/example}{xml}
\end{frame}

\begin{frame}{Utilisation en Python}
  La bibliothèque standard contient un ensemble de paquets dédiés~:

  \begin{description}
    \item[\texttt{xml.etree.ElementTree}] API basique qui construit un arbre représentant le document XML
    \item[\texttt{xml.dom}] Implémentation du DOM (pages HTML)
    \item[\texttt{xml.sax}] Paquet SAX (Simple API for XML), gestion des fichiers par traitement d'événements
    \item[\texttt{xml.parsers.expat}] Binding vers la libexpat, un parseur extrêmement rapide
  \end{description}
\end{frame}

\begin{frame}{Alternative plus simple et très performante}
  La bibliothèque externe \bluelink{https://github.com/martinblech/xmltodict}{\texttt{xmltodict}} permet de~:

  \begin{itemize}
    \item récupérer les objets d'un document au fur et à mesure (pas de chargement du document entier en mémoire)
    \item récupérer les objets sous forme de dictionnaire, souvent plus simples à manipuler que l'API ElementTree
    \item produire des fichiers XML à partir de dictionnaires directement
  \end{itemize}

  Cette bibliothèque utilise le parseur expat de la bibliothèque standard
\end{frame}

\begin{frame}{Alternative plus simple et très performante}
  \mintedcustomcode{python/xml/xmltodict}{pycon}
\end{frame}
