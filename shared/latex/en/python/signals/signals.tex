\begin{frame}{Introduction}
  Les signaux sont un mécanisme de communication entre processus hérité du C.

  En particulier, les systèmes d'exploitation communiquent avec les programmes par ce mécanisme.
\end{frame}

\begin{frame}{Module \texttt{signal} de la bibliothèque standard}
  En Python, le module \texttt{signal} est utilisé pour intercepter et envoyer des signaux. Il définit~:

  \begin{itemize}[<+(1)->]
    \item Une variable pour chaque signal (\texttt{SIGINT}, \texttt{SIGTERM}, …)
    \item Une fonction \texttt{signal} pour exécuter une fonction à chaque réception de signal
    \item Plusieurs fonctions connexes moins utilisées à étudier dans la \bluelink{https://docs.python.org/fr/3/library/signal.html}{documentation officielle}
  \end{itemize}
\end{frame}

\begin{frame}{Signification de chaque signal}
  Description des \bluelink{https://fr.wikipedia.org/wiki/Signal_(informatique)}{signaux POSIX} sur Wikipédia et des \bluelink{https://docs.python.org/fr/3/library/signal.html\#module-contents}{signaux du module \texttt{signal}} dans la documentation Python.
\end{frame}

\begin{frame}{Définition d'un gestionnaire de signal}
  \mintedcustom{python/signals/signal}{py}{console}
\end{frame}

\begin{frame}{Points à noter}
  \begin{itemize}[<+->]
    \item Les signaux diffèrent entre les systèmes d'exploitation
    \item Les signaux arrivent dans le thread principal
    \item On ne peut pas ignorer ou redéfinir \texttt{SIGKILL}
    \item Possibilités de situations de compétition entre signaux
  \end{itemize}
\end{frame}

\begin{frame}{Cadre d'utilisation}
  \begin{exampleblock}{Conseillé}
    Traiter les signaux du système d'exploitation ou envoyés depuis des scripts d'admin ou des programmes C.

    Suivre la norme POSIX pour réagir aux signaux.
  \end{exampleblock}

  \begin{alertblock}{Déconseillé}
    Pour faire de la communication à l'intérieur d'une application ou entre applications. Préférez~:

    \begin{itemize}
      \item Exposer une interface RPC (\emph{Remote Procedure Call})
      \item Exposer un serveur HTTP
      \item Utiliser une queue d'événements
      \item …
    \end{itemize}
  \end{alertblock}
\end{frame}
