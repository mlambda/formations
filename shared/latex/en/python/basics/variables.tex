\begin{frame}{Introduction to variables}
  A variable is a \emph{label} assigned to a Python object.

  We can use this label instead of the object it refers to at any time in our programs.
\end{frame}

\begin{frame}{Variable creation}
  To create a variable, we use the following syntax:
  \mintedpycode{python/basics/variables/creation}  
\end{frame}

\begin{frame}{Variable usage}
  To use a variable, we just refer to it by its name:
  \pycon{python/basics/variables/usage}
\end{frame}

\begin{frame}{Variable naming}
  There are two rules to respect when it comes to variables names:

  \begin{itemize}
    \item The names should start by a lowercase letter, uppercase letter or an underscore
    \item The names should continue by those same characters and you can also mix in digits
  \end{itemize}

  \begin{alertblock}{Unicode}
    From Python 3 on, it's possible to use any unicode letter in variable names. \emph{It's strongly discouraged. Do yourself a favor and stick to ASCII!}
  \end{alertblock}
\end{frame}

\begin{frame}{Basic types}
  To represent simple values, we mainly use:

  \begin{description}
    \item[\texttt{int}] Integers
    \item[\texttt{float}] Reals
    \item[\texttt{str}] Strings
    \item[\texttt{bool}] Booleans
  \end{description}
\end{frame}

\begin{frame}{The \texttt{type} function}
  Useful to know the type of an object:

  \pycon{en/python/basics/variables/type}
\end{frame}

\begin{frame}{Type conversion}
  We can use the classes that represent basic values to convert \emph{to} them:

  \pycon{python/basics/variables/type-conversion}
\end{frame}
