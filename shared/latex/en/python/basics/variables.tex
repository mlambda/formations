\begin{frame}{Introduction aux variables}
  Une variable est un \emph{label} assigné à un objet Python.

  On peut utiliser ce label à n'importe quel moment en lieu et place de l'objet lui-même.
\end{frame}

\begin{frame}{Création d'une variable}
  Pour créer une variable, on utilise la forme suivante~:
  \mintedpycode{python/basics/variables/creation}  
\end{frame}

\begin{frame}{Utilisation d'une variable}
  Pour utiliser une variable, on utilise son nom~:
  \pycon{python/basics/variables/usage}
\end{frame}

\begin{frame}{Nommage}
  On peut nommer ses variables en respectant ces deux règles~:

  \begin{itemize}
    \item Le nom commence par une lettre minuscule ou majuscule ou un tiret bas
    \item Le nom continue par ces mêmes caractères ou des chiffres
  \end{itemize}

  \begin{alertblock}{Unicode}
    Depuis Python 3, on peut utiliser de l'unicode. \emph{C'est toutefois très déconseillé}
  \end{alertblock}
\end{frame}

\begin{frame}{Types de base}
  Pour représenter des valeurs simples, on utilise principalement~:

  \begin{description}
    \item[\texttt{int}] Nombres entiers
    \item[\texttt{float}] Nombres réels
    \item[\texttt{str}] Chaînes de caractères
    \item[\texttt{bool}] Booléens
  \end{description}
\end{frame}

\begin{frame}{La fonction \texttt{type}}
  Permet de connaître le type d'un objet~:

  \pycon{python/basics/variables/type}
\end{frame}

\begin{frame}{Conversion de types}
  On peut utiliser les classes représentants les types pour convertir vers ceux-ci~:

  \pycon{python/basics/variables/type-conversion}
\end{frame}
