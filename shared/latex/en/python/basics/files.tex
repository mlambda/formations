\begin{frame}{Main tools}
  \begin{itemize}
    \item \texttt{open} function to retrieve an object to manipulate a file
    \item Usage of a \texttt{with} statement to ensure that file closing is handled properly
  \end{itemize}
\end{frame}

\begin{frame}{\texttt{with} statement}
  Introduces a context:
  \begin{itemize}
    \item Entering and leaving the context will execute some code, here the opening and closing of the file
    \item Errors can be handled
  \end{itemize}

\end{frame}

\begin{frame}{Most important \texttt{open} parameters}
  \begin{description}
    \item[\texttt{mode}] By default, \texttt{"r"} for reading. See \bluelink{https://docs.python.org/3/library/functions.html\#open}{here} for all the options.
    \item[\texttt{encoding}] By default, locale dependent. Nowaydas, \texttt{encoding="utf8"} is almost always what you want.
  \end{description}
\end{frame}

\begin{frame}{Reading}
  \pycon{en/python/basics/files/read}
\end{frame}

\begin{frame}{Writing}
  \mintedpycode{en/python/basics/files/write}
\end{frame}

\begin{frame}{Manipulating two files at once}
  \mintedpycode{python/basics/files/two-files}
\end{frame}

\begin{frame}{Newlines}
  By default, Python 3 understands newlines from all platforms.

  When writing, it will use mostly \texttt{\textbackslash n} as a default option.
\end{frame}
