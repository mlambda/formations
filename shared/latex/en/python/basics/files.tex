\begin{frame}{Mécanismes principaux}
  \begin{itemize}
    \item Fonction \texttt{open} pour récupérer un objet pour manipuler le fichier
    \item Utilisation d'un bloc \texttt{with} pour garantir la fermeture du fichier
  \end{itemize}
\end{frame}

\begin{frame}{Bloc \texttt{with}}
  Définit un contexte~:
  \begin{itemize}
    \item L'entrée et la sortie du contexte sont gérés par le gestionnaire de contexte
    \item Les erreurs peuvent aussi l'être
  \end{itemize}

  Très utile pour la gestion de fichier~: \texttt{open} est un gestionnaire de contexte.
\end{frame}

\begin{frame}{Paramètres les plus importants d'\texttt{open}}
  \begin{description}
    \item[\texttt{mode}] Par défaut, \texttt{"r"} pour lecture. Voir \bluelink{https://docs.python.org/fr/3/library/functions.html\#open}{ici} pour la liste complète.
    \item[\texttt{encoding}] Par défaut, dépend de la locale. Quasi toujours mettre \texttt{encoding="utf8"}.
  \end{description}
\end{frame}

\begin{frame}{Lecture}
  \pycon{python/basics/files/read}
\end{frame}

\begin{frame}{Écriture}
  \mintedpycode{python/basics/files/write}
\end{frame}

\begin{frame}{Manipulation de deux fichiers à la fois}
  \mintedpycode{python/basics/files/two-files}
\end{frame}

\begin{frame}{Sauts de ligne}
  Par défaut, Python 3 comprend en lecture les sauts de ligne de Windows, Mac OS et GNU/Linux sans travail additionnel.

  En écriture il utilise \texttt{\textbackslash n} (sauts de ligne GNU/Linux).
\end{frame}
