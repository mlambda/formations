\begin{frame}{Fonction d'affichage}
  \texttt{print} est la principale fonction pour afficher du texte en Python~:

  \pycon{python/basics/display/basic}
\end{frame}

\begin{frame}{Arguments de \texttt{print}}
  \begin{description}
    \item[\texttt{sep}] Affiché entre chaque chaîne argument. Défaut~: un espace
    \item[\texttt{end}] Affiché en fin d'affichage. Défaut~: un retour chariot
    \item[\texttt{file}] Fichier de sortie. Défaut~: \texttt{sys.stdout}
    \item[\texttt{flush}] Faut-il flush le buffer ? Défaut~: \texttt{False}
  \end{description}

  \pycon{python/basics/display/args}
\end{frame}

\begin{frame}{Chaînes formatées}
  Il est possible de construire des chaînes complexes avec la méthode \texttt{format}~:

  \pycon{python/basics/display/format}

  Le site \bluelink{https://pyformat.info/}{PyFormat} donne des précisions sur toutes les options disponibles.  
\end{frame}

\begin{frame}{\texttt{f-strings}}
  On peut aussi utiliser un raccourci syntaxique pour appeler \texttt{format}~:

  \pycon{python/basics/display/f-strings}
\end{frame}
