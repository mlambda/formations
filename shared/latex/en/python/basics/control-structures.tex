\begin{frame}{Introduction}
  Python has 3 main control structures: \texttt{if}, \texttt{while} and \texttt{for}.

  We'll talk about \texttt{try}/\texttt{except} later to handle errors.

  From Python 3.10 on, there's also a \texttt{match} structure but it's still a bit early to use it.
\end{frame}

\begin{frame}{\texttt{if}}
  \pycon{en/python/basics/control-structures/if}
\end{frame}

\begin{frame}{\texttt{for} ⋅ With indices}
  The \texttt{range} function can be used to generate indices for iteration:

  \mintedpycode{python/basics/control-structures/range}

  Usage in a loop:

  \pycon{python/basics/control-structures/for-indices}
\end{frame}

\begin{frame}{\texttt{for} ⋅ With a list}
  \pycon{python/basics/control-structures/for-list}
\end{frame}

\begin{frame}{\texttt{for} ⋅ With a dict}
  \pycon{python/basics/control-structures/for-dict}
\end{frame}

\begin{frame}{\texttt{for} ⋅ With several iterables}
  \pycon{python/basics/control-structures/for-zip}

  Works with more than 2 iterables.
\end{frame}

\begin{frame}{\texttt{while}}
  \pycon{en/python/basics/control-structures/while}
\end{frame}

\begin{frame}{\texttt{break} \& \texttt{continue}}
  \begin{description}
    \item[\texttt{break}] Used to skip to after the most inner loop
    \item[\texttt{continue}] Used to skip to the next iteration of the most inner loop
  \end{description}

  \pycon{python/basics/control-structures/break-continue}
\end{frame}

\begin{frame}{\texttt{else} branch for loops}
  The \texttt{else} branch is executed if no \texttt{break} was used in the loop body:

  \pycon{python/basics/control-structures/else}
\end{frame}
