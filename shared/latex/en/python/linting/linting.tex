\begin{frame}{Introduction}
  Goals:
  \begin{itemize}
    \item Ensuring standards within a team
    \item Minimizing time spent on code review
    \item Producing more maintainable software
  \end{itemize}
\end{frame}

\begin{frame}{Possibilities}
  Major points usually checked include:
  
  \begin{itemize}
    \item Types
    \item Unused code elements
    \item Missing or incomplete documentation
    \item Common bug patterns
    \item Code style
  \end{itemize}
\end{frame}

\begin{frame}{Tools}
  The PyCQA organization maintains several linting tools, including:
  
  \begin{itemize}
    \item \texttt{flake8}
    \item \texttt{pylint}
    \item \texttt{RedBaron}
  \end{itemize}
  
  \texttt{mypy} is commonly used for type checking, and \texttt{black} for code formatting.
\end{frame}

\begin{frame}{Type Checking}
  Type checking is arguably the most important aspect to verify:
  
  \begin{itemize}
    \item Allows aggressive code modifications
    \item Resolves a significant number of bugs before integration
    \item Documents the code
    \item Enables automatic code generation (e.g., CLI with Typer)
  \end{itemize}
  
  The transition to a typed codebase can be done gradually.
\end{frame}

\begin{frame}{Type Notations}
  An expression is typed by preceding the type annotation with a \texttt{:}.
  
  For function returns, \texttt{->} is used.
  
  \mintedpycode{python/linting/mypy}
  
  Additional types are available in the \texttt{typing} module.
\end{frame}

\begin{frame}{General Checks}
  The essential tool is \texttt{flake8}. Depending on the installed extensions, it checks for:
  
  \begin{itemize}
    \item Syntax
    \item Unused elements
    \item Imports
    \item Code complexity
    \item ...
  \end{itemize}
\end{frame}

\begin{frame}{\texttt{flake8} Extensions}
  Some useful extensions include:
  
  \begin{itemize}
    \item \texttt{flake8-bugbear}
    \item \texttt{flake8-docstrings}
    \item \texttt{flake8-import-order}
    \item \texttt{flake8-black}
    \item \texttt{flakehell}
  \end{itemize}
  
  You can find a more comprehensive list in this \bluelink{https://github.com/DmytroLitvinov/awesome-flake8-extensions}{\texttt{git} repository}.
\end{frame}

\begin{frame}{\texttt{flake8} Configuration}
  \texttt{flake8} can be configured in the \texttt{.flake8} file at the root of your project.
  
  \mintedcustomcode{python/linting/flake8}{ini}
\end{frame}

\begin{frame}{Style Checking}
  For code style, I highly recommend using \texttt{black}:
  
  \begin{itemize}
    \item Automatic formatting (source code transformed into syntax tree and back $\Rightarrow$ original formatting unused)
    \item Ability to check codebase compliance
    \item Consistent style
    \item No more style issues in code reviews!
 

 \end{itemize}
\end{frame}

\begin{frame}{Documentation Checking}
  Use \texttt{pylint} with these options:
  
  \begin{itemize}
    \item \texttt{missing-param-doc}
    \item \texttt{differing-param-doc}
    \item \texttt{differing-type-doc}
    \item \texttt{missing-return-doc}
  \end{itemize}
  
  Also, consider using \texttt{flake8-docstrings} for docstring style.
\end{frame}

\begin{frame}{Recommendations}
  Recommended approach: group the checks into a single command (e.g., using a Makefile).
  
  \begin{itemize}
    \item Run these checks before each pull request (Git hooks can be used)
    \item Integrate them into the CI pipeline
    \item \alert{A pull request should not be reviewed until these issues are addressed}
  \end{itemize}
\end{frame}

\begin{frame}{Configuration}
  \begin{itemize}
    \item Define the points to check as a team
    \item Regularly analyze the cost and benefits of each check
    \item Add or remove checks based on these analyses
  \end{itemize}
\end{frame}

\begin{frame}{Reproducibility}
  Manage the linting tools as dependencies:
  
  \begin{itemize}
    \item Each team member should have consistent results given the same codebase
    \item The tools should be easy to install, including on the CI system
  \end{itemize}
  
  However, be careful not to install these dependencies in production.
\end{frame}