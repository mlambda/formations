\begin{frame}{Introduction}
  Les interfaces sont un des concepts les plus importants en ingénierie logicielle objet.
\end{frame}

\begin{frame}{Définition}
  Une interface est un ensemble de signatures de méthodes publiques.

  Le point crucial est que l'implémentation \textbf{ne} fait \textbf{pas} partie de l'interface.
\end{frame}

\begin{frame}{Principe fondamental}
  En programmation orientée objet, on doit toujours \emph{coder pour l'interface}.

  Cela respecte le principe d'encapsulation.
\end{frame}

\begin{frame}{Types d'interfaces en Python}
  \begin{itemize}
    \item Interfaces informelles
    \item Interfaces formelles
  \end{itemize}
\end{frame}

\begin{frame}{Interfaces informelles}
  Définies par une simple classe, avec des méthodes qui retournent NotImplementedError.

  Les classes filles en héritent.

  \mintedpycode{python/advanced/interfaces/simple-class}
\end{frame}

\begin{frame}{Interfaces formelles}
  Utilisent le module \bluelink{https://docs.python.org/fr/3/library/abc.html}{\texttt{abc}}.

  \mintedpycode{python/advanced/interfaces/abc}
\end{frame}

\begin{frame}{Enregistrement d'implémentations}
  \mintedpycode{python/advanced/interfaces/register}
\end{frame}

\begin{frame}{Choix entre les deux types d'interface}
  \begin{itemize}
    \item On utilise plus souvent l'interface informelle : moins lourde, moins de surprises
    \item Dès que l'on a besoin de l'enregistrement d'implémentations, on utilise le module \texttt{abc}
  \end{itemize}
\end{frame}