\begin{frame}{Introduction}
  Buts~:
  \begin{itemize}[<+->]
    \item Exécuter du code pendant la définition d'une fonction
    \item Disposer d'une grande flexibilité d'extension des fonctions
  \end{itemize}
\end{frame}

\begin{frame}{Exemple}
  \mintedpycode{python/advanced/decorators/example}

  Le décorateur \texttt{command} est exécuté pendant la définition de \texttt{my\_command}.
\end{frame}

\begin{frame}{Concept}
  \begin{itemize}[<+->]
    \item Décorateur = fonction
    \item Prend en entrée une fonction
    \item Rend usuellement en sortie une fonction
  \end{itemize}
\end{frame}

\begin{frame}{Exemple reformulé}
  \begin{center}
    \mintedpycode{python/advanced/decorators/example}
    {\Huge =}
    \mintedpycode{python/advanced/decorators/example-reformulated}
  \end{center}
\end{frame}

\begin{frame}{Possibilités}
  \begin{itemize}[<+->]
    \item Modification des arguments ou du résultat d'une fonction
    \item Effets de bord (logging, mesure des performances, …)
  \end{itemize}
\end{frame}

\begin{frame}{Définition d'un décorateur}
  \mintedpycode{python/advanced/decorators/definition}
\end{frame}

\begin{frame}{Définition d'un décorateur avec arguments}
  \mintedpycode{python/advanced/decorators/definition-args}
\end{frame}

\begin{frame}{Utilisation de plusieurs décorateurs}
  \begin{center}
    \mintedpycode{python/advanced/decorators/multiple}
    {\Huge =}
    \mintedpycode{python/advanced/decorators/multiple-equivalent}
  \end{center}
\end{frame}

\begin{frame}{Quelques décorateurs utiles}
  \begin{description}[<+->]
    \item[\texttt{classmethod}] Déclare une méthode comme étant une méthode de classe
    \item[\texttt{staticmethod}] Déclare une méthode comme étant statique
    \item[\texttt{property}] Définit une ou des méthode(s) qui se comporte(nt) comme un attribut
    \item[\texttt{functools.cache}] Permet de stocker le résultat d'un appel de fonction pour ne pas le recalculer
  \end{description}
\end{frame}

\begin{frame}{Un outil indispensable}
  Pour créer des décorateurs, l'utilisation de \texttt{functools.wraps} est \alert{très} recommandée.
\end{frame}

\begin{frame}{Décorateurs de classes}
  Exactement de la même manière que pour les fonctions, il est possible de décorer des classes.
\end{frame}
