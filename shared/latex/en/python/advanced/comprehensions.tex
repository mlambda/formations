\begin{frame}{Introduction}
  \begin{itemize}[<+->]
    \item Simple mechanism to create a new collection from an existing one
    \item Readable and flexible
  \end{itemize}
\end{frame}

\begin{frame}{Syntax}
  The syntax is \texttt{<result for element in original\_collection if condition>}

  It uses the common symbols of different data structures:

  \begin{center}
    \begin{tabular}{ccc}
      \textbf{Delimiters} & \textbf{Example} & \textbf{Result Type} \\
      \toprule
      \texttt{[]} & \texttt{[n * 2 for n in l]} & List \\
      \texttt{\{\}} & \texttt{\{n * 2 for n in l\}} & Set \\
      \texttt{\{\}} & \texttt{\{n: n * 2 for n in l\}} & Dictionary \\
      \texttt{()} & \texttt{(n * 2 for n in l)} & Generator \\
    \end{tabular}
  \end{center}
\end{frame}

\begin{frame}{Equivalence}
  A comprehension corresponds to a \texttt{for} loop:

  \mintedpycode{python/advanced/comprehensions/correspondence}
\end{frame}

\begin{frame}{Nested Comprehensions}
  \mintedpycode{python/advanced/comprehensions/nested}
\end{frame}
