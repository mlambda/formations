\begin{frame}{Introduction}
  \begin{itemize}[<+->]
    \item Mécanisme simple pour créer une nouvelle collection à partir d'une collection existante
    \item Lisible et flexible
  \end{itemize}
\end{frame}

\begin{frame}{Syntaxe}
  La syntaxe est \texttt{<resultat for element in collection\_originale if condition>}

  Reprend les symboles usuels des différentes structures de données~:

  \begin{center}
    \begin{tabular}{ccc}
      \textbf{Délimiteurs} & \textbf{Exemple} & \textbf{Type résultat} \\
      \toprule
      \texttt{[]} & \texttt{[n * 2 for n in l]} & Liste \\
      \texttt{\{\}} & \texttt{\{n * 2 for n in l\}} & Ensemble \\
      \texttt{\{\}} & \texttt{\{n: n * 2 for n in l\}} & Dictionnaire \\
      \texttt{()} & \texttt{(n * 2 for n in l)} & Générateur \\
    \end{tabular}
  \end{center}
\end{frame}

\begin{frame}{Équivalence}
  Une compréhension correspond à une boucle \texttt{for}~:

  \mintedpycode{python/advanced/comprehensions/correspondence}
\end{frame}

\begin{frame}{Compréhensions imbriquées}
  \mintedpycode{python/advanced/comprehensions/nested}
\end{frame}
