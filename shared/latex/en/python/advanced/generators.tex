\begin{frame}{Introduction}
  Buts~:

  \begin{itemize}[<+->]
    \item Transformer des données sans stocker le résultat en mémoire
    \item Donc traiter de grands volumes de données facilement
    \item Remplacer avantageusement les listes pour certaines utilisations
    \item Plus simples à créer qu'un itérateur classique (supertype de générateur)
    \item Offre des possibilités avancées de communication
  \end{itemize}
\end{frame}

\begin{frame}{Intuition}
  Les générateurs \alert{stockent un algorithme} pour fabriquer des données.

  Cet algorithme n'est utilisé \alert{que lorsque la valeur est demandée}, pas avant (\emph{lazy evaluation}).
\end{frame}

\begin{frame}{Hiérarchie de l'itération en Python}

  Du plus général au plus spécifique~:

  \begin{description}[<+->]
    \item[Iterable] Méthode \texttt{\_\_iter\_\_}
    \item[Iterator] Méthodes \texttt{\_\_iter\_\_} et \texttt{\_\_next\_\_}
    \item[Generator] Méthodes \texttt{send}, \texttt{throw}, \texttt{close}, \texttt{\_\_iter\_\_} et \texttt{\_\_next\_\_}
  \end{description}  

  \onslide<+->{Ces méthodes doivent implémenter les contrats suivants~:}

  \begin{description}[<+->]
    \item[\texttt{\_\_iter\_\_}] Renvoie un itérateur
    \item[\texttt{\_\_next\_\_}] Demande l'objet suivant à un itérateur
  \end{description}

  \onslide<+->{Nous verrons plus loin ce que font \texttt{send}, \texttt{throw} \& \texttt{close}.}
\end{frame}

\begin{frame}{Création d'un générateur}
  Deux options~:

  \begin{itemize}[<+->]
    \item Par compréhension
    \item Avec une fonction et les mots clefs \texttt{yield} et \texttt{yield from}
  \end{itemize}
\end{frame}

\begin{frame}{Exemple --- Par compréhension}
  \mintedpycode{python/advanced/generators/comprehension}
\end{frame}

\begin{frame}{Exemple --- Par une fonction utilisant yield}
  \mintedpycode{python/advanced/generators/function}
\end{frame}

\begin{frame}{Fonctions qui acceptent des générateurs}
  Beaucoup de fonctions acceptent des générateurs. Quand c'est le seul argument, pas besoin de parenthèses~:

  \mintedpycode{python/advanced/generators/builtins}
\end{frame}

\begin{frame}{Utilisation d'un générateur}

  Comme pour un itérateur~:

  \begin{itemize}[<+->]
    \item Itération à l'aide d'une boucle \texttt{for} \mintedpycode{python/advanced/generators/for}
    \item Manuellement avec la méthode \texttt{next} \mintedpycode{python/advanced/generators/manual}
    \item \alert{Utilisation unique}, une fois épuisé, il faut recréer l'itérateur pour ré-itérer
  \end{itemize}
\end{frame}

\begin{frame}{Chaînage de générateurs}
  \mintedpycode{python/advanced/generators/chain}

  Ou avec une compréhension~:

  \mintedpycode{python/advanced/generators/chain-short}
\end{frame}

\begin{frame}{Fonctions avancées}

  \texttt{yield} retourne les valeurs qu'on lui envoie avec \texttt{send}~:

  \mintedpycode{python/advanced/generators/send}

  \texttt{next(generator)} est équivalent à \texttt{generator.send(None)}

  De la même manière, les méthodes \texttt{throw} et \texttt{close} envoient des exceptions au point de pause.
\end{frame}

\begin{frame}{Mot clef \texttt{yield from}}
  Le mot clef \texttt{yield from} permet de \texttt{yield} depuis un autre générateur~:

  \mintedpycode{python/advanced/generators/yieldfrom}
\end{frame}