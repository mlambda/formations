\begin{frame}{Introduction}
  Goals:
  
  \begin{itemize}[<+->]
  \item Transform data without storing the result in memory
  \item Easily process large volumes of data
  \item Provide advantageous replacements for lists in certain use cases
  \item Simpler to create than a traditional iterator (generator's supertype)
  \item Offer advanced communication possibilities
  \end{itemize}
  \end{frame}
  
  \begin{frame}{Intuition}
  Generators \alert{store an algorithm} for producing data.
  
  This algorithm is \alert{only used when the value is requested}, not before (lazy evaluation).
  \end{frame}
  
  \begin{frame}{Python Iteration Hierarchy}
  
  From the most general to the most specific:
  
  \begin{description}[<+->]
  \item[Iterable] Method \texttt{\_\_iter\_\_}
  \item[Iterator] Methods \texttt{\_\_iter\_\_} and \texttt{\_\_next\_\_}
  \item[Generator] Methods \texttt{send}, \texttt{throw}, \texttt{close}, \texttt{\_\_iter\_\_}, and \texttt{\_\_next\_\_}
  \end{description}
  
  \onslide<+->{These methods must implement the following contracts:}
  
  \begin{description}[<+->]
  \item[\texttt{\_\_iter\_\_}] Returns an iterator
  \item[\texttt{\_\_next\_\_}] Requests the next object from an iterator
  \end{description}
  
  \onslide<+->{We will see later what \texttt{send}, \texttt{throw}, and \texttt{close} do.}
  \end{frame}
  
  \begin{frame}{Creating a Generator}
  Two options:
  
  \begin{itemize}[<+->]
  \item Using comprehension
  \item With a function and the \texttt{yield} and \texttt{yield from} keywords
  \end{itemize}
  \end{frame}
  
  \begin{frame}{Example --- Using Comprehension}
  \mintedpycode{python/advanced/generators/comprehension}
  \end{frame}
  
  \begin{frame}{Example --- Using a Function with yield}
  \mintedpycode{python/advanced/generators/function}
  \end{frame}
  
  \begin{frame}{Functions that Accept Generators}
  Many functions accept generators. When it's the only argument, no need for parentheses:
  
  \mintedpycode{python/advanced/generators/builtins}
  \end{frame}
  
  \begin{frame}{Using a Generator}
  
  Similar to an iterator:
  
  \begin{itemize}[<+->]
  \item Iteration using a \texttt{for} loop \mintedpycode{python/advanced/generators/for}
  \item Manually with the \texttt{next} method \mintedpycode{python/advanced/generators/manual}
  \item \alert{Single-use}, once exhausted, the iterator needs to be recreated to iterate again
  \end{itemize}
  \end{frame}
  
  \begin{frame}{Chaining Generators}
  \mintedpycode{python/advanced/generators/chain}
  
  Or using a comprehension:
  
  \mintedpycode{python/advanced/generators/chain-short}
  \end{frame}
  
  \begin{frame}{Advanced Functions}
  
  \texttt{yield} returns values that are sent to it using \texttt{send}:
  
  \mintedpycode{python/advanced/generators/send}
  
  \texttt{next(generator)} is equivalent to \texttt{generator.send(None)}.
  
  Similarly, the \texttt{throw} and \texttt{close} methods send exceptions to the pause point.
  \end{frame}
  
  \begin{frame}{\texttt{yield from} Keyword}
  The \texttt{yield from} keyword allows \texttt{yielding} from another generator:
  
  \mintedpycode{python/advanced/generators/yieldfrom}
  \end{frame}