\begin{frame}{Introduction}
  Design patterns are recognized solutions to common problems.
\end{frame}

\begin{frame}{Origin}
  Book \textit{Design Patterns – Elements of Reusable Object-Oriented Software} by the "Gang of Four" (Erich Gamma, Richard Helm, Ralph Johnson, and John Vlissides), published in 1994.
\end{frame}

\begin{frame}{In Python}
  The required design patterns in Python are fewer compared to Java or other similar languages. However, there are several patterns that are frequently used in Python, such as:

  \begin{itemize}
    \item Composite (same name in English)
    \item Builder
    \item Adapter
    \item Factory
    \item Iterator
    \item Observer
  \end{itemize}
\end{frame}

\begin{frame}{Formalism}
  In their book, the GoF presents each pattern in the form of a UML diagram. This is also how they are most commonly presented today.

  However, this approach is often not suitable for the Python language, which
  does not have all the features of Java (e.g., visibility).
\end{frame}

\begin{frame}{Demonstration}
  Study of some patterns on the website \bluelink{https://refactoring.guru/design-patterns/python}{refactoring.guru}.
\end{frame}