\begin{frame}{Objectif}
  Vous pouvez être amené⋅e à utiliser Python sur GNU/Linux même si vous travaillez habituellement sur Windows~:

  \begin{itemize}[<+(1)->]
    \item Sur un serveur
    \item Dans un conteneur
    \item Pour un projet en particulier
  \end{itemize}

  \onslide<+(1)->{Cette courte partie (re-)présente quelques idées clefs pour comprendre ce système.}
\end{frame}

\begin{frame}{Historique}
  Système dérivé de UNIX, basé sur un noyau Linux et des logiciels GNU.
\end{frame}

\begin{frame}{Principes}
  GNU/Linux~:
  \begin{itemize}[<+(1)->]
    \item Suit une organisation de fichiers normalisée
    \item Représente tous les objets par des fichiers
    \item Fournit des petits programmes qui intéragissent bien entre eux plutôt que de gros programmes monolithiques
    \item Est très adapté au développement
    \item A un dépôt de logiciels centralisé adaptés pour bien fonctionner ensemble
  \end{itemize}
\end{frame}

\begin{frame}{Hiérarchie standard de fichiers}
  Le rôle de chaque dossier dans un système GNU/Linux est expliqué \bluelink{https://fr.wikipedia.org/wiki/Filesystem_Hierarchy_Standard}{dans cet article Wikipédia}.
\end{frame}

\begin{frame}{Flux standards}
  Trois flux sont accessibles facilement depuis n'importe quelle invite de commande ou programme~:

  \begin{description}[<+(1)->]
    \item[\texttt{stdin}] Le flux d'entrée standard
    \item[\texttt{stdout}] Le flux de sortie standard
    \item[\texttt{stderr}] Le flux d'erreur standard
  \end{description}

  \onslide<+(1)->{Par exemple, \texttt{print} en Python écrit par défaut dans la sortie standard et \texttt{input} lit dans l'entrée standard.}
\end{frame}

\begin{frame}{Quelques différences notables avec Windows}
  \begin{itemize}[<+->]
    \item Les chemins utilisent des \texttt{/} comme séparateurs
    \item Il y a une seule racine au système de fichier (\texttt{/})
    \item Un système de permission graduellement porté sur Windows est présent depuis UNIX
    \item Il n'y a pas de base de registre pour le système de base. Tout se gère à l'aide de fichiers de configuration
    \item Les signaux envoyés par le noyau ne sont pas les mêmes
    \item Les liens symboliques ou appareils montés ne se comportent pas de la même manière
  \end{itemize}

  \onslide<+->{
    \begin{alertblock}{Attention}
      Si vous concevez une application système multi plates-formes, testez la rigoureusement sur tous les OS visés pour vérifier les comportements des différents modules utilisés.
    \end{alertblock}
  }
\end{frame}
