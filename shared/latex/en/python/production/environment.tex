\begin{frame}{Introduction}
  Goals of controlling your software environment~:
  \begin{itemize}
    \item Make your dev environment reproducible
    \item Control dependencies for production
    \item Deploy easily on any cloud
  \end{itemize}
\end{frame}

\begin{frame}{Environment isolation --- \texttt{venv}}
  \texttt{venv} allows you to isolate your Python environment~:
  \begin{itemize}
    \item It will not interact with your system installation
    \item It allows multiple and non-compatible environments on the same machine
    \item It is faster than container-based solutions
    \item It works by copying a Python installation (\textit{e.g.} the system one) and customizing it
  \end{itemize}
\end{frame}

\begin{frame}{\texttt{venv} usage}
  \mintedcustomcode{python/production/environment/venv}{console}
\end{frame}

\begin{frame}{Dependencies management}
  Goal~: describe precisely which libraries (including their version) are used.

  Several good tools exist~:

  \begin{description}
    \item[\texttt{pip} + \texttt{setup.py}] Traditional tool, defines a Python package
    \item[\texttt{pip} + \texttt{requirements.txt}] Simple dependencies list
    \item[\texttt{pip} + \texttt{pip-compile}] Same as above and handles dependencies version freeze
    \item[\texttt{Pipenv}] Modern tool to define a Python \textbf{client} but not a library
    \item[\texttt{poetry}] Modern tool to define a Python client or library
  \end{description}
\end{frame}

\begin{frame}{\texttt{poetry} usage}
  A few useful \bluelink{https://python-poetry.org/docs/cli/}{commands} to manage dependencies~:
  \begin{description}
    \item[\texttt{poetry new}] Creates a Python package skeleton that follows \texttt{poetry} standards
    \item[\texttt{poetry add}] Adds a Python package as dependency
    \item[\texttt{poetry remove}] Removes a Python package as dependency
    \item[\texttt{poetry install}] Installs dependencies as described in the \texttt{poetry.lock} file (which is handled by \texttt{poetry})
    \item[\texttt{poetry export}] Exports the \texttt{poetry.lock} to other formats (\textit{e.g.} \texttt{requirements.txt})
  \end{description}
\end{frame}

\begin{frame}{Towards deployment}
  A few platforms will directly deploy Python packages.

  More often, you will have to perform an additional step before deploying, containerization~:

  \begin{itemize}
    \item It allows to manage native dependencies on top of Python dependencies
    \item It has a greater robustness if the software is deployed on several OSes
    \item It is heavier to setup than \texttt{venv}
    \item It is overall more suited for production (\textit{e.g.} easy deployment on a Kubernetes cluster)
  \end{itemize}

  $\Rightarrow$ Dockerization of a Python package with a \texttt{Dockerfile}
\end{frame}

\begin{frame}{\texttt{Dockerfile} example}
  \mintedcustomcode{python/production/environment/Dockerfile}{docker}
\end{frame}
