\begin{frame}{Introduction}
  Buts~:
  \begin{itemize}[<+->]
    \item Mettre à disposition le logiciel au reste de l'entreprise / au public
    \item Rendre le processus automatique
    \item Déployer souvent, pour éviter l'effet tunnel
  \end{itemize}
\end{frame}

\begin{frame}{Niveaux de déploiement}
  On déploie habituellement sur plusieurs niveaux~:

  \begin{description}[<+->]
    \item[Local] Machine de la personne qui développe
    \item[Développement] Machine ou cluster pour des premiers tests de déploiement
    \item[Intégration] Environnement de l'intégration continue
    \item[Staging] Réplique à l'identique de la production
    \item[Production] Système utilisé par les utilisateurs finaux
  \end{description}
\end{frame}

\begin{frame}{Livraison continue et déploiement continu}
  En anglais, les deux se disent CD (\emph{Continous Delivery} et \emph{Continous Deployment}).

  \begin{block}{Livraison continue}
    \begin{itemize}[<+->]
      \item Cycles courts de production du logiciel
      \item Production régulière de livrables
      \item Déploiement à la main
    \end{itemize}
  \end{block}

  \onslide<+->{\begin{block}{Déploiement continu}
    Comme la livraison continue, mais déploiement régulier et automatique.
  \end{block}}
\end{frame}

\begin{frame}{Bonnes pratiques}
  \begin{itemize}[<+->]
    \item Utiliser un environnement contrôlé
    \item Déployer automatiquement sur tag git d'une certaine forme
    \item Utiliser pour cela Github Actions, Travis, Jenkins, CircleCI, …
  \end{itemize}
\end{frame}

\begin{frame}{Exemple de configuration}
  \mintedcustomcode{python/production/deployment/example}{yaml}
\end{frame}
