\begin{frame}{Introduction}
  En Python, la manipulation de fichiers se fait principalement depuis 4 modules~:

  \begin{description}[<+(1)->]
    \item[\texttt{shutil}] Opérations de haut niveau de copie et déplacement
    \item[\texttt{os}] Opérations de bas niveau
    \item[\texttt{pathlib}] Remplacement moderne pour une partie de \texttt{os}
    \item[\texttt{tempfile}] Création de fichiers et dossiers temporaires
  \end{description}
\end{frame}

\begin{frame}{Navigation ⋅ Module \texttt{pathlib}}
  \pycon{python/files/navigation}
\end{frame}

\begin{frame}{Utilisation de jokers}
  Il est possible d'utiliser des jokers grâce aux fonctions \texttt{pathlib.Path.glob} \& \texttt{pathlib.Path.rglob}~:

  \pycon{python/files/search}
\end{frame}

\begin{frame}{Reste de l'API \texttt{pathlib}}
  Consultez l'excellente \bluelink{https://docs.python.org/fr/3/library/pathlib.html}{documentation officielle} pour découvrir l'API complète.  
\end{frame}

\begin{frame}{Copie, déplacement \& suppression de fichiers et dossiers}
  \begin{description}[<+->]
    \item[\texttt{shutil.copy}] Copie sans les méta-données
    \item[\texttt{shutil.copy2}] Copie avec les méta-données
    \item[\texttt{shutil.copytree}] Copie un dossier
    \item[\texttt{shutil.move}] Déplace un fichier ou un dossier
    \item[\texttt{os.remove}] Supprime un fichier
    \item[\texttt{shutil.rmtree}] Supprime un dossier
  \end{description}
\end{frame}

\begin{frame}{Exemple d'utilisation de \texttt{shutil}}
  \mintedpycode{python/files/backup}
\end{frame}

\begin{frame}{Fichiers et dossiers temporaires}
  \begin{description}[<+->]
    \item[\texttt{tempfile.TemporaryFile}] Crée un fichier temporaire
    \item[\texttt{tempfile.NamedTemporaryFile}] Crée un fichier temporaire dont le nom est accessible (et pas seulement l'objet)
    \item[\texttt{tempfile.TemporaryDirectory}] Crée un dossier temporaire
  \end{description}

  \onslide<+->{Toutes ces méthodes s'utilisent dans un bloc \texttt{with} pour s'assurer de l'effacement des fichiers/dossiers.}

  \onslide<+->{\mintedpycode{python/files/tmp}}
\end{frame}

\begin{frame}{Création d'archives}
  \texttt{shutil.make\_archive} crée des archives avec compression si les modules correspondant sont disponibles.
\end{frame}

\begin{frame}{Manipulation de permissions, groupes et propriétaires}
  Le module \texttt{os} contient toutes ces fonctions quand elles sont exposées par le système d'exploitation.
\end{frame}
