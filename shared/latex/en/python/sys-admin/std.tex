\begin{frame}{Introduction}
  Les flux d'entrée, de sortie et d'erreur sont des canaux de communication privilégiés pour les outils systèmes.
\end{frame}

\begin{frame}{Accès en Python}
  Par le module \texttt{sys}~:

  \begin{itemize}[<+(1)->]
    \item \texttt{sys.stdin}
    \item \texttt{sys.stdout}
    \item \texttt{sys.stderr}
  \end{itemize}

  \onslide<+->{Ces flux sont exposés comme des fichiers.}
\end{frame}

\begin{frame}{Gestion facile de \texttt{stdin} pour créer des outils système}
  Le module \texttt{fileinput} permet de créer des outils qui travaillent au choix sur~:

  \begin{itemize}[<+(1)->]
    \item Les lignes de l'entrée standard
    \item Les lignes contenues dans les fichiers dont les noms sont données en argument
  \end{itemize}

  \onslide<+->{$\Rightarrow$ Définition d'outils standards type UNIX facile.}
\end{frame}

\begin{frame}{Redirection de des sortie et erreur standards}
  Le module \texttt{contextlib} propose deux fonctions utiles~:

  \begin{itemize}[<+(1)->]
    \item \texttt{redirect\_stdout}
    \item \texttt{redirect\_stderr}
  \end{itemize}
\end{frame}
