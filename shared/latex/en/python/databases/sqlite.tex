\begin{frame}{Introduction to DB API}
  \alert{DB API} is a specification of an API to access databases.

  This API (\textit{Python Database API Specification v2.0}) is defined by the \bluelink{https://peps.python.org/pep-0249/}{PEP 249}.
  
  Modules that want to conform to this specification should implement the interface provided.
\end{frame}

\begin{frame}{DB API implementation}
  Despite being a so called \textit{standard}, all modules do not necessarily conform to the DB API specification, but most of them do:
  \begin{itemize}
    \item \textit{sqlite}
    \item \textit{psycopg}
    \item \textit{mysql-python}
    \item \dots
  \end{itemize}

  As a counter example, SQLAlchemy, in his first version, was not compliant to the DB API interface (fixed in 2.0).
\end{frame}

\begin{frame}{The DB API interface}
    Briefly, DB API requires the implementation of two specific objects:
    \begin{description}
      \item[\texttt{Connection}] Manage the connection to the database
      \item[\texttt{Cursor}] Manage the context of a fetch operation, allowing to get information and execute queries 
    \end{description}
    Each object should implement dedicated methods (see p.\ref{dbapi:connection} and p.\ref{dbapi:cursor})

\end{frame}

\begin{frame}{\texttt{Connection} Objects}
  \texttt{Connection} objects are created with the method \texttt{connect}

  \mintedpycode{python/databases/sqlite-connect}

\end{frame}

\begin{frame}{\texttt{Connection} Objects}\label{dbapi:connection}
  Main methods of \texttt{Connection} Object are:
  \begin{description}
    \item[\texttt{close}] Close the connection immediately (called when object is deleted \texttt{\_\_del\_\_()})
    \item[\texttt{commit}] Commit pending transactions (not auto-commit by default)
    \item[\texttt{cursor}] Provide a \texttt{Cursor} object on the database
  \end{description}
  
\end{frame}

\begin{frame}{\texttt{Cursor} Objects}\label{dbapi:cursor}
  Main methods of \texttt{Cursor} Object are:
  \begin{description}
    \item[\texttt{execute}] Execute a given query on the dataset (\texttt{executemany} for multiple insertions)
    \item[\texttt{fetchone}] Fetch the next row of a query result (also \texttt{fetchmany}, \texttt{fetchall})
  \end{description} 

\end{frame}

\begin{frame}{A Simple Example}
  \mintedpycode{python/databases/sqlite-full-example}
\end{frame}

\begin{frame}{Demo SQLite}
  \bluelink{https://codespaces.new/NyxAether/python_sqlite_formation}{Demo}
\end{frame}
