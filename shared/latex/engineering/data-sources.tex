\fmg{
  \begin{frame}{Sources de données}
    \V{["img/trends-bigdata-ml", "tw", 1.0] | image}
  \end{frame}
}

\begin{frame}{Formats de données}
  \begin{itemize}
    \item Fichiers (textes, CSV, XML, JSON, HDF5, Parquet, …)
    \item BDDs relationnelles : SQL
    \item BDDs ``pas que'' relationnelles : NoSQL (représentation non-tabulaire des données)
  \end{itemize}
\end{frame}

\begin{frame}{Types de données}
  \begin{itemize}
    \item Logs d'activité (tabulaire)
    \item IOT (capteurs)
    \item Réseaux sociaux
    \item Open data (data-gouv.fr, geonames.org, ...)
    \item Web crawling 
  \end{itemize}
\end{frame}

\begin{frame}{Lac de données (\textit{data lake})}

  Caractéristiques

  \begin{itemize}
    \item Espace de stockage global pour les informations de l'entreprise
    \item Pas de schéma de données
    \item Pas de format imposé
    \item Pas de transactions
    \item Potentiellement volatile
  \end{itemize}

  Utilité

  \begin{itemize}
    \item[\textcolor{green}{+}] Facilité \& faible coût de stockage, scalable
    \item[\textcolor{red}{-}] Pas de gouvernance, de vérification d'intégrité
    \item[\textcolor{red}{-}] Pas de transactionnalité
  \end{itemize}
\end{frame}

\begin{frame}{Entrepôt de données (\textit{data warehouse})}
  Caractéristiques

  \begin{itemize}
    \item Espace de stockage global pour les informations de l'entreprise
    \item Schéma de données
    \item Base de données relationnelle
    \item Transactions
    \item Non volatile
  \end{itemize}

  Utilité

  \begin{itemize}
    \item[\textcolor{green}{+}] Gouvernance \& vérifications d'intégrité
    \item[\textcolor{red}{-}] Peu scalable
    \item[\textcolor{red}{-}] Travail important de normalisation
  \end{itemize}
\end{frame}

\begin{frame}{\textit{Lakehouse}}
  Combinaison de data lake \& data warehouse~: utilisation des normes de la warehouse sur du stockage big data
  \begin{itemize}
    \item Espace de stockage global pour les informations de l'entreprise
    \item Schéma de données
    \item Format imposé (par exemple Parquet)
    \item Transactions
    \item Non volatile
  \end{itemize}

  Utilité

  \begin{itemize}
    \item[\textcolor{green}{+}] Gouvernance \& vérifications d'intégrité
    \item[\textcolor{green}{+}] Scalable
    \item[\textcolor{red}{-}] Travail important de normalisation
  \end{itemize}
\end{frame}

\begin{frame}{Extraction Transformation Chargement (\textit{ETL})}
  \V{["img/etl-schema", "th", 0.45] | image}
  \begin{minipage}[l]{0.49\linewidth}
    Orientés \textbf{Synchronisation}
    \begin{itemize}
      \item Blendo
      \item Fivetran
      \item Segment
      \item Stitch
    \end{itemize}
  \end{minipage}\hfill
  \begin{minipage}[l]{0.49\linewidth}
    Orientés \textbf{Transformation}
    \begin{itemize}
      \item DataVirtuality
      \item Alooma
      \item Etleap
      \item Xplenty
    \end{itemize}
  \end{minipage}\hfill
\end{frame}

\begin{frame}{Sources de données}
  Bases de données sur les clients disponibles~:
  \begin{itemize}
    \item Instituts de sondage
    \item Réseaux sociaux
    \item Profils clients
    \item ...
  \end{itemize}
\end{frame}
    
\begin{frame}{Sources de données}
  Des places de marché en ligne de données :
  \begin{itemize}
    \item Microsoft Azure Marketplace
    \item Datamarket
    \item Data Publica
    \item ...
  \end{itemize}
\end{frame}
