\begin{frame}{Pipeline d'un projet de Machine Learning}
  \V{["ml-craftmanship", "tw", 0.8] | image}
\end{frame}

\begin{frame}{Complexification des solutions : retours dégressifs}
  \V{["ml-effort-vs-value", "tw", 0.8] | image}
\end{frame}

\begin{frame}{Caractéristiques et buts différents}
  Projet orientés académiques Vs orientés industriels
  \begin{itemize}
  \item $\neq$ Développement logiciel
  \item $\neq$ Infrastructure
  \item $\neq$ Performances
  \end{itemize}
\end{frame}

\begin{frame}{Différentes caractéristiques --- Développement logiciel}
  Orienté académique

  \begin{tabularx}{\textwidth}{rX}
    \textbf{Type de code} & Pile de scripts \\
    \onslide<2->{\textbf{Documentation} & Strict nécessaire à la collaboration immédiate \\}
    \onslide<3->{\textbf{Maintenabilité} & Fonctionne le temps de l'expérience \\}
    \onslide<4->{\textbf{Licence} & « Fair use » \\}
  \end{tabularx}

  \vfill
  Orienté industriel

  \begin{tabularx}{\textwidth}{rX}
    \textbf{Type de code} & Code hiérarchisé et déployable en production \\
    \onslide<2->{\textbf{Documentation} & Documentation fournie, testée \\}
    \onslide<3->{\textbf{Maintenabilité} & Code maintenable et robuste, CI/CD \\}
    \onslide<4->{\textbf{Licence} & Galaxies de licences à respecter \\}
  \end{tabularx}
\end{frame}

\begin{frame}{Différentes caractéristiques --- Infrastructure}
  Orienté académique

  \begin{tabularx}{\textwidth}{rX}
    \textbf{Données} & Benchmarks, souvent quelques fichiers \\
    \onslide<2->{\textbf{Hardware} & Limité, ou contraignant (clusters publics) \\}
  \end{tabularx}

  \vfill
  Orienté industriel

  \begin{tabularx}{\textwidth}{rX}
    \textbf{Données} & \og{}Data lake\fg{} \\
    \onslide<2->{\textbf{Hardware} & Cloud computing \& clusters locaux \\}
  \end{tabularx}
\end{frame}

\begin{frame}{Différentes caractéristiques --- Performance}
  Orienté académique

  Citations, rapidité, ouverture de collaborations, …

  \vfill
  Orienté industriel

  Plus value, avantage compétitif, gain d'image, …
\end{frame}

\begin{frame}{Qualité des données}
  Un problème d'ingénérie avant d'être un problème de machine learning~:

  \begin{center}
    \textbf{Données et prétraitements de qualité > Algorithme de qualité}
  \end{center}

  C'est une conséquence du fameux~:

  \begin{center}
    \textbf{Trash in $\rightarrow$ trash out}
  \end{center}
\end{frame}

\begin{frame}{Développer un projet en Machine Learning}
  Une approche en 4 étapes :
  \begin{enumerate}
  \item Créer un pipeline robuste de bout en bout (sans ML)
  \item Intégrer du ML simple
  \item Étendre et améliorer les datasets
  \item Améliorer les modèles
  \end{enumerate}

  \vfill

  \begin{alertblock}{Attention}
    Toujours conserver une pipeline robuste \& s'arrêter dès que les objectifs sont atteints (\textbf{potentiellement à l'étape 1.})
  \end{alertblock}
\end{frame}

\begin{frame}{Étape 1~: Créer un pipeline robuste de bout en bout (sans ML)}
  \begin{itemize}[<+->]
  \item Acquérir et nettoyer les données
  \item Créer un modèle baseline avec une heuristique simple
  \item Mettre en place l'évaluation (métrique, statistiques, visualisations, logs)
  \item Si nécessaire (industrie), mettre en place le déploiement
  \end{itemize}

  \onslide<+->{Buts~:}

  \begin{itemize}[<+->]
    \item Régler tous les détails importants d'ingénierie \textbf{avant} d'introduire du ML
    \item S'assurer d'une base solide pour comparer les modèles
  \end{itemize}

  \onslide<+->{\begin{alertblock}{Important}
    S'arrêter là si l'heuristique simple suffit à atteindre les objectifs.
  \end{alertblock}}
\end{frame}

\begin{frame}{Étape 2~: Intégrer du ML simple}
  \begin{itemize}[<+->]
    \item Implémenter le modèle avec le meilleur retour coût/bénéfice attendu
    \item Si nécessaire, mettre en place une recherche d'hyper-paramètres
  \end{itemize}

  \begin{alertblock}{Important}
    Comparer à la baseline. Comme auparavant, s'arrêter là si les objectifs sont atteints.
  \end{alertblock}
\end{frame}

\begin{frame}{Étape 3~: Étendre et améliorer les datasets}
  Étendre et améliorer les datasets~:

  \begin{itemize}[<+->]
    \item Ajouter de nouvelles sources de données
    \item Étendre les sources existantes
    \item Travailler le pré-processing et la normalisation des différentes sources
    \item Versionner les données autant que le code
  \end{itemize}
\end{frame}

\begin{frame}{Étape 3~: Ajouter des caractéristiques sensées}
  \begin{itemize}[<+->]
    \item Beaucoup de caractéristiques simples > peu de caractéristiques complexes
    \item Des caractéristiques répandues plutôt que rares
    \item Regarder les erreurs pour imaginer les caractéristiques qui aideraient
    \item Communiquer avec les experts métiers
    \item Versionner les caractéristiques autant que le code
  \end{itemize}
\end{frame}

\begin{frame}{Étape 4~: Améliorer les modèles}
  Si les objectifs ne sont toujours pas atteints, retour à l'étape 2 avec un modèle plus complexe~:

  \begin{itemize}[<+->]
    \item Implémenter le modèle avec le meilleur retour coût/bénéfice attendu
    \item Si nécessaire, mettre en place une recherche d'hyper-paramètres
  \end{itemize}
\end{frame}

\begin{frame}{Puis itérer}
  Améliorer autant de fois que nécessaire les données et les modèles.
\end{frame}

\begin{frame}{Évolution dans le temps}
  \begin{itemize}[<+->]
    \item Apprendre un modèle tous les $n$ jours
    \item Évaluer la dégradation des performances en fonction de l'âge du modèle
    \item Vérifier les performances en test avant de déployer en production
    \item Modèle appris sur des données jusqu'au jour $n$, tester sur les données après le jour $n$
    \item Mesurer la différence entre performance en apprentissage et test
    \item Plateau de performance $\Rightarrow$ trouver de nouvelles caractéristiques/augmenter la puissance du modèle
    \item En fonction du modèle, supprimer des caractéristiques pas déterminantes
  \end{itemize}
\end{frame}
