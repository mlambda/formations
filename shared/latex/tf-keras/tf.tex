\begin{frame}{Introduction}
  \V{["logos/tensorflow", "th", 0.7] | image}
\end{frame}

\begin{frame}{Structure de l'API}
  \centering
  \V{["tf-api", "tw", 1] | image}
\end{frame}

\begin{frame}{Deux styles}
  Mode graphe~: on définit un graphe de calcul qu'on exécute ensuite.

  Mode «~eager~»: on définit des fonctions python qui opèrent directement sur des valeurs.
\end{frame}

\begin{frame}{TensorFlow 1 vs TensorFlow 2}
  \centering
  Principales différences

  \begin{minipage}[l]{0.5\linewidth}
    \centering
    \textbf{TF1}
    \begin{itemize}[<+->]
      \item Graphes de calcul explicites par défaut
      \item API de haut niveau Keras indépendante
    \end{itemize}
  \end{minipage}\hfill
  \begin{minipage}[l]{0.49\linewidth}
    \setcounter{beamerpauses}{1}
    \centering
    \textbf{TF2}
    \begin{itemize}[<+->]
      \item Graphes de calcul implicites par défaut
      \item API de haut niveau Keras intégrée à TF
    \end{itemize}
  \end{minipage}\hfill
\end{frame}

\begin{frame}{Tenseurs --- Objets au cœur de TensorFlow}
  Tableaux multidimensionnels, utilisés pour~:

  \begin{itemize}[<+->]
    \item Poids des réseaux
    \item Données
  \end{itemize}

  \onslide<+->{\texttt{tf.Tensor} $\approx$ \texttt{numpy.ndarray} dans l'univers TensorFlow.}
\end{frame}

\begin{frame}{Tenseurs --- Création}
  \begin{itemize}[<+->]
    \item Directement à partir de tableaux NumPy ou de listes
    \item Échantillonnés aléatoirement
    \item Appel à une fonction tensorflow avec un tableau NumPy
    \item Pipeline \texttt{tf.data}
  \end{itemize}
\end{frame}

\begin{frame}{\texttt{tf.Variable}}
  Surcouche mutable de \texttt{tf.Tensor} pour les poids des modèles.
\end{frame}

\begin{frame}{Graphe de calcul}
  \begin{minipage}[l]{0.15\linewidth}
      $\;$
  \end{minipage}\hfill
  \begin{minipage}[l]{0.35\linewidth}
    \huge
    \begin{center}
      $\pi*r^2\;\;\;\rightarrow$
    \end{center}
  \end{minipage}\hfill
  \begin{minipage}[l]{0.49\linewidth}
    \V{["tensorflow", "th", 0.9] | image}
  \end{minipage}\hfill
\end{frame}

\begin{frame}{Graphe de calcul}
  Conversion du mode «~eager~» au mode graphe par \texttt{tf.function}. Permet :

  \begin{itemize}[<+->]
    \item L'optimisation du graphe des opérations
    \item Le déploiement vers Android, iOS, TPU, GPU, …
    \item De faire tourner le réseau sans interpréteur Python !
  \end{itemize}
\end{frame}

\begin{frame}{Différenciation automatique}
  Calcul automatique des gradients pour faciliter la rétropropagation des gradients pendant la descente de gradient.

  \texttt{tf.GradientTape} enregistre les opérations effectuées sur une variable~: permet l'autodiff.
\end{frame}

\begin{frame}{Organisation du code}
  \texttt{tf.Module} regroupe du code d'une même couche ou unité logique.

  Un modèle = une combinaison de \texttt{tf.Module}s.

  API principale~:

  \begin{itemize}[<+->]
    \item Accès aux variables du modèle par \texttt{model.variables}
    \item Accès aux variables entraînables du modèle par \texttt{model.trainable\_variables}
    \item Sauvegarde des poids par \texttt{tf.train.Checkpoint}
    \item Chargement des poids par \texttt{tf.train.Checkpoint.restore}
    \item Sauvegarde du modèle complet par \texttt{tf.saved\_model.save}
    \item Chargement du modèle complet par \texttt{tf.saved\_model.load}
  \end{itemize}
\end{frame}