\begin{frame}
  \frametitle{TF-IDF}
  Un document = un vecteur de la taille d'un dictionnaire. (donc à dimension fixe) \\
  \begin{center}
    $\boxed{w_{ij} = tf_{ij}\log{\frac{N}{df_i}}}$
  \end{center}
  où :

  \begin{description}[<+->]
    \item[$tf_{ij}$] Nombre d'occurrences du mot $i$ dans le document $j$
    \item[$df_i$] Nombre de documents contenant le mot $i$
    \item[$N$] Nombre total de documents
  \end{description}
  \onslide<+->{$\Rightarrow$ Produit scalaire, SVM, arbres, réseaux de neurones, …}
\end{frame}

\begin{frame}{TF-IDF --- Intérêt de $df_i$}
  Loi de Zipf, justifiant l'utilisation du terme $df_i$
  \V{"img/zipf-law" | image("tw", 0.9)}
\end{frame}

\begin{frame}{TF-IDF}
  Utilisation de $n$-grammes de mots ou de caractères.
  \begin{center}
  Exemple~: «~Le chien mange de la viande~»
  \end{center}

  Bigrammes de mots~:

  \begin{itemize}
  \item le-chien, chien-mange, mange-de, de-la, la-viande
  \end{itemize}

  Trigrammes de caractères~:

  \begin{itemize}
  \item le\_, e\_c, \_ch, chi, hie, ien, en\_, n\_m, \_ma, man, …
  \end{itemize}
\end{frame}

\begin{frame}{Analyse sémantique latente}
  Famille algorithmique des modèles thématiques (topic models)~:

  \begin{itemize}[<+(1)->]
    \item $\approx$ PCA sur la matrice des documents $\rightarrow$ relations entre les mots
    \item Composantes principales $\approx$ topics
  \end{itemize}

  \onslide<+(1)->{Par exemple~: un axe correspond au champ lexical du sport, un autre à celui de l'économie, etc.}
\end{frame}
