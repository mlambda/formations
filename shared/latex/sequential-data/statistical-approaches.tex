\begin{frame}{Données séquentielles - Premières approches }
  \begin{itemize}[<+->]
  \item Calcul des différences du premier ou second ordre.
  \item Modèles autoregressifs à moyenne mobile (ARMA, ARIMA, ...)
  \end{itemize}
\end{frame}

\begin{frame}{Outils --- Chronogramme}
  Chronogramme : tracé de t $\rightarrow$ X(t) 

  \V{["time-serie-chronogramme", "tw", 0.9] | image}
\end{frame}

\begin{frame}{Outils --- Lag plot}
  Tracé des points $X_{t - k}$ \& $X_t$ pour détecter les corrélations temporelles

  \begin{minipage}[l]{0.499\linewidth}
    \V{["lag-plot-linear", "tw", 0.7] | image}
    \V{["lag-plot-sinusoidal", "tw", 0.7] | image}
  \end{minipage}\hfill
  \begin{minipage}[l]{0.499\linewidth}
    \V{["lag-plot-white-noise", "tw", 0.6] | image}
    \V{["lag-plot-uniform-noise", "tw", 0.6] | image}
  \end{minipage}\hfill
\end{frame}

\begin{frame}{Outils --- Opérateur de lag L}
  \begin{itemize}
  \item $L(X_{t}) = X_{t-1}$
  \item $L^2(X_{t}) = X_{t-2}$
  \item $L^{-1}(X_{t}) = X_{t+1}$
  \item …
  \end{itemize}
\end{frame}

\begin{frame}{Premières approches}
  $x_t=g(t)+\phi_t$ \\
  où :
  \begin{itemize}
  \item $g(t)$ est déterministe (tendance globale du signal)
  \item $\phi(t)$ est un bruit stochastique
  \end{itemize}
  Les modèles statistiques «~classiques~» vont alors essayer de décomposer $g(t)$ en plusieurs fonctions dépendant de t ou bien de fonction récurrentes (comme dans ARMA)
\end{frame}

\begin{frame}{Premières approches }
  \V{["time-serie-decomposition", "th", 0.9] | image}
\end{frame}

\begin{frame}{ARMA --- Introduction}
  Modèlisation statistique de processus (faiblement) stationnaires \\
  \newline
  Processus faiblement stationnaire si :
  \begin{itemize}
    \item sa moyenne ne dépend pas de t
    \item la covariance ne dépend pas de t
  \end{itemize}
\end{frame}

\begin{frame}{ARMA --- Introduction}
  Exemple de processus non-stationnaires :
  \V{["processus-non-stationnaires", "tw", 0.9] | image}
\end{frame}

\begin{frame}{ARMA --- Composante autoregressive}
  \[
  X_{t}=c+\varepsilon _{t}+\sum _{i=1}^{p}\varphi _{i}X_{t-i}
  \]

  où

  \begin{itemize}
    \item $c$ est une constante
    \item $\varphi_i$ sont les paramètres du modèle
    \item $\varepsilon_i$ sont les termes d'erreurs considéré comme un bruit blanc
  \end{itemize}
\end{frame}

\begin{frame}{ARMA --- Composante moyenne mobile}
  \[
  X_{t}=\mu +\varepsilon _{t}+\sum _{i=1}^{q}\theta _{i}\varepsilon _{t-i}
  \]

  où

  \begin{itemize}
    \item $\mu$ est la moyenne attendue
    \item $\theta_i$ sont les paramètres du modèle
    \item $\varepsilon_i$ sont les termes d'erreurs considéré comme un bruit blanc
  \end{itemize}
\end{frame}

\begin{frame}{ARMA --- Formulation finale}
  \[
  X_{t}=c +\varepsilon _{t}+\sum _{i=1}^{p}\varphi _{i}X_{t-i}+\sum _{i=1}^{q}\theta _{i}\varepsilon _{t-i}
  \]
\end{frame}

% \begin{frame}{ARMA}
%   ARMA exprimé avec l'opérateur L: \\
%   \[
%   \left(I-\sum _{i=1}^{p}\varphi _{i}L^{i}\right)X_{t}=\left(I+\sum _{i=1}^{q}\theta _{i}L^{i}\right)\varepsilon _{t}
%   \]
% \end{frame}

% \begin{frame}{ARMA}
%   Autocorrélation de lag $k\in\mathbb{Z}$ (ACF): \\
%   $\Rightarrow$ Corrélation des deux variables aléatoires $X_t$ et $X_{t-k}$ \\
%   \newline
%   Autocorrélation partielle de lag $k\in\mathbb{Z}$ (PACF): \\
%   $\Rightarrow$ Corrélation des deux variables aléatoires $X_t$ et $X_{t-k}$ où les dépendances linéaires entre $X_t$ et les variables $X_{t-k'}$ on été enlevées (pour $k'\in[1 .. (k-1)]$)  
% \end{frame}

% \begin{frame}{ARMA}
%   Apprentissage des paramètres :

%   \begin{enumerate}[<+->]
%     \item se rapporter à un processus stationnaire : Transformation de box-cox des données, élimination des tendances et saisonnalité
%     \item on fixe p et q
%       \begin{itemize}
%         \item plot des autocorrélations partielles pour p
%         \item plot des autocorrélations  pour q
%         \item la méthode Akaike Information Criterion (AIC) est cependant recommandée
%       \end{itemize}
%     \item estimation des paramètres $\varphi$ et $\theta$ en utilisant par exemple le maximum de vraisemblance ou le critère des moindres carrés
%   \end{enumerate}
% \end{frame}

\begin{frame}{ARIMA}
  Peut être vu comme une cascade de deux modèles :

  \begin{itemize}[<+->]
    \item Processus de suppression de la tendance par différenciation temporelle
    \item Puis traitement comme un processus ARMA(p, q)
  \end{itemize}
\end{frame}

\begin{frame}{SARIMA}
  Peut être vu comme une cascade de deux modèles :

  \begin{itemize}[<+->]
    \item Processus de suppression des tendance et saisonnalité
    \item Puis traitement comme un processus ARMA(p, q)
  \end{itemize}
\end{frame}

\begin{frame}{VARMAX}
  Modèle encore plus général que les précédents~:

  \begin{itemize}
  \item Généralisation aux séries de vecteurs
  \end{itemize}
\end{frame}

\begin{frame}{Implémentations}
  \begin{itemize}[<+->]
    \item Module \texttt{tsa} de la librarie \texttt{statsmodels}
    \item Module \texttt{Prophet} de Facebook, variations sur ces algorithmes
  \end{itemize}
\end{frame}
