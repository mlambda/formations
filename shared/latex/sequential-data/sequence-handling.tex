\begin{frame}{Introduction}
  Plusieurs critères pour décider de comment traiter les séquences~:

  \begin{itemize}[<+->]
    \item Rapidité de traitement
    \item Modèle utilisé
    \item Volume de données
  \end{itemize}
\end{frame}

\begin{frame}{Fenêtrage}
  \begin{itemize}[<+->]
    \item Prendre en compte seulement les $n$ derniers éléments
    \item Taille de la fenêtre $\approx$ ordre d'un modèle de Markov
  \end{itemize}
\end{frame}

\begin{frame}{Aggrégation}
  Aussi appelé bucketing~:

  \begin{itemize}[<+->]
    \item Créer des valeurs aggrégées à partir des éléments précédents de la séquence
    \item Possibilité de combiner différentes échelles
  \end{itemize}
\end{frame}

\begin{frame}{Traitement complet}
  Certains modèles (comme les réseaux récurrents \& transformeurs), peuvent traiter des séquences entières~:

  \begin{itemize}[<+->]
    \item Modélisation des dépendances longue distance
    \item Pas forcément meilleur sur de petits datasets
  \end{itemize}
\end{frame}