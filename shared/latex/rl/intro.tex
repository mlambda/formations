\begin{frame}{Paradigme}
  \V{["img/rl", "tw", 0.5] | image}

  Où

  \begin{description}
    \item[$S_t$] État de l'environnement au temps $t$
    \item[$A_t$] Action choisie par l'agent au temps $t$
    \item[$R_t$] Récompense reçue par l'agent au temps $t$
  \end{description}
\end{frame}

\begin{frame}{But}
  Maximiser la récompense cumulative attendue~:

  \[
    G_t = \sum_{k=0}^T R_{t+k+1}
  \]

  where

  \begin{description}
    \item[$G_t$] Récompense cumulative attendue
    \item[$R_t$] Récompense reçue au temps $t$
  \end{description}
\end{frame}

\begin{frame}{Types de tâches}
  \begin{description}[<+->]
    \item[Épisodique] Début et fin clairs (\textit{i.e.} jusqu'au Game Over d'un jeu)
    \item[Continue] Début non clair, pas de fin (\textit{i.e.} bourse)
  \end{description}
\end{frame}

\begin{frame}{Familles principales}
  Approches~:

  \begin{itemize}
    \item Orientées valeur d'états ou états/actions
    \item Orientées politique
    \item Orientées modèle
  \end{itemize}
\end{frame}

\begin{frame}{Approches orientées valeur}
  Calculer une valeur pour chaque état possible~: la récompense cumulative (potentiellement \red{réduite}) attendue depuis cet état.

  \[
    V_\pi(s) = \mathbb{E}\left[
      R_{t + 1}
      + \red{\gamma} R_{t + 2}
      + \red{\gamma^2} R_{t + 3}
      + \dots
      \mid S_t = s
    \right]
  \]

  \pause

  Pour calculer $a$ depuis $V_\pi$~: choisir l'action $a$ qui mène à l'état avec la plus haute valeur ($V_\pi(s)$).

  \pause

  Bonne approche quand le nombre d'état est raisonnable.

  \pause

  Variante~: calculer une valeur pour une combinaison état/action.
\end{frame}

\begin{frame}{Exemple --- Problème}
  Aller du départ à l'arrivée. Chaque mouvement coûte $1$ (récompense de $-1$).

  \begin{tabular}{c|c|c|c|c|c|c}
    \cline{2-6}
     & & \cellcolor{black} & & & & \\
    \cline{2-6}
     & & \cellcolor{black} & & \cellcolor{black} & \cellcolor{black} & \\
    \cline{2-6}
    Entrée $\rightarrow$ & \phantom{-5} & \phantom{-5} & \phantom{-5} & \phantom{-5} & \phantom{-5} & $\leftarrow$ Sortie \\
    \cline{2-6}
     & & \cellcolor{black} & \cellcolor{black} & \cellcolor{black} & \cellcolor{black} & \\
    \cline{2-6}
     & & \cellcolor{black} & \cellcolor{black} & \cellcolor{black} & \cellcolor{black} & \\
    \cline{2-6}
  \end{tabular}
\end{frame}

\begin{frame}{Exemple --- Solution orientée valeur}
  Calculer $V(s)$ pour tous les états:

  \begin{tabular}{c|c|c|c|c|c|c}
    \cline{2-6}
      & -7 & \cellcolor{black} & -5 & -6 & -7 & \\
      \cline{2-6}
      & -6 & \cellcolor{black} & -4 & \cellcolor{black} & \cellcolor{black} & \\
      \cline{2-6}
      Entrée $\rightarrow$ & -5 & -4 & -3 & -2 & -1 & $\leftarrow$ Sortie \\
      \cline{2-6}
      & -6 & \cellcolor{black} & \cellcolor{black} & \cellcolor{black} & \cellcolor{black} & \\
      \cline{2-6}
      & -7 & \cellcolor{black} & \cellcolor{black} & \cellcolor{black} & \cellcolor{black} & \\
      \cline{2-6}
  \end{tabular}

  \pause

  Politique correspondante~: choisir l'action qui mène à l'état qui a la plus haute valeur~:

  \begin{tabular}{c|c|c|c|c|c|c}
    \cline{2-6}
      & $\downarrow$ & \cellcolor{black} & $\downarrow$ & $\leftarrow$ & $\leftarrow$ & \\
      \cline{2-6}
      & $\downarrow$ & \cellcolor{black} & $\downarrow$ & \cellcolor{black} & \cellcolor{black} & \\
      \cline{2-6}
      Entrée $\rightarrow$ & $\rightarrow$ & $\rightarrow$ & $\rightarrow$ & $\rightarrow$ & $\rightarrow$ & $\leftarrow$ Sortie \\
      \cline{2-6}
      & $\uparrow$ & \cellcolor{black} & \cellcolor{black} & \cellcolor{black} & \cellcolor{black} & \\
      \cline{2-6}
      & $\uparrow$ & \cellcolor{black} & \cellcolor{black} & \cellcolor{black} & \cellcolor{black} & \\
      \cline{2-6}
  \end{tabular}
\end{frame}

\begin{frame}{Approches orientées politique}
  Définir une politique (Distribution de probabilités sur les actions étant donné un état).

  \[
    \pi(a \mid s) = \mathbb{P}[A_t = a \mid S_t = s]
  \]

  Choisir une politique avec la plus haute récompense cumulative attendue.

  Bon choix quand le nombre d'états est grand.
\end{frame}

\begin{frame}{Exemple --- Solution orientée politique}
  Calculer \textbf{directement} une politique qui choisit les actions qui mènent à la plus haute récompense cumulative attendue~:

  \begin{tabular}{c|c|c|c|c|c|c}
    \cline{2-6}
      & $\downarrow$ & \cellcolor{black} & $\downarrow$ & $\leftarrow$ & $\leftarrow$ & \\
      \cline{2-6}
      & $\downarrow$ & \cellcolor{black} & $\downarrow$ & \cellcolor{black} & \cellcolor{black} & \\
      \cline{2-6}
      Entrée $\rightarrow$ & $\rightarrow$ & $\rightarrow$ & $\rightarrow$ & $\rightarrow$ & $\rightarrow$ & $\leftarrow$ Sortie \\
      \cline{2-6}
      & $\uparrow$ & \cellcolor{black} & \cellcolor{black} & \cellcolor{black} & \cellcolor{black} & \\
      \cline{2-6}
      & $\uparrow$ & \cellcolor{black} & \cellcolor{black} & \cellcolor{black} & \cellcolor{black} & \\
      \cline{2-6}
  \end{tabular}
\end{frame}

\begin{frame}{Hybrides valeur/politique}
  Certaines approches combinent les deux mécanismes (\textit{i.e.} méthodes acteur-critique ou Alpha Zero).
\end{frame}

\begin{frame}{Approches orientées modèles}
  Autoriser l'agent à construire une représentation de son environnement.

  Pour comparaison, une approche sans modèle n'apprend que les dynamiques état/action.
\end{frame}

\begin{frame}{Approches sur et hors politique}
  Certaines approches n'utilisent pas la même politique pour choisir les actions (simulation) et apprendre (fonction cible). Elles sont dites \frquote{hors politique}.
\end{frame}

\begin{frame}{Contraintes techniques \& théoriques}
  \begin{itemize}[<+->]
    \item L'environnement ne peut souvent pas être simulé parfaitement
    \item La récompense peut être lointaine
    \item L'espace à explorer peut être très grand
  \end{itemize}
\end{frame}

\begin{frame}{Contraintes techniques \& théoriques --- Bourse}
  \V{["img/stock-market", "th", 0.7] | image}
\end{frame}

\begin{frame}{Contraintes techniques \& théoriques --- Échecs}
  \V{["img/chess", "th", 0.5] | image}
  \pause
  \begin{center}
    $\approx 10^{120}$ parties possibles $\ggg 6 x 10^{85}$
    \newline
    (nombre d'atomes dans l'univers observable)
  \end{center}
\end{frame}

\begin{frame}{Contraintes techniques \& théoriques --- Go}
  \V{["img/goban", "th", 0.5] | image}
  \begin{center}
    \huge{$\approx 10^{600}$}
  \end{center}
\end{frame}
