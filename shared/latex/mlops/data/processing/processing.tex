\begin{frame}{Obtention des données}
  \begin{itemize}
    \item Viser une première itération courte pour avoir des retours des phases ultérieures
    \item Lister les sources potentielles et leur budget temps / argent
    \item Potentiellement faire faire une première annotation par la team ML
    \item Définir les profils compétents
  \end{itemize}
\end{frame}

\begin{frame}{Itérations sur les données}
  \begin{itemize}
    \item Ordre de grandeur~: pas plus de x10 d'un coup
    \item Travailler conjointement la qualité de la source, des processus, le volume
    \item Très différent en phase de prototypage et de production~:
      \begin{description}
        \item[Prototypage] Réunion d'assez de données pour décider go~/~nogo
        \item[Production] Travail de fond comme principal vecteur pour atteindre le plafond de performance
      \end{description}
  \end{itemize}
\end{frame}

\begin{frame}{Méta-données}
  Parce que les données sont le cœur d'un système de ML~:
  \begin{itemize}
    \item Conservation de leur provenance
    \item Traçage de leurs transformations
    \item Notion proche, la reproductibilité de l'obtention des données et des transformations~:
      \begin{itemize}
        \item De plus en plus importante pour la réglementation
        \item Primordiale pour le débogage
      \end{itemize}
  \end{itemize}
\end{frame}

\begin{frame}{Méta-données — Conséquences}
  La conservation de méta-données requiert~:

  \begin{itemize}
    \item Une définition stricte des processus d'obtention des données
    \item Une définition stricte des transformations
    \item La mise en place de systèmes dédiés
  \end{itemize}

  → Contraint beaucoup de décisions d'architecture
\end{frame}

\begin{frame}{Chaînes de traitements}
  \begin{itemize}
    \item Réponse aux problématiques de reproductibilité \& d'automatisation de processus
    \item Comme souvent en informatique~: graphes dirigés acycliques d'opérations
    \item Plusieurs solutions existent en fonction des besoins
  \end{itemize}
\end{frame}

\begin{frame}{Caractéristiques souhaitables}
  Points forts d'une chaîne de traitements
  \begin{itemize}
    \item Adaptation au batch (développement) comme au temps réel (production)
    \item Fidélité des traitements développement~/~production
    \item Passage à l'échelle
    \item Déploiement sur des cibles diverses
    \item Facilité de développement
    \item Intégration à l'écosystème
  \end{itemize}
\end{frame}

\begin{frame}{Principales solutions de chaînes de traitements}
  Point fort de chaque solution~:

  \begin{description}
    \item[\bluelink{https://www.tensorflow.org/tfx}{TFX}] Options de déploiement \& parité dev~/~prod
    \item[\bluelink{https://www.kubeflow.org/}{KubeFlow}] Intégration Kubernetes
    \item[\bluelink{https://dvc.org/}{DVC}] Reproductibilité \& solution «~low tech~»
    \item[\bluelink{https://mlflow.org/}{MLFlow}] Flexible, facile à adopter, supporté par les clouds
    \item[\bluelink{https://www.pachyderm.com/}{Pachyderm}] Intégration Kubernetes
  \end{description}
\end{frame}
