\begin{frame}{Introduction}
  \bluelink{https://dvc.org/}{\texttt{dvc}} (\textit{Data Version Control}) est né d'un constat~:

  \begin{itemize}
    \item \texttt{git} est très répandu
    \item Toutes les équipes de développement versionnent leur code
    \item L'équivalent en ML~/~data science n'est pas vrai~:
    \begin{itemize}
      \item Pas de \texttt{git} pour les données \& modèles
      \item Quasiment aucune équipe de ML~/~data science ne versionne ses données et ses modèles
    \end{itemize}
  \end{itemize}
\end{frame}

\begin{frame}{Concepts clefs}
  \begin{itemize}
    \item Construit sur \texttt{git}
    \item Seuls le hash md5 d'un fichier de données est géré par \texttt{git}
    \item Le fichier lui-même est stocké sur un serveur dédié
    \item \texttt{dvc} fait le lien entre le hash md5 et le fichier stocké
    \item Reproductibilité et mise en cache par les md5 des données et résultats intermédiaires
    \item Registre de modèles utilisant les tags \texttt{git}
    \item Traçage de métriques par \texttt{dvclive}
  \end{itemize}
\end{frame}

\begin{frame}{Serveurs de stockage}
  \begin{description}
    \item[Cloud]
      \begin{itemize}
        \item Amazon S3 (AWS) et outils compatibles (MinIO par exemple)
        \item Microsoft Azure Blob Storage
        \item Google Cloud Storage (GCP)
        \item Google Drive
        \item Aliyun OSS
      \end{itemize}
    \item[À héberger]
      \begin{itemize}
        \item SSH \& SFTP
        \item HDFS \& WebHDFS
        \item HTTP
        \item WebDAV
      \end{itemize}
  \end{description}

  Pendant le développement, on utilise souvent un «~serveur de stockage~» local~: un simple dossier.
\end{frame}

\begin{frame}{Commandes utiles}
  Parmi \bluelink{https://dvc.org/doc/command-reference}{toutes les commandes} \texttt{dvc}, les principales sont~:
  \begin{description}
    \item[\texttt{dvc add}] Gérer un fichier dans \texttt{dvc}~:
      \begin{itemize}
        \item L'ajoute à \texttt{.gitignore}
        \item Crée un fichier de méta-données \texttt{.dvc}
        \item Ajoute le fichier au cache \texttt{dvc}
      \end{itemize}
    \item[\texttt{dvc checkout}] Récupère dans le cache le fichier qui correspond à un fichier de méta-données \texttt{.dvc}
    \item[\texttt{dvc pull}] Récupère les fichiers de données depuis un serveur de stockage
    \item[\texttt{dvc push}] Envoie les fichiers de données vers un serveur de stockage
  \end{description}
\end{frame}

\begin{frame}{Chaînes de traitements}
  Définition par étapes avec~:
  \begin{itemize}
    \item Dépendances
    \item Paramètres
    \item Sorties
    \item Commande à exécuter
  \end{itemize}

\end{frame}

\begin{frame}{Définition d'une pipeline}
  Dans le terminal~:

  \mintedcustomcode{mlops/data/processing/dvc-pipeline}{console}

  En yaml~:

  \mintedcustomcode{mlops/data/processing/dvc-pipeline}{yaml}
\end{frame}

\begin{frame}{Exécution}
  \begin{itemize}
    \item Lancement d'une pipeline par \texttt{dvc repro} ou \texttt{dvc exp run}
    \item Crée un fichier \texttt{dvc.lock} qui contient les md5 de toutes les entrées \& sorties
    \item \texttt{dvc exp run} crée un commit hors de l'arbre de commits
    \item Transformation de ce commit en vraie branche si l'expérimentation était intéressante
  \end{itemize}
\end{frame}

\begin{frame}{Site avec intégration DVC}
  \bluelink{https://dagshub.com/}{DagsHub} permet de disposer d'une remote \texttt{dvc} gratuite pour se familiariser avec l'outil (et aussi d'une remote MLFlow).
\end{frame}
