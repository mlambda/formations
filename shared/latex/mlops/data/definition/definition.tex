\begin{frame}{Buts}
  On cherche à répondre à principalement deux questions~:
  \begin{itemize}
    \item Quelles sont les entrées et les sorties pertinentes~?
    \item Quel niveau de performance peut-on attendre~?
  \end{itemize}
\end{frame}

\begin{frame}{Principe \textit{trash in trash out}}
  Principe fondamental du Machine Learning~: la définition des données est le point \alert{central} d'un système de Machine Learning.
\end{frame}

\begin{frame}{Recherche vs Industrie}
  \begin{description}
    \item[Recherche] $\text{Système} = \text{Données} + \overbrace{\text{Paramètres} + \text{Modèle}}^{\text{Travail}}$
    \item[Industrie] $\text{Système} = \overbrace{\text{Données} + \text{Paramètres}}^{\text{Travail}} + \text{Modèle}$
  \end{description}
\end{frame}

\begin{frame}{Exemple de définition – Traduction}
  Comment définir la sortie~?
  \begin{enumerate}
    \item I was overwhelmed with joy. → J'ai été submergé par la joie.
    \item I was overwhelmed with joy. → Je fus submergé de joie.
    \item I was overwhelmed with joy. → Je fus terrassé par la joie.
  \end{enumerate}
\end{frame}

\begin{frame}{Exemple de définition – Audio}
  Comment définir l'entrée~?
  \begin{enumerate}
    \item Hum… J'arrive dans 5 minutes
    \item Hum, J'arrive dans 5 minutes
    \item J'arrive dans 5 minutes
  \end{enumerate}
\end{frame}

\begin{frame}{Exemple de définition – Fusion d'identités}
  Comment définir la sortie~?
  \begin{itemize}
    \item Martin Durant, 44000, …, <martin@durant.fr>
    \item Martin Durant, 44000, …, <martin.durant@gmail.com>
  \end{itemize}
\end{frame}

\begin{frame}{Enjeu}
  Toutes ces décisions changent la fonction que le modèle va approximer.
\end{frame}

\begin{frame}{Types de données}
  Deux critères primordiaux~:

  \begin{itemize}
    \item Taille du jeu de données
    \item Données structurées ou non-structurées
  \end{itemize}
\end{frame}

\begin{frame}{Taille du jeu de données}
  \begin{description}
    \item[Petit] Qualité des annotations primordiale
    \item[Grand] Qualité des processus de traitement des données primordiale
  \end{description}

  \begin{block}{Attention}
    Un sous-ensemble d'un grand jeu de données peut se comporter comme un petit jeu de données (en particulier une tranche critique).
  \end{block}
\end{frame}

\begin{frame}{Données structurées~/~non-structurées}
  \begin{description}
    \item[Structurées] Dures à annoter pour des humains. Dures à augmenter
    \item[Non-structurées] Probablement faciles à annoter pour des humains. Souvent augmentables
  \end{description}
\end{frame}

\begin{frame}{Guide d'annotation}
  À destination des annotateurs~:
  \begin{itemize}
    \item Doit être le plus robuste possible
    \item Idéalement, écrit par un mélange d'experts métier / ML
    \item Écrit itérativement~:
      \begin{itemize}
        \item Écriture d'une version
        \item Annotation
        \item Détection des points ambigus
        \item Écriture d'une nouvelle version
      \end{itemize}
  \end{itemize}
\end{frame}

\begin{frame}{Couverture des données d'entrée}
  \begin{itemize}
    \item Tous les cas que l'on traitera doivent être représentés dans les données
    \item Dans des quantités suffisantes
    \item Particulièrement important pour éviter les discriminations sur les attributs protégés
  \end{itemize}
\end{frame}

\begin{frame}{Calibrage}
  Estimation des performances attendues~:

  \begin{itemize}
    \item Recherche bibliographique sur des approches existantes
    \item Estimation des performances humaines
  \end{itemize}
\end{frame}

\begin{frame}{Performances humaines}
  Aussi appelées HLP (\textit{Human Level Performance})~:

  \begin{itemize}
    \item Servent à estimer l'erreur de Bayes (l'aléa intrinsèque des données)
    \item En particulier quand l'annotation n'est pas définie par un autre humain mais par un processus externe
    \item Particulièrement intéressantes pour les données non-structurées
    \item Donnent une idée du niveau de performances atteignable
  \end{itemize}
\end{frame}

\begin{frame}{Amélioration des performances humaines}
  \begin{itemize}
    \item Tendance parfois à miniser les HLP pour les battre facilement
    \item De mauvaises HLP empêchent pourtant de bien orienter le travail à effectuer
  \end{itemize}
\end{frame}

\begin{frame}{Amélioration des performances humaines — Exemple}
  Supposons que quand une hésitation est formulée dans un audio, deux transcriptions soient possibles~:
  \begin{enumerate}
    \item Hum… J'arrive dans 5 minutes
    \item J'arrive dans 5 minutes
  \end{enumerate}
  Si 1. est préférée par $80\%$ des annotateurs et 2. par $20\%$~: deux annotateurs aléatoires ont $0,8^2 + 0,2^2 = 68\%$ d'être d'accord.

  Un algorithme trivial a $80\%$ d'accord.

  → Des erreurs plus graves sont « cachées » par ces gains sur les HLPs.
\end{frame}

\begin{frame}{Documentation}
  Pendant le processus de définition \& de calibrage, garder une trace des~:

  \begin{itemize}
    \item Biais potentiels des données que vous soupçonnez
    \item Problèmes de couverture des données réelles
    \item Enjeux réglementaires liés aux données
  \end{itemize}

  Ces informations sont cruciales pour bien documenter le modèle à terme.
\end{frame}
