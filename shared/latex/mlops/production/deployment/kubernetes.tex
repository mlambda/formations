\begin{frame}{Qu'est-ce que Kubernetes~?}
  Kubernetes (abrégé k8s), est~:
  \begin{itemize}
    \item Un gestionnaire de clusteur
    \item Un moyen d'orchestrer des conteneurs
  \end{itemize}
\end{frame}

\begin{frame}{Concepts clefs}
  La compréhension de quelques concepts clefs suffit pour aborder k8s dans un premier temps~:
  \begin{itemize}
    \item Réconciliation
    \item Pods
    \item Services
    \item Déploiements
  \end{itemize}
\end{frame}

\begin{frame}{Réconciliation}
  k8s fonctionne en confrontant deux états~:

  \begin{itemize}
    \item État déclaré dans la configuration du cluster
    \item État constaté dans le cluster
  \end{itemize}

  Le but de k8s est de faire en sorte que l'état constaté soit l'état déclaré.
\end{frame}

\begin{frame}{Réconciliation — Exemple}
  Si votre configuration déclare qu'un certain conteneur est présent 3 fois, k8s va~:

  \begin{itemize}
    \item Regarder dans le cluster combien de fois le conteneur est exécuté
    \item Si ce nombre est plus petit que 3, il va lancer le nombre nécessaire de conteneurs
    \item Si ce nombre est plus grand que 3, il va arrêter le nombre nécessaire de conteneurs
    \item Si ce nombre est 3, il ne va rien faire
  \end{itemize}
\end{frame}

\begin{frame}{Pods}
  Les Pods sont une surcouche des conteneurs. Ils ajoutent aux conteneurs des~:
  \begin{itemize}
    \item étiquettes et annotations
    \item politiques de redémarrage
    \item sondes (de démarrage, de préparation, de santé, …)
    \item politiques d'affinité et d'anti-affinité
    \item politiques d'arrêt
    \item politiques de sécurité
    \item politiques de gestion des ressources
    \item possibilités de couplage (regrouper plusieurs conteneurs liés dans un seul Pod)
  \end{itemize}

  → C'est à dire tout le nécessaire pour la bonne orchestration d'un grand nombre de conteneurs.
\end{frame}

\begin{frame}{Services}
  \begin{itemize}
    \item Les services sont des points d'interaction
    \item Ils permettent de s'abstraire de l'adresse de pods individuels
    \item Ils répartissent le travail à des pods
    \item Ils fournissent une adresse stable
  \end{itemize}
\end{frame}

\begin{frame}{Déploiements}
  Les Deployments k8s permettent de décrire les conditions de déploiement d'un type de conteneur.

  Ils peuvent être écrits directement ou abstraits par une surcouche (comme KubeFlow ou autre).
\end{frame}

\begin{frame}{Aller plus loin}
  \begin{itemize}
    \item \bluelink{https://kubernetes.io/docs/home/}{Documentation officielle de k8s}
    \item \bluelink{https://www.amazon.com/Kubernetes-Book-Version-November-2018-ebook/dp/B072TS9ZQZ/ref=tmm\_kin\_swatch\_0?\_encoding=UTF8&qid=1688546329&sr=8-1}{Ma recommandation personnelle de livre sur k8s}
  \end{itemize}
\end{frame}
