\begin{frame}{Qu'est-ce que Docker ?}
  \begin{itemize}
    \item Docker est une plateforme de conteneurisation
    \item Elle permet de regrouper et distribuer des applications et leurs dépendances dans des conteneurs légers
  \end{itemize}
\end{frame}

\begin{frame}{Pourquoi Docker ?}
  \begin{description}
    \item[Isolation] Les conteneurs fournissent des environnements isolés pour les applications
    \item[Portabilité] Les conteneurs s'exécutent de manière cohérente dans différents environnements
    \item[Efficacité] Les conteneurs sont légers et démarrent rapidement. Surcoût limité par rapport à une exécution native
  \end{description}
\end{frame}

\begin{frame}{Concepts importants de Docker}
  \begin{itemize}
    \item Image
    \item Conteneur
    \item Registre
  \end{itemize}
\end{frame}

\begin{frame}{Image}
  Deux manières de comprendre une image docker~:
  \begin{itemize}
    \item « Recette » pour produire un conteneur, comme une classe l'est pour un objet en programmation orientée objet
    \item État d'un conteneur à son lancement
  \end{itemize}
\end{frame}

\begin{frame}{Conteneur}
  \begin{itemize}
    \item Proche d'une machine virtuelle (mais ne réimplémente pas le système d'exploitation)
    \item C'est qui est exécuté pour lancer une application dockerisée
    \item Instance d'une image.
  \end{itemize}
\end{frame}

\begin{frame}{Registre}
  \begin{itemize}
    \item Dépôt d'images
    \item Public ou privé
    \item Par défaut~: \bluelink{https://hub.docker.com/}{Docker Hub}
  \end{itemize}
\end{frame}

\begin{frame}{Exécution avec docker}
  Quand on demande à lancer un conteneur docker~:
  \begin{enumerate}
    \item Docker vérifie si l'image du conteneur est déjà téléchargée localement. Sinon il la télécharge depuis Docker Hub par défaut
    \item Un conteneur est produit depuis l'image
    \item Ce conteneur est exécuté
  \end{enumerate}
\end{frame}

\begin{frame}{Options d'exécution avec docker}
  \begin{enumerate}
    \item Redirection de ports
    \item Montage de volumes de fichiers (pour rendre accessibles des fichiers de la machine hôte au conteneur)
    \item Suppression ou non du conteneur à la fin de l'exécution
    \item …
  \end{enumerate}
\end{frame}

\begin{frame}{Dockerfile}
  \begin{itemize}
    \item Un Dockerfile est un script qui permet de créer une image Docker personnalisée
    \item Il définit les étapes nécessaires pour configurer l'environnement de votre application
  \end{itemize}
\end{frame}

\begin{frame}{Structure d'un Dockerfile}
  \begin{itemize}
    \item Un Dockerfile commence généralement par une image de base
    \item Ensuite, il spécifie les étapes pour copier des fichiers, installer des dépendances, etc
  \end{itemize}
\end{frame}

\begin{frame}{Exemple de Dockerfile}
  \mintedcustomcode{mlops/deployment/docker/Dockerfile}{docker}
\end{frame}

\begin{frame}{Docker compose — Exécution complexe}
  \begin{itemize}
    \item Un outil pour définir et exécuter des applications Docker multi-conteneurs.
    \item Utilise un fichier YAML pour définir les services, les réseaux et les volumes.
  \end{itemize}
\end{frame}

\begin{frame}{Exemple Docker Compose}
  \mintedcustomcode{mlops/deployment/docker/docker-compose}{yaml}
\end{frame}

\begin{frame}{Commandes utiles}
  \begin{description}
    \item[\texttt{docker build}] Construit une image
    \item[\texttt{docker run}] Exécute un conteneur
    \item[\texttt{docker images}] Liste les images présentes localement
    \item[\texttt{docker ps}] Liste les conteneurs actifs
    \item[\texttt{docker rm/rmi}] Supprime un conteneur / une image
    \item[\texttt{docker attach/detach}] Permet de simuler screen / tmux
  \end{description}
\end{frame}

\begin{frame}{Image de base}
  Pour la mise en production de modèles de deep learning, bien choisir son image de base~:

  \begin{itemize}
    \item L'installation de TensorFlow~/~PyTorch~/~autre doit être optimisée pour la production
    \item Maîtriser la taille de l'image et les performances est délicat
    \item Préférable d'avoir le support GPU si il est nécessaire dès l'image de base
  \end{itemize}
\end{frame}

\begin{frame}{Images dédiées}
  Certaines images sont utilisables en l'état pour la mise en production~: \bluelink{https://www.tensorflow.org/tfx/serving/docker}{\texttt{tensorflow/serving}} pour TensorFlow par exemple.
\end{frame}

\begin{frame}{Conclusion}
  \begin{itemize}
    \item Docker est un outil puissant de conteneurisation
    \item Les Dockerfiles vous permettent de personnaliser vos images Docker
    \item Vous pouvez ensuite exécuter des conteneurs à partir de ces images pour déployer vos applications
  \end{itemize}
\end{frame}
