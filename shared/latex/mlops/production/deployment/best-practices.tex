\begin{frame}{Stratégies de déploiement}
  Plusieurs critères permettent de choisir une stratégie de déploiement~:
  \begin{itemize}
    \item Volume d'utilisation du service
    \item Public (tout public, interne, autres services, …)
    \item Fréquence de déploiement
    \item …
  \end{itemize}

  Concepts clefs~:
  \begin{itemize}
    \item Déploiement progressif
    \item Rollback (capacité de revenir à l'état antérieur du système)
  \end{itemize}
\end{frame}

\begin{frame}{Déploiement shadow}
  \begin{itemize}
    \item Déploiement du modèle en double du système en place
    \item Les sorties du modèle ne sont pas utilisées par l'application
    \item Analyse des sorties du modèle et décision de poursuivre le déploiement ou non
  \end{itemize}
\end{frame}

\begin{frame}{Déploiement bleu/vert}
  \begin{itemize}
    \item Déploiement du nouveau modèle (vert) en parallèle du système en place (bleu)
    \item Quand les tests sont concluants sur le système vert, on y redirige traffic
    \item Permet un rollback si des problèmes remontent
  \end{itemize}
\end{frame}

\begin{frame}{Déploiement canari}
  \begin{itemize}
    \item Similaire à bleu/vert, deux systèmes parallèles
    \item Montée en puissance progressive du système vert
  \end{itemize}
\end{frame}

\begin{frame}{Mise en place}
  Les deux options les plus populaires~:
  \begin{itemize}
    \item Kubernetes + Istio
    \item Outillage interne
  \end{itemize}
\end{frame}
