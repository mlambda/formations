\begin{frame}{Qu'est-ce que LIME~?}
  \begin{itemize}
    \item Méthode d'explication locale d'un modèle
    \item L'explication est un modèle interprétable
    \item Peut être utilisée sur n'importe quelle famille de modèles
  \end{itemize}
\end{frame}

\begin{frame}{Méthode}
  \begin{itemize}
    \item Transformation des données pour qu'elles soient interprétables, par exemple~:
      \begin{itemize}
        \item Superpixels pour des images
        \item Sac de mots pour du texte
        \item …
      \end{itemize}
    \item Entraînement d'un modèle simple qui approxime le modèle complexe autour du point à expliquer. Minimisation de la distance au modèle complexe
  \end{itemize}
\end{frame}

\begin{frame}{Illustration des modèles}
  \V{"img/mlops/technics/xai/local-model" | image("th", 0.5)}
\end{frame}

\begin{frame}{Illustration des explications}
  \V{"img/mlops/technics/xai/explain-dog-guitar" | image("tw", 1)}
\end{frame}

\begin{frame}{Méthode — Quelques détails}
  \begin{itemize}
    \item Des données perturbées sont générées pour peupler l'espace autour du point à expliquer~:
      \begin{itemize}
        \item Pour les données structurées, par échantillonnage d'une distribution apprise sur les données d'entrée
        \item Pour du texte, en enlevant au hasard des mots
        \item Pour des images, en segmentant en super-pixels et en en sélectionnant au hasard
      \end{itemize}
    \item Les données générées sont pondérées par leur distance au point
    \item Le modèle simple est souvent k-LASSO dans LIME → k caractéristiques sont non nulles après entraînement
  \end{itemize}
\end{frame}

\begin{frame}{Implémentation}
  En Python, le paquet \bluelink{https://github.com/marcotcr/lime/}{\texttt{lime}} contient tout le nécessaire.
\end{frame}

\begin{frame}{Limitations}
  \begin{itemize}
    \item Les données perturbées peuvent être invalides
    \item Le voisinnage d'un point peut être mal modélisé → explication incorrecte
    \item Globalement, insuffisant pour un besoin réglementaire d'explicabilité
  \end{itemize}
\end{frame}

\begin{frame}{Pour aller plus loin}
  \bluelink{https://youtu.be/CYl172IwqKs}{Vidéo explicative qui expose plus de détails}
\end{frame}
