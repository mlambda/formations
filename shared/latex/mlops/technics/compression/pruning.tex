\begin{frame}{Qu'est-ce que l'élagage (\textit{pruning})~?}
  \begin{itemize}
    \item Suppression de parties du modèle pour le rendre plus léger \& rapide
    \item Technique complémentaire de la quantization
  \end{itemize}
\end{frame}

\begin{frame}{Buts}
  \begin{itemize}
    \item Simplifier le modèle appris jusqu'à une complexité prédéfinie
    \item Accélérer l'exécution du modèle
    \item Réduire sa taille pour faciliter le stockage
  \end{itemize}
\end{frame}

\begin{frame}{Techniques}
  Comme pour la quantization, les techniques diffèrent en fonction des familles de modèles.

  Ici nous parlerons des réseaux de neurones car ce sont les modèles
qui ont le plus besoin d’être élagués.
\end{frame}

\begin{frame}{Méthode}
  \begin{itemize}
    \item À chaque itération on fixe à 0 une partie des poids les plus faibles
    \item La proportion passée à 0 est décidée par un plan d'élagage
    \item Le plan d'élagage fixe la densité de 0 à atteindre et la cadence à suivre
  \end{itemize}
\end{frame}

\begin{frame}{Implémentation}
  De manière très similaire à la quantization~:
  \begin{itemize}
    \item Utilisation du paquet \texttt{tensorflow-model-optimization}, en particulier~:
      \begin{itemize}
        \item La fonction \texttt{tfmot.sparsity.keras.prune\_low\_magnitude} pour l'entraînement
        \item Le callback \texttt{tfmot.sparsity.keras.UpdatePruningStep} pour l'entraînement
        \item La fonction \texttt{tfmot.sparsity.keras.strip\_pruning} pour l'export
      \end{itemize}
    \item Passage d'un modèle Keras à cette fonction et entraînement Keras normal (ne pas oublier le callback)
    \item Appel à \texttt{strip\_pruning}
    \item Conversion en modèle TFLite
  \end{itemize}
\end{frame}

\begin{frame}{Pour aller plus loin}
  \bluelink{https://www.tensorflow.org/model\_optimization/guide/pruning}{Guide sur l'élagage en TensorFlow}
\end{frame}