\begin{frame}{Introduction}
  L'analyse d'erreur est cruciale pendant le développement~:

  \begin{itemize}
    \item Oriente le travail futur
    \item Détermine le progrès potentiel
  \end{itemize}
\end{frame}

\begin{frame}{Lien entre analyse d'erreurs \& interprétabilité}
  \begin{itemize}
    \item L'interprétabilité permet l'analyse d'erreurs
    \item Parfois une obligation réglementaire ou fonctionnelle en production → transparence
    \item Continuum interprétabilité → performance~:
      \begin{itemize}
        \item Les modèles interprétables ne suffisent souvent pas à approximer les fonctions recherchées
        \item Les modèles performants modernes sont des boîtes noires
      \end{itemize}
  \end{itemize}

  Souvent un dilemme pour choisir un modèle.
\end{frame}

\begin{frame}{Approche conseillée}
  \begin{itemize}
    \item Déterminer des tranches des données pertinentes
    \item Estimer les performances du modèle et les performances atteignables sur chacune des tranches
    \item Prioriser le travail sur les tranches qui offriront le plus d'impact
  \end{itemize}
\end{frame}

\begin{frame}{Approche conseillée — Exemple}
  Imaginons travailler sur un système de classification d'images.

  Supposons qu'on distingue les tranches suivantes dans nos données d'images~:

  \begin{itemize}
    \item Présence de montagnes
    \item Présence d'humain
    \item Présence de voiture
  \end{itemize}
\end{frame}

\begin{frame}{Approche conseillée — Exemple, suite}
  On estime les performances suivantes~:

  \begin{tabular}{rcc}
    \toprule
    Tranche   & HLP  & Modèle \\
    \midrule
    Montagnes & 65\% & 60\%   \\
    Humains   & 96\% & 90\%   \\
    Voitures  & 80\% & 40\%   \\
    \bottomrule
  \end{tabular}

  Quelle est la tranche la plus importante pour les améliorations de la prochaine itération~?
\end{frame}

\begin{frame}{Approche conseillée — Exemple, fin}
  Utilisation des proportions dans le jeu de données pour quantifier l'impact~:

  \begin{tabular}{rcccc}
    \toprule
    Tranche   & HLP  & Modèle & Proportion & Impact potentiel \\
    \midrule
    Montagnes & 65\% & 60\%   & 10\%       & 0.5\% \\
    Humains   & 96\% & 90\%   & 85\%       & 5.1\% \\
    Voitures  & 80\% & 40\%   & 1\%        & 0.4\% \\
    \bottomrule
  \end{tabular}
\end{frame}

\begin{frame}{Analyse d'erreurs}
  Une fois les tranches pertinentes détectées~:

  \begin{itemize}
    \item Trouver des exemples explicatifs du modèle
    \item Utiliser un modèle plus simple qui explique globalement le modèle
    \item Ou localement
  \end{itemize}
\end{frame}

\begin{frame}{Exemples explicatifs}
  Trouver des~:
  \begin{description}
    \item[Prototypes] Exemples représentatifs du comportement du modèle
    \item[Exemples contrefactuels] Modification d'instances existantes pour voir comment évolue la prédiction
    \item[Exemples adversariaux] Exemples contrefactuels qui ont un grand impact sur la prédiction
    \item[Exemples influents] Exemples qui ont le plus eu d'impact sur le modèle
  \end{description}
\end{frame}

\begin{frame}{Modèles explicatifs}
  Entraînement de modèles~:
  \begin{itemize}
    \item Simples
    \item Interprétables (régression linéaire, arbres simples, etc)
    \item Approximant le modèle complexe
    \item Permettant de comprendre les caractéristiques importantes
    \item Globalement (plusieurs exemples, grande couverture)
    \item Ou localement (un exemple)
  \end{itemize}
\end{frame}

\begin{frame}{Mesures}
  Une fois l'analyse faite~:

  \begin{itemize}
    \item Augmentation des données
    \item Ingénierie de caractéristiques
    \item Exploration de nouveaux paramètres
  \end{itemize}
\end{frame}

\begin{frame}{Pour aller plus loin}
  \bluelink{https://christophm.github.io/interpretable-ml-book/}{Livre «~Interpretable Machine Learning~»}
\end{frame}