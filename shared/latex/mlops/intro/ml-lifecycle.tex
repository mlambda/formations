
\begin{frame}{Cycle de vie d'un projet ML}
  Plusieurs phases, toutes nécessitant des interventions spécifiques~:

  \begin{enumerate}
    \item Définition du périmètre projet
    \item Gestion des données
    \item Modélisation
    \item Déploiement
  \end{enumerate}

  Chaque phase a ses enjeux particuliers.
\end{frame}

\begin{frame}{1. Définition du périmètre projet}
  Points cruciaux~:

  \begin{itemize}
    \item Délimitation de la tâche à accomplir
    \item Définition des indicateurs de réussite
    \item Budget en temps, personnel, etc
  \end{itemize}
\end{frame}

\begin{frame}{2. Gestion des données}
  Points cruciaux~:

  \begin{itemize}
    \item Méthode non-biaisée de récolte des données
    \item Définition des entrées et des sorties claire
    \item Chaîne de traitement des données robuste sans disparité entraînement/production
    \item Reproductibilité et gestion des expériences
  \end{itemize}
\end{frame}

\begin{frame}{3. Modélisation}
  Points cruciaux~:

  \begin{itemize}
    \item Entraînement qui prend en compte les besoins de la production (taille du modèle, vitesse, etc)
    \item Analyse des erreurs, souvent par tranches importantes des données
    \item Reproductibilité et gestion des expériences
  \end{itemize}
\end{frame}

\begin{frame}{4. Déploiement}
  Points cruciaux~:

  \begin{itemize}
    \item Passage à l'échelle
    \item Détection des dérives des données et concepts
    \item Monitoring
    \item Processus de réentraînement
  \end{itemize}
\end{frame}
