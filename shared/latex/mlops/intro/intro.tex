\begin{frame}{Qu'est-ce que le MLOps~?}
  Plusieurs définitions~:

  \begin{itemize}
    \item DevOps adapté au ML
    \item Outils et pratiques d'ingénierie autour du ML
    \item Ce qui comble le fossé entre le prototype et la production (\textit{PoC to production gap}) en ML
  \end{itemize}
\end{frame}

\begin{frame}{Fossé entre le prototype et la production}
  \V{"tikz/mlops/intro/technical-debt" | image("tw", 1)}
\end{frame}

\begin{frame}{Fossé entre le prototype et la production — Exemple}
  Quelques problématiques pour une appli de traduction vocale sur téléphone~:

  \begin{itemize}
    \item Exécution du modèle sur téléphone ou en ligne
    \item Traitement pour homogénéiser les signaux d'entrée de différents micros
    \item Méthode de transmission des cas d'erreur pour améliorer l'application
    \item Niveaux de performance pour les différents accents
    \item Consommation des ressources du téléphone
    \item …
  \end{itemize}

  Ces problèmes ne sont pas des problèmes de ML pur.
\end{frame}

\begin{frame}{Deux grandes familles de problématiques}
  Par rapport au prototypage, la mise en production soulève~:

  \begin{itemize}
    \item Des problèmes techniques supplémentaires
    \item Des problèmes de réglementation et d'éthique supplémentaires
  \end{itemize}
\end{frame}

\begin{frame}{Niveaux de MLOps}
  Grandes lignes~:
  \begin{description}
    \item[0] Processus manuels
    \item[1] Chaînes de traitement automatisées
    \item[2] CI/CD pour le ML
  \end{description}

  Le niveau 1 est bien exploré, pas le 2.
\end{frame}

\begin{frame}{État du domaine MLOps}
  \begin{itemize}
    \item Évolution rapide, beaucoup de bibliothèques concurrentes
    \item Beaucoup moins stable que le domaine DevOps
    \item Adoption beaucoup moins homogène que le DevOps
    \item Évolution conjointe à la législation et réglementation
  \end{itemize}

  Étant donné l'instabilité du domaine, les grands principes sont plus importants que les techniques spéficiques.
\end{frame}
