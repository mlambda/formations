\begin{frame}
  \frametitle{Reproductibilité}
  \begin{itemize}
  \item extrêmement importante pour compléter les analyses après les retours business
  \item ensemble de bonnes pratiques d'ingénierie
  \end{itemize}
\end{frame}

\begin{frame}
  \frametitle{Reproductibilité}
  \begin{itemize}[<+->]
  \item garder une trace exacte du préprocessing
  \item de préférence utiliser des notebooks
  \item faire attention au random (utiliser des seeds)
  \item définir les datasets utilisés, dates comprises
  \item garder une trace de l'environnement
  \end{itemize}
\end{frame}

\begin{frame}
  \frametitle{Régularisation}
  \begin{minipage}[l]{0.49\linewidth}
    \begin{center}
      Régularisation \\
      $\approx$\\
      empêcher le surapprentissage
    \end{center}
  \end{minipage}\hfill
  \begin{minipage}[l]{0.49\linewidth}
    \V{["overfitting", "th", 0.3] | image}
  \end{minipage}\hfill
  Techniques variées en fonction du modèle :
  \begin{itemize}
  \item Pénalisation de la norme des paramètres
  \item Bruitage
  \item Dropout
  \item …
  \end{itemize}
\end{frame}

\begin{frame}
  \frametitle{Optimisation des méta-paramètres}
  Méta-paramètres : paramètres \alert{non appris} par le modèle.
  \begin{exampleblock}{Exemples}
  \begin{description}
  \item[Forme] Nombre de couches ? De quelles tailles ? …
  \item [Optimisation] SGD, AdaBoost, Adam, …
  \item [Régularisation] Pénalisation de la Norme des paramètres dans la loss, bruitage, dropout, …
  \end{description}
  \end{exampleblock}
  Optimisation par recherche aléatoire ou processus gaussien.
\end{frame}

