\subsection{Détection d'Anomalies}

\begin{frame}
  \frametitle{Détection d'Anomalies}
  Détection :
  \begin{itemize}
  \item de Fraude
  \item d'Intrusion/Fuite (physique ou électronique)
  \item Santé (biologique, geologique, machine, ...)
  \end{itemize}
\end{frame}

\begin{frame}
  \frametitle{Définition}
  \begin{itemize}
  \item une anomalie diffère de la norme par ses features
  \item les anomalies sont rares comparées aux instances normales
  \end{itemize}
\end{frame}

\begin{frame}
  \frametitle{Modes de détection d'anomalie}
  \imgtw[.8]{anomaly-modes}
\end{frame}

\begin{frame}
  \frametitle{Détection d'Anomalies : Supervisé}
  Problème de classification normal. \\
  Réseaux de neurones et SVM très performants.
\end{frame}

\begin{frame}
  \frametitle{Détection d'Anomalies : Semi-Supervisé}
  Détection de nouveauté. \\
  Pas traité ici. \\
  One-class SVM très utilisé.
\end{frame}

\begin{frame}
  \frametitle{Détection d'Anomalies : Non-Supervisé}
  De nombreuses méthodes : 
  \begin{itemize}
  \item Local Outlier Factor (LOF)
  \item Unweighted Cluster-Based Outlier Factor
  \item Isolation Forest
  \item Autoencoder
  \item ...
  \end{itemize}
\end{frame}

\begin{frame}
  \frametitle{Détection d'Anomalies}
  \imgtw[1.0]{voronoi-airports}
\end{frame}

\begin{frame}
  \frametitle{Local Outlier Factor}
  \begin{columns}
    \begin{column}{.7\tw}
      \begin{itemize}[<+->]
      \item anomalies locales
      \item basé sur les k voisins du point
      \item définit une \og atteignabilité\fg{} par les distances de
        ces voisins
      \item calcule un ratio moyen d'atteignabilité du point et de ses
        voisins
      \end{itemize}
    \end{column}
    \begin{column}{.3\tw}
      \imgtw{lof-density}
    \end{column}
  \end{columns}
  \onslide<+->{→ Anomalie si le ratio moyen d'atteignabilité est
    beaucoup plus faible que celui de ses plus proches voisins}
\end{frame}

\begin{frame}
  \frametitle{Local Outlier Factor}
  \imgtw[.7]{lof}
\end{frame}

\begin{frame}
  \frametitle{Désavantages}
  \begin{itemize}
  \item lent (quadratique)
  \item a des à priori sur la distribution des données
  \end{itemize}
\end{frame}

\begin{frame}
  \frametitle{Isolation tree}
  \begin{itemize}
  \item arbre aléatoire (comme random forest mais le split est
    aléatoire)
  \item but : isoler une anomalie plus vite qu'un exemple normal
  \item petit chemin pour arriver à une feuille : anomalie
  \end{itemize}
  → Se sert du fait que les features des anomalies ne sont pas
  distribuées comme les autres.
\end{frame}

\begin{frame}
  \frametitle{Isolation forest}
  \imgtw[.9]{isolation-forest}
\end{frame}

