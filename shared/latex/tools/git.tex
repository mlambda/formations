\begin{frame}{Introduction}
  Buts de \texttt{git}~:
  \begin{itemize}[<+->]
    \item Collaborer sur une base de code
    \item Contrôler le source code
    \item Ne pas perdre d'information
  \end{itemize}
\end{frame}

\begin{frame}{Principales caractéristiques}
  \begin{itemize}[<+->]
    \item Décentralisé
    \item Branchage facile
    \item Adapté même aux gros dépôts de code (rapide)
  \end{itemize}
\end{frame}

\begin{frame}{Principales caractéristiques --- Décentralisation}
  \begin{itemize}[<+->]
    \item Un utilisateur = une copie personnelle du dépôt
    \item Opérations explicites de synchronisation entre les dépôts
  \end{itemize}
\end{frame}

\begin{frame}{Principales caractéristiques --- Branchage facile}
  \begin{itemize}[<+->]
    \item Une idée ou une feature = une branche
    \item Merge de branches très facilité par rapport aux outils précédents (svn)
  \end{itemize}
\end{frame}

\begin{frame}{Principales caractéristiques --- Performant}
  \begin{itemize}[<+->]
    \item Compression de tous les objets de l'historique
    \item Gère sans problème des dépôts de plusieurs Go
  \end{itemize}
\end{frame}

\begin{frame}{Récupérer un dépôt}
  Utilisation de la commande \texttt{git clone}

  \begin{exampleblock}{Exemple}
    \texttt{git clone git@github.com:m09/langage-python.git}
  \end{exampleblock}
\end{frame}

\begin{frame}{Enregistrer les changements}
  Utilisation de \texttt{git add} et \texttt{git commit}

  \begin{exampleblock}{Exemple}
    \texttt{git add README.md}\\
    \texttt{git commit -m "Improve the README file with external links"}
  \end{exampleblock}
\end{frame}

\begin{frame}{Créer des branches}
  Créer des branches est très facile avec git en utilisant \texttt{git branch} ou \texttt{git checkout -b}.

  \begin{exampleblock}{Exemple}
    \texttt{git checkout -b authentification}
  \end{exampleblock}
\end{frame}

\begin{frame}{Combiner deux branches}
  Trois options principales~:

  \begin{itemize}[<+->]
    \item \texttt{git merge} pour combiner deux branches en conservant l'historique
    \item \texttt{git rebase} pour combiner deux branches sans conserver l'historique
    \item \texttt{git cherry-pick} pour récupérer des commits particuliers
  \end{itemize}

  \begin{exampleblock}{Exemple}
    \texttt{git checkout main \&\& git merge feature}\\
    \texttt{git rebase main}\\
    \texttt{git cherry-pick 6ffb416}
  \end{exampleblock}
\end{frame}

\begin{frame}{Excellent tutoriel}
  Recommandé pour approfondir le concept des branches (et pas que)~: cet excellent \bluelink{https://learngitbranching.js.org/}{tutoriel en ligne}.
\end{frame}

\begin{frame}{Récupérer des changements}
  Pour mettre à jour le dépôt avec les changements introduits par des collègues, deux options~:

  \begin{itemize}[<+->]
    \item \texttt{git pull} récupère les changements et fait un merge
    \item \texttt{git fetch} se contente de récupérer les changements
  \end{itemize}

  \begin{exampleblock}{Exemple}
    \texttt{git pull}\\
    \texttt{git fetch origin feature}
  \end{exampleblock}
\end{frame}

\begin{frame}{Publier des changements}
  Pour mettre à jour un dépôt distant (remote) avec vos changements, utilisation de \texttt{git push}.

  \begin{exampleblock}{Exemple}
    \texttt{git push}\\
    \texttt{git push origin feature}\\
    \texttt{git push origin feature:remotefeaturebranch}
  \end{exampleblock}
\end{frame}

\begin{frame}{Annuler des changements}
  Pour revenir en arrière avec git, deux options~:

  \begin{itemize}[<+->]
    \item \texttt{git revert} si les changements ont été publiés (\texttt{git push})
    \item \texttt{git reset} sinon
  \end{itemize}

  \begin{exampleblock}{Exemple}
    \texttt{git revert}\\
    \texttt{git reset 67f0670}
  \end{exampleblock}
\end{frame}

\begin{frame}{Spécifier des révisions}
  Pour parler d'un commit en particulier dans git, on peut utiliser~:

  \begin{itemize}[<+->]
    \item son hash (\texttt{2209291045201b303a087c0387317b17607fc66e})
    \item son hash écourté (\texttt{2209291})
    \item une référence (\texttt{main}, \texttt{HEAD})
    \item des modificateurs (\texttt{main\^{}}, \texttt{main\~{}5\^{}2})
  \end{itemize}

  \begin{exampleblock}{Exemple}
    \texttt{git cherry-pick feature\~{}5}\\
    \texttt{git reset HEAD\^{}}
  \end{exampleblock}
\end{frame}

\begin{frame}{Aller plus loin}
  Référence très didactique~: \url{https://git-scm.com/book/en/v2}
\end{frame}