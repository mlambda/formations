\begin{frame}{Introduction}
  Contrôler son environnement logiciel pour~:
  \begin{itemize}[<+->]
    \item Rendre son environnement de développement reproductible
    \item Contrôler les dépendences pour la prod
    \item Déployer son environnement facilement sur différents clouds
  \end{itemize}
\end{frame}

\begin{frame}{\texttt{virtualenv}}
  \texttt{virtualenv} permet d'isoler un environnement Python~:
  \begin{itemize}[<+->]
    \item N'intéragit pas avec l'environnement système
    \item Permet la cohabitation de plusieurs environnements incompatibles
    \item Plus rapide et natif que les solutions basées sur les conteneurs
    \item Copie d'une distribution Python de base + customisation
  \end{itemize}
\end{frame}

\begin{frame}{Solutions basées sur \texttt{virtualenv}}
  \begin{description}[<+->]
    \item[pip + setup.py] Définition traditionnelle d'une librarie Python
    \item[pip + requirements.txt] Liste de dépendences
    \item[pip + pip-compile] Lister les dépendences puis les geler pour la stabilité
    \item[Pipenv] Définition moderne d'une \textbf{application} Python (pas librairie)
    \item[poetry] Définition moderne d'application ou librairie
  \end{description}
\end{frame}

\begin{frame}{Solutions basées sur les conteneurs}
  Docker~: conteneurs qui embarquent un système d'exploitation en plus de l'environnement Python~:
  \begin{itemize}[<+->]
    \item Permet de contrôler les librairies natives en plus des librairies Python
    \item Plus grande robustesse si le logiciel s'exécute sur plusieurs OS
    \item Plus lourd à mettre en place que \texttt{virtualenv}
    \item Plus adapté pour la prod (déploiement facile par Kubernetes)
  \end{itemize}
\end{frame}