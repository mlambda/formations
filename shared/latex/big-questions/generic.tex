\begin{frame}{Ingénierie \& théorie}
  \begin{itemize}[<+->]
    \item Données trouées, incomplètes, non-représentatives, biaisées, …
    \item Fonction de perte (objectif) pas adaptée au besoin
    \item Fuite d'information dans le modèle
    \item Quasi non-interprétabilité du modèle
    \item Non-adaptation du modèle au changement
    \item Sur-apprentissage
  \end{itemize}
\end{frame}

\begin{frame}{Données}
  Les \textbf{données} coûtent cher (récolte, nettoyage, \#Data~>~1M). \\
  Exemple d'outil : "Amazon Mechanical Turc"
  \V{["img/amazon-mechanical-turc", "th", 0.5] | image}
\end{frame}

\begin{frame}{Temps long de l'apprentissage}
  L'apprentissage d'un modèle prend \textbf{beaucoup} de \textbf{temps}
  \begin{itemize}[<+->]
    \item Reconnaissance parole : 4 GPU => plusieurs jours, semaines
    \item AlphaGo : 3 semaines sur 5000 TPU ( $\approx$ 30M\$)
    \item Pas de résultats concluants au premier run
  \end{itemize}
  \onslide<+->{$\rightarrow$ Peu de visibilité sur le temps d'obtention d'une plus value
  \V{["img/sablier-windows", "th", 0.6] | image}}
\end{frame}

\begin{frame}{Infrastructure}
  L'utilisation de GPU/TPU coûte cher… et ça ne s'arrange pas avec l'explosion du besoin par le minage de cryptomonnaies
  \V{["plt/nvidia-stock-history", "th", 0.6] | image}
\end{frame}

\begin{frame}{Faisabilité}
  Discussion sur le temps de développement :
  \begin{itemize}
    \item Test dans une app si un téléphone est dans un parc national
    \item Un solver de sudoku
    \item Savoir si il y a un panneau indicateur sur une photo
    \item Savoir si il y a un cycliste sur une photo
  \end{itemize}
\end{frame}

\begin{frame}{Faisabilité}
  Il est compliqué de faire comprendre la différence entre le facile et le très difficile.
  $\Rightarrow$ Problèmes de communication dans une équipe.
\end{frame}

\begin{frame}{Recrutement}
  \begin{minipage}{0.62\linewidth}
    Les \textbf{ingénieurs en machine learning} (compétents) coûtent cher :
    \begin{itemize}
      \item git clone d'un papier + jouer avec data $\neq$ data-scientist
      \item au croisement de l'ingénérie (traitement de gros volumes de données, standard de développement) et des mathématiques (statistiques, algèbre, optimisation)
      \item Le nombre de nouveaux diplomés ne suit pas la demande
      \item Les meilleurs sont/vont dans une poignée d'entreprises ! (ou presque...)
    \end{itemize}
  \end{minipage}
  \begin{minipage}{0.37\linewidth}
    \V{["img/data-scientist-star", "tw", 1] | image}
    \bluelink{https://www.lefigaro.fr/secteur/high-tech/2017/09/29/32001-20170929ARTFIG00274-du-parfum-a-la-voiture-l-intelligence-artificielle-sert-a-vendre-de-tout.php}{Article \emph{Le Figaro} : Du parfum à la voiture, l'intelligence artificielle sert à vendre de tout}
  \end{minipage}
\end{frame}

\begin{frame}{Acceptation}
  Utilisateurs prêts à accepter un algorithme qui fait des erreurs ?
  Xiaoyi, le docteur chinois :
  \begin{itemize}
    \item est un robot
    \item décroche l'examen de médecine en 2017
    \item système d’assistance au diagnostic testé en centre de santé
  \end{itemize}
\end{frame}

\begin{frame}{Acceptation}
  Utilisateurs prêts à accepter un algorithme qui fait des erreurs ?
  \V{["img/google-ceo-congress", "th", 0.65] | image}
\end{frame}

\begin{frame}{Interprétabilité}
  Utilisateurs prêts à accepter un algorithme qui n'est pas interprétable ?

  Réseaux de neurones $\iff$ enchaînement de multiplications matricielles avec des millions de paramètres.

  \V{["tikz/neural-networks/ae/deep-auto-encoder", "tw", 1] | image}
\end{frame}
