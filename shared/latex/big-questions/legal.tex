\begin{frame}{CNIL}
  \V{"img/logos/cnil" | image("tw", 0.8)}
\end{frame}

\begin{frame}{RGPD}
  Introduit en 2016 par le parlement européen, appliqué en 2018.

  Le consentement est requis pour tout traitement qui ne concerne pas l’objet principal du service.

  \V{"img/logos/rgpd" | image("th", 0.5)}
\end{frame}

\begin{frame}{La procédure ordinaire avec RGPD}
  À l’issue de contrôle ou de plaintes, 
  en cas de méconnaissance des dispositions du RGPD ou de la loi de la part 
  des responsables de traitement et des sous-traitants, \\
  la formation restreinte de la CNIL  peut :
  \begin{itemize}
    \item prononcer un rappel à l’ordre
    \item enjoindre de mettre le traitement en conformité, y compris sous astreinte
    \item limiter temporairement ou définitivement un traitement
    \item suspendre les flux de données
    \item ordonner de satisfaire aux demandes d’exercice des droits des personnes, 
          y compris sous astreinte
    \item prononcer une amende administrative
  \end{itemize}
\end{frame}

\begin{frame}{RGPD}
  \begin{itemize}
    \item amende jusqu’à \bluelink{https://www.cnil.fr/fr/definition/sanction}{4 \% du chiffre d’affaires annuel mondial}
    \item exception dans le droit français : \bluelink{https://www.cnil.fr/fr/la-loi-informatique-et-libertes\#article80}{la presse, l’art et la littérature} (Ex: refus des cookies $\rightarrow$ abonnement pour avoir accès aux contenus)\\
  \end{itemize}
\end{frame}

\begin{frame}{Loi informatique et liberté, procédure simplifiée}
  En cas de jurisprudence, des sanctions :
  \begin{itemize}
    \item moins nombreuses et moins sévères
    \item ne peuvent jamais être rendues publiques
  \end{itemize}
  Dans ce cadre, le président de la formation restreinte de la CNIL peut :
  \begin{itemize}
      \item prononcer un rappel à l’ordre 
      \item enjoindre de mettre le traitement en conformité, 
            y compris sous astreinte d’un montant maximal de 100 € par jour de retard 
      \item prononcer une amende administrative d’un montant maximal de 20 000 €
  \end{itemize}
\end{frame}

\begin{frame}{Délégué à la protection des données (DPO)}
  Personne désignée par un organisme
  \begin{itemize}
    \item relais entre la CNIL, l’organisme et les usagers concernés
    \item garant du respect de la RGPD (droits d’accès, retractation, ...)
    \item pilote de la démarche de mise en conformité permanente et dynamique
  \end{itemize}
\end{frame}

\begin{frame}{Délégué à la protection des données (DPO) --- Missions}
  Obligation de désignation pour toute organisation :
  \begin{itemize}
    \item publique
    \item amenée à réaliser un suivi régulier et systématique de personnes à grande échelle
    \item amenée à traiter à grande échelle des données 
          \bluelink{https://www.cnil.fr/fr/definition/donnee-sensible}{sensibles} 
          ou relatives à des condamnations pénales et infractions
  \end{itemize}
\end{frame}

\begin{frame}{Loi bioéthique}
  À propos du «~traitement algorithmique de données massives~» pour des 
  «~actes à visée préventive, diagnostique ou thérapeutique~»~: 
  le professionnel de santé sera tenu d’informer le patient 
  «~de cette utilisation et des modalités d’action de ce traitement~».
\end{frame}
