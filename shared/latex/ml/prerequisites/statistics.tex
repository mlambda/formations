
\begin{frame}
  \frametitle{Utilité}
  \begin{itemize}
  \item description et compréhension des données
  \item correction pour faciliter les traitements
  \end{itemize}
\end{frame}

\begin{frame}
  \frametitle{Types de variables}
  \V{["tikz/variables", "tw", 1] | image}
\end{frame}

\begin{frame}
  \frametitle{Hypothèse}

  Pré-requis pour les mesures statistiques qui suivent (et la plupart
  du machine learning) :
  \begin{itemize}
  \item les données \textbf{doivent être issues d'une même loi}
  \item chaque échantillon doit être \textbf{indépendant} des autres
  \item \red{pas évident en pratique !} \visible<+(1)->{\question{Pourquoi ?}}
  \end{itemize}
\end{frame}

\begin{frame}[fragile]
  \frametitle{Variance}
  Mesure la dispersion d'une série statistique (ou d'une variable) :

  \[
    V(X) = \mathbb{E}\left[(X - \mathbb{E}[X])^2\right]
  \]

  Pour la calculer :

  \[
    V(X) = \frac{1}{n}\sum_{i = 1}^{n}(x_i - \bar{x})^2
  \]
\end{frame}

\begin{frame}[fragile]
  \frametitle{Écart-type}
  Racine carrée de la variance

  \[
    \sigma(X) = \sqrt{V(X)}
  \]

\end{frame}

\begin{frame}[fragile]
  \frametitle{Écart-type — règle des 68, 95 et 99,7}

  Pour les lois normales :

  \V{["img/68-95-99,7", "th", 0.7] | image}
\end{frame}

\begin{frame}[fragile]
  \frametitle{Quartile}

  Les quartiles ($Q_1$, $Q_2$ et $Q_3$) divisent les données en 4
  intervalles contenant le même nombre d'observations.

  Déclinable en quantile de taille arbitraire (décile, percentile).

  \question{Que veut dire être dans le 95\ieme{} percentile ?}
\end{frame}

\begin{frame}[fragile]
  \frametitle{Boxplot}

  \V{["img/boxplot", "th", 0.9] | image}
\end{frame}

\begin{frame}[fragile]
  \frametitle{Covariance}
  Mesure la variabilité jointe de deux variables aléatoires :

  \[
    V(X) = \mathbb{E}\left[(X - \mathbb{E}[X])(X - \mathbb{E}[X])\right]
  \]
  \[
    \cov(X, Y) = \mathbb{E}\left[(X - \mathbb{E}[X])(Y - \mathbb{E}[Y])\right]
  \]

  Pour la calculer :

  \[
    \cov(X, Y) = \frac{1}{n}\sum_{i = 1}^{n}(x_i - \bar{x})(y_i - \bar{y})
  \]
\end{frame}

\begin{frame}[fragile]
  \frametitle{Corrélation}
  Covariance divisée par le produit des écart-types :

  \[
    \corr(X, Y) = \frac{\cov(X, Y)}{\sigma_X\sigma_Y}
  \]

  \question{Intérêt ?}{Pas d'unité.}
\end{frame}

\begin{frame}
  \frametitle{Test de normalité}

  Pour tester (et corriger) la normalité d'une distribution, on
  utilise deux mesures :

  \begin{itemize}
  \item l'asymétrie (\textit{skew})
  \item le kurtosis
  \end{itemize}
\end{frame}

\begin{frame}
  \frametitle{Asymétrie}

  \V{["img/skew", "tw", 1] | image}

  \[
    \asym(X) = \mathbb{E} \left[\left(\frac{X - \bar{X}}{\sigma}\right)^3\right]
  \]
\end{frame}

\begin{frame}
  \frametitle{Kurtosis}

  \V{["img/kurtosis", "th", 0.5] | image}

  \[
    \kurt(X) = \mathbb{E} \left[\left(\frac{X - \mu}{\sigma}\right)^4\right]
  \]
\end{frame}

\begin{frame}
  \frametitle{Transformation de Box-Cox}
  Asymétrie et kurtosis peuvent se corriger avec la transformation de
  Box-Cox ou des transformations log.
\end{frame}
