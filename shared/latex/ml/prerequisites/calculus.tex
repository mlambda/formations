
\begin{frame}
  \frametitle{Utilité}
  Souvent besoin de minimiser une fonction en machine learning.
\end{frame}

\begin{frame}
  \frametitle{Idée clef}

  Décider d'un $x$ de départ puis suivre la pente jusqu'au minimum.

  \onslide<2->{Pente = dérivée}

  \onslide<3->{→ Modifier itérativement $x$ par un pas vers l'opposé
    de la dérivée.}
\end{frame}

\begin{frame}
  \frametitle{Pente positive}

  \V{["positive-slope", "tw", 0.8] | image}

  Opposé de la pente = $-2$. Avec un pas de $0,1$, on passe de $1$ à $0,8$.
\end{frame}

\begin{frame}
  \frametitle{Pente négative}

  \V{["negative-slope", "tw", 0.8] | image}

  Opposé de la pente = $2$. Avec un pas de $0,1$, on passe de $-1$ à $-0,8$.
\end{frame}

\begin{frame}
  \frametitle{Exemple en 2 dimensions}
  dérivée → gradient
  \V{["gradient-nzmog", "th", 0.65] | image}
\end{frame}
