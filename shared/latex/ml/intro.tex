\begin{frame}{Un domaine vaste}
  \V{["plt/data-science", "th", 0.7] | image}
\end{frame}

\begin{frame}{Hiérarchie des noms}
  \V{["tikz/ml/ia-ml-deep", "th", 0.8] | image}
\end{frame}

\begin{frame}{Machine Learning}
  Nouvelle manière d'aborder la conception logicielle.
  \vfill
  \begin{block}{Changement de paradigme}
  Programmation explicite $\rightarrow$ programmation implicite
  \end{block}
\end{frame}

\begin{frame}{Ingénierie}
  \V{["tikz/ml/ml-ingenierie", "th", 0.75] | image}
\end{frame}

\begin{frame}{Matière première~: les données}
  On cherche une fonction qui a~:
  \begin{itemize}
    \item en entrée : N données en dimension D
    \item en sortie : N sorties en dimension K
  \end{itemize}
\end{frame}

\begin{frame}{Matière première~: les données}
  Chaque dimension du problème est représentée par une variable : 
  \V{["tikz/variables", "tw", 0.8] | image}
\end{frame}

\begin{frame}{Grandes familles}

  Apprentissage supervisé, non-supervisé, par renforcement, …
\end{frame}

\begin{frame}{Apprentissage supervisé}
  \textbf{Prédire} une valeur numérique (régression) ou l'appartenance à une classe (classification).
\end{frame}

\begin{frame}{Apprentissage non-supervisé}
  Faire émerger des profils, des groupes.
  \vfill
  \begin{exampleblock}{Exemple}
    Groupes de clients pour adapter sa stratégie marketing
  \end{exampleblock}
\end{frame}

\begin{frame}{Apprentissage par Renforcement}
  Apprendre une stratégie efficace dans un univers où les actions fournissent des récompenses (possiblement négatives).
  \vfill
  \begin{columns}[T]
    \begin{column}{0.5\textwidth}
      \V{["img/chess", "tw", 0.9] | image}
    \end{column}
    \begin{column}{0.5\textwidth}
      \V{["img/darpa-2007", "tw", 0.9] | image}
    \end{column}
  \end{columns}
\end{frame}

\begin{frame}{Exemple --- Régression linéaire}
  Prédire une valeur en fonction d'une autre.

  \V{["plt/reg-plot", "th", 0.65] | image}
\end{frame}

\begin{frame}{Exemple --- Régression linéaire multiple}
  Prédire une valeur en fonction de plusieurs autres.

  \V{["plt/regression-hyperplan", "th", 0.65] | image}
\end{frame}

\begin{frame}{Exemple --- Classification avec des réseaux à convolutions}
  \V{["img/cnn-schema", "tw", 1] | image}
\end{frame}

\begin{frame}{Exemple --- Classification avec des réseaux à convolutions}
  \V{["img/caltech256", "tw", 0.8] | image}
\end{frame}

\begin{frame}{Exemple --- Apprentissage par renforcement}
  \V{["img/car-dashboard", "tw", 0.8] | image}
\end{frame}

\begin{frame}{Topologie du domaine}
  \V{["tikz/ml-illustration", "tw", 0.8] | image}
\end{frame}

\begin{frame}{Points de vue}
  Beaucoup de façons de voir le machine learning. Basées sur~:
  \begin{itemize}[<+->]
    \item les paradigmes (supervisé, non supervisé, renforcement, en
      ligne, …)
    \item les modèles (arbres, grammaires, automates, réseaux de
      neurones …)
    \item les données (tabulaire, image, texte, vidéo, graphe, …)
    \item les techniques (statistiques, symboliques, probabilistes, …)
    \item les contraintes (real time, embarqué, big data, multilingue,
      …)
  \end{itemize}

  \onslide<+->{→ Domaine \textbf{extrêmement} vaste.}
\end{frame}

\begin{frame}{Choisir la bonne facette}
  Critères pour s'orienter dans les approches de machine learning:
  \begin{itemize}[<+->]
  \item quantité de données à disposition
  \item qualité du signal d'apprentissage dans les données
  \item difficulté du problème à résoudre
  \item besoin d'interprétabilité
  \item contraintes techniques
  \item contraintes de délai
  \item … et d'autres en fonction des domaines métiers
  \end{itemize}
\end{frame}

\begin{frame}{Conclusion}
  \begin{itemize}
  \item le machine learning est un champ vaste.
  \item il existe sûrement un modèle/paradigme pour vos besoins
  \item l'important est de définir les bons critères
  \end{itemize}
\end{frame}

\begin{frame}{Discussion}
  \begin{itemize}
  \item à quelles données allez-vous appliquer le machine learning ? À
    quels besoins ?
  \item aurez-vous besoin de modèles interprétables ou simplement très
    performants en prédiction ?
  \item quelles sont vos contraintes ?
  \end{itemize}
\end{frame}
