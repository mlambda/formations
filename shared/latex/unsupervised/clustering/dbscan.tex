
\begin{frame}{DBSCAN --- Introduction}
  Density-Based Spatial Clustering of Applications with Noise…

  \begin{itemize}[<+->]
    \item Gère le bruit (points sans clusters)
    \item Fonctionne bien dans la pratique
    \item Ne nécessite pas de nombre de clusters prédéfini
  \end{itemize}
\end{frame}

\begin{frame}{DBSCAN --- Procédé}
  Tant qu'il reste des points non étiquetés~:

  \begin{enumerate}
  \item Définir une distance $\epsilon$ et un nombre minimum de voisins $v$
  \item Prendre un point non-étiqueté au hasard et regarder son $\epsilon$-voisinage
  \item Si il y a $v$ voisins ou plus on crée un nouveau cluster
    \begin{enumerate}
    \item Expansion du cluster de proche en proche dans le voisinage
    \end{enumerate}
  \item Sinon (Bruit)
  \end{enumerate}
\end{frame}

\begin{frame}{DBSCAN --- Démonstration}
  \V{"img/dbscan-1" | image("tw", 0.9)}
\end{frame}

\begin{frame}{DBSCAN --- Démonstration}
  \V{"img/dbscan-2" | image("tw", 0.9)}
\end{frame}

\begin{frame}{DBSCAN --- Démonstration}
  \V{"img/dbscan-3" | image("tw", 0.9)}
\end{frame}

\begin{frame}{DBSCAN --- Démonstration}
  \V{"img/dbscan-4" | image("tw", 0.9)}
\end{frame}

\begin{frame}{DBSCAN --- Démonstration}
  \V{"img/dbscan-5" | image("tw", 0.9)}
\end{frame}

\begin{frame}{DBSCAN --- Démonstration}
  \V{"img/dbscan-6" | image("tw", 0.9)}
\end{frame}

\begin{frame}{DBSCAN --- Démonstration}
  \V{"img/dbscan-7" | image("tw", 0.9)}
\end{frame}

\begin{frame}{DBSCAN --- Démonstration}
  \V{"img/dbscan-8" | image("tw", 0.9)}
\end{frame}

\begin{frame}{DBSCAN --- Démonstration}
  \V{"img/dbscan-9" | image("tw", 0.9)}
\end{frame}
