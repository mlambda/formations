\begin{frame}{Présentation du problème}
    \begin{minipage}{0.49\linewidth}
        \textbf{Clustering de graphe}

        Faire du clustering sur de données qui présente des données structurées en graphe.

        Dans un premier temps, supposons que nous voulions diviser le graphe en deux.
        \\

        \alert{\textbf{Comment séparer ce graphe ?}}
    \end{minipage}
    \begin{minipage}{0.49\linewidth}
        \V{"tikz/graph/spectral-clustering-1" | image("tw", 1)}
    \end{minipage}
\end{frame}
\begin{frame}{Présentation du problème}
    \begin{minipage}{0.49\linewidth}
        \textbf{Clustering de graphe}
        \begin{itemize}
            \item L'objectif est de trouver une séparation qui minimise la \alert{conductance} ($\approx$ suppression de peu d'arêtes et clusters de tailles similaires)
            \item Calculer la conductance minimale est NP-complet, on lui préfère une analyse spectrale qui fournit une bonne approximation.
    \end{itemize}
        
    \end{minipage}
    \begin{minipage}{0.49\linewidth}
        \V{"tikz/graph/spectral-clustering-2" | image("tw", 1)}
    \end{minipage}
\end{frame}

\begin{frame}{Matrices de graphe}
    Il existe différentes matrices permettant de décrire un graphe :
    \begin{itemize}
        \item La matrice de degrée
        \item La matrice de d'adjacence
        \item La matrice de d'incidence
        \item La matrice laplacienne
    \end{itemize}
\end{frame}

\begin{frame}{Matrice de graphe : Degrée}
    \begin{minipage}{0.49\linewidth}
        \begin{itemize}
            \item Le \alert{degrée d'un nœud} indique le nombre d'arêtes connectées à celui-ci.
            \item La \alert{matrice de degrée} $D$ représente le degrée de chaque nœud dans une matrice diagonale :
            $$
            D = \begin{pmatrix}
                3 & 0 & 0 & 0 & 0 & 0\\
                0 & 2 & 0 & 0 & 0 & 0\\
                0 & 0 & 2 & 0 & 0 & 0\\
                0 & 0 & 0 & 3 & 0 & 0\\
                0 & 0 & 0 & 0 & 2 & 0\\
                0 & 0 & 0 & 0 & 0 & 2
            \end{pmatrix}
            $$
        \end{itemize}
    \end{minipage}
    \begin{minipage}{0.49\linewidth}
        \V{"tikz/graph/spectral-clustering-1" | image("tw", 1)}
    \end{minipage}
\end{frame}

\begin{frame}{Matrice de graphe : Adjacence}
    \begin{minipage}{0.49\linewidth}
        \begin{itemize}
            \item La \alert{matrice d'adjacence} $A$ représente les arêtes entre les nœuds. 
            \item Un 0 représente une absence d'arête entre deux nœuds, un 1 sa présence :
            $$
            A = \begin{pmatrix}
                0 & 1 & 1 & 1 & 0 & 0\\
                1 & 0 & 1 & 0 & 0 & 0\\
                1 & 1 & 0 & 0 & 0 & 0\\
                1 & 0 & 0 & 0 & 1 & 1\\
                0 & 0 & 0 & 1 & 0 & 1\\
                0 & 0 & 0 & 1 & 1 & 0
            \end{pmatrix}
            $$
        \end{itemize}
    \end{minipage}
    \begin{minipage}{0.49\linewidth}
        \V{"tikz/graph/spectral-clustering-1" | image("tw", 1)}
    \end{minipage}
\end{frame}

\begin{frame}{Matrice de graphe : Incidence}
    \begin{minipage}{0.49\linewidth}
        \begin{itemize}
            \item La \alert{matrice d'incidence} $I$ représente la relation entre les nœuds et les arêtes
            \item La matrice est de dimension $n \times p$ avec $n$ le nombe de nœuds et $p$ le nombre d'arêtes
            \begin{itemize}
                \item -1 si un arc sort du nœud
                \item 1 si un arc rentre dans le nœud
                \item 0 si le nœud n'est pas connecté à l'arc
            \end{itemize}
            \item Si le graphe est non orienté, on choisi arbitrairement un graphe orienté correspondant et valide.
        \end{itemize}
    \end{minipage}
    \begin{minipage}{0.49\linewidth}
        \V{"tikz/graph/spectral-clustering-3" | image("tw", 1)}
    \end{minipage}
\end{frame}

\begin{frame}{Matrice de graphe : Incidence}
    \begin{minipage}{0.49\linewidth}
        \begin{itemize}
            \item La matrice est de dimension $n \times p$ avec $n$ le nombe de nœuds et $p$ le nombre d'arêtes
            \begin{itemize}
                \item -1 si un arc sort du nœud
                \item 1 si un arc rentre dans le nœud
                \item 0 si le nœud n'est pas connecté à l'arc
            \end{itemize}
        \end{itemize}
        \scriptsize{
        $$
        I = \begin{pmatrix}
            0 & -1 & -1 & -1 & 0 & 0 & 0\\
            -1 & 1 & 0 & 0 & 0 & 0 & 0\\
            1 & 0 & 1 & 0 & 0 & 0 & 0\\
            0 & 0 & 0 & 1 & -1 & -1 & 0\\
            0 & 0 & 0 & 0 & 1 & 0 & -1\\
            0 & 0 & 0 & 0 & 0 & 1 & 1
        \end{pmatrix}
        $$}
    \end{minipage}
    \begin{minipage}{0.49\linewidth}
        \V{"tikz/graph/spectral-clustering-3" | image("tw", 1)}
    \end{minipage}
\end{frame}

\begin{frame}{Matrice de graphe : Laplacienne}
    \begin{minipage}{0.49\linewidth}
        \begin{itemize}
            \item La \alert{laplacienne} $L$ est une matrice particulière de représentation d'un graphe
            \item $L = I^{T}I$
            \item $L = D - A$
        \end{itemize}
        \footnotesize{$$
            L = \begin{pmatrix}
                3 & -1 & -1 & -1 & 0 & 0\\
                -1 & 2 & -1 & 0 & 0 & 0\\
                -1 & -1 & 2 & 0 & 0 & 0\\
                -1 & 0 & 0 & 3 & -1 & -1\\
                0 & 0 & 0 & -1 & 2 & -1\\
                0 & 0 & 0 & -1 & -1 & 2
            \end{pmatrix}
        $$}
    \end{minipage}
    \begin{minipage}{0.49\linewidth}
        \V{"tikz/graph/spectral-clustering-1" | image("tw", 1)}
    \end{minipage}
\end{frame}

\begin{frame}{Spécification du problème}
    \begin{minipage}{0.49\linewidth}
        \begin{itemize}
            \item Le but est de trouver un sous-graphe $G'\subset G$ defini par un vecteur $x$ qui représente l'ensemble des noeuds de $G'$
            \item $G'$ conserve l'ensemble des arêtes possibles entre les noeuds de $G'$
        \end{itemize}
    \end{minipage}
    \begin{minipage}{0.49\linewidth}
            \V{"tikz/graph/spectral-clustering-1" | image("tw", 1)}
    \end{minipage}
\end{frame}
\begin{frame}{Spécification du problème}
    \begin{minipage}{0.49\linewidth}
        \begin{itemize}
            \item Le but est de trouver un sous-graphe $G'\subset G$ defini par un vecteur $x$ qui représente l'ensemble des noeuds de $G'$
            \item $G'$ conserve l'ensemble des arêtes possibles entre les noeuds de $G'$
            \item Exemple : $x =(1,1,0,1,0,1)$
        \end{itemize}
    \end{minipage}
    \begin{minipage}{0.49\linewidth}
            \V{"tikz/graph/spectral-clustering-4" | image("tw", 1)}
    \end{minipage}
\end{frame}

\begin{frame}{Spécification du problème}
    Étant donné deux sous graphes $G_1,G_2 \subset G$ tels que l'ensemble des nœuds $x_1$ et $x_2$, respectivement, de $G_1$ et $G_2$ soient mutuellement exclusifs, on cherche $x_1, x_2$ tels que~:
    $$
        \min_{x_1, x_2} L.x_1 + L.x_2
    $$
\end{frame}

\begin{frame}{Spécification du problème}
    Exemple pour la séparation suivante:

    \V{"tikz/graph/spectral-clustering-2" | image("tw", 0.3)}

    $$
        L.x_1 + L.x_2 = 2
    $$

    avec $x_1 = (1,1,1,0,0,0)$ et $x_2 = (0,0,0,1,1,1)$
\end{frame}

\begin{frame}{Spécification du problème (Avancé)}
    Étant donné un graphe $G$ de laplacienne $L$, on cherche le vecteur $x$ de partitionnage tels que~:
    $$
        \min_{x} \frac{x^T L x}{x^T x}
    $$
    car L est symétrique semi definie positive.

    Ce qui est équivalent à une erreur quadratique :
    $$
        \min_x \frac{\sum_{(i,j)}(x_i-x_j)^2}{||x||^2}
    $$
    
    On pourra simplifier l'expression en garantissant que $||x||^2 = 1$ :
    $$
        \min_x \sum_{(i,j)}(x_i-x_j)^2
    $$
\end{frame}



\begin{frame}{Vecteurs propres de la laplacienne}
    Les vecteurs propres de la matrice laplacienne vont nous permettre de trouver une solution à notre problème.
    
    \begin{alertblock}{Rappel}
        Si $x$ est un vecteur propre de $L$ de valeur propre $\lambda$ alors~:
        $$
            Lx = \lambda x
        $$
    \end{alertblock} 

    On va chercher un vecteur propre $x$ tel que $\lambda$ est minimal et différent de 0.
\end{frame}

\begin{frame}{Vecteurs propres de la laplacienne}
    On peut voir un vecteur propre de la laplacienne comme une solution de séparation du graphe et sa valeur propre comme une estimation de la conductance suite à la séparation.
\end{frame}

\begin{frame}{Vecteurs propres de la laplacienne : Vecteurs nuls}
    Un vecteur propre évident est le vecteur $x=(1,1,\dots ,1 )$.
    
    C'est le principe de la laplacienne puisqu'elle contient à la fois le degré et la liste des adjacences pour chaque nœud.
    \vfill

    Il s'agit d'un vecteur nul non trivial ($\neq 0$).
    
    \alert{Il y aura autant de vecteurs nuls non triviaux qu'il y a de composantes connexes dans le graphe.}

    \centering \textbf{Pourquoi ?}
\end{frame}
 
\begin{frame}{Vecteur de Fiedler}
    \begin{itemize}
        \item Le \alert{vecteur de Fiedler} est le vecteur non nul ($\lambda \neq 0$) où $\lambda$ est minimal. 
        \item Le vecteur de Fiedler résout le problème de minimisation de la conductance à \textbf{une} séparation. [Fiedler 73]
    \end{itemize} 
\end{frame}
 
\begin{frame}{Vecteur de Fiedler~: Exemple}
    Voici les vecteurs propres (en colonnes) de notre graphe~:
    {\footnotesize
    $$
    \begin{pmatrix}
        0.41& \alert{-0.26}& -0.12&  0.11&  0.55& -0.66\\
        0.41& \alert{-0.46}&  0.62&  0.37& -0.25&  0.18\\
        0.41& \alert{-0.46}& -0.51& -0.48& -0.31&  0.18\\
        0.41&  \alert{0.26}& -0.12&  0.11&  0.55&  0.66\\
        0.41&  \alert{0.46}&  0.46& -0.6 & -0.08& -0.18\\
        0.41&  \alert{0.46}& -0.35&  0.49& -0.47& -0.18\\
    \end{pmatrix}
    $$}
    Le vecteur de Fiedler est indiqué en \alert{orange}.

    \V{"tikz/graph/spectral-clustering-1" | image("th", 0.35)}
\end{frame}

\begin{frame}{Décomposition en cluster}
    \begin{minipage}{0.64\linewidth}
        \begin{itemize}
            \item La décomposition en cluster utilise le seul vecteur de Fiedler.
            \item Les valeurs négatives forment les nœuds du premier groupe, les valeurs positives ceux du second groupe.
        \end{itemize}
        $x_F = ({\color{IndianYellow}-0.26, -0.46, -0.46}, {\color{NavyBlue}{0.26, 0.46, 0.46}})$
    \end{minipage}
    \begin{minipage}{0.34\linewidth}
        \V{"tikz/graph/spectral-clustering-2" | image("tw", 1)}
    \end{minipage}
\end{frame}

\begin{frame}{Résumé}
    Rappel du protocole complet :
    \begin{enumerate}
        \item Calculer la matrice laplacienne du graphe
        \item Calculer les vecteurs propres de cette matrice
        \item Conserver le vecteur de plus faible valeur propre supérieure à zéro
        \item Séparer les nœuds en deux groupes selon que la valeur associée au nœud soit positive ou négative
    \end{enumerate}
\end{frame}

\begin{frame}{Notes et compléments}
    \begin{itemize}
        \item Le nombre de valeurs propres nulles indique le nombre de clusters naturels (c-à-d les composantes connexes du graphe)
        \item Il n'est pas obligatoire de séparer au niveau de zéro. Il est possible d'utiliser une autre valeur de séparation comme par exemple celle qui maximise l'écart de rang entre les nœuds.
    \end{itemize}
    \V{"plt/clustering/spectral-clustering-fiedler-rank" | image("th", 0.5)}
\end{frame}

\begin{frame}{Cas non-binaire}
    L'algorithme présenté permet seulement de séparer un graphe en deux sous-graphe.

    Dans le cas de plusieurs clusters :
    \begin{itemize}
        \item On sépare les sous-graphes récursivement, jusqu'a atteindre le nombre de clusters souhaités (\textit{ancienne approche})
        \item On utilise les vecteurs propres de la laplacienne avec un algorithme de clustering type k-means (\textit{recommandée}). On parle alors de \alert{Spectral k-means clustering}
    \end{itemize}
\end{frame}

\begin{frame}{Cas non-binaire : Spectral k-means clustering}
    Au lieu de ne garder que le vecteur de Fiedler, on conserve les $k$ premiers vecteurs propres de plus petites valeurs propres non-nulles.

    Cela donne une matrice de dimension $(n,k)$, sur laquelle on applique un algorithme k-means, avec $k$ clusters.

    \alert{\textbf{Attention}} : 
    
    On utilise la laplacienne normalisée dans ce cas 
    $$L_{norm} = D^{-\frac{1}{2}}LD^{-\frac{1}{2}}$$
\end{frame}

\begin{frame}{Avantages et inconvénients}
    \begin{minipage}[t]{0.49\linewidth}
        \textbf{Avantages : }
        \begin{itemize}
            \item Pas d'hypothèse statistique dépendante d'une distance
            \item Implémentable en quelques lignes d'algèbre
            \item Produit de bons clusters avec des propriétés mathématiques garanties sur l'optimalité
        \end{itemize}
    \end{minipage}
    \begin{minipage}[t]{0.49\linewidth}
        \textbf{Inconvénients : }
        \begin{itemize}
            \item La dernière phase utilisant k-means peut rendre les résultats aléatoires
            \item La recherche de valeurs propres peut être très couteuses sur des datasets de grandes tailles. Il existe cependant des algorithmes très performants pour ce type de problème (LOBPCG).
        \end{itemize}
    \end{minipage}
\end{frame}

\begin{frame}{Clustering sur données non graphe}
    Pour les données qui ne sont pas des graphes, il est toujours possible de faire du clustering spectral en créant un graphe associé aux données~:
    \begin{itemize}
        \item Graphe des $\epsilon$-voisins : les nœuds à une distance de $\epsilon$ sont connectés
        \item Graphe KNN : Les $k$ plus proches voisins d'un point sont connectés
        \item Graphe dense de similarités : Graphe où tous les nœuds sont connectés mais les arêtes sont pondérés par 
        $$
            s(x_i,x_j) = \exp \left( -\frac{||x_i-x_j||^2}{2\sigma^2} \right)
        $$
        avec $\sigma$ l'hyperparametre qui controle la taille d'un \og voisinage \fg{}
    \end{itemize}
\end{frame}