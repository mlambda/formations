\begin{frame}{Présentation du problème}
    \begin{minipage}{0.49\linewidth}
        \textbf{Clustering de graphe}

        Faire du clustering sur de données qui présente des données structurées en graphe.

        Dans un premier temps, supposons que nous voulions diviser le graphe en deux.
        \\

        \alert{\textbf{Comment séparer ce graphe ?}}
    \end{minipage}
    \begin{minipage}{0.49\linewidth}
        \only<1>{\V{["tikz/graph/spectral-clustering-1", "tw", 1] | image}}
        \only<2>{\V{["tikz/graph/spectral-clustering-2", "tw", 1] | image}}
    \end{minipage}
\end{frame}

\begin{frame}{Matrices de graphe}
    Il existe différentes matrices permettant de décrire un graphe :
    \begin{itemize}
        \item La matrice de degrée
        \item La matrice de d'adjacence
        \item La matrice de d'incidence
        \item La matrice laplacienne
    \end{itemize}
\end{frame}

\begin{frame}{Matrice de graphe : Degrée}
    \begin{minipage}{0.49\linewidth}
        \begin{itemize}
            \item Le \alert{degrée d'un nœud} indique le nombre d'arêtes connectées à celui-ci.
            \item La \alert{matrice de degrée} représente le degrée de chaque nœud dans une matrice diagonale :
            $$
            D = \left[\begin{array}{cccccc}
                3 & 0 & 0 & 0 & 0 & 0\\
                0 & 2 & 0 & 0 & 0 & 0\\
                0 & 0 & 2 & 0 & 0 & 0\\
                0 & 0 & 0 & 3 & 0 & 0\\
                0 & 0 & 0 & 0 & 2 & 0\\
                0 & 0 & 0 & 0 & 0 & 2
            \end{array}\right]
            $$
        \end{itemize}
    \end{minipage}
    \begin{minipage}{0.49\linewidth}
        \V{["tikz/graph/spectral-clustering-1", "tw", 1] | image}
    \end{minipage}
\end{frame}

\begin{frame}{Matrice de graphe : Adjacence}
    \begin{minipage}{0.49\linewidth}
        
    \end{minipage}
    \begin{minipage}{0.49\linewidth}
        \V{["tikz/graph/spectral-clustering-1", "tw", 1] | image}
    \end{minipage}
\end{frame}

\begin{frame}{Matrice de graphe : Incidence}
    \begin{minipage}{0.49\linewidth}
        
    \end{minipage}
    \begin{minipage}{0.49\linewidth}
        \V{["tikz/graph/spectral-clustering-1", "tw", 1] | image}
    \end{minipage}
\end{frame}

\begin{frame}{Matrice de graphe : Lapiacienne}
    \begin{minipage}{0.49\linewidth}
        
    \end{minipage}
    \begin{minipage}{0.49\linewidth}
        \V{["tikz/graph/spectral-clustering-1", "tw", 1] | image}
    \end{minipage}
\end{frame}

\begin{frame}{Vecteurs propres de la laplacienne}
\end{frame}
 
\begin{frame}{Fiedler vector}
\end{frame}

\begin{frame}{Décomposition en cluster}
\end{frame}

\begin{frame}{Cas non-binaire}
\end{frame}

\begin{frame}{Avantages et inconvénients}
\end{frame}