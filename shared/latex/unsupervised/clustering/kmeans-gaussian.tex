\begin{frame}{K-Means --- Introduction}
  Algorithme EM
  \V{["img/cluster-2", "th", 0.7] | image}
\end{frame}

\begin{frame}{K-Means --- Algorithme}
  \begin{itemize}[<+->]
    \item Choisir $k$ points comme clusters initiaux
    \item Itérer les phases E (expectation) et M (maximisation)~:
      \begin{description}[<+->]
        \item[E] Attribuer à chaque point le cluster dont le centroïde est le plus proche
        \item[M] Réajuster les centroïdes pour qu'ils soient au milieu des points qui leur est attribué
      \end{description}
  \end{itemize}  
\end{frame}

\begin{frame}{K-Means --- Exemple}
  \bluelink{https://youtu.be/BVFG7fd1H30}{Vidéo de plusieurs apprentissages}
\end{frame}

\begin{frame}{K-Means --- Distance}
  distance euclidienne, Manhattan, Chebychev
  \V{["img/distances-euclidienne-manahattan-chebychev", "tw", 0.9] | image}
\end{frame}

\begin{frame}{K-Means++}
  Améliore l'initialisation~:

  \begin{itemize}[<+->]
    \item Choisir un point comme cluster initial
    \item Pour les points suivants, itérer~:
      \begin{itemize}[<+->]
        \item Calculer la distance de chaque point à son cluster le plus proche
        \item Choisir un point avec une probabilité proportionnelle à la distance calculée
      \end{itemize}
  \end{itemize}

  \onslide<+->{$\rightarrow$ Bien meilleures propriétés de convergence~!}
\end{frame}

\begin{frame}{Gaussian Mixture Model --- Introduction}
  Les clusters sont représentés par un centre et une matrice de covariance.
  \V{["img/gaussian-mixture", "tw", 0.7] | image}
\end{frame}

\begin{frame}{Gaussian Mixture Model --- Caractéristiques}
  \begin{itemize}[<+->]
    \item \og{}Soft\fg{} clustering~: chaque cluster est une fonction de densité de probabilité
    \item Algorithme EM, souvent initialisé avec K-Means
  \end{itemize}
\end{frame}
