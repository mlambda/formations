\begin{frame}{Principe}
  Local outlier factor en Anglais.
  \begin{columns}
    \begin{column}{.7\tw}
      \begin{itemize}[<+->]
        \item anomalies locales
        \item basé sur les k voisins du point
        \item définit une \og atteignabilité\fg{} par les distances de ces voisins
        \item calcule un ratio moyen d'atteignabilité du point et de ses voisins
      \end{itemize}
    \end{column}
    \begin{column}{.3\tw}
      \V{["img/lof-density", "tw", 1] | image}
    \end{column}
  \end{columns}

  \onslide<+->{→ Anomalie si le ratio moyen d'atteignabilité est beaucoup plus faible que celui de ses plus proches voisins.}
\end{frame}

\begin{frame}{Exemple}
  \V{["img/lof", "tw", 0.7] | image}
\end{frame}

\begin{frame}{Désavantages}
  \begin{itemize}
  \item lent (quadratique)
  \item a des à priori sur la distribution des données
  \end{itemize}
\end{frame}
