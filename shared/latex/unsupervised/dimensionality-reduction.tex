\begin{frame}
  \frametitle{Problématique}
    \begin{center}
      Comment appréhender des données en grande dimension ?
    \end{center}
    \[
    X = \begin{bmatrix}
      X_{1,1} & X_{1,2} & \dots  & X_{1,D} \\
      X_{2,1} & X_{2,2} & \dots  & X_{2,D} \\
      \vdots & \vdots & \ddots & \vdots \\
      X_{N,1} & X_{N,2} & \dots  & X_{N,D}
    \end{bmatrix}
    \]
\end{frame}

\begin{frame}{La malédiction des grandes dimensions}
    \begin{center}
      Nombre d'extrémités dans une espace de dimension : \\
      $\;$ \\
      \begin{tabular}{cc}
        Dimensions & Points d'un hypercube \\
        \midrule
        1 & 2 \\
        2 & 4 \\
        3 & 8 \\
        4 & 16 \\
        5 & 32 \\
      \end{tabular}
    \end{center}

  \begin{alertblock}{Besoin en données}
    Corrélé exponentiellement à la taille des espaces.
  \end{alertblock}
\end{frame}

\begin{frame}
  \frametitle{Approches}
    \begin{itemize}
    \item Séléction de dimensions
    \item Projections linéaires (ACP, LDA, …)
    \item Projections non-linéaires (t-SNE, UMAP, kernels, embeddings, …)
    \end{itemize}
\end{frame}
