\begin{frame}{Projections linéaires}
  Deux méthodes principales~:
  \begin{itemize}
    \item Principal Component Analysis (Non-supervisée)
    \item Linear Discriminant Analysis (Supervisée)
  \end{itemize}
\end{frame}

\begin{frame}{PCA --- Caractéristiques}
  \begin{itemize}[<+->]
    \item Crée de nouvelles features (combinaisons linéaires des originales)
    \item Ces features sont décorrélées
    \item Les premières sont les plus informatives
    \item Pour réduire la dimensionnalité~: ne conserver que les $k$ premières
  \end{itemize}
\end{frame}

\begin{frame}{PCA --- Exemple 1}
  \imgth[0.9]{pca-nuage}
\end{frame}

\begin{frame}{PCA --- Exemple 2}
  \imgtw[0.9]{pca-fish}
\end{frame}

\begin{frame}{PCA --- Point de départ}
  \[
  X = \begin{bmatrix}
    X_{1,1} & \dots  & X_{1,D} \\
    \vdots & \ddots & \vdots \\
    X_{N,1} & \dots  & X_{N,D}
  \end{bmatrix}
  \]
\end{frame}

\begin{frame}{PCA --- Procédé}
  Chaque dimension est centrée:
  \[
  \overline{X} =
  \begin{bmatrix}
    X_{1,1}-\overline{X_1} & \dots  & X_{1,D}-\overline{X_D} \\
    \vdots & \ddots & \vdots \\
    X_{N,1}-\overline{X_1} & \dots  & X_{N,D}-\overline{X_D}
  \end{bmatrix}
  \]

  puis optionnellement réduite~:

  \[
  \widetilde{X} =
  \begin{bmatrix}
    \frac{X_{1,1}-\overline{X_1}}{\sigma(X_1)} & \dots  & \frac{X_{1,D}-\overline{X_D}}{\sigma(X_D)} \\
    \vdots & \ddots & \vdots \\
    \frac{X_{N,1}-\overline{X_1}}{\sigma(X_1)} & \dots  & \frac{X_{N,D}-\overline{X_D}}{\sigma(X_D)}
  \end{bmatrix}
  \]
\end{frame}

\begin{frame}{PCA --- Procédé (suite)}
  \begin{itemize}[<+->]
    \item Calcul de la matrice de covariance (resp corrélation)~:
      \[
      \frac{1}{N} \times \overline{X^T} \times \overline{X}
      \quad
      ( \frac{1}{N} \times \widetilde{X^T} \times \widetilde{X} )
      \]
    \item Calcul des vecteurs propres de cette matrice~: ce sont les combinaisons linéaires des nouvelles features
    \item Tri des vecteurs par valeur propre décroissante
    \item Réduction de dimension : on ne conserve que les $k$ premiers vecteurs propres
  \end{itemize}
\end{frame}

\begin{frame}{PCA --- Linear Discriminant Analysis}
  Maximisation de la variance inter-classes.

  \begin{minipage}[l]{0.49\linewidth}
    \imgtw[0.8]{pca-nuage}
  \end{minipage}\hfill
  \begin{minipage}[l]{0.49\linewidth}
    \imgtw[1.0]{lda}
  \end{minipage}\hfill
\end{frame}
