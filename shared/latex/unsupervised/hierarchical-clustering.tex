
\begin{frame}{Clustering Hiérarchique}
  Deux approches~:

  \begin{itemize}[<+->]
    \item Agglomératives (bottom-up)
    \item Divisantes (top-down), moins utilisées
  \end{itemize}
\end{frame}

\begin{frame}{Clustering Hiérarchique --- Procédé}
  Méthode Agglomérative

  \begin{description}[<+->]
    \item[Initialisation] Chaque élément est dans une classe distincte
    \item[Aggrégation] Itérativement, fusion deux par deux les classes les plus similaires
    \item[Exploitation] On choisit le nombre de clusters à exploiter
  \end{description}
\end{frame}

\begin{frame}{Clustering Hiérarchique --- Résultat}
  \imgth[0.8]{dendogram}
\end{frame}

\begin{frame}{Clustering Hiérarchique --- Procédé}

  Quelques distances de dissimilarités, après avoir défini une distance D dans l'espace :

  \begin{itemize}[<+->]
  \item saut minimum : $dissim(C_1,C_2) = \underset{x \in C_1,y \in C_2}{\min}{D(x,y)}$
  \item saut maximum : $dissim(C_1,C_2) = \underset{x \in C_1,y \in C_2}{\max}{D(x,y)}$
  \item saut moyen : $dissim(C_1,C_2) = \underset{x \in C_1,y \in C_2}{moyenne\;}{D(x,y)}$
  \item distance de Ward qui vise à maximiser l'inertie inter-classe
  \item ...
  \end{itemize}

  $O(n^2)$ < complexité < $O(n^3)$ !
\end{frame}
