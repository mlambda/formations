\begin{frame}{Introduction}
  Les régressions linéaire et logistique partagent le même cœur~: une fonction affine des données.
  
  La régression logistique ajoute à ce cœur un mécanisme pour prédire des probabilités.
\end{frame}

\begin{frame}{Le cas à 2 classes}
  \begin{center}
    \V{["tikz/activations/sigmoid", "tw", 1] | image}
  \end{center}

  Ajout d'une sigmoïde au mécanisme de régression linéaire pour prédire des probabilités d'appartenance à une classe~: 
  \begin{center}
    $P(Y = 1 \mid X) = \sigma(\theta_0 + \theta_1 x_1 + \dots + \theta_d x_d)$
  \end{center}

  On décide ensuite d'un seuil pour déterminer la classe prédite (par exemple $0,5$).
\end{frame}

\begin{frame}{Régression logistique à plusieurs classes}
  Avec $k$ classes~:

  \begin{description}
  \item[Un contre tous] $k$ régressions logistiques~: chaque classe contre toutes les autres
  \item[Un contre un] $\frac{k(k - 1)}{2}$ régressions logistiques~: chaque classe contre chaque autre classe
  \end{description}
\end{frame}
