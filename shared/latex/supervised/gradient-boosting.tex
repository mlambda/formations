\begin{frame}
  \frametitle{Introduction}
  Arbres qui s'améliorent successivement.
  \V{["tikz/tree/tree-training-modes", "tw", 0.9] | image}
\end{frame}

\begin{frame}
  \frametitle{Idée à retenir}
  \begin{itemize}
  \item séquence d'arbres qui s'entraînent à corriger les erreurs de
    l'arbre d'avant
  \item modélisation de la correction de l'erreur par un pas de
    descente de gradient pour plus de flexibilité.
  \end{itemize}
   
\end{frame}

\begin{frame}
  \frametitle{Extensions}
  \begin{itemize}
  \item row sampling
  \item column sampling
  \item tree structure cost
  \end{itemize}
\end{frame}

\begin{frame}
  \frametitle{Avantages}
  \begin{itemize}
  \item modèle extrêmement performant et polyvalent
  \item entraînement parallélisable
  \end{itemize}
\end{frame}

\begin{frame}
  \frametitle{Désavantages}
  Sujet à l'overfit si pas assez régularisé (tree structure cost) et
  randomisé (row \& column sampling)
\end{frame}
