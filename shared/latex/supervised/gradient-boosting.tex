\begin{frame}{Introduction}
  Arbres qui s'améliorent successivement.
  \V{"tikz/tree/tree-training-modes" | image("tw", 0.9)}
\end{frame}

\begin{frame}{Idées à retenir}
  \begin{itemize}
    \item Séquence d'arbres qui s'entraînent à corriger les erreurs de l'arbre d'avant
    \item Modélisation de la correction de l'erreur par un pas de descente de gradient pour plus de flexibilité.
  \end{itemize}
\end{frame}

\begin{frame}{Extensions}
  \begin{itemize}
    \item Bagging (échantillonnage des lignes)
    \item Échantillonnage des colonnes
    \item Coût de structure d'arbre
  \end{itemize}
\end{frame}

\begin{frame}{Avantages}
  \begin{itemize}
    \item Modèle extrêmement performant et polyvalent
    \item Entraînement parallélisable
  \end{itemize}
\end{frame}

\begin{frame}{Désavantages}
  Sujet à l'overfit si pas assez régularisé (coût de structure d'arbre, échantillonnage de lignes \& de colonnes)
\end{frame}
