\begin{frame}{Évaluation de la qualité d'un split}
  En fonction de la tâche :
  \begin{description}
  \item[Régression] coût si on rendait la moyenne des instances
    comme résultat
    \[
      Loss = \sum|\overline{y}-y| \approx variance
    \]
  \item[Classification] Entropie de Shannon :
    \[
    Loss = -\sum_{x \in X}P_x*\log_2(P_x)
    \]
    $= 0 \Rightarrow$ il n'y a pas d'incertitude \\
    maximale quand on a une distribution uniforme
  \end{description}
\end{frame}

\begin{frame}{Exemple — démarrage}
  \begin{columns}
    \begin{column}{.5\textwidth}
      ID, jardinage, jeux vidéos, chapeaux, âge
      \[
        \begin{bmatrix}
          1 & 0 & 1 & 1 & 13  \\
          2 & 0 & 1 & 0 & 14 \\
          3 & 0 & 1 & 0 & 15 \\
          4 & 1 & 1 & 1 & 25 \\
          5 & 0 & 1 & 1 & 35 \\
          6 & 1 & 0 & 0 & 49 \\
          7 & 1 & 1 & 1 & 68 \\
          8 & 1 & 0 & 0 & 71 \\
          9 & 1 & 0 & 1 & 73 \\
        \end{bmatrix}
      \]
    \end{column}
    \begin{column}{.5\textwidth}
      Première étape : création du nœud de départ
      \\[1cm]
      \begin{forest}
        [{1, 2, 3, 4, 5, 6, 7, 8, 9} [{}]]
      \end{forest}
    \end{column}
  \end{columns}
\end{frame}

\begin{frame}{Exemple — split}
  \begin{columns}
    \begin{column}{.5\textwidth}
      ID, jardinage, jeux vidéos, chapeaux, âge
      \[
        \begin{bmatrix}
          1 & 0 & 1 & 1 & 13  \\
          2 & 0 & 1 & 0 & 14 \\
          3 & 0 & 1 & 0 & 15 \\
          4 & 1 & 1 & 1 & 25 \\
          5 & 0 & 1 & 1 & 35 \\
          6 & 1 & 0 & 0 & 49 \\
          7 & 1 & 1 & 1 & 68 \\
          8 & 1 & 0 & 0 & 71 \\
          9 & 1 & 0 & 1 & 73 \\
        \end{bmatrix}
      \]
    \end{column}
    \begin{column}{.5\textwidth}
      Split du premier nœud. Il faut tester 3 splits. Split sur
      jardinage :
      \\[1cm]
      \begin{forest}
        [{1, 2, 3, 4, 5, 6, 7, 8, 9}
          [{4, 6, 7, 8, 9},
          edge label={node[midway,left,font=\scriptsize]{jardinage}}
            [{$\overline{y}=57,2$} [{$L=80,8$}]]]
          [{1, 2, 3, 5},
          edge label={node[midway,right,font=\scriptsize]{$\neg$ jardinage}}
            [{$\overline{y}=19,25$} [{$L=31,5$}]]]
        ]
      \end{forest}\\
      Loss totale : $122,3$
    \end{column}
  \end{columns}
\end{frame}

\begin{frame}{Exemple — split}
  \begin{columns}
    \begin{column}{.5\textwidth}
      ID, jardinage, jeux vidéos, chapeaux, âge
      \[
        \begin{bmatrix}
          1 & 0 & 1 & 1 & 13  \\
          2 & 0 & 1 & 0 & 14 \\
          3 & 0 & 1 & 0 & 15 \\
          4 & 1 & 1 & 1 & 25 \\
          5 & 0 & 1 & 1 & 35 \\
          6 & 1 & 0 & 0 & 49 \\
          7 & 1 & 1 & 1 & 68 \\
          8 & 1 & 0 & 0 & 71 \\
          9 & 1 & 0 & 1 & 73 \\
        \end{bmatrix}
      \]
    \end{column}
    \begin{column}{.5\textwidth}
      Split du premier nœud. Il faut tester 3 splits. Split sur
      jeux vidéos :
      \\[1cm]
      \begin{forest}
        [{1, 2, 3, 4, 5, 6, 7, 8, 9}
          [{1, 2, 3, 4, 5, 7},%
          edge label={node[midway,left,font=\scriptsize]{jeux vidéos}}
            [{$\overline{y}=28,3$} [{$L=92,6$}]]]
          [{6, 8, 9},
          edge label={node[midway,right,font=\scriptsize]{$\neg$ jeux vidéos}}
            [{$\overline{y}=64,3$} [{$L=30,7$}]]]
        ]
      \end{forest}\\
      Loss totale : $123,3$
    \end{column}
  \end{columns}
\end{frame}

\begin{frame}{Exemple — split}
  \begin{columns}
    \begin{column}{.5\textwidth}
      ID, jardinage, jeux vidéos, chapeaux, âge
      \[
        \begin{bmatrix}
          1 & 0 & 1 & 1 & 13  \\
          2 & 0 & 1 & 0 & 14 \\
          3 & 0 & 1 & 0 & 15 \\
          4 & 1 & 1 & 1 & 25 \\
          5 & 0 & 1 & 1 & 35 \\
          6 & 1 & 0 & 0 & 49 \\
          7 & 1 & 1 & 1 & 68 \\
          8 & 1 & 0 & 0 & 71 \\
          9 & 1 & 0 & 1 & 73 \\
        \end{bmatrix}
      \]
    \end{column}
    \begin{column}{.5\textwidth}
      Split du premier nœud. Il faut tester 3 splits. Split sur
      chapeaux :
      \\[1cm]
      \begin{forest}
        [{1, 2, 3, 4, 5, 6, 7, 8, 9}
          [{1, 4, 5, 7, 9},%
          edge label={node[midway,left,font=\scriptsize]{chapeaux}}
            [{$\overline{y}=42,8$} [{$L=110,8$}]]]
          [{2, 3, 6, 8},
          edge label={node[midway,right,font=\scriptsize]{$\neg$ chapeaux}}
            [{$\overline{y}=37,25$} [{$L=91$}]]]
        ]
      \end{forest}\\
      Loss totale : $201,8$
    \end{column}
  \end{columns}
\end{frame}

\begin{frame}{Exemple — split}
  \begin{columns}
    \begin{column}{.5\textwidth}
      ID, jardinage, jeux vidéos, chapeaux, âge
      \[
        \begin{bmatrix}
          1 & 0 & 1 & 1 & 13  \\
          2 & 0 & 1 & 0 & 14 \\
          3 & 0 & 1 & 0 & 15 \\
          4 & 1 & 1 & 1 & 25 \\
          5 & 0 & 1 & 1 & 35 \\
          6 & 1 & 0 & 0 & 49 \\
          7 & 1 & 1 & 1 & 68 \\
          8 & 1 & 0 & 0 & 71 \\
          9 & 1 & 0 & 1 & 73 \\
        \end{bmatrix}
      \]
    \end{column}
    \begin{column}{.5\textwidth}
      \begin{description}
      \item[122,3] jardinage
      \item[123,3] jeux vidéos
      \item[201,8] chapeaux
      \end{description}
      → On split donc sur jardinage
    \end{column}
  \end{columns}
\end{frame}

\begin{frame}{Exemple — split}
  \begin{columns}
    \begin{column}{.5\textwidth}
      ID, jardinage, jeux vidéos, chapeaux, âge
      \[
        \begin{bmatrix}
          1 & 0 & 1 & 1 & 13  \\
          2 & 0 & 1 & 0 & 14 \\
          3 & 0 & 1 & 0 & 15 \\
          4 & 1 & 1 & 1 & 25 \\
          5 & 0 & 1 & 1 & 35 \\
          6 & 1 & 0 & 0 & 49 \\
          7 & 1 & 1 & 1 & 68 \\
          8 & 1 & 0 & 0 & 71 \\
          9 & 1 & 0 & 1 & 73 \\
        \end{bmatrix}
      \]
    \end{column}
    \begin{column}{.5\textwidth}
      Résultat après le premier split :\\[1cm]

      \begin{forest}
        [{1, 2, 3, 4, 5, 6, 7, 8, 9}
          [{4, 6, 7, 8, 9},
          edge label={node[midway,left,font=\scriptsize]{jardinage}}]
          [{1, 2, 3, 5},
          edge label={node[midway,right,font=\scriptsize]{$\neg$ jardinage}}]
        ]
      \end{forest}\\
      \onslide<+->{À vous de jouer !}
    \end{column}
  \end{columns}
\end{frame}

\begin{frame}{Limiter l'overfit}
  Fait par :
  \begin{itemize}
  \item la profondeur maximum
  \item le nombre minimum d'instances dans chaque feuille
  \item une baisse d'entropie maximale à chaque split
  \item le nombre minimum d'instances pour split
  \item le pruning
  \end{itemize}
\end{frame}
