\begin{frame}{Boosted Trees}
    Boosted trees rely on the same concept than random forest to use weak learners to produce a strong learner.

    The main difference being how the weak learners are used.

    While random forests learn trees in parallal, Boosted trees learn trees sequentially.
\end{frame}

\begin{frame}{Boosted Trees -- General concept}
Let $F_n$ be a \textit{strong learner}, build over \textit{weak learners} $\{f_0, f_1,\dots , f_n\}$. Each $f_k$ weak learn, might be any potential algorithm, but it is commonly a decision tree. 

$F_k$ can be define, given an input $x$ as an intermediate strong learner build over the subset of weak learners $\{f_0, f_1, \dots , f_k\}$ as follows:
\begin{align*}
    F_0(x) &= f_0(x)\\ 
    F_k(x) &= F_{k-1}(x) - f_k(x) \quad | 0 < k \leq n\\ 
\end{align*}
\end{frame}

\begin{frame}{Boosted Trees -- Learning}
    So what each learner learns ?
    \begin{itemize}
        \item \textbf{Strong learners} still try to learn a desired output $y$ by minimizing the error from your prediction $$\hat{y}_k = F_k(x)$$
        \item \textbf{First weak learner $f_0$} do the same and try to predict $y$ by minimizing the error from your prediction $$\hat{y}_0 = f_0(x) = F_0(x)$$
        \item \textbf{an other weak learner $f_k$} try to learn the residuals (error) from the previous strong learner $F_{k-1}$. Formally, $f_k$ tries to learn:$$F_{k-1}(x) - y$$. 
    \end{itemize}

\end{frame}

\begin{frame}{Boosted Trees -- Learning}
   Consequently, the final strong learner $F_n$ can be define such that:
   $$F_n(x) = f_0(x) - \sum_{k=1}^{n}f_k(x)$$
\end{frame}

\begin{frame}{Boosted Trees -- Visualization}
    \V{"img/boosted-trees" | image("tw", 1)}
\end{frame}

\begin{frame}{Boosted Trees -- Pros \& Cons}
    Boosted trees has been showned to be extremely efficient structures to learn from data, especially on tabular data.

    But, by contrast with random forest, they have a strong tendancy to overfit, which require to put a lot of works on regularization.
\end{frame}