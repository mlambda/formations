\begin{frame}{Markov Model}
  \V{["markov/simple", "th", 0.8] | image}
\end{frame}



\begin{frame}{Introduction}
  \begin{itemize}[<+->]
    \item Les états représentent les observations
    \item Les transitions représentent les probabilités qu'une observation en suive une autre
  \end{itemize}
\end{frame}

\begin{frame}{Estimation des probabilités}
  Le modèle est estimé par ``comptage'' des transitions d'états

  \begin{enumerate}[<+->]
    \item ATGCGATCTATCGCTAGCCGCGCTATACGCA
    \item GATTATAGCTAGCTCGCGCTATATCGCTAGCTAGCTAGCTAGC
  \end{enumerate}

  \begin{align*}
    \onslide<+->{P(A|T) & = \frac{\#TA}{\#TA+\#TC+\#TG+\#TT} \\}
    \onslide<+->{P(A|C) & = \frac{\#CA}{\#CA+\#CC+\#CG+\#CT}}
  \end{align*}
 \end{frame}

\begin{frame}{Ordre du modèle}
  On peut considérer $n$ éléments passés pour décider de la transition.

  Le modèle sera alors d'ordre $n$.

  On compte dans ce cas les transitions depuis autant d'éléments~:

  \begin{align*}
    P(A|TG) & = \frac{\#TGA}{\#TGA+\#TGC+\#TGG+\#TGT} \\
    P(A|CT) & = \frac{\#CTA}{\#CTA+\#CTC+\#CTG+\#CTT}
  \end{align*}
\end{frame}

\begin{frame}{Utilisations}
  \begin{itemize}
  \item Modèle génératif de séquence
  \item Prédiction de classe : un modèle par classe
  \item Découverte de pattern
  \end{itemize}
  \V{["markov/simple", "th", 0.6] | image}
\end{frame}

\begin{frame}
  \frametitle{Markov Model}
  Des applications où c'est efficace (malgré des limitations évidentes):
  \begin{itemize}
  \item Premières approximations météo
  \item Thermodynamique
  \item Théorie des files d'attente (télécommunication)
  \end{itemize}
  \V{["markov/simple", "th", 0.6] | image}
\end{frame}
