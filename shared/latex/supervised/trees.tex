\begin{frame}{Introduction}
  Modèle de classification ou regréssion qui classe un exemple dans une de ses feuilles pour rendre sa prédiction~:
  \V{"tikz/tree/decision" | image("th", 0.6)}
\end{frame}

\begin{frame}{Avantages}
  Les arbres de décision

  \begin{itemize}
    \item gèrent les inputs numériques comme catégoriels
    \item sont interprétables
    \item sont très rapides durant l'inférence
    \item ne nécessitent pas de normalisation des données
    \item leur apprentissage est hautement parallèlisable
  \end{itemize}

  $\Rightarrow$ Couteau-suisse du machine learning tabulaire.
\end{frame}

\begin{frame}{Désavantages}
  Les arbres de décision

  \begin{itemize}
    \item surapprentissage des données, mais les méthodes d'ensembles résoud ce problème
    \item sont sensibles aux déséquilibres de classe
  \end{itemize}
  $\Rightarrow$ Si les classes ne sont pas équilibrées, peut-être les ré-échantillonner.
\end{frame}

\begin{frame}{Arbres de classification}
  \V{"tikz/tree/decision" | image("th", 0.7)}
\end{frame}

\begin{frame}{Arbres de régression}
  \V{"tikz/tree/regression" | image("th", 0.7)}
\end{frame}

\begin{frame}{Apprendre un arbre de décision}
  Approche descendante, procédure récursive~:

  \begin{itemize}
    \item Créer un nœud de départ qui contient toutes les instances de la base d'entraînement
    \item Tant qu'il reste des nœuds non-traités~:
      \begin{itemize}
        \item Choisir un nœud non traité
        \item Si le nœud remplit des conditions de feuille finale, ne rien faire
        \item Sinon, créer deux branches à partir du nœud non traité pour répartir les instances dans deux nouveaux nœuds
      \end{itemize}
  \end{itemize}

  Conditions de feuilles finales~: contient $n_{min}$ éléments, est déjà à profondeur $p_{max}$, splitterait sans décroître assez l'entropie…
\end{frame}

\begin{frame}{Décision rendue}
  En fonction de la tâche, une fois arrivé dans la feuille de fin~:

  \begin{description}
  \item[Classification] Classe majoritaire
  \item[Régression] Moyenne des valeurs cibles
  \end{description}
\end{frame}

\begin{frame}{Splits possibles}
  Splits possibles d'une caractéristique donnée~:
  \begin{description}
    \item[Catégorielle] Chaque catégorie contre le reste
    \item[Ordinale/Continue] Milieu de chaque paire de valeurs consécutives ou quantiles
  \end{description}
\end{frame}
