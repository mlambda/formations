\begin{frame}{Introduction}
  La classification bayésienne naïve (Naive Bayes) est~:
  \begin{itemize}[<+(1)->]
    \item simple à calculer
    \item simple dans sa formulation
    \item une bonne première baseline
    \item limitée par ses hypothèses fortes
  \end{itemize}
\end{frame}

\begin{frame}{Raisonnement bayésien}
  \begin{itemize}
    \item Formulation d'une hypothèse sans regarder les annotations (probabilité antiérieure)
    \item Prise en compte des données pour mettre à jour l'hypothèse (probabilité postérieure)
  \end{itemize}
\end{frame}

\begin{frame}{Notation}
  Pour la suite, nous utiliserons la notation suivante~:

  \begin{description}[<+->]
    \item[$C$] Variable aléatoire représentant la classe d'un exemple
    \item[$C_k$] Une valeur possible de $C$
    \item[$F$] Variable aléatoire représentant les caractéristiques des exemples
    \item[$F_i$] Valeur des caractéristiques de l'exemple $i$
  \end{description}
\end{frame}

\begin{frame}{Méthode de calcul}
  Utilisation du théorème de Bayes~:
  \[
    P(C = C_k \mid F_i) = \frac{P(C = C_k) \times P(F_i \mid C = C_k)}{P(F_i)}
  \]

  Où~:

  \begin{description}
    \item[$P(C = C_k \mid F_i)$] Probabilité postérieure
    \item[$P(C = C_k)$] Probabilité antérieure
    \item[$P(F_i \mid C = C_k)$] Vraisemblance
    \item[$P(F_i)$] Évidence
  \end{description}
\end{frame}

\begin{frame}{Exemple --- But}
  \def\banane{\text{banane}}%
  \def\orange{\text{orange}}%
  \def\tomate{\text{tomate}}%
  \def\jaune{\text{jaune}}%
  \def\longg{\text{long}}%
  \def\sucre{\text{sucré}}%

  Base de données de fruits (bananes, oranges et tomates).

  Caractéristiques~: \text{couleur}, \text{taille}, \text{sucré}.

  Pour un élément \jaune{}, \longg{} et \sucre{} on cherche le maximum parmi~:
  \begin{itemize}
  \item $P(\banane \mid \jaune, \longg, \sucre)$
  \item $P(\orange \mid \jaune, \longg, \sucre)$
  \item $P(\tomate \mid \jaune, \longg, \sucre)$
  \end{itemize}
\end{frame}

\begin{frame}{Exemple --- Calcul}
  \def\banane{\text{banane}}%
  \def\orange{\text{orange}}%
  \def\tomate{\text{tomate}}%
  \def\jaune{\text{jaune}}%
  \def\longg{\text{long}}%
  \def\sucre{\text{sucré}}%
  Classification naïve~: variables considérées indépendantes, donc~:

  \begin{center}
    $P(\banane \mid \jaune, \longg, \sucre) =$\\
    $\;$\\
    $ \frac{P(\jaune \mid \banane) \times P(\longg \mid \banane) \times P(\sucre \mid \banane) \times P(\banane)}%
    {P(\jaune) \times P(\longg) \times P(\sucre)}$
  \end{center}
  Pour estimer les différentes probabilités, on «~compte~» dans notre base de donnée de fruits :\\
  \begin{center}
    $P(\sucre \mid \banane) = \frac{\card{\banane \land \sucre}}{\card{\banane}}$
  \end{center}
\end{frame}
