\begin{frame}{Un terme ambigu}
  \begin{center}
    \huge{Intelligence}
  \end{center}
\end{frame}

\begin{frame}{Étudions deux des définitions de l'intelligence}
  \epigraph{%
    \begin{enumerate}
      \item Ensemble des fonctions mentales ayant pour objet la connaissance conceptuelle et rationnelle.
      \item Aptitude d'un être humain à s'adapter à une situation, à choisir des moyens d'action en fonction des circonstances.
    \end{enumerate}%
  }{Larousse}

  Qu'en est-il de l'intelligence collective~? Émotionnelle~? Animale~?

  Comment appliquer ces définitions pour écrire des programmes les implémentant ?
\end{frame}

\begin{frame}{Reformulation limitée pour l'intelligence artificielle}
  \begin{center}
    Situation $\Rightarrow$ \green{\underline{décision}} $\Leftarrow$ Optimisation d'une récompense
  \end{center}

  Quelques éléments nécessaires pour y parvenir~:

  \begin{itemize}[<+->]
    \item mémoire
    \item conceptualisation
    \item prévision
    \item curiosité
  \end{itemize}
\end{frame}

\begin{frame}{Avancement}
  Consensus actuel~:

  \begin{itemize}[<+->]
    \item Tous les blocs de base nécessaires existent
    \item Les combiner pour former un programme doté d'une intelligence forte prendra un temps inconnu
  \end{itemize}
\end{frame}

\begin{frame}{Acception courante}
  \emph{Intelligence artificielle} dénote un système qui implémente au moins un des mécanismes nécessaires à une intelligence artificielle forte.
\end{frame}
