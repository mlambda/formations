\documentclass[beamer,crop,tikz]{standalone}

\usepackage{formation}

\begin{document}
  \begin{tikzpicture}
    \node[output,circle] (O) at (4 , 3) {$\sigma(\sum{w_i . x_i} + b)$};
    \node[input] (S) at (7 , 3) {Y};
    \node[input] (X) at (-3, 3) {X};
    \node(I) at (-1.8, 3) {};
    \node[input] (I1) at (0, 6) {$x_1$};
    \node[input] (I2) at (0, 4) {$x_2$};
    \node[input] (I3) at (0, 2) {$x_3$};
    \node[input] (I4) at (0, 0) {$x_4$};
    \node (T) at (2, -1) {où $\sigma$ est une fonction émulant  une activation à seuil};
    \draw[->,>=stealth] (O) to (S);
    \path[->,>=stealth] (I1) edge [bend  left=20] node [pos=0.45,label distance=1.5,below] {$w_1$} (O);
    \path[->,>=stealth] (I2) edge [bend  left=20] node [pos=0.45,label distance=1.5,below] {$w_2$} (O);
    \path[->,>=stealth] (I3) edge [bend right=20] node [pos=0.45,label distance=1.5,above] {$w_3$} (O);
    \path[->,>=stealth] (I4) edge [bend right=20] node [pos=0.45,label distance=1.5,above] {$w_4$} (O);
    \draw[dashed,-] (X) |- (I);
    \draw[dashed,-] (I) |- (I1);
    \draw[dashed,-] (I) |- (I2);
    \draw[dashed,-] (I) |- (I3);
    \draw[dashed,-] (I) |- (I4);
  \end{tikzpicture}
\end{document}
