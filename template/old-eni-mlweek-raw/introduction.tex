\documentclass{formation}
\title{Introduction au machine learning}
\subtitle{Module 1}

\begin{document}

\maketitle

\section{Objectifs}

\begin{frame}
  \frametitle{Objectifs}
  \begin{itemize}
  \item cerner ce qu'est le machine learning
  \item appréhender les différentes facettes du domaine
  \end{itemize}
\end{frame}

\section{Machine Learning}

\begin{frame}
  \frametitle{Machine learning}

  \green{Qu'est-il pour vous ?}
\end{frame}

\begin{frame}
  \frametitle{Machine learning}
  \begin{itemize}
  \item pas de définition exacte
  \item idée transversale: éviter la programmation \textbf{explicite}.
  \item création de programmes qui utilisent des données ou des
    algorithmes généraux pour apprendre à réaliser leurs tâches
  \end{itemize}
\end{frame}

\begin{frame}
  \frametitle{Points de vue}
  Beaucoup de façons de voir le machine learning. Basées sur :
  \begin{itemize}[<+->]
  \item les paradigmes (supervisé, non supervisé, renforcement, en
    ligne, …)
  \item les modèles (arbres, grammaires, automates, réseaux de
    neurones …)
  \item les données (tabulaire, image, texte, vidéo, graphe, …)
  \item les techniques (statistiques, symboliques, probabilistes, …)
  \item les contraintes (real time, embarqué, big data, multilingue,
    …)
  \end{itemize}

  \onslide<+->{→ Domaine \textbf{extrêmement} vaste.}
\end{frame}

\begin{frame}
  \frametitle{Facettes du machine learning — supervised learning}
  \imgtw[.9]{gnmt}
  Demande beaucoup de données, parfois couteuses. Modèles performants
  en sortie.
\end{frame}

\begin{frame}
  \frametitle{Facettes du machine learning — unsupervised learning}
  \imgtw[0.7]{cyclegan}
  Pas besoin d'annotation → données moins couteuses. Limite les
  possibilités des modèles.
\end{frame}

\begin{frame}
  \frametitle{Facettes du machine learning — reinforcement learning}
  \imgtw[0.8]{alphazero} Paradigme de d'acquisition des données
  différent. Modèles potentiellement extrêmement performants.
\end{frame}

\begin{frame}
  \frametitle{Facettes du machine learning — bayesian networks}
  \imgtw[.75]{bayesian}
  Très interprétable, requêtable.
\end{frame}

\begin{frame}
  \frametitle{Facettes du machine learning — decision trees}
  \imgth[0.6]{dietanic} Assez interprétable, robuste, couteau-suisse
  du machine learning tabulaire.
\end{frame}

\begin{frame}
  \frametitle{Calculabilité vs expressivité}

  Un modèle \green{facilement calculable} est souvent \red{peu
    expressif}.

  Inversement, un modèle \red{peu calculable} est souvent
  \green{expressif} (sinon mauvais modèle).
\end{frame}

\begin{frame}
  \frametitle{Choisir la bonne facette}
  Critères pour s'orienter dans les approches de machine learning:
  \begin{itemize}[<+->]
  \item quantité de données à disposition
  \item qualité du signal d'apprentissage dans les données
  \item difficulté du problème à résoudre
  \item besoin d'interprétabilité
  \item contraintes techniques
  \item contraintes de délai
  \item … et d'autres en fonction des domaines métiers
  \end{itemize}
\end{frame}

\begin{frame}
  \frametitle{Conclusion}
  \begin{itemize}
  \item le machine learning est un champ vaste.
  \item il existe sûrement un modèle/paradigme pour vos besoins
  \item l'important est de définir les bons critères
  \end{itemize}
\end{frame}

\begin{frame}
  \frametitle{Discussion}
  \begin{itemize}
  \item à quelles données allez-vous appliquer le machine learning ? À
    quels besoins ?
  \item aurez-vous besoin de modèles interprétables ou simplement très
    performant en prédiction ?
  \item quelles sont vos contraintes ?
  \end{itemize}
\end{frame}

\end{document}
%%% Local Variables:
%%% mode: latex
%%% TeX-master: t
%%% End:
