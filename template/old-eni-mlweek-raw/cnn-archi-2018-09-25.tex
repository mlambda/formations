\documentclass{formation}

\title{Architectures de réseaux de traitement d'images}

\begin{document}

\maketitle

\begin{frame}
  \frametitle{Objectifs}
  \begin{itemize}
  \item Comprendre l'architecture des réseaux les plus efficaces de la
    littérature
  \end{itemize}
\end{frame}

\begin{frame}
  \frametitle{AlexNet}
  \begin{itemize}
  \item Réseau qui a révélé le Deep Learning~\cite{Krizhevsky2012}
  \item Vainqueur d'ILSVRC (ImageNet Large Scale Visual Recognition
    Challenge) en 2012
  \item Taux d'erreur: 25.8\% en 2011 → 16.4\% en 2012
  \item Entraîné sur deux GPUs
  \item Connections entre couches compliquées en conséquence
  \end{itemize}
\end{frame}

\begin{frame}
  \frametitle{AlexNet}
    \begin{figure}
      \centering
      \begin{tabular}{lccccc}
        \toprule
        Type        & Nombre & Taille         & Stride & Padding \\
        \midrule
        Input       & ---    & ---            & ---    & ---     \\
        \rowcolor{green}
        Convolution   & $96$   & $11 \times 11$ & $4$    & $0$     \\
        \rowcolor{red}
        Max Pooling   & ---    & $3 \times 3$   & $2$    &         \\
        \rowcolor{cyan}
        Normalisation & ---    & ---            & ---    & ---     \\
        \rowcolor{green}
        Convolution   & $256$  & $5 \times 5$   & $1$    & $2$     \\
        \rowcolor{red}
        Max Pooling   & ---    & $3 \times 3$   & $2$    &         \\
        \rowcolor{cyan}
        Normalisation & ---    & ---            & ---    & ---     \\
        \rowcolor{green}
        Convolution   & $384$  & $3 \times 3$   & $1$    & $1$     \\
        \rowcolor{green}
        Convolution   & $384$  & $3 \times 3$   & $1$    & $1$     \\
        \rowcolor{green}
        Convolution   & $256$  & $3 \times 3$   & $1$    & $1$     \\
        \rowcolor{red}
        Max Pooling   & ---    & $3 \times 3$   & $2$    &         \\
        \rowcolor{yellow}
        Linear        & 4096   & ---            & ---    & ---     \\
        \rowcolor{yellow}
        Linear        & 4096   & ---            & ---    & ---     \\
        \rowcolor{yellow}
        Linear        & 1000   & ---            & ---    & ---     \\
        \bottomrule
      \end{tabular}
    \end{figure}
\end{frame}

\begin{frame}
  \frametitle{Exercice}
  {\small
    \begin{figure}
      \centering
      \begin{tabular}{lccccc}
        \toprule
        Type        & Nombre & Taille         & Stride & Padding \\
        \midrule
        Input       & ---    & ---            & ---    & ---     \\
        \rowcolor{green}
        Convolution   & $96$   & $11 \times 11$ & $4$    & $0$     \\
        \rowcolor{red}
        Max Pooling   & ---    & $3 \times 3$   & $2$    &         \\
        \rowcolor{cyan}
        Normalisation & ---    & ---            & ---    & ---     \\
        \rowcolor{green}
        Convolution   & $256$  & $5 \times 5$   & $1$    & $2$     \\
        \rowcolor{red}
        Max Pooling   & ---    & $3 \times 3$   & $2$    &         \\
        \rowcolor{cyan}
        Normalisation & ---    & ---            & ---    & ---     \\
        \rowcolor{green}
        Convolution   & $384$  & $3 \times 3$   & $1$    & $1$     \\
        \rowcolor{green}
        Convolution   & $384$  & $3 \times 3$   & $1$    & $1$     \\
        \rowcolor{green}
        Convolution   & $256$  & $3 \times 3$   & $1$    & $1$     \\
        \rowcolor{red}
        Max Pooling   & ---    & $3 \times 3$   & $2$    &         \\
        \rowcolor{yellow}
        Linear        & 4096   & ---            & ---    & ---     \\
        \rowcolor{yellow}
        Linear        & 4096   & ---            & ---    & ---     \\
        \rowcolor{yellow}
        Linear        & 1000   & ---            & ---    & ---     \\
        \bottomrule
      \end{tabular}
    \end{figure}
  }
  Quelles couches contiennent le plus de paramètres ?
\end{frame}

\begin{frame}
  \frametitle{Solution}
  Les couches linéaires. Les couches de convolutions nécessitent peu
  de paramètres mais plus d'opérations par paramètre.
\end{frame}
\begin{frame}
  \frametitle{VGG}
  \begin{itemize}
  \item Vainqueur d'ILSVRC 2014 (localisation)~\cite{Simonyan2015}
  \item Plus profond
  \item Filtres plus petits
  \end{itemize}
  → Calculs plus simples, plus hiérarchisés
\end{frame}

\begin{frame}
  \begin{figure}
    \centering
    \begin{tabular}{lccccc}
      \toprule
      Type            & Nombre & Taille       & Stride & Padding \\
      \midrule
      Input           & ---    & ---          & ---    & ---     \\
      \rowcolor{green}
      Convolution * 2 & $64$   & $3 \times 3$ & $1$    & $1$     \\
      \rowcolor{red}
      Max Pooling     & ---    & $2 \times 2$ & $2$    & $0$     \\
      \rowcolor{green}
      Convolution * 2 & $128$  & $3 \times 3$ & $1$    & $1$     \\
      \rowcolor{red}
      Max Pooling     & ---    & $2 \times 2$ & $2$    & $0$     \\
      \rowcolor{green}
      Convolution * 2 & $256$  & $3 \times 3$ & $1$    & $1$     \\
      \rowcolor{red}
      Max Pooling     & ---    & $2 \times 2$ & $2$    & $0$     \\
      \rowcolor{green}
      Convolution * 3 & $512$  & $3 \times 3$ & $1$    & $1$     \\
      \rowcolor{red}
      Max Pooling     & ---    & $2 \times 2$ & $2$    & $0$     \\
      \rowcolor{green}
      Convolution * 3 & $512$  & $3 \times 3$ & $1$    & $1$     \\
      \rowcolor{red}
      Max Pooling     & ---    & $2 \times 2$ & $2$    & $0$     \\
      \rowcolor{yellow}
      Linear          & 4096   & ---          & ---    & ---     \\
      \rowcolor{yellow}
      Linear          & 4096   & ---          & ---    & ---     \\
      \rowcolor{yellow}
      Linear          & 1000   & ---          & ---    & ---     \\
      \bottomrule
    \end{tabular}
  \end{figure}
\end{frame}

\begin{frame}
  \frametitle{GoogleNet}
  \begin{itemize}
  \item Vainqueur d'ILSVRC 2014 (classification)~\cite{Szegedy2015}
  \item Plus profond
  \item Introduction d'un bloc astucieux
  \end{itemize}
\end{frame}

\begin{frame}
  \frametitle{Bloc Inception}
  \imgth[.6]{logo_eni} (inception-block-basic)
\end{frame}

\begin{frame}
  \frametitle{Exercice}
  \imgth[.6]{logo_eni} (inception-block-basic)
  Quels paddings pour les différents blocs ?
\end{frame}

\begin{frame}
  \frametitle{Solution}
  \imgth[.4]{logo_eni} (inception-block-basic)
  Quels paddings pour les différents blocs ?

  Il faut maintenir des dimensions stables pour pouvoir concaténer:
  \pause
  \begin{description}[<+->]
  \item[$1 \times 1$] → $0$
  \item[$3 \times 3$] → $1$
  \item[$5 \times 5$] → $2$
  \end{description}
\end{frame}

\begin{frame}
  \frametitle{Problème du bloc Inception}
  \imgth[.4]{logo_eni} (inception-block-basic)
  Les profondeurs deviennent prohibitives avec le nombre de couches
\end{frame}

\begin{frame}
  \frametitle{Bloc Inception \og Bottleneck\fg}
  \imgth[.6]{logo_eni} (inception-block)

  Rajouter des convolutions $1 \times 1$ pour contrôler la profondeur.
\end{frame}

\begin{frame}
  \frametitle{Exercice}
  \imgth[.6]{logo_eni} (inception-block)

  Donner un exemple de contrôle de la profondeur par convolutions $1
  \times 1$
\end{frame}

\begin{frame}
  \frametitle{Architecture complète}
  \begin{itemize}
  \item Principalement 9 blocs Inception empilés
    
  \item Un \og petit\fg{} réseau classique pour l'input
  \item 3 \og petits \fg{} réseaux de prédiction aux blocs 3, 6 et 9
  \end{itemize}
\end{frame}

\begin{frame}
  \frametitle{Architecture complète}
  \begin{itemize}
  \item Principalement 9 blocs Inception empilés
    
  \item Un \og petit\fg{} réseau classique pour l'input
  \item 3 \og petits \fg{} réseaux de prédiction aux blocs
    3, 6 et 9 \pause \\
    \green{\textbf{Pourquoi pas seulement en couche finale ?}}
  \end{itemize}
\end{frame}

\begin{frame}
  \frametitle{ResNet}
  \begin{itemize}
  \item Microsoft Research~\cite{He2016}
  \item Vainqueur d'ILSVRC 2015
  \item \textbf{Beaucoup, beaucoup } plus profond (jusqu'à
    \textbf{1000} couches)
  \item Utilisation de connections résiduelles
  \item Très peu de pooling
  \end{itemize}
\end{frame}

\begin{frame}
  \frametitle{Bloc résiduel}
  \imgth[0.6]{logo_eni} (residual-block-basic)

  Intuition: le problème d'optimisation classique est trop dur à
  résoudre quand il y a beaucoup de couches.
\end{frame}

\begin{frame}
  \frametitle{Bloc résiduel}
  \imgth[0.6]{logo_eni} (residual-block)

  Existe aussi avec un \og bottleneck \fg{} pour améliorer les
  performances des réseaux les plus profonds.
\end{frame}

\begin{frame}
  \frametitle{Comparaison des différentes architectures}
  Comparaison précise des temps de calculs, performance dans
  l'étude~\cite{Canziani2016}.
\end{frame}

\begin{frame}[allowframebreaks]
  \bibliography{papers}
\end{frame}
\end{document}
%%% Local Variables:
%%% mode: latex
%%% TeX-master: t
%%% End:
