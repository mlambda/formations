\documentclass{formation}

\title{Apprentissage Automatique}

\begin{document}

\maketitle

\begin{frame}
  \frametitle{Objectifs}
  \begin{itemize}
  \item Comprendre le positionnement du Deep Learning dans le champ
    plus large de l'apprentissage automatique
  \item Savoir quand le Deep Learning n'est pas approprié
  \end{itemize}
\end{frame}

\begin{frame}
  \frametitle{Apprentissage automatique}
  \begin{itemize}
  \item Pas de définition précise (champ très vaste)
  \item Idée transversale: éviter la programmation \textbf{explicite}.
  \item Création de programmes qui utilisent des données ou des
    algorithmes généraux pour apprendre à réaliser leurs tâches
  \end{itemize}
\end{frame}

\begin{frame}
  \frametitle{Deep Learning}
  \begin{itemize}
  \item Désigne l'ensemble des méthodes qui utilisent des réseaux de
    neurones avec de nombreuses couches
  \item Nombreuses couches = hiérarchisation = grande expressivité
  \end{itemize}
\end{frame}

\begin{frame}
  \frametitle{Types d'apprentissage automatique}
  \begin{itemize}
  \item Symbolique (grammaire, automate, …)
  \item Probabiliste (modèle bayésien)
  \item Statistique (SVM, arbre de décision, réseaux de neurones, …)
  \end{itemize}
\end{frame}

\begin{frame}
  \frametitle{Apprentissage automatique symbolique}
  \begin{itemize}
  \item Première branche explorée historiquement
  \item Analyse stricte de symboles
  \item Tolère souvent mal les erreurs
  \item Difficultés à généraliser à l'inconnu
  \end{itemize}
  Exemple: apprentissage de grammaires.
\end{frame}

\begin{frame}
  \frametitle{Apprentissage automatique bayésien}
  \begin{itemize}
  \item Branche popularisée par Judea Pearl
  \item Aussi appelée approche probabiliste
  \item Tolère mieux les erreurs dans les données
  \item Interprétable
  \end{itemize}
  Exemple: modélisation d'un score de crédit par un graphe
  probabiliste.
\end{frame}

\begin{frame}
  \frametitle{Apprentissage automatique statistique}
  \begin{itemize}
  \item Ne cherche pas à modéliser exactement comme l'apprentissage symbolique
  \item Dégage des tendances statistiques
  \item Tolère bien les données bruitées
  \item Peut être interprétable
  \end{itemize}
  Exemple: Classification de documents avec des SVM.
\end{frame}

\begin{frame}
  \frametitle{Apprentissage automatique statistique --- Deep Learning}
  Par rapport à l'apprentissage automatique statistique \og classique
  \fg:
  \begin{itemize}
  \item Nécessite plus de données
  \item Plus dur à entraîner
  \item Moins interprétable
  \item Plus expressif
  \end{itemize}
  Exemple: Reconnaissance vocale.
\end{frame}

\begin{frame}
  \frametitle{Calculabilité vs expressivité}

  Un modèle \green{facilement calculable} est souvent \red{peu
    expressif}.

  Inversement, un modèle \red{peu calculable} est souvent
  \green{expressif}.

  Critères pour savoir si l'on doit choisir un modèle calculable ou
  expressif:
  \begin{itemize}
  \item Quantité de données à disposition
  \item Qualité du signal d'apprentissage dans les données
  \item Difficulté du problème à résoudre
  \end{itemize}
\end{frame}

\begin{frame}
  \frametitle{Tableau récapitulatif}
  {
    \scriptsize
  \begin{tabular}{lcccc}
    \toprule
    Famille       & Expressivité & Calculabilité & Quantité de données
                  & Interprétabilité                                         \\
    \midrule
    Symbolique    & -/+          & ++/-          & -{}-      & +/-           \\
    Probabiliste  & +            & +/-           & +         & ++
                                                                             \\
    Statistique   & ++           & +/-           & -/++      & -/+           \\
    Deep Learning & \green{+++}  & \red{-{}-}    & \red{+++} & \red{-{}-{}-} \\
    \bottomrule
  \end{tabular}
  }
\end{frame}

\begin{frame}
  \frametitle{Conclusion}
  Le Deep Learning rassemble des modèles \green{expressifs},
  \red{difficiles à entraîner}, \red{peu interprétables} et
  nécessitant \red{beaucoup de données}.

  Il faut donc majoritairement l'appliquer à des \textbf{problèmes
    difficiles} où \textbf{beaucoup de données} sont disponibles.
\end{frame}

\begin{frame}
  \frametitle{Un parallèle}
  Les approches Big Data :

  \begin{itemize}
  \item Permettent de traiter des données plus volumineuses
  \item Sont \textbf{beaucoup} plus lourdes à mettre en place
  \end{itemize}
  → Les utiliser avant des volumes de données vraiment importants est
  une perte de temps.
\end{frame}

\begin{frame}
  \frametitle{Discussion}
  \begin{itemize}
  \item Faire l'inventaire de vos problèmes d'apprentissage récents
  \item Les discuter en termes d'expressivité requise et de quantité
    de données
  \end{itemize}
\end{frame}

\end{document}
%%% Local Variables:
%%% mode: latex
%%% TeX-master: t
%%% End:
