\documentclass{formation}

\title{Introduction}

\begin{document}

\maketitle

\begin{frame}{Introduction}
  \begin{multicols}{2}
    Hugo Mougard\\
    + 33 (0)6 37 63 82 71\\
    \href{mailto:hugo@mougard.fr}{\url{hugo@mougard.fr}}\\

    \columnbreak

    \begin{itemize}
    \item R\&D Deep Learning
    \item Formation
    \item Consulting
    \end{itemize}
  \end{multicols}
\end{frame}

\begin{frame}
  \frametitle{Petite remarque sur la langue de Shakespeare}
  Je fais un effort pour traduire un maximum de termes.

  Malgré cela il peut m'arriver d'abuser des anglicismes. N'hésitez
  pas à signaler si cela vous gêne.
\end{frame}

\begin{frame}
  \frametitle{Profiter d'une formation}
  Éviter autant que possible l'apprentissage passif. Y préférer
  l'apprentissage \textbf{actif}:
  \begin{itemize}
  \item Se poser des questions
  \item Manipuler les concepts
  \item Explorer les limites de sa compréhension
  \end{itemize}
  Le plus efficace est de voir cette formation comme une
  \textbf{discussion}.
\end{frame}

\begin{frame}
  \frametitle{Pas vraiment une règle mais…}
  \textbf{La participation avant tout}.

  Je peux vous demander de délaisser vos appareils électroniques si
  cela améliore la qualité de votre apprentissage.
\end{frame}

\begin{frame}
  \frametitle{Plan de bataille}
  Comment apprendre à créer des modèles en Deep Learning ?

  On se concentrera sur 3 aspects:
  \begin{enumerate}
  \item Avoir des fondamentaux solides
  \item Connaître les idées clefs du domaine
  \item Savoir les appliquer
  \end{enumerate}
\end{frame}

\begin{frame}{Programme}
  \textbf{Fondamentaux}
  \begin{itemize}
  \item Apprentissage Automatique
  \item Réseaux de neurones
  \item Modèles à convolutions
  \item Modèles récurrents
  \item Mécanismes d'attention
  \item Modèles à mémoire
  \end{itemize}
  \textbf{Objectifs}
  \begin{itemize}
  \item Maîtriser l'alphabet du Deep Learning
  \item Éclaircir tous les concepts clefs derrière les briques de
    bases utilisées dans les modèles récents
  \end{itemize}
\end{frame}

\begin{frame}{Programme}
  \textbf{Tour des idées clefs du domaine}
  \begin{itemize}
  \item Question/Réponse
  \item Traduction automatique
  \item Génération de son
  \item Apprentissage de programmes
  \item AlphaZero
  \item Transformation d'images
  \end{itemize}
  \textbf{Objectifs}
  \begin{itemize}
  \item Voir quelles idées marchent
  \item Connaître les méthodes principales
  \item Emprunter des idées à d'autres domaines
  \end{itemize}
\end{frame}

\begin{frame}{Programme}
  \textbf{Application des concepts}
  \begin{itemize}
  \item Présentation de PyTorch
  \item Création d'un réseau de génération d'images
  \item Bonnes pratiques
  \end{itemize}
  \textbf{Objectifs}
  \begin{itemize}
  \item Mettre en pratique
  \item Démystifier les modèles complexes
  \end{itemize}
\end{frame}

\end{document}
%%% Local Variables:
%%% mode: latex
%%% TeX-master: t
%%% End:
