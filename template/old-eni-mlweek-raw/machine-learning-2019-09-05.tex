\documentclass{formation}
\title{Intelligence Artificielle}

\begin{document}

\maketitle

\begin{frame}
  \frametitle{Intelligence Artificielle}
  \imgtw{data-science}
\end{frame}

\begin{frame}
  \frametitle{Machine Learning}
  \imgth{ia-ml-deep}
\end{frame}

\begin{frame}
  \frametitle{Machine Learning}
  \begin{minipage}[c]{0.41\linewidth}
    À l'intérieur du \textbf{Modèle}:
    \begin{itemize}
    \item \textbf{Algèbre linéaire}
    \item Théorie de l'Optimisation
    \item Calcul différentiel
    \item Probabilités
    \item Statistiques
    \end{itemize}
  \end{minipage}\hfill
  \begin{minipage}[c]{0.58\linewidth}
    \imgtw[1]{linear-algebra-matrices}
  \end{minipage}\hfill
\end{frame}

\begin{frame}
  \frametitle{Machine Learning}
  \begin{minipage}[c]{0.45\linewidth}
    À l'intérieur du \textbf{Modèle}:
    \begin{itemize}
    \item Algèbre linéaire
    \item \textbf{Théorie de l'Optimisation}
    \item Calcul différentiel
    \item Probabilités
    \item Statistiques
    \end{itemize}
  \end{minipage}\hfill
  \begin{minipage}[c]{0.55\linewidth}
    \imgtw[1]{surface-parametre-erreur1}
  \end{minipage}\hfill
\end{frame}

\begin{frame}
  \frametitle{Machine Learning}
  \begin{minipage}[c]{0.41\linewidth}
    À l'intérieur du \textbf{Modèle}:
    \begin{itemize}
    \item Algèbre linéaire
    \item Théorie de l'Optimisation
    \item \textbf{Calcul différentiel}
    \item Probabilités
    \item Statistiques
    \end{itemize}
  \end{minipage}\hfill
  \begin{minipage}[c]{0.58\linewidth}
    \imgtw[0.7]{derive-tangente}
  \end{minipage}\hfill
\end{frame}

\begin{frame}
  \frametitle{Machine Learning}
  \begin{minipage}[c]{0.41\linewidth}
    À l'intérieur du \textbf{Modèle}:
    \begin{itemize}
    \item Algèbre linéaire
    \item Théorie de l'Optimisation
    \item Calcul différentiel
    \item \textbf{Probabilités}
    \item \textbf{Statistiques}
    \end{itemize}
  \end{minipage}\hfill
  \begin{minipage}[c]{0.58\linewidth}
    \imgtw[1]{gaussian-full}
  \end{minipage}\hfill
\end{frame}

\begin{frame}
  \frametitle{Machine Learning}
  Nouvelle manière d'aborder la \textbf{conception logicielle}.
  \newline
  \newline
  Programmation explicite $\neq$ programmation implicite
\end{frame}

\begin{frame}
  \frametitle{Machine Learning}
  \imgtw[0.8]{ml-craftmanship}
\end{frame}

\begin{frame}
  \frametitle{Machine Learning}
  \underline{Définition du besoin} :
  \newline
  \newline
  Apprentissage \textbf{supervisé}, \textbf{non-supervisé} ou par \textbf{renforcement}?
\end{frame}

\begin{frame}
  \frametitle{Machine Learning}
  \textbf{Apprentissage supervisé}
  \newline \newline
  \textbf{Prédire} une valeur numérique ou l'appartenance à une classe
  \newline
  Données d'entrainement \textbf{annotées} !
  \newline
  Ex : prédire une note sur un film (Netflix)
\end{frame}

\begin{frame}
  \frametitle{Machine Learning}
  \textbf{Apprentissage non-supervisé}
  \newline \newline
  Faire \textbf{émerger} des \textbf{profils}, des \textbf{groupes}
  \newline
  Ex : groupes de clients pour adapter sa stratégie marketing
\end{frame}

\begin{frame}
  \frametitle{Machine Learning}
  \textbf{Apprentissage par Renforcement}
  \newline \newline
  Apprendre une \textbf{stratégie} efficace dans un \textbf{univers} où les \textbf{actions} fournissent des \textbf{récompenses} (possiblement négatives)
  \newline
  Ex : Les échecs, le Go , conduire une voiture,...
\end{frame}

\begin{frame}
  \frametitle{Machine Learning}
  \imgtw[0.8]{ml-illustration}
\end{frame}

\begin{frame}
  \frametitle{Machine Learning}
  AlphaGo Vs Lee sedol
  \imgtw[0.8]{alphago-sedol}
\end{frame}

\end{document}
%%% Local Variables:
%%% mode: latex
%%% TeX-master: t
%%% End:
