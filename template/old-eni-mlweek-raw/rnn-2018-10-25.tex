\documentclass{formation}

\title{Séries temporelles par réseaux de neurones récurrents}

\subtitle{Module 8}

\begin{document}

\maketitle

\begin{frame}
  \frametitle{Objectifs}
  \begin{itemize}
  \item savoir appliquer les réseaux de neurones aux séquences
  \item comprendre les contraintes et avantages des réseaux récurrents
    dans le cadre des séries temporelles
  \end{itemize}
\end{frame}

\begin{frame}
  \frametitle{Problème à résoudre}
  Traiter une séquence d'inputs:
  \begin{itemize}
  \item séquence d'appels dans le cas GFI
  \item séquence de mots dans une phrase
  \item séquence de traces dans un log
  \item séquence d'actions sur une page web
  \item séquence de mesures en géologie / océanographie / météorologie
  \item …
  \end{itemize}
\end{frame}

\begin{frame}
  \frametitle{Première solution: padding}
  \textbf{Idée}
  \begin{itemize}
  \item utilisation d'un réseau standard
  \item taille d'input fixée à la taille maximale des séquences d'input
  \item padding des séquences plus petites avec des inputs neutres
  \end{itemize}
  \textbf{Problèmes}
  \begin{itemize}
  \item très couteux en mémoire et temps de calcul.
  \item l'entraînement des poids n'est pas optimal. \pause
    \green{Pourquoi ?}
  \end{itemize}
\end{frame}

\begin{frame}
  \frametitle{Deuxième solution: fenêtre glissante}
  \textbf{Idée}
  \begin{itemize}
  \item taille d'input fixe
  \item parcours de la séquence en décalant une fenêtre de la taille
    de l'input
  \end{itemize}
  
  \textbf{Problème}\\[.2cm]
  Modélisation des dépendances longues extrêmement simpliste, parfois
  non apprise. \pause \green{Comment peut-on les modéliser, même
    simplement ?}
\end{frame}

\begin{frame}
  \frametitle{Troisième solution: réseaux récurrents}
  \begin{itemize}
  \item taille d'input variable
  \item utilisation d'un même réseau pour chaque élément de l'input,
    séquentiellement
  \end{itemize}
\end{frame}

\begin{frame}
  \frametitle{Réseau récurrent enroulé}
  \imgth[.5]{rolled-rnn}
  Pour l'élement $i$ d'une séquence de $n$ inputs:
  \[
    h_i = \RNN(x_i, h_{i-1})
  \]
\end{frame}

\begin{frame}
  \frametitle{Réseau récurrent déroulé}
  \imgth[.5]{unrolled-rnn}
  Réseau déroulé très proche d'un réseau standard mais:
  \begin{itemize}
  \item poids partagés par les neurones correspondants des différentes
    étapes temporelles.
  \item connections horizontales (calcul pas à pas).
  \end{itemize}
  
\end{frame}

\begin{frame}
  \frametitle{Adaptation de la rétropropagation des gradients}

  \begin{itemize}
  \item aucune adaptation nécessaire. Le modèle standard fonctionne.
  \item seule différence: poids du réseau déroulé modifiés autant de
    fois qu'il y a d'étapes temporelles.
  \item de manière équivalente: poids du réseau déroulé modifiés par
    la somme des modifications des étapes temporelles (règle de
    chainage).
  \end{itemize}

\end{frame}

\begin{frame}
  \frametitle{Où trouver des RNNs implémentés}

  \begin{itemize}
  \item keras (de loin le plus facile)
  \item tensorflow
  \end{itemize}
\end{frame}

\begin{frame}
  \frametitle{Première conclusion}

  Une adaptation simple des réseaux de neurones permet de manipuler
  des séquences d'input

\end{frame}

\begin{frame}
  \frametitle{Problème principal}

  Modèle simple dysfonctionnel en pratique: les gradients ne
  permettent pas l'apprentissage de dépendances longues:

  \imgtw[0.7]{gradient-rnn}

\end{frame}

\begin{frame}
  \frametitle{Solution: LSTM}

  Ajouter un mécanisme de mémoire et de portes pour protéger le flot
  d'information (et de gradient) :

  \imgtw[0.7]{gradient-lstm}

  Approche appelée Long Short-Term Memory Networks.
\end{frame}

\begin{frame}
  \frametitle{Cellule LSTM}
  \imgtw{colah-complete} Des portes sont multipliées à l'input et à
  l'ouput pour limiter leur impact sur le flot d'information au strict
  nécessaire.
\end{frame}

\begin{frame}
  \frametitle{LSTM: étape 1/4}
  Conserver ou non les informations en mémoire:
  \imgtw[0.8]{colah-forget}
  \[
    f_t = \sigma(W_f \cdot [h_{t-1}, x_t] + b_f)
  \]
\end{frame}

\begin{frame}
  \frametitle{LSTM: étape 2/4}
  Prendre en compte ou non l'input:
  \imgtw[0.7]{colah-input}
    \begin{align*}
    i_t & = \sigma\left(W_i \cdot [h_{t-1}, x_t] + b_i\right) \\
    \tilde{C}_t & = \tanh\left(W_C \cdot [h{t-1}, x_t] + b_C\right) \\
    \end{align*}
\end{frame}

\begin{frame}
  \frametitle{LSTM: étape 3/4}
  Mettre à jour l'état caché:
  \imgtw[0.8]{colah-state}
  \[
    C_t = f_t * C_{t-1} + i_t * \tilde{C}_t
  \]
\end{frame}

\begin{frame}
  \frametitle{LSTM: étape 4/4}
  Contrôler si l'on produit ou non une valeur de sortie:
  \imgtw[0.6]{colah-output}
  \begin{align*}
    o_t & = \sigma(W_o\cdot [h_{t-1}, x_t] + b_o) \\
    h_t & = o_t * \tanh(C_t) \\    
  \end{align*}
\end{frame}

\begin{frame}
  \frametitle{Extension: judas}
  
  \imgtw{colah-peepholes}
\end{frame}

\begin{frame}
  \frametitle{Variante: GRU}
  
  \imgtw[0.6]{colah-gru}
  \begin{align*}
    \text{update gate: } z_t         =           & \sigma(W_z\cdot [h_{t-1}, x_t])         \\
    \text{reset gate: }  r_t         =           & \sigma(W_r\cdot [h_{t-1}, x_t])         \\
    \text{input candidat: } \tilde{h}_t =        & \tanh(W \cdot [r_t * h_{t-1}, x_t])     \\
    \text{mise à jour de l'état: } h_t         = & (1 - z_t) * h_{t-1} + z_t * \tilde{h}_t \\
  \end{align*}
\end{frame}

\begin{frame}
  \frametitle{Deuxième conclusion}
  \begin{itemize}
  \item les modèles modernes permettent avec des mécanismes simples de
    gérer les dépendances longues
  \item les séquences sont donc des citoyens de premier ordre dans les
    modèles de machine learning
  \end{itemize}
\end{frame}

\end{document}
%%% Local Variables:
%%% mode: latex
%%% TeX-master: t
%%% End:
