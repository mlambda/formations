\documentclass{formation}
\title{Machine Learning}
\subtitle{Generative Adversarial Network (\textbf{GAN})}

\begin{document}

\maketitle

\begin{frame}
  \frametitle{GAN}
  - Modèle génératif $\neq$ Monte Carlos
  \newline
  - Théorie des jeux : minimax
  \begin{itemize}
  \item $\min_G\max_DV(D,G) = \mathbb{E}_{x\simeq p_{data}(x)}[ln(D(x))] + \mathbb{E}_{z\simeq p_{z}(z)}[ln(1 - D(G(z)))]$
  \item où :
  \item $p_{z}(z)$ est la distribution du bruit et
  \item $p_{data}(x)$ est la distribution des vraies données
  \end{itemize}
  \imgtw[0.8]{gan-schema}
\end{frame}

\begin{frame}
  \frametitle{GAN}
  \href{https://reiinakano.github.io/gan-playground/}{\blue{Démo MNIST}}
\end{frame}

\begin{frame}
  \frametitle{GAN}
  Le générateur n'a jamais vu les vraies données !
  \imgtw[0.8]{gan-schema}
\end{frame}

\end{document}
%%% Local Variables:
%%% mode: latex
%%% TeX-master: t
%%% End:
