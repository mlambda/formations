\documentclass{formation}
\title{Big Data Analytics}
\subtitle{Introduction à Python}

\begin{document}

\maketitle

\begin{frame}
  \frametitle{Introduction à Python}
  Python est créé en 1989 par Guido Van Russum. \\
  En 2001 création de la Python Software Foundation. \\
  Python est sous licence GPL depuis 2001. \\
  Python 3 depuis 2009
  \imgtw[0.8]{python_logo}
\end{frame}

\begin{frame}
  \frametitle{Introduction à Python}
  \begin{itemize}
  \item \underline{Langage \textbf{interprété}}
    \begin{itemize}
    \item Compilé à la volée
    \end{itemize}
  \item \underline{Orienté \textbf{Objet}}
    \begin{itemize}
    \item Paradigme objet (mais pas que)
    \end{itemize}
  \item \underline{\textbf{Portable}}
    \begin{itemize}
    \item Compatible  avec toutes les plateformes actuelles
    \end{itemize}
  \item \underline{Un couteau suisse \textbf{puissant} et \textbf{populaire}}
    \begin{itemize}
    \item À chaque besoin, une librairie
    \item Les librairies importés sont compilées en C/C++
    \end{itemize}
  \end{itemize}
\end{frame}

\begin{frame}
  \frametitle{Introduction à Python}
  \begin{minipage}[l]{0.49\linewidth}
    Atouts
    \begin{itemize}
    \item Stable
    \item multi-plateforme
    \item Facile à apprendre
    \item Grande communauté (le plus utilisé depuis 2019)
    \item un besoin, un module
    \end{itemize}
  \end{minipage}\hfill
  \begin{minipage}[l]{0.49\linewidth}
    Inconvénients
    \begin{itemize}
    \item Non-compilé
      \begin{itemize}
      \item Plus lent qu'un langage bas-niveau
      \item Optimiser une opération $\Rightarrow$ pas facile à apprendre
      \end{itemize}
    \end{itemize}
  \end{minipage}\hfill
\end{frame}

\begin{frame}
  \frametitle{Introduction à Python}
  Différents interpréteurs :
  \begin{itemize}
  \item Python/CPython $\Rightarrow$ C
  \item Jython $\Rightarrow$ Java
  \item IronPython $\Rightarrow$ .Net
  \end{itemize}
\end{frame}

\begin{frame}
  \frametitle{Introduction à Python}
  Domaines d'applications :
  \begin{itemize}
  \item Web (Django ,Flask, ...)
  \item Sciences (Data mining, Machine learning, Physique, ...)
  \item OS (Linux, Raspberry, Script administration système, ...)
  \item Éducation (Initiation à la programmation)
  \item CAO 3D (FreeCAD, pythonCAD, ...)
  \item Multimédia (Kodi, ...)
  \end{itemize}
\end{frame}

\begin{frame}
  \frametitle{Introduction à Python}
  Syntaxe à typage dynamique sans délimiteurs de blocs :
  \inputminted[linenos,fontsize=\small,bgcolor=pythonbg]{python}{code-illustration/python-blocs.py}
  \inputminted[linenos,fontsize=\small,bgcolor=returnbg]{python}{code-illustration/python-blocs.txt}
\end{frame}

\end{document}
%%% Local Variables:
%%% mode: latex
%%% TeX-master: t
%%% End:
