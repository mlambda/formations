\documentclass{formation}
\title{Machine Learning, méthodes et solutions}
\subtitle{Réduction de la dimensionalité : Projections linéaires}

\begin{document}

\maketitle

\begin{frame}
  \frametitle{Réduction de la dimensionalité : Projections linéaires}
    \begin{itemize}
    \item Principal Component Analysis (Non-supervisée)
    \item Linear Discriminant Analysis (Supervisée)
    \end{itemize}
\end{frame}

\begin{frame}
  \frametitle{Réduction de la dimensionalité : PCA}
  \imgth[0.9]{pca-nuage}
\end{frame}

\begin{frame}
  \frametitle{Réduction de la dimensionalité : PCA}
  \imgtw[0.9]{pca-fish}
\end{frame}

\begin{frame}
  \frametitle{Réduction de la dimensionalité : PCA}
  \[
  X = \begin{bmatrix}
    X_{1,1} & \dots  & X_{1,D} \\
    \vdots & \ddots & \vdots \\
    X_{N,1} & \dots  & X_{N,D}
  \end{bmatrix}
  \]
\end{frame}

\begin{frame}
  \frametitle{Réduction de la dimensionalité : PCA}
  chaque dimension est centrée (et réduite):
  \[
  \bar{X} =
  \begin{bmatrix}
    X_{1,1}-\bar{X_1} & \dots  & X_{1,D}-\bar{X_D} \\
    \vdots & \ddots & \vdots \\
    X_{N,1}-\bar{X_1} & \dots  & X_{N,D}-\bar{X_D}
  \end{bmatrix}
  \]
  ou
  \[
  \tilde{X} =
  \begin{bmatrix}
    \frac{X_{1,1}-\bar{X_1}}{\sigma(X_1)} & \dots  & \frac{X_{1,D}-\bar{X_D}}{\sigma(X_D)} \\
    \vdots & \ddots & \vdots \\
    \frac{X_{N,1}-\bar{X_1}}{\sigma(X_1)} & \dots  & \frac{X_{N,D}-\bar{X_D}}{\sigma(X_D)}
  \end{bmatrix}
  \]
\end{frame}

\begin{frame}
  \frametitle{Réduction de la dimensionalité : PCA}
  Matrice de covariance (resp. corrélation) :
  \[
  \frac{1}{N} * \bar{X}^T * \bar{X} \;,\;( \frac{1}{N} * \tilde{X}^T * \tilde{X} )
  \]
  \underline{ACP} : \\
  Retrouver les valeurs et vecteurs propres de de la matrice de covariance (resp. corrélation), donc diagonaliser la matrice carrée obtenue. \\
  \\
  Vecteur propre : vecteur permettant de projeter les données \\
  Valeur propre : ``proportion d'information'' conservée par la projection suivant le vecteur propre correspondant \\
  Réduction de dimension : On ne projette que suivant le nombre de vecteurs propres voulus
\end{frame}

\begin{frame}
  \frametitle{Réduction de la dimensionalité : LDA}
  \begin{center}
    Linear Discriminant Analysis
  \end{center}
  \begin{minipage}[l]{0.49\linewidth}
    \imgtw[0.8]{pca-nuage}
  \end{minipage}\hfill
  \begin{minipage}[l]{0.49\linewidth}
    \imgtw[1.0]{lda}
  \end{minipage}\hfill
\end{frame}

\end{document}
%%% Local Variables:
%%% mode: latex
%%% TeX-master: t
%%% End:
