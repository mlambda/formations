\documentclass{formation}
\title{Machine Learning, méthodes et solutions}
\subtitle{Random Forest}

\begin{document}

\maketitle

\begin{frame}
  \frametitle{Introduction}
  \begin{itemize}
  \item les arbres de décision overfit facilement
  \item ils sont rapides à apprendre
  \item en combiner beaucoup est faisable et réduit la variance
  \end{itemize}
  → création d'une forêt (ensemble d'arbres) aléatoire
\end{frame}

\begin{frame}
  \frametitle{But}
  Produire des arbres décorrélés et moyenner leurs prédictions pour
  réduire la variance.
\end{frame}

\begin{frame}
  \frametitle{Outil 1 — bagging (row sampling)}
  \textbf{B}oostrap \textbf{agg}regat\textbf{ing} (Bagging) :
  \begin{itemize}
  \item tirer un échantillon du dataset avec replacement
  \item entraîner un arbre sur cet échantillon
  \item répéter $B$ fois
  \end{itemize}
  Le bagging s'appelle aussi row sampling.
\end{frame}

\begin{frame}
  \frametitle{Outil 2 — random subspace method (column sampling)}
  \begin{itemize}
  \item à chaque split, considérer seulement un sous-ensemble des
    features
  \item valeurs conseillées :
    \begin{itemize}
    \item classification : $\lvert\sqrt m\rvert$ features par split
    \item regréssion: $\lvert\frac{m}{3}\rvert$ features par split, 5
      exemples par node minimum
    \end{itemize}
  \end{itemize}
\end{frame}

\begin{frame}
  \frametitle{Random Forest}
  \imgtw[1.05]{random-forest-full}
\end{frame}

\begin{frame}
  \frametitle{Random Forest}
  \begin{itemize}
  \item Pas de sur-apprentissage en augmentant le nombre d'arbres
  \item Une fois appris, le modèle est très rapide
  \end{itemize}
\end{frame}

\begin{frame}
  \frametitle{Conclusion}
  \begin{itemize}
  \item les arbres sont interprétables, rapides à entraîner,
    combinables.
  \item random forest combine des arbres faibles en un prédicteur
    versatile
  \end{itemize}
\end{frame}

\end{document}

%%% Local Variables:
%%% mode: latex
%%% TeX-master: t
%%% End:
