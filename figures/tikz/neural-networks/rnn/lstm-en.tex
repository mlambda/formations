\documentclass[beamer,crop,tikz]{standalone}

\usepackage{formation}
\usepackage{tikz}
\usetikzlibrary{positioning, fit, arrows.meta, shapes, decorations.pathreplacing,decorations.markings}

% used to avoid putting the same thing several times...
% Command \empt{var1}{var2}
\newcommand{\empt}[2]{$#1^{\langle #2 \rangle}$}

\begin{document}

\begin{tikzpicture}[
    % GLOBAL CFG
    font=\sf \scriptsize,
    >=LaTeX,
    % Styles
    cell/.style={% For the main box
        rectangle, 
        rounded corners=5mm, 
        draw,
        very thick,
        fill=cyan
        },
    operator/.style={%For operators like +  and  x
        circle,
        draw,
        inner sep=-0.5pt,
        minimum height =.5cm,
        fill=green
        },
    function/.style={%For functions
        ellipse,
        draw,
        inner sep=1pt,
        minimum width=0.8cm,
        minimum height=0.5cm,
        fill=orange,
        },
    mylabel/.style={% something new that I have learned
        font=\scriptsize\sffamily
        },
    ArrowC1/.style={% Arrows with rounded corners
        rounded corners=.25cm,
        thick,
        },
    ArrowC2/.style={% Arrows with big rounded corners
        rounded corners=.5cm,
        thick,
        },
    ]

%Start drawing the thing...    
    % Draw the cell: 
    \node [cell, minimum height =4cm, minimum width=6cm] at (0,0){} ;

    % Draw inputs named ibox#
    \node [function] (ibox1) at (-2.5,-0.75) {$\sigma$};
    \node [function] (ibox2) at (-1.5,-0.75) {$\sigma$};
    \node [function, minimum width=1cm] (ibox3) at (-0.5,-0.75) {Tanh};
    \node [function] (ibox4) at (0.5,-0.75) {$\sigma$};

   % Draw opérators   named mux# , add# and func#
    \node [operator] (mux1) at (-2.5,1.5) {$\times$};
    \node [operator] (add1) at (-0.5,1.5) {$+$};
    \node [operator] (mux2) at (-0.5,0) {$\times$};
    \node [operator] (mux3) at (1.5,0) {$\times$};
    \node [function] (func1) at (1.5,0.75) {Tanh};

    % Draw External inputs? named as basis c,h,x
    \node[hencoder, label={[mylabel]Memory}] (c) at (-4,1.5){$\vec{c}_{t-1}$};
    \node[hencoder, label={[mylabel]Hidden}] (h) at (-4,-1.5)     {$\vec{h}_{t-1}$};
    \node[input] (x) at (-3.2,-3) {$\vec{x}_{t  }$};

    % Draw External outputs? named as basis c2,h2,x2
    \node[hencoder, label={[mylabel]Memory}] (c2) at (4,1.5)      {$\vec{c}_{t}$};
    \node[hencoder, label={[mylabel]Hidden}] (h2) at (4,-1.5)     {$\vec{h}_{t}$};
    \node[output] (x2) at (2.5,3) {$\vec{y}_{t}$};

% Start connecting all.
    %Intersections and displacements are used. 
    % Drawing arrows    
    \draw [->, ArrowC1] (c) -- (mux1) -- (add1) -- (c2);

    % Inputs
    \draw [ArrowC2] (h) -| (ibox4);
    \draw [ArrowC1] (h -| ibox1)++(-0.5,0) -| (ibox1); 
    \draw [ArrowC1] (h -| ibox2)++(-0.5,0) -| (ibox2);
    \draw [ArrowC1] (h -| ibox3)++(-0.5,0) -| (ibox3);
    \draw [ArrowC1] (x) -- (x |- h)-| (ibox3);

    % Internal
    \draw [->, ArrowC2] (ibox1) -- node[right, pos=0.8] {Forget} (mux1);
    \draw [->, ArrowC2] (ibox2) |- node[left, pos=1] {Input    } (mux2);
    \draw [->, ArrowC2] (ibox3) -- (mux2);
    \draw [->, ArrowC2] (ibox4) |-  node[left, pos=1] {Output    } (mux3);
    \draw [->, ArrowC2] (mux2) -- (add1);
    \draw [->, ArrowC1] (add1 -| func1)++(-0.5,0) -| (func1);
    \draw [->, ArrowC2] (func1) -- (mux3);

    %Outputs
    \draw [->, ArrowC2] (mux3) |- (h2);
    \draw (c2 -| x2) ++(0,-0.1) coordinate (i1);
    \draw [-, ArrowC2] (h2 -| x2)++(-0.5,0) -| (i1);
    \draw [->, ArrowC2] (i1)++(0,0.2) -- (x2);

\end{tikzpicture}
\end{document}