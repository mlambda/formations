\documentclass[beamer,crop,tikz]{standalone}

\usepackage{formation}
\usepackage{tikz}
\usetikzlibrary{positioning, fit, arrows.meta, shapes, decorations.pathreplacing,decorations.markings}

% used to avoid putting the same thing several times...
% Command \empt{var1}{var2}
\newcommand{\empt}[2]{$#1^{\langle #2 \rangle}$}

\begin{document}

\begin{tikzpicture}[
    % GLOBAL CFG
    font=\sf \scriptsize,
    >=LaTeX,
    % Styles
    cell/.style={% For the main box
        rectangle, 
        rounded corners=5mm, 
        draw,
        very thick,
        fill=cyan
        },
    operator/.style={%For operators like +  and  x
        circle,
        draw,
        inner sep=-0.5pt,
        minimum height =.5cm,
        fill=green
        },
    function/.style={%For functions
        ellipse,
        draw,
        inner sep=1pt,
        minimum width=0.8cm,
        minimum height=0.5cm,
        fill=orange,
        },
    ct/.style={% For external inputs and outputs
        circle,
        draw,
        line width = .75pt,
        minimum width=1cm,
        inner sep=1pt,
        },
    gt/.style={% For internal inputs
        rectangle,
        draw,
        minimum width=4mm,
        minimum height=3mm,
        inner sep=1pt,
        },
    mylabel/.style={% something new that I have learned
        font=\scriptsize\sffamily
        },
    ArrowC1/.style={% Arrows with rounded corners
        rounded corners=.25cm,
        thick,
        },
    ArrowC2/.style={% Arrows with big rounded corners
        rounded corners=.5cm,
        thick,
        },
    ]

%Start drawing the thing...    
    % Draw the cell: 
    \node [cell, minimum height =4cm, minimum width=6cm] at (0,0){} ;

    % Draw inputs named ibox#
    \node [function] (ibox1) at (-0.5,-0.75) {$\sigma$};
    \node [function] (ibox2) at (0.6,-0.75) {$\sigma$};
    \node [function] (ibox3) at (2.1,-1) {Tanh};

   % Draw opérators   named mux# , add# and func#
    \node [operator] (mux1) at (0.6,1.5) {$\times$};
    \node [operator] (add1) at (2.1,1.5) {$+$};
    \node [operator] (mux2) at (-1.5,0.25) {$\times$};
    \node [operator] (minus1) at (1.2,0.25) {$1-$};
    \node [operator] (mux3) at (2.1,0.25) {$\times$};

    % Draw inputs, hidden, output (x,h,o)
    \node[input] (x) at (-2.5,-3) {$\vec{x}_{t  }$};
    \node[hencoder] (h) at (-4,1.5)     {$\vec{h}_{t-1}$};
    \node[hencoder] (h2) at (4,1.5)     {$\vec{h}_{t}$};
    \node[output] (x2) at (2.6,3) {$\vec{y}_{t}$};

    %Invisible node
    \node (i1) at (-1.5, -1.5) {};

% Start connecting all.
    %Intersections and displacements are used. 
    % Drawing arrows    
    \draw [->, ArrowC1] (h) -- (mux1) -- (add1) -- (h2);
    \draw [->, ArrowC1] (h) -- (-2,1.5) |- (i1) -| (ibox1);
    \draw [->, ArrowC1] (h) -- (-2,1.5) |- (i1) -| (ibox2);
    \draw [->, ArrowC1] (h) -- (-2,1.5) -| (mux2);
    \draw [->, ArrowC1] (x) |- (i1) -| (ibox1);
    \draw [->, ArrowC1] (x) |- (i1) -| (ibox2);
    \draw [->, ArrowC1] (x) |- (0,-1.8) -| (ibox3);
    \draw [->, ArrowC1] (mux2) |- (0,-1.8) -| (ibox3);
    \draw [->, ArrowC2] (ibox1) |- node[right, pos=0.8] {$\;\;R_t$} (mux2);
    \draw [->, ArrowC1] (ibox2) -- node[left, pos=0.43] {$Z_t$} (mux1);
    \draw [->, ArrowC1] (ibox2) |- (minus1) -- (mux3);
    \draw [->, ArrowC1] (ibox3) -- (mux3) -- (add1);
    \draw [->, ArrowC1] (add1) -| (x2);


\end{tikzpicture}
\end{document}