\documentclass[beamer,crop,tikz]{standalone}

\usepackage{formation}
\usepackage{tikz}
\usetikzlibrary{positioning, fit, arrows.meta, shapes, decorations.pathreplacing,decorations.markings}

% used to avoid putting the same thing several times...
% Command \empt{var1}{var2}
\newcommand{\empt}[2]{$#1^{\langle #2 \rangle}$}

\begin{document}

\begin{tikzpicture}
%Start drawing the thing...    
    % Draw the cell: 
    %\node [rectangle, rounded corners=5mm, draw, very thick, minimum height =4cm, minimum width=9cm] at (1,0){} ;

    % Draw input
    \node[input] (x1) at (1,0) {$\vec{x}_{1}$};
    \node[input] (x2) at (5,0) {$\vec{x}_{2}$};
    \node[input] (x3) at (9,0) {$\vec{x}_{3}$};
    
    \node[feedforward] (E11) at ( 0,1) {$Q$};
    \node[feedforward] (E12) at ( 4,1) {$Q$};
    \node[feedforward] (E13) at ( 8,1) {$Q$};
    \node[feedforward] (E21) at ( 1,1) {$K$};
    \node[feedforward] (E22) at ( 5,1) {$K$};
    \node[feedforward] (E23) at ( 9,1) {$K$};
    \node[feedforward] (E31) at ( 2,1) {$V$};
    \node[feedforward] (E32) at ( 6,1) {$V$};
    \node[feedforward] (E33) at (10,1) {$V$};
    

    \node[hencoder] (q1) at ( 0,2) {$\vec{q}_{1}$};
    \node[hencoder] (q2) at ( 4,2) {$\vec{q}_{2}$};
    \node[hencoder] (q3) at ( 8,2) {$\vec{q}_{3}$};
    \node[hencoder] (k1) at ( 1,2) {$\vec{k}_{1}$};
    \node[hencoder] (k2) at ( 5,2) {$\vec{k}_{2}$};
    \node[hencoder] (k3) at ( 9,2) {$\vec{k}_{3}$};
    \node[hencoder] (v1) at ( 2,2) {$\vec{v}_{1}$};
    \node[hencoder] (v2) at ( 6,2) {$\vec{v}_{2}$};
    \node[hencoder] (v3) at (10,2) {$\vec{v}_{3}$};
    
    \node[self-attention,minimum width=12cm] (self-att) at (5,3.5) {auto-attention};
    % \node[feedforward,minimum width=7cm] (feedforward) at (1,1.25) {couches denses};
    % 
    % % Draw Value vectors
    % \node[hencoder] (v1) at (-2,0) {$\vec{v}_{1}$};
    % \node[hencoder] (v2) at (1, 0) {$\vec{v}_{2}$};
    % \node[hencoder] (v3) at (4, 0) {$\vec{v}_{3}$};
% 
    % % Draw returned vectors
    \node[output] (r1) at (1,5) {$\vec{r}_{1}$};
    \node[output] (r2) at (5,5) {$\vec{r}_{2}$};
    \node[output] (r3) at (9,5) {$\vec{r}_{3}$};

% Start connecting all.
    \foreach \y in {1,...,3} {
        \draw[->,thick,>=stealth] (x1) to (E\y1);    
    }
    \foreach \y in {1,...,3} {
        \draw[->,thick,>=stealth] (x2) to (E\y2);    
    }
    \foreach \y in {1,...,3} {
        \draw[->,thick,>=stealth] (x3) to (E\y3);    
    }

    \foreach \y in {1,...,3} {
        \draw[->,thick,>=stealth] (E1\y) to (q\y);    
    }
    \foreach \y in {1,...,3} {
        \draw[->,thick,>=stealth] (E2\y) to (k\y);    
    }
    \foreach \y in {1,...,3} {
        \draw[->,thick,>=stealth] (E3\y) to (v\y);    
        }
        
    \foreach \y in {1,...,3} {
        \draw[<-,thick,>=stealth] (self-att.south) to[*|] (q\y.north);    
    }
    \foreach \y in {1,...,3} {
        \draw[<-,thick,>=stealth] (self-att.south) to[*|] (k\y.north);    
    }
    \foreach \y in {1,...,3} {
        \draw[<-,thick,>=stealth] (self-att.south) to[*|] (v\y.north);    
    }

    \foreach \y in {1,...,3} {
        \draw[->,thick,>=stealth] (self-att.north) to[*|] (r\y.south);    
    }

%Intersections and displacements are used. 
    % Drawing arrows    
    % \draw [->, ArrowC1,>=stealth] (p1) |- (s1);
    % \draw [->, ArrowC1,>=stealth] (p2) |- (s2);
    % \draw [->, ArrowC1,>=stealth] (p3) |- (s3);
    % \draw [->, ArrowC1,>=stealth] (x1) to (s1);
    % \draw [->, ArrowC1,>=stealth] (x2) to (s2);
    % \draw [->, ArrowC1,>=stealth] (x3) to (s3);
    % \foreach \y in {1,...,3} {
    %     \draw[<-,thick,>=stealth] (feedforward.south) to[*|] (v\y.north);    
    % }
    % \foreach \y in {1,...,3} {
    %     \draw[->,thick,>=stealth] (feedforward.north) to[*|] (r\y.south);    
    % }

\end{tikzpicture}
\end{document}