\documentclass{formation}
\title{Big Data Analytics}
\subtitle{Optimisation}

\begin{document}

\maketitle

\begin{frame}
  \frametitle{Optimisation}
  Calcul du gradient de l'erreur par rapport aux paramètres:
  \[
  \frac{\partial{Err}}{\partial{w_i}}
  \]
  Mise à jour :
  \[
  w_i' = w_i - \gamma * grad 
  \]
  où : $0 < \gamma < 1$ (learning rate)
  \imgtw[0.6]{gradient}
\end{frame}

\begin{frame}
  \frametitle{Optimisation}
  1 - initialisation aléatoire du modèle
  \newline
  2 - Tant que(critère arret == 0)
  \begin{itemize}
  \item Selection aléatoire d'un \textbf{batch} de données
  \item \textbf{Forward} : Passe avant du \textbf{batch} dans le modèle
  \item Calcul de l'erreur par rapport aux sorties attendues
  \item \textbf{Backward} : Rétropropagation du gradient de l'erreur en fonction des paramèrtres dans le modèle (mise à jour du modèle)
  \item Calcul critère arret
  \end{itemize}
\end{frame}

\end{document}
%%% Local Variables:
%%% mode: latex
%%% TeX-master: t
%%% End:
