\documentclass{formation}
\title{Big Data Analytics}
\subtitle{Introduction à la Modélisation de Données}

\begin{document}

\maketitle

\begin{frame}
  \frametitle{Votre formateur}
  \begin{description}
  \item[Nom] Giraud François-Marie
  \item[Courriel] giraud.francois@gmail.com
  \item[Activité] Consultant/Formateur indépendant
  \item[Spécialité] Intelligence Artificielle
  \item[Parcours] Master Intelligence Artificielle et Décision (Paris 6)
  \end{description}
\end{frame}

\begin{frame}
  \frametitle{Votre formation — description}
  Cette formation présente les \textbf{fondamentaux} de la \textbf{Modélisation Statistique} à travers des travaux pratiques. \\
\end{frame}

\begin{frame}
  \frametitle{Votre formation — connaissances préalables}
  \begin{itemize}
  \item Connaissances de base en statistiques et algèbre
  \item Connaissances de base en python
  \item Avoir un \textbf{compte Google} afin de pouvoir faire les TPs dans \blue{\underline{\href{https://colab.research.google.com/notebooks/welcome.ipynb}{Google colaboratory}}}
  \end{itemize}
\end{frame}

\begin{frame}
  \frametitle{Votre formation — objectifs à atteindre}
  \begin{itemize}
  \item Comprendre les principes de la modélisation statistique
  \item Comprendre les différents types de régressions
  \item Évaluer les performances d'un algorithme prédictif
  \item Sélectionner et classer des données dans de grands volumes de données
  \item Se familiariser avec les librairies scientifiques python (NumPy, seaborn, sikit-learn, ... )
  \end{itemize}
  En bref, se familiariser avec les concepts et techniques de bases du ``Machine Learning''
\end{frame}

\begin{frame}
  \frametitle{Votre formation — programme}
  \begin{itemize}
  \item 4 jours de 9h à 12h30 et de 14h à 17h30
  \item Le dernier jour on finit à 15h30. 
  \item À 15h on commence à remplir les documents administratifs.
  \end{itemize}
\end{frame}

\begin{frame}
  \frametitle{Votre formation — programme}
  \begin{itemize}
  \item Introduction à la modélisation
  \item Evaluation de modèles prédictifs
  \item Les algorithmes supervisés/non-supervisés
  \item Projection de données par composantes
  \item Analyse de données textuelles
  \end{itemize}
\end{frame}

\begin{frame}
  \frametitle{Votre formation — ressources}
  \begin{itemize}
  \item Je vous fourni ressources utilisées à chaque début de cours.
  \item Elles sont aussi accessibles via \blue{\underline{\href{https://myorsys.orsys.fr/}{https://myorsys.orsys.fr/}}}
  \end{itemize}
\end{frame}

\begin{frame}
  \frametitle{Tour de table — présentez-vous}
  \begin{itemize}
  \item Votre nom
  \item Votre métier
  \item Vos compétences dans les domaines liés à cette formation
  \item Vos objectifs et vos attentes vis-à-vis de cette formation
  \end{itemize}
\end{frame}

\end{document}
%%% Local Variables:
%%% mode: latex
%%% TeX-master: t
%%% End:
