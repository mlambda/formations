\begin{frame}
  \frametitle{Rappels - Probabilités}
  \begin{minipage}[c]{0.60\linewidth}
    \underline{Rappels} :
    \begin{itemize}
    \item Une variable aléatoire \\ $A\in\mathbb{R}$
    \item Probabilité \\ $0 \leq P(A\in[a_1\;a_2]) \leq 1$
    \item Probabilité conditionnelle \\ $P(\;A>0\;|\;B<-3\;)$
    \item Évènements indépendants \\ $P(A|B)=P(A)$ et $P(B|A)=P(B)$
    \item Probabilité jointe \\ $P(A,B)=P(B|A)*P(A)$ \\ $P(A,B)=P(A|B)*P(B)$
    \item A et B Indépendants \\ $\iff P(A,B) = P(A)*P(B)$ 
    \end{itemize}
  \end{minipage}\hfill
  \begin{minipage}[c]{0.39\linewidth}
    \begin{center}
      \underline{Théorème de Bayes}
      \[
      \boxed{P(A|B)=\frac{P(B|A).P(A)}{P(B)}}
      \]
    \end{center}
  \end{minipage}\hfill
\end{frame}

\begin{frame}
  \frametitle{Naive Bayes}
  Prenons un exemple : \\
  Soit une base de donnée de fruits contenant uniquement des bananes ,oranges ,courgettes. \\
  Chaque élément possède des caractéristiques couleur, taille, sucré. \\
  Appliquer Naive Bayes, c'est chercher le maximum de vraisemblance d'un élèments dont on ne connait pas la nature mais dont on connait les caractéristiques. \\
  On cherche donc quelle est la plus grande probabilité :
  \begin{itemize}
  \item P(banane | jaune,long,sucré)
  \item P(orange | jaune,long,sucré)
  \item P(tomate | jaune,long,sucré)
  \end{itemize}
\end{frame}

\begin{frame}
  \frametitle{Naive Bayes}
  Naive = toutes les variables sont considérée indépendantes, donc : \\
  \begin{center}
    $P(banane | jaune,long,sucre) =$\\
    $\;$\\
    $ \frac{P(jaune|banane)*P(long|banane)*P(sucre|banane)*P(banane)}{P(jaune)*P(long)*P(sucre)}$
  \end{center}
  Pour estimer les différentes probabilités, on 'compte' dans notre base de donnée de fruits :\\
  \begin{center}
    $P(sucre|banane) = \frac{card(banane\;ET\;sucre)}{card(banane)}$
  \end{center}
\end{frame}
