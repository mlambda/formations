\subsection{Mettre en place un transition IA}
\subsection{Développer un projet en Machine Learning}

\begin{frame}
  \frametitle{Développer un projet en Machine Learning}
  \imgtw[0.8]{ml-craftmanship}
\end{frame}

\begin{frame}
  \frametitle{Développer un projet en Machine Learning}
  \begin{itemize}
  \item \underline{Séparer} les données en TRAIN/VALIDATION/TEST (i.e 60/20/20)
  \item \underline{Apprendre} sur \textbf{TRAIN}
  \item Optimizer les \underline{hyperparamètres} sur \textbf{VALIDATION}
  \item Observer la \underline{performance} finale sur \textbf{TEST}
  \end{itemize}
\end{frame}

\begin{frame}
  \frametitle{Développer un projet en Machine Learning}
  \imgtw[0.8]{ml-effort-vs-value}
\end{frame}

\begin{frame}
  \frametitle{Développer un projet en Machine Learning}
  Projet académiques Vs Industriels
  \begin{itemize}
  \item $\neq$ Développement logiciel
  \item $\neq$ Infrastructure
  \item $\neq$ Performances
  \end{itemize}
\end{frame}

\begin{frame}
  \frametitle{Développer un projet en Machine Learning}
  \begin{center}
    Développement logiciel
  \end{center}
  \begin{minipage}[c]{0.49\linewidth}
    \begin{center}
      Académique
      \newline
    \end{center}
  \end{minipage}\hfill
  \begin{minipage}[c]{0.49\linewidth}
    \begin{center}
      Industriel
      \newline
    \end{center}
  \end{minipage}\hfill
  \begin{minipage}[c]{0.49\linewidth}
    \begin{itemize}
    \item Pile de scripts
    \item Peu de documentation
    \item Fonctionne le temps de l'expérience
      \item ``Fair use''
    \end{itemize}
  \end{minipage}\hfill
  \vrule{}
  \begin{minipage}[c]{0.49\linewidth}
    \begin{itemize}
    \item Code hiérarchisé et déployable en production
    \item Documentation
    \item Code maintenable et robuste
    \item Galaxies de licences à respecter
    \end{itemize}
  \end{minipage}\hfill
\end{frame}

\begin{frame}
  \frametitle{Développer un projet en Machine Learning}
  \begin{center}
    Infrastructure
  \end{center}
  \begin{minipage}[c]{0.49\linewidth}
    \begin{center}
      Académique
      \newline
    \end{center}
  \end{minipage}\hfill
  \begin{minipage}[c]{0.49\linewidth}
    \begin{center}
      Industriel
      \newline
    \end{center}
  \end{minipage}\hfill
  \begin{minipage}[c]{0.49\linewidth}
    \begin{itemize}
    \item Données = un fichier
    \item Hardware limité
    \item Performance = Précision
    \end{itemize}
  \end{minipage}\hfill
  \vrule{}
  \begin{minipage}[c]{0.49\linewidth}
    \begin{itemize}
    \item Données = cloud
    \item Cloud computing
    \item Performance = Plus-value
    \end{itemize}
  \end{minipage}\hfill
\end{frame}

\begin{frame}
  \frametitle{Développer un projet en Machine Learning}
  Un problème d'ingénérie avant d'être un problème de machine learning : \\
  \begin{center}
    \textbf{Données et prétraitements de qualité > algorithme de qualité}
  \end{center}
\end{frame}

\begin{frame}
  \frametitle{Développer un projet en Machine Learning}
  Une approche en 4 étapes :
  \begin{itemize}
  \item Créer un pipeline robuste de bout en bout (sans ML)
  \item Intégrer du ML simple
  \item Ajouter des caractéristiques sensées
  \item Conserver un pipeline robuste
  \end{itemize}
\end{frame}

\begin{frame}
  \frametitle{Développer un projet en Machine Learning}
  Créer un pipeline robuste de bout en bout (sans ML) :
  \begin{itemize}
  \item Une baseline avec une heuristique
  \item Mettre en place des statistiques d'évaluation
  \end{itemize}
\end{frame}

\begin{frame}
  \frametitle{Développer un projet en Machine Learning}
  Intégrer du ML simple :
  \begin{itemize}
  \item Obtenir des données
  \item Définir UNE métrique d'évaluation facile à observer
  \item Définir des caractéristiques sensées et faciles à obtenir
  \item Considérer les heuristiques comme des caractéristiques
  \item Documenter TOUTES les caractéristiques utilisées
  \end{itemize}
\end{frame}

\begin{frame}
  \frametitle{Développer un projet en Machine Learning}
  Intégrer du ML simple :
  \begin{itemize}
  \item Apprendre un modèle tous les n-jours
  \item Évaluer la dégradation des performances en fonction de l'âge du modèle
  \item Vérifier les performances en test avant de déployer en production
  \item Modèle appris sur des données jusqu'au jour N, tester sur les données après le jour N
  \item Mesurer la différence entre performance en apprentissage et test
  \item Plateau de performance $\Rightarrow$ trouver des nouvelles caractéristiques/augmenter la puissance du modèle
  \item Supprimer des caractéristiques pas déterminantes
  \end{itemize}
\end{frame}

\begin{frame}
  \frametitle{Développer un projet en Machine Learning}
  Ajouter des caractérstiques sensées :
  \begin{itemize}
  \item Beaucoup de caractéristiques simples > peu de caractéristiques complexes
  \item Des caractéristiques répandues plutôt que rares
  \item Regarder les erreurs pour imaginer les caractéristiques qui aideraient
  \item Communiquer avec les experts métiers
  \end{itemize}
\end{frame}

\begin{frame}
  \frametitle{Développer un projet en Machine Learning}
  Des questions à garder en tête :
  \begin{itemize}
  \item Ajouter des statistiques d'évaluation ?
  \item Revoir/Complexifier la métrique d'évaluation ?
  \item Les données sont-elle ``stables'' ?
  \end{itemize}
\end{frame}
