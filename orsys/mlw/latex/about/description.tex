
\begin{frame}{Description}
  \begin{description}
    \item[Devops] Pratique de développement logiciel continu pour déployer avec efficacité et fiabilité les nouveautés
    \item[Machine Learning] Création et maintient des modèles pour améliorer l'avenir
    \item[Association des deux] \alert{MLOps} pour gérer le cycle de vie des projets de data science, s'appuyant sur la conteneurisation
  \end{description}
\end{frame}

\begin{frame}{Prérequis}
  \begin{itemize}
    \item Bonne pratique du langage Python
    \item Connaissances en Machine learning / Deep learning
    \item Utilisation de Docker
    \item Avoir un \textbf{compte Google} afin de pouvoir faire les TPs dans \bluelink{https://colab.research.google.com/notebooks/welcome.ipynb}{Google Colaboratory}
  \end{itemize}
\end{frame}

\begin{frame}{Objectifs pédagogiques}
  \begin{itemize}
    \item Connaître les différentes étapes de vie du modèle et de la donnée après le POC
    \item Connaître les méthodes de réduction de dimensions d'un modèle pour le passage à l'échelle
    \item Connaître les différentes plateformes de production
    \item Savoir mettre en place des algorithmes d'explicabilité d'un modèle
    \item Avoir des notions sur l'embarquabilité
    \item Avoir des notions sur l'entrainement de larges modèles de façon distribuée
  \end{itemize}
\end{frame}

\begin{frame}{Programme}
  \begin{itemize}
    \item La vie après le PoC (Proof of Concept)
    \item Les étapes de mise en production de modèles de Deep Learning
    \item Intégration de Docker et Kubernetes
  \end{itemize}
\end{frame}
