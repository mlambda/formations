\documentclass{formation}
\title{Machine Learning}
\subtitle{Sources de données}

\begin{document}

\maketitle

\begin{frame}
  \frametitle{Sources de données}
  \imgtw[1.0]{trends-bigdata-ml}
\end{frame}

\begin{frame}
  \frametitle{Sources de données}
  Dans un premier temps, les données internes de l'entreprise
  \imgtw[0.9]{datalake}
\end{frame}

\begin{frame}
  \frametitle{Sources de données}
  Entrepot de données sur les clients (Datawarehouse)
  \imgtw[0.9]{data-warehouse}
\end{frame}

\begin{frame}
  \frametitle{Sources de données}
  Processus de gestion (ERP)
  \imgtw[0.9]{erp}
\end{frame}

\begin{frame}
  \frametitle{Sources de données}
  Forme des données :
  \begin{itemize}
  \item logs d'activité (tabulaire)
  \item IOT (capteurs)
  \item Réseaux sociaux
  \item Open-data (data-gouv.fr, geonames.org, ...)
  \item Web crawling 
  \end{itemize}

\end{frame}

\begin{frame}
  \frametitle{Sources de données}
  Bases de données sur les clients dsponibles :
  \begin{itemize}
  \item Institut de sondage
  \item Réseaux sociaux
  \item Profils clients
  \item ...
  \end{itemize}
\end{frame}
    
\begin{frame}
  \frametitle{Sources de données}
  Des places de marché en ligne de données :
  \begin{itemize}
  \item Microsoft Azure Marketplace
  \item Datamarket
  \item Data Publica
  \item ...
  \end{itemize}
\end{frame}

\end{document}
%%% Local Variables:
%%% mode: latex
%%% TeX-master: t
%%% End:
