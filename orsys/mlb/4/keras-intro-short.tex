\documentclass{formation}
\title{Machine Learning, méthodes et solutions}
\subtitle{Introduction à Keras}

\begin{document}

\maketitle

\begin{frame}
  \frametitle{Introduction à Keras}
  Keras est une API de haut niveau qui permet de prototyper des réseaux de neurones de toute sorte.
  \imgtw[0.5]{keras-tensorflow-logo}
\end{frame}

\begin{frame}
  \frametitle{Introduction à Keras}
  User-friendly :
  \begin{itemize}
  \item Interface simple
  \item Accès facilité aux métriques d'évaluation
  \end{itemize}
\end{frame}

\begin{frame}
  \frametitle{Introduction à Keras}
  Modulaire :
  \begin{itemize}
  \item Les réseaux se 'branchent' facilement les uns avec les autres
  \item Tous les réseaux se configurent facilement
  \end{itemize}
\end{frame}

\begin{frame}
  \frametitle{Introduction à Keras}
  Etat de l'art :
  \begin{itemize}
  \item Les modèles et optimiseurs pertinents sont rapidement ajouté à Keras
  \item Reproduction facile de résultats récents
  \end{itemize}
\end{frame}

\begin{frame}
  \frametitle{Introduction à Keras}
  Facile de dévellopper de nouvelles :
  \begin{itemize}
  \item Couche de réseau (Layers)
  \item Fonction de perte (Loss)
  \item Métriques d'évaluation
  \end{itemize}
\end{frame}

\begin{frame}
  \frametitle{Réseau de neurones en Keras}
  \inputminted[linenos,fontsize=\small,bgcolor=pythonbg]{python}{code-illustration/tf-keras-mlp.py}
\end{frame}

\begin{frame}
  \frametitle{Affichage du modèle}
  \inputminted[linenos,fontsize=\small,bgcolor=pythonbg]{python}{code-illustration/tf-keras-print-model.py}
  \inputminted[linenos,fontsize=\small,bgcolor=returnbg]{python}{code-illustration/tf-keras-print-model.txt}
\end{frame}

\begin{frame}
  \frametitle{Apprentissage}
  \inputminted[linenos,fontsize=\small,bgcolor=pythonbg]{python}{code-illustration/tf-keras-learn-1.py}
\end{frame}

\end{document}
%%% Local Variables:
%%% mode: latex
%%% TeX-master: t
%%% End:
