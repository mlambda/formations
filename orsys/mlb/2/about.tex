\documentclass{formation}
\title{Machine Learning}
\subtitle{État de l'art}

\begin{document}

\maketitle

\begin{frame}
  \frametitle{Votre formateur}
  \begin{description}
  \item[Nom] Giraud François-Marie
  \item[Courriel] giraud.francois@gmail.com
  \item[Activité] Formateur / Consultant indépendant
  \item[Spécialité] Intelligence Artificielle
  \item[Parcours] Master Intelligence Artificielle et Décision (Paris 6)
  \end{description}
\end{frame}

\begin{frame}
  \frametitle{Votre formation — description}
  Cette formation présente les \textbf{fondamentaux du machine learning} ainsi que les principales \textbf{techniques utilisées dans l’industrie}.
\end{frame}

\begin{frame}
  \frametitle{Votre formation — profil des stagiaires}
  Toute personne intéressée par l'Intelligence Artificielle souhaitant obtenir une vision claire de ses capacités et de son impact sur les systèmes existants.
\end{frame}

\begin{frame}
  \frametitle{Votre formation — connaissances préalables}
  Bonne culture en informatique. \\
  Algèbre et Satatistique si vous voulez comprendre les formules mais ce n'est pas rédhibitoire
\end{frame}

\begin{frame}
  \frametitle{Votre formation — objectifs à atteindre}
  \begin{itemize}
  \item Comprendre les possibilités offertes par le machine learning (et ses contraintes)
  \item Cerner les acteurs du domaine
  \item Apercevoir les tendances en cours et à venir
  \item Savoir intégrer le machine learning dans une entreprise
  \end{itemize}
\end{frame}

\begin{frame}
  \frametitle{Votre formation — programme}
  \begin{itemize}
  \item Les Données et le Big Data
  \item Introduction et Histoire du Machine Learning
  \item Les Algorithmes du Machine Learning et leurs évaluations
  \item Mise en production et description des outils existants
  \item Aspect Éthiques et Juridiques
  \item Transformations récentes de secteurs économiques, tendances
  \end{itemize}
\end{frame}

\begin{frame}
  \frametitle{Votre formation — ressources}
  Je vous ferai parvenir les ressources utilisées pendant ce cours en fin de formation.
  Si je n'arrive pas à répondre à une question pendant la formation, je m'éfforcerais d'y répondre par mail.
\end{frame}

\begin{frame}
  \frametitle{Tour de table}
  \begin{itemize}
  \item Votre nom
  \item Votre métier
  \item Votre société client si appliquable
  \item Vos compétences dans les domaines liés à cette formation
  \item Vos objectifs et vos attentes vis-à-vis de cette formation
  \end{itemize}
\end{frame}

\begin{frame}
  \frametitle{Pour le bon déroulement de la formation}
  Merci d'éteindre vos téléphones portables.
\end{frame}

\begin{frame}
  \frametitle{Préambule à l'intelligence artificielle}
  \begin{center}
    \huge{\red{INTELLIGENCE ?}}
  \end{center}
\end{frame}

\begin{frame}
  \frametitle{Préambule à l'intelligence artificielle}
  \imgtw[0.2]{larousse}
  \underline{\textbf{Intelligence}} :
  \begin{itemize}
  \item Ensemble des fonctions mentales ayant pour objet la connaissance conceptuelle et rationnelle
  \item Aptitude d'un être humain à s'adapter à une situation, à choisir des moyens d'action en fonction des circonstances
  \end{itemize}
\end{frame}

\begin{frame}
  \frametitle{Préambule à l'intelligence artificielle}
  \imgtw[0.2]{larousse}
  \underline{\textbf{Intelligence}} :
  \begin{itemize}
  \item Ensemble des fonctions mentales ayant pour objet la connaissance conceptuelle et rationnelle
  \item Aptitude d'un être humain à s'adapter à une situation, à choisir des moyens d'action en fonction des circonstances
  \end{itemize}
  \begin{center}
    \red{intelligence animale ? intelligence collective ?}
  \end{center}
\end{frame}

\begin{frame}
  \frametitle{Préambule à l'intelligence artificielle}
  \underline{\textbf{Intelligence}} :
  \newline
  \newline
  \begin{center}
    Situation $\Rightarrow$ \green{\underline{choix action}} $\Leftarrow$ optimisation récompense
  \end{center}
\end{frame}

\begin{frame}
  \frametitle{Préambule à l'intelligence artificielle}
  \underline{\textbf{Intelligence}} :
  \newline
  \newline
  \begin{center}
    Situation $\Rightarrow$ \green{\underline{choix action}} $\Leftarrow$ optimisation récompense
    \newline
    \newline
    \red{Quelles caractéristiques indispensables ?}
  \end{center}
\end{frame}

\begin{frame}
  \frametitle{Préambule à l'intelligence artificielle}
  \imgth[0.6]{memoire-fun}
\end{frame}

\begin{frame}
  \frametitle{Préambule à l'intelligence artificielle}
  \underline{\textbf{Intelligence}} :
  \newline
  \newline
  Situation $\Rightarrow$ \green{\underline{choix action}} $\Leftarrow$ optimisation récompense
  \newline
  \newline
  grâce à :
  \begin{itemize}
  \item \green{la mémoire}
  \end{itemize}
\end{frame}

\begin{frame}
  \frametitle{Préambule à l'intelligence artificielle}
  \imgth[0.6]{conceptualisation}
\end{frame}

\begin{frame}
  \frametitle{Préambule à l'intelligence artificielle}
  \underline{\textbf{Intelligence}} :
  \newline
  \newline
  Situation $\Rightarrow$ \green{\underline{choix action}} $\Leftarrow$ optimisation récompense
  \newline
  grâce à :
  \begin{itemize}
  \item la mémoire
  \item \green{la conceptualisation}
  \end{itemize}
\end{frame}

\begin{frame}
  \frametitle{Préambule à l'intelligence artificielle}
  \imgtw[0.8]{foot-keeper-penalty}
\end{frame}

\begin{frame}
  \frametitle{Préambule à l'intelligence artificielle}
  \underline{\textbf{Intelligence}} :
  \newline
  \newline
  Situation $\Rightarrow$ \green{\underline{choix action}} $\Leftarrow$ optimisation récompense
  \newline
  grâce à :
  \begin{itemize}
  \item la mémoire
  \item la conceptualisation
  \item \green{la prédiction}
  \end{itemize}
\end{frame}

\begin{frame}
  \frametitle{Préambule à l'intelligence artificielle}
  \begin{center}
    \huge{\red{INTELLIGENCE ARTIFICIELLE ?}}
  \end{center}
\end{frame}

\begin{frame}
  \frametitle{Préambule à l'intelligence artificielle}
  \underline{\textbf{Intelligence Artificielle}} :
  \newline
  \newline
  Situation $\Rightarrow$ \green{\underline{choix action}} $\Leftarrow$ optimisation récompense
  \newline
  grâce à :
  \begin{itemize}
  \item la mémoire
  \item la conceptualisation
  \item la prédiction
  \end{itemize}
\end{frame}

\end{document}
%%% Local Variables:
%%% mode: latex
%%% TeX-master: t
%%% End:
