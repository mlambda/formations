
\begin{frame}{Description}
  Cette formation présente les \textbf{fondamentaux} de la \textbf{Modélisation Statistique} à travers des travaux pratiques.
\end{frame}

\begin{frame}{Prérequis}
  \begin{itemize}
  \item Connaissances de base en statistiques
  \item Connaissances de base en python
  \item Avoir un \textbf{compte Google} afin de pouvoir faire les TPs dans \bluelink{https://colab.research.google.com/notebooks/welcome.ipynb}{Google Colaboratory}
  \end{itemize}
\end{frame}

\begin{frame}{Objectifs pédagogiques}
  \begin{itemize}
  \item Comprendre les différents modèles d'apprentissage
  \item Modéliser un problème pratique sous forme abstraite
  \item Identifier les méthodes d'apprentissage pertinentes pour résoudre un problème
  \item Appliquer et évaluer les méthodes identifiées sur un problème
  \item Faire le lien entre les différentes techniques d'apprentissage
  \end{itemize}
\end{frame}

\begin{frame}{Programme}
  \begin{itemize}
  \item Introduction à la modélisation
  \item Evaluation de modèles prédictifs
  \item Les algorithmes supervisés/non-supervisés
  \item Réduction de dimension
  \item Données séquentielles (temporelles)
  \end{itemize}
\end{frame}
