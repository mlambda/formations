\documentclass{formation}
\title{Big Data Analytics}
\subtitle{Analyse en composantes - variantes spécifiques}

\begin{document}

\maketitle

\begin{frame}
  \frametitle{Analyse des Correspondances Multiples (ACM)}
  ACP sur des données qualitatives (Ex : enquètes d'opinions avec QCM) \\
  Chaque variable qualitative est transformé en vecteur sparse. \\
  On obtient une matrice binaire sur laquelle on procède à l'ACP.\\
\end{frame}

\begin{frame}
  \frametitle{Analyse Factorielle pour données mixtes (AFDM)}
  Quand on a des variables qualitative ET quantitatives pour décrires nos échantillons, on discrétise chaque variable quantitative. On peut ainsi procéder à l'Analyse en Composantes Multiples
\end{frame}

\begin{frame}
  \frametitle{Analyse Factorielle des Correspondances (AFC)}
  Méthode sur un tableau de contingence :
  \imgtw[0.8]{tableau-contingence}
  On procède alors à une double ACP (une sur le profil ligne, l'autre sur le profil colonne) en utilisant une métrique particulière : le $\chi^2$
\end{frame}

\end{document}
%%% Local Variables:
%%% mode: latex
%%% TeX-master: t
%%% End:
