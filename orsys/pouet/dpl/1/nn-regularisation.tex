\documentclass{formation}
\title{Deep Learning par la Pratique}
\subtitle{Régularisation de l'apprentissage}

\begin{document}

\maketitle

\begin{frame}
  \frametitle{Régularisation}
  \imgtw[0.6]{over-underfitting}
\end{frame}

\begin{frame}
  \frametitle{Régularisation}
  L1 et L2 régularisation par la loss : \\
  \[
  Loss = Loss + \frac{\lambda}{2m}*\sum||w||^2
  \]
  où $\lambda$ est un hyperparamètre
\end{frame}

\begin{frame}
  \frametitle{Régularisation}
  Early stopping :
  \imgtw[0.6]{early-stopping}
\end{frame}

\begin{frame}
  \frametitle{Régularisation}
  Augmentation/bruitage des données :
  \imgtw[0.8]{data-augmentation}
\end{frame}

\begin{frame}
  \frametitle{Régularisation}
  Dropout :
  \imgtw[0.8]{dropout-2}
\end{frame}

\begin{frame}
  \frametitle{Régularisation}
  Apprentissage de tâches différentes :
  \imgtw[0.8]{multi-task-v2}
\end{frame}

\begin{frame}
  \frametitle{Régularisation}
  Réseaux adverses :
  \imgtw[0.8]{gan-schema}
\end{frame}

\end{document}
%%% Local Variables:
%%% mode: latex
%%% TeX-master: t
%%% End:
