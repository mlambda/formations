\documentclass{formation}
\title{Machine Learning, méthodes et solutions}
\subtitle{Exploration de données}

\begin{document}

\maketitle

\begin{frame}
  \frametitle{Travaux Pratiques}
  \blue{\underline{\href{https://drive.google.com/open?id=18UC6cySq3loE7S2t9-Fu_ihMT2nDeJjk}{Quelques fonctions utiles pour les Notebook dans Colaboratory}}} \\
  \newline
  Après avoir ouvert le lien dans Colaboratory :
  \begin{center}
    Fichier > Enregistrer une copie dans Drive...
  \end{center}
  Sinon vous ne pourrez pas éditer le notebook.
\end{frame}

\begin{frame}
  \frametitle{Travaux Pratiques}
  \blue{\underline{\href{https://drive.google.com/open?id=1b93dMeJ1wubsVYyNkiCzg_4eaRqFgl51}{Exploration de données-TP}}} \\
  $\;$ \\
  $\;$ \\
  \blue{\underline{\href{https://perso.limsi.fr/pointal/_media/python:cours:mementopython3-english.pdf}{python-help}}} \\
  \blue{\underline{\href{https://github.com/pandas-dev/pandas/blob/master/doc/cheatsheet/Pandas_Cheat_Sheet.pdf}{pandas-help}}} \\
  \blue{\underline{\href{https://s3.amazonaws.com/assets.datacamp.com/blog_assets/Python_Matplotlib_Cheat_Sheet.pdf}{matplotlib-help}}} \\
  (Mais on n'oublie pas la doc ;-) )
\end{frame}

\end{document}
%%% Local Variables:
%%% mode: latex
%%% TeX-master: t
%%% End:
