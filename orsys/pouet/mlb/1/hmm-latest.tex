\documentclass{formation}
\title{Machine Learning}
\subtitle{Hidden Markov Model}

\begin{document}

\maketitle

\begin{frame}
  \frametitle{Hidden Markov Model}
  Un modèle de markov où les états générants les observations sont cachés \\
  Par exemple : diagnostic, capteur bruité, bourse, achats clients, ...
  \imgth[0.7]{hmm}
\end{frame}

\begin{frame}
  \frametitle{Hidden Markov Model}
  Sachant une espace d'oservation O, un HMM est un quadruplet $\{S,\Pi,A,B\}$ où :
  \begin{itemize}
  \item $O = \{o_1, \dotsc, o_K \}$ est l'espace des observations,
  \item $S = \{s_1, \dotsc, s_n \}$ l'espace des états,
  \item $\pi = \{\pi_1, \dotsc, \pi_n \}$ ,les probabilités des états de départ,
  \item $A$ la matrice de transition de taille $n\times n$ où $a_{ij} = P(S_j|S_i)$,
  \item $B$ la matrice d'émission de taille $n\times K$ où $b_{ik} = P(o_k|S_i)$,
  \item Une séquence d'observations : $X=\{x_1, \dotsc x_T \}$
  \end{itemize}
\end{frame}

\begin{frame}
  \frametitle{Hidden Markov Model}
  Exemple sur l'ADN : 
  \begin{itemize}
  \item $O = \{A,C,T,G\}$ 
  \item $S =  \{cheveux, oeil, peau, foie, coeur,\dotsc\,organe_{78}\}$
  \item $A$ est de taille $78 \times 78$ 
  \item $B$ est de taille $78\times 4$
  \item Une séquence d'observations : $X=ATGCGATCTATCGCTAGCCGCGCTATACGCA$
  \end{itemize}
\end{frame}

\begin{frame}
  \frametitle{Hidden Markov Model}
  Pour en savoir plus sur l'apprentissage de HMM :
  \begin{itemize}
  \item Algorithme Forward-Backward (Expectation-Maximisation)
  \item Algorithme de Baum-Welch
  \item Viterbi et théorème de Bayes
  \end{itemize}
\end{frame}

\begin{frame}
  \frametitle{Hidden Markov Model}
  Utilisations :
  \begin{itemize}
  \item Modèle génératif de séquence
  \item Prédiction de classe : un modèle par classe
  \item Découverte de patterns
  \end{itemize}
  \imgth[0.6]{hmm}
\end{frame}

\begin{frame}
  \frametitle{Hidden Markov Model}
  Des imperfections :
  \begin{itemize}
  \item L'observation émise par le modèle ne dépend que de l'état courant
  \item L'état ne dépend que de l'état précédent
  \item Algorithme EM donc très sensible à l'initialisation
  \end{itemize}
  \imgth[0.6]{hmm}
\end{frame}

\end{document}
%%% Local Variables:
%%% mode: latex
%%% TeX-master: t
%%% End:
