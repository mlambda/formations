\documentclass{formation}
\title{Machine Learning, méthodes et solutions}
\subtitle{DBSCAN}

\begin{document}

\maketitle

\begin{frame}
  \frametitle{DBSCAN}
  Density-Based Spatial Clustering of Applications with Noise...\\
  \\
  Tant qu'il reste des points non-étiquettés : \\
  \begin{enumerate}
  \item Prend un point non-étiquettés au hasard et on regarde son voisinage
  \item Si (densité > seuil) alors (Nouveau cluster)
    \begin{enumerate}
    \item Expansion du cluster de proche en proche dans le voisinage
    \end{enumerate}
  \item Sinon (Bruit)
  \end{enumerate}
\end{frame}

\begin{frame}
  \frametitle{DBSCAN}
  \imgtw[0.9]{dbscan-1}
\end{frame}

\begin{frame}
  \frametitle{DBSCAN}
  \imgtw[0.9]{dbscan-2}
\end{frame}

\begin{frame}
  \frametitle{DBSCAN}
  \imgtw[0.9]{dbscan-3}
\end{frame}

\begin{frame}
  \frametitle{DBSCAN}
  \imgtw[0.9]{dbscan-4}
\end{frame}

\begin{frame}
  \frametitle{DBSCAN}
  \imgtw[0.9]{dbscan-5}
\end{frame}

\begin{frame}
  \frametitle{DBSCAN}
  \imgtw[0.9]{dbscan-6}
\end{frame}

\begin{frame}
  \frametitle{DBSCAN}
  \imgtw[0.9]{dbscan-7}
\end{frame}

\begin{frame}
  \frametitle{DBSCAN}
  \imgtw[0.9]{dbscan-8}
\end{frame}

\begin{frame}
  \frametitle{DBSCAN}
  \imgtw[0.9]{dbscan-9}
\end{frame}



\end{document}
%%% Local Variables:
%%% mode: latex
%%% TeX-master: t
%%% End:
