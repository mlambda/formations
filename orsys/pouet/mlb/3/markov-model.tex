\documentclass{formation}
\title{Machine Learning}
\subtitle{Markov Model}

\begin{document}

\maketitle

\begin{frame}
  \frametitle{Markov Model}
  \imgth[0.9]{markov}
\end{frame}

\begin{frame}
  \frametitle{Markov Model}
  Le modèle est estimé par ``comptage'' des transitions d'états \\
  1: ATGCGATCTATCGCTAGCCGCGCTATACGCA \\
  2: GATTATAGCTAGCTCGCGCTATATCGCTAGCTAGCTAGCTAGC \\
  $P(A|T) = \frac{\#TA}{\#TA+\#TC+\#TG+\#TT}$ \\
  $P(A|C) = \frac{\#CA}{\#CA+\#CC+\#CG+\#CT}$ \\
  ...
\end{frame}

\begin{frame}
  \frametitle{Markov Model}
  Utilisations :
  \begin{itemize}
  \item Modèle génératif de séquence
  \item Prédiction de classe : un modèle par classe
  \item Découverte de pattern
  \end{itemize}
  \imgth[0.6]{markov}
\end{frame}

\begin{frame}
  \frametitle{Markov Model}
  Des applications où c'est efficace (malgrès des limitations évidentes):
  \begin{itemize}
  \item Premières approximations météo
  \item Thermodynamique
  \item Théorie des files d'attente (télécommunication)
  \item ...
  \end{itemize}
  \imgth[0.6]{markov}
\end{frame}


\end{document}
%%% Local Variables:
%%% mode: latex
%%% TeX-master: t
%%% End:
