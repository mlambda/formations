\documentclass{formation}
\title{Machine Learning, méthodes et solutions}
\subtitle{Support Vector Machine}

\begin{document}

\maketitle

\begin{frame}
  \frametitle{Support Vector Machine}
  \imgtw[0.8]{separable-problem-linear}
\end{frame}

\begin{frame}
  \frametitle{Support Vector Machine}
  \imgtw[0.8]{separable-problem-linear-solution}
\end{frame}

\begin{frame}
  \frametitle{Support Vector Machine}
  \imgtw[0.8]{separable-problem-linear-solution-bad}
\end{frame}

\begin{frame}
  \frametitle{Support Vector Machine}
  \imgtw[0.8]{separable-problem-linear-svm}
\end{frame}

\begin{frame}
  \frametitle{Support Vector Machine}
  \imgtw[0.9]{svm-schema}
\end{frame}

\begin{frame}
  \frametitle{Support Vector Machine}
  \imgtw[0.8]{separable-problem-nonlinear}
\end{frame}

\begin{frame}
  \frametitle{Support Vector Machine}
  \imgtw[0.8]{separable-problem-nonlinear-solution}
\end{frame}

\begin{frame}
  \frametitle{Support Vector Machine}
  \imgtw[0.9]{separable-problem-nonlinear-kernelsvm}
\end{frame}

\begin{frame}
  \frametitle{Support Vector Machine}
  Généralisation à un problème de régression logistique à $K>2$ classes :
  \begin{itemize}
  \item One Vs All : K modèles. Agréagation par meilleur score.
  \item One Vs One : $\frac{K(K-1)}{2}$ modèles. Vote majoritaire.
  \end{itemize}
\end{frame}

\end{document}
%%% Local Variables:
%%% mode: latex
%%% TeX-master: t
%%% End:
