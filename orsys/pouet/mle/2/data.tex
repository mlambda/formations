\documentclass{formation}
\title{Possibilités offertes par le machine learning}
\subtitle{De l'importance des \textbf{données}}

\begin{document}

\maketitle

\begin{frame}
  \frametitle{Données}
  Les \textbf{données} sont l'élément central du machine learning, bien plus que l'algorithme d'apprentissage lui-même !
  (trash in, trash out)
\end{frame}

\begin{frame}
  \frametitle{Données}
  \red{Quelle bonne méthode de séparation ?}
  \imgtw{double-moon-part}
\end{frame}

\begin{frame}
  \frametitle{Données}
  \red{Quelle bonne méthode de séparation ?}
  \imgth[0.8]{double-moon-full}
\end{frame}

\begin{frame}
  \frametitle{Données}
  Les données d'apprentissage doivent être représentatives des données qui passeront en production, sinon on risque une catastrophe.
  \imgtw[0.8]{black-gorilla-fail}
\end{frame}

\begin{frame}
  \frametitle{Données}
  Données à dimension fixe ou dimension variable ?
\end{frame}

\begin{frame}
  \frametitle{Données à dimension fixe}
  \begin{itemize}
  \item Image
  \item Texte (tf-idf : ~50000 mots)
  \item ligne dans une table
  \item ...
  \end{itemize}
\end{frame}

\begin{frame}
  \frametitle{Données à dimension variable}
  Données ancrées dans le temps.
  \newline
  Séquence de données à dimension fixe
  \begin{itemize}
  \item Son
  \item Video
  \item Texte
  \item ...
  \end{itemize}
\end{frame}

\end{document}
%%% Local Variables:
%%% mode: latex
%%% TeX-master: t
%%% End:
