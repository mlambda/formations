\documentclass{formation}
\title{Machine Learning, méthodes et solutions}
\subtitle{Gradient boosted trees}

\begin{document}

\maketitle

\begin{frame}
  \frametitle{Introduction}
  Arbres qui s'améliorent successivement.
  \begin{figure}
    \centering \imgth[.4]{tree-training-modes}
    \scriptsize{\href{https://quantdare.com/what-is-the-difference-between-bagging-and-boosting/}%
        {quantdare.com/what-is-the-difference-between-bagging-and-boosting/}}
  \end{figure}
\end{frame}

\begin{frame}
  \frametitle{Algorithme grossier}
  \begin{enumerate}
  \item partir d'un arbre grossier
  \item entraîner un nouvel arbre sur les résiduels du premier
  \item concaténer le nouvel arbre au premier
  \item \texttt{goto 2.}
  \end{enumerate}
\end{frame}

\begin{frame}
  \frametitle{Idée à retenir}
  \begin{itemize}
  \item séquence d'arbres qui s'entrainent à corriger les erreurs de
    l'arbre d'avant
  \item modélisation de la correction de l'erreur par un pas de
    descente de gradient pour plus de flexibilité.
  \end{itemize}
\end{frame}

\end{document}
%%% Local Variables:
%%% mode: latex
%%% TeX-master: t
%%% End:
