\begin{frame}
  \frametitle{Introduction}
  Caputre et manipulation de chaînes de caractères par le biais de motifs génériques. \\
  \newline
  Exemple : Rajouter un espace après chaque ponctuation (si il n'y en as pas déjà) \\
  \newline
  "Le chat mange,délicatement, ses croquettes.Jean est attendri." \\
  $\rightarrow$ \\
  "Le chat mange, délicatement, ses croquettes. Jean est attendri. " \\
\end{frame}

\begin{frame}
  \frametitle{Utilisations}
  Une fois le motif capturé, on peut :
  \begin{itemize}
  \item supprimer
  \item modifier
  \item réutiliser
  \item ...
  \end{itemize}
\end{frame}

\begin{frame}
  \frametitle{Regex en Python : Les caractères spéciaux}
  \begin{itemize}
  \item \makebox[1cm][c]{\^{}} début de ligne
  \item \makebox[1cm][c]{\$} fin de ligne
  \item \makebox[1cm][c]{.} joker (capture n'importe quel caractère) 
  \item \makebox[1cm][c]{[} ouverture d'un ensemble de caractère
  \item \makebox[1cm][c]{]} fermeture d'un ensemble de caractère
  \item \makebox[1cm][c]{(} ouverture de groupe de capture
  \item \makebox[1cm][c]{)} fermeture de groupe de capture
  \item \makebox[1cm][c]{?} de 0 à 1 occurence du motif
  \item \makebox[1cm][c]{*} de 0 à $\infty$ occurences du motif
  \item \makebox[1cm][c]{+} de 1 à $\infty$ occurences du motif
  \item \makebox[1cm][c]{\{} ouverture d'intervalle de répétition du motif
  \item \makebox[1cm][c]{\}} fermeture d'intervalle de répétition du motif
  \item \makebox[1cm][c]{\textbackslash} caractère spécialement spécial
  \item \makebox[1cm][c]{\textbar} OU logique dans un groupe de capture
  \end{itemize}
\end{frame}

\begin{frame}
  \frametitle{re.search}
  \inputminted[linenos,fontsize=\small,bgcolor=pythonbg,framesep=1mm]{python}
  {code/regex/demo-intro-search.py}
  \vspace{-1cm}
  \inputminted[linenos,fontsize=\small,bgcolor=returnbg,framesep=1mm]{python}
  {code/regex/demo-intro-search.out}
\end{frame}

\begin{frame}
  \frametitle{re.match}
  \inputminted[linenos,fontsize=\small,bgcolor=pythonbg]{python}
  {code/regex/demo-intro-match.py}
  \vspace{-1cm}
  \inputminted[linenos,fontsize=\small,bgcolor=returnbg]{python}
  {code/regex/demo-intro-match.out}
\end{frame}

\begin{frame}
  \frametitle{Utilisations}
\end{frame}

\begin{frame}
  \frametitle{Utilisations}
\end{frame}

\begin{frame}
  \frametitle{Utilisations}
\end{frame}

\begin{frame}
  \frametitle{Utilisations}
\end{frame}

\begin{frame}
  Beaucoup de gros fichiers ? `sed` reste la solution optimale depuis les années 70.
\end{frame}

\begin{frame}
  \frametitle{template}

  \imgtw[0.1]{logo_mlweek}
  
  \begin{itemize}
  \item 
  \item 
  \item 
  \item 
  \end{itemize}

  \begin{minipage}[l]{0.49\linewidth}
  \end{minipage}\hfill
  \begin{minipage}[l]{0.49\linewidth}
  \end{minipage}\hfill

  \begin{minipage}[l]{0.49\linewidth}
      \begin{itemize}
      \item 
      \item 
      \item 
      \item 
      \end{itemize}
  \end{minipage}\hfill
  \begin{minipage}[l]{0.49\linewidth}
  \end{minipage}\hfill

\end{frame}
