\begin{frame}
  \frametitle{Description}
  Cette formation présente les techniques de traitement automatique du 
  \textbf{texte} et du \textbf{langage} à travers des travaux pratiques.
\end{frame}

\begin{frame}
  \frametitle{Prérequis}
  \begin{itemize}
  \item Connaissances de base en statistiques
  \item Connaissances de base en python
  \item Avoir un \textbf{compte Google} afin de pouvoir faire les TPs dans \bluelink{https://colab.research.google.com/notebooks/welcome.ipynb}{Google colaboratory}
  \end{itemize}
\end{frame}

\begin{frame}
  \frametitle{Objectifs pédagogiques}
  \begin{itemize}
  \item Comprendre les méthodes de la statistique textuelle
  \item Mettre en oeuvre l'extraction de caractéristiques textuelles
  \item Choisir un algorithme de classification
  \item Évaluer les performances prédictives d'un algorithme
  \end{itemize}
\end{frame}

\begin{frame}{Programme}
  \begin{itemize}
  \item Text Mining par la Pratique
  \item Traitements Automatique du Langage
  \item Introduction au Machine Learning
  \item Apprentissage supervisé
  \item NLP et Deep Learning
  \end{itemize}
\end{frame}
