\begin{frame}
  \frametitle{Reproductibilité}
  \begin{itemize}
  \item extrêmement importante pour compléter les analyses après les retours business
  \item ensemble de bonnes pratiques d'ingénierie
  \end{itemize}
\end{frame}

\begin{frame}
  \frametitle{Reproductibilité}
  \begin{itemize}[<+->]
  \item garder une trace exacte du préprocessing
  \item de préférence utiliser des notebooks
  \item faire attention au random (utiliser des seeds)
  \item définir les datasets utilisés, dates comprises
  \item garder une trace de l'environnement
  \end{itemize}
\end{frame}

\begin{frame}
  \frametitle{Méta-paramètres et Régularisation}
  Les méta-paramètres forment l'ensemble des prétraitements,la forme et les contraintes appliquées au modèle \textbf{AVANT} son apprentissage.
  \begin{itemize}
  \item Forme : Nombre de couches ?, de quelle taille ? ...
  \item L'algorithme d'optimisation (SGD, adaboost, adam,...)
  \item Méthodes de régularisation (norme des paramètres dans la loss, bruitage, dropout, ...)
  \end{itemize}
  \begin{minipage}[l]{0.49\linewidth}
    \begin{center}
      Régularisation \\
      $\approx$\\
      empêcher le surapprentissage
    \end{center}
  \end{minipage}\hfill
  \begin{minipage}[l]{0.49\linewidth}
    \imgth[.3]{overfitting}
  \end{minipage}\hfill
\end{frame}

