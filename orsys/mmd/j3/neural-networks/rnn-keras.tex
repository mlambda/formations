\begin{frame}
  \frametitle{RNN avec Keras}
  \inputminted[linenos,fontsize=\small,bgcolor=pythonbg]{python}{code/tf-keras-rnn.py}
  \inputminted[linenos,fontsize=\small,bgcolor=returnbg]{python}{code/tf-keras-rnn.txt}
\end{frame}

\begin{frame}
  \frametitle{LSTM avec Keras}
  \inputminted[linenos,fontsize=\small,bgcolor=pythonbg]{python}{code/tf-keras-lstm.py}
  \inputminted[linenos,fontsize=\small,bgcolor=returnbg]{python}{code/tf-keras-lstm.txt}
\end{frame}

\begin{frame}
  \frametitle{GRU avec Keras}
  \inputminted[linenos,fontsize=\small,bgcolor=pythonbg]{python}{code/tf-keras-gru.py}
  \inputminted[linenos,fontsize=\small,bgcolor=returnbg]{python}{code/tf-keras-gru.txt}
\end{frame}

\begin{frame}
  \frametitle{Deep Encoder avec Keras}
  On peut demander à un layer récurrent à de fournir une output pour chaque timestep :
  \inputminted[linenos,fontsize=\small,bgcolor=pythonbg]{python}{code/tf-keras-rnn-all-timestep.py}
  \inputminted[linenos,fontsize=\small,bgcolor=returnbg]{python}{code/tf-keras-gru.txt}
\end{frame}

