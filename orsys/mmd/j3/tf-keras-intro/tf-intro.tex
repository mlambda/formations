\subsection{TensorFlow}


\begin{frame}
  \frametitle{TensorFlow}
  \imgth[0.7]{tensorflow_logo}
\end{frame}

\begin{frame}
  \frametitle{TensorFlow}
  \imgth[0.7]{tensorflow-hierarchy}
\end{frame}

\begin{frame}
  \frametitle{TensorFlow}
  \begin{minipage}[l]{0.15\linewidth}
      $\;$
  \end{minipage}\hfill
  \begin{minipage}[l]{0.35\linewidth}
    \huge
    \begin{center}
      $\pi*r^2\;\;\;\rightarrow$
    \end{center}
  \end{minipage}\hfill
  \begin{minipage}[l]{0.49\linewidth}
    \imgth[0.9]{tensorflow}
  \end{minipage}\hfill
\end{frame}

\begin{frame}
  \frametitle{TensorFlow}
  Créer un graphe :
  \inputminted[linenos,fontsize=\small,bgcolor=pythonbg]{python}{code/tf-create_graph.py}
  \inputminted[linenos,fontsize=\small,bgcolor=returnbg]{text}{code/tf-create_graph.txt}
\end{frame}

\begin{frame}
  \frametitle{TensorFlow}
  Exécuter un graphe :
  \inputminted[linenos,fontsize=\small,bgcolor=pythonbg]{python}{code/tf-run_graph.py}
  \inputminted[linenos,fontsize=\small,bgcolor=returnbg]{text}{code/tf-run_graph.txt}
\end{frame}

\begin{frame}
  \frametitle{TensorFlow}
  Créer un graphe tensorflow avec des ``placeholder'':
  \inputminted[linenos,fontsize=\small,bgcolor=pythonbg]{python}{code/tf-create_graph_placeholder.py}
  \inputminted[linenos,fontsize=\small,bgcolor=returnbg]{text}{code/tf-create_graph_placeholder.txt}
\end{frame}

\begin{frame}
  \frametitle{TensorFlow}
  Executer un graphe avec des ``placeholder'':
  \inputminted[linenos,fontsize=\small,bgcolor=pythonbg]{python}{code/tf-run_graph_placeholder.py}
  \inputminted[linenos,fontsize=\small,bgcolor=returnbg]{text}{code/tf-run_graph_placeholder.txt}
\end{frame}

\begin{frame}
  \frametitle{TensorFlow}
  Matrices et opérations usuelles :
  \inputminted[linenos,fontsize=\small,bgcolor=pythonbg]{python}{code/tf-matrices.py}
  \inputminted[linenos,fontsize=\small,bgcolor=returnbg]{text}{code/tf-matrices.txt}
\end{frame}

\begin{frame}
  \frametitle{TensorFlow}
  Régression Linéaire (Définition des variables)
  \inputminted[linenos,fontsize=\small,bgcolor=pythonbg]{python}{code/tf-regression_lineaire-0.py}
\end{frame}

\begin{frame}
  \frametitle{TensorFlow}
  Régression Linéaire (Définition de la régression et de son erreur) :
  \inputminted[linenos,fontsize=\small,bgcolor=pythonbg]{python}{code/tf-regression_lineaire-1.py}
\end{frame}

\begin{frame}
  \frametitle{TensorFlow}
  Régression Linéaire :
  \inputminted[linenos,fontsize=\small,bgcolor=pythonbg]{python}{code/tf-regression_lineaire_run.py}
  \inputminted[linenos,fontsize=\small,bgcolor=returnbg]{text}{code/tf-regression_lineaire_run.txt}
\end{frame}

\begin{frame}
  \frametitle{TensorFlow}
  Enregistrer un graphe de calcul tensorflow :
  \inputminted[linenos,fontsize=\small,bgcolor=pythonbg]{python}{code/tf-save_models.py}
  \inputminted[linenos,fontsize=\small,bgcolor=returnbg]{text}{code/tf-save_models.txt}
\end{frame}

\begin{frame}
  \frametitle{TensorFlow}
  Charger un graphe de calcul tensorflow :
  \inputminted[linenos,fontsize=\small,bgcolor=pythonbg]{python}{code/tf-load_models.py}
  \inputminted[linenos,fontsize=\small,bgcolor=returnbg]{text}{code/tf-load_models.txt}
\end{frame}

\begin{frame}
  \frametitle{TensorFlow}
  Réutiliser le graphe pour ajouter de nouveaux calculs :
  \inputminted[linenos,fontsize=\small,bgcolor=pythonbg]{python}{code/tf-modify_models.py}
  \inputminted[linenos,fontsize=\small,bgcolor=returnbg]{text}{code/tf-modify_models.txt}
\end{frame}

\begin{frame}
  \frametitle{TensorFlow}
  \bluelink{https://colab.research.google.com/drive/13g9krUBFSxMPKVQwiDOIz9bNUOXlCYdi}{Le code présenté, executable dans colaboratory}
\end{frame}

