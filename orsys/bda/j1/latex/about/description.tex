\begin{frame}{Description}
  Cette formation présente les \textbf{fondamentaux} de la \textbf{Modélisation Statistique} à travers des travaux pratiques.
\end{frame}

\begin{frame}{Prérequis}
  \begin{itemize}
  \item Connaissances de base en statistiques et algèbre
  \item Connaissances de base en python
  \item Avoir un \textbf{compte Google} afin de pouvoir faire les TPs dans \bluelink{https://colab.research.google.com/notebooks/welcome.ipynb}{Google Colaboratory}
  \end{itemize}
\end{frame}

\begin{frame}{Objectifs pédagogiques}
  \begin{itemize}
  \item Comprendre les principes de la modélisation statistique
  \item Comprendre les différents types de régressions
  \item Évaluer les performances d'un algorithme prédictif
  \item Sélectionner et classer des données dans de grands volumes de données
  \item Se familiariser avec les librairies scientifiques python (NumPy, seaborn, scikit-learn, ... )
  \end{itemize}
  En bref, se familiariser avec les concepts et techniques de bases du \og{}Machine Learning\fg
\end{frame}

\begin{frame}{Programme}
  \begin{itemize}
  \item Introduction à la modélisation
  \item Évaluation de modèles prédictifs
  \item Les algorithmes supervisés/non-supervisés
  \item Projection de données par composantes
  \item Analyse de données textuelles
  \end{itemize}
\end{frame}
