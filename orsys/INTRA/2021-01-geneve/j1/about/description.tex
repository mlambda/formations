
\begin{frame}{Description}
  Cette formation aborde les points non couverts par la dernière formation : le deep learning, le déploiement et git ainsi que des travaux pratiques sur la détection d'anomalie.
\end{frame}

\begin{frame}{Prérequis}
  \begin{itemize}
  \item Connaissances de base en statistiques
  \item Connaissances de base en python
  \item Avoir un \textbf{compte Google} afin de pouvoir faire les TPs dans \bluelink{https://colab.research.google.com/notebooks/welcome.ipynb}{Google Colaboratory}
  \end{itemize}
\end{frame}

\begin{frame}{Objectifs pédagogiques}
  \begin{itemize}
  \item Comprendre les fondamentaux du deep learning
  \item Identifier les méthodes d'apprentissage pertinentes pour résoudre un problème
  \item Savoir mettre en production un modèle
  \item Manipuler les méthodes de détection d'anomalie
  \item Comprendre la gestion de code source par git
  \end{itemize}
\end{frame}

\begin{frame}{Programme}

  \begin{description}
    \item[Jour 1]
      \begin{itemize}
        \item Deep learning
      \end{itemize}
    \item[Jour 2]
      \begin{itemize}
        \item Introduction à git
        \item Ingénierie en machine learning
        \item Techniques de déploiement
      \end{itemize}
  \end{description}
\end{frame}
