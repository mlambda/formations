\documentclass{formation}
\title{Big Data Analytics}
\subtitle{Réduction de la dimensionalité}

\begin{document}

\maketitle

\begin{frame}
  \frametitle{Réduction de la dimensionalité}
    \begin{center}
      Comment appréhender des données en grande dimension ?
    \end{center}
    \[
    X = \begin{bmatrix}
      X_{1,1} & X_{1,2} & \dots  & X_{1,D} \\
      X_{2,1} & X_{2,2} & \dots  & X_{2,D} \\
      \vdots & \vdots & \ddots & \vdots \\
      X_{N,1} & X_{N,2} & \dots  & X_{N,D}
    \end{bmatrix}
    \]
\end{frame}

\begin{frame}
  \frametitle{Réduction de la dimensionalité}
    \begin{center}
      La malédiction des grandes dimensions !\\
      (Curse of dimensionality)
    \end{center}
\end{frame}

\begin{frame}
  \frametitle{Réduction de la dimensionalité}
    \begin{itemize}
    \item Séléction de dimensions
    \item Projections linéaires (ACP, LDA, ...)
    \item Projections non-linéaires (kernels, neural network embeddings, ...)
    \end{itemize}
\end{frame}

\begin{frame}
  \frametitle{Réduction de la dimensionalité}
  Sélection de dimensions :
  \begin{itemize}
  \item Random forest
  \item SVM
  \item ...
  \end{itemize}
\end{frame}

\end{document}
%%% Local Variables:
%%% mode: latex
%%% TeX-master: t
%%% End:
