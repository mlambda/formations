\documentclass{formation}
\title{Machine Learning}
\subtitle{Autoregressive Moving Average Model : ARMA}

\begin{document}

\maketitle

\begin{frame}
  \frametitle{ARMA}
  Modèlisation statistique de processus (faiblement) stationnaires \\
  \newline
  Processus faiblement stationnaire si :
  \begin{itemize}
    \item sa moyenne ne dépend pas de t
    \item la covariance ne dépend pas de t
  \end{itemize}
\end{frame}

\begin{frame}
  \frametitle{ARMA}
  Exemple de processus non-stationnaires :
  \imgtw[0.9]{processus-non-stationnaires}
\end{frame}

\begin{frame}
  \frametitle{ARMA}
  Une composante \textbf{autoregressive} : \\
  \[
  X_{t}=c+\varepsilon _{t}+\sum _{i=1}^{p}\varphi _{i}X_{t-i}
  \]
  où \\
  $c$ est une constante, 
  $\varphi_i$ sont les paramètres du modèle, \\
  $\varepsilon_i$ les termes d'erreurs considéré comme un bruit blanc
\end{frame}

\begin{frame}
  \frametitle{ARMA}
  Un composante \textbf{moyenne mobile} : \\
  \[
  X_{t}=\mu +\varepsilon _{t}+\sum _{i=1}^{q}\theta _{i}\varepsilon _{t-i}
  \]
  $\mu$ est la moyenne attendue, \\
  $\theta_i$ sont les paramètres du modèle, \\
  $\varepsilon_i$ les termes d'erreurs considéré comme un bruit blanc
\end{frame}

\begin{frame}
  \frametitle{ARMA}
  Au final pour ARMA : \\
  \[
  X_{t}=c +\varepsilon _{t}+\sum _{i=1}^{p}\varphi _{i}X_{t-i}+\sum _{i=1}^{q}\theta _{i}\varepsilon _{t-i}
  \]
\end{frame}

\begin{frame}
  \frametitle{ARMA}
  Opérateur de ``Lag'' L : \\
  \begin{itemize}
  \item $L(X_{t}) = X_{t-1}$
  \item $L^2(X_{t}) = X_{t-2}$
  \item $L^{-1}(X_{t}) = X_{t+1}$
  \item ...
  \end{itemize}
\end{frame}

\begin{frame}
  \frametitle{ARMA}
  ARMA exprimé avec l'opérateur L: \\
  \[
  \left(I-\sum _{i=1}^{p}\varphi _{i}L^{i}\right)X_{t}=\left(I+\sum _{i=1}^{q}\theta _{i}L^{i}\right)\varepsilon _{t}
  \]
\end{frame}

\begin{frame}
  \frametitle{ARMA}
  Autocorrélation de lag $k\in\mathbb{Z}$ (ACF): \\
  $\Rightarrow$ Corrélation des deux variables aléatoires $X_t$ et $X_{t-k}$ \\
  \newline
  Autocorrélation partielle de lag $k\in\mathbb{Z}$ (PACF): \\
  $\Rightarrow$ Corrélation des deux variables aléatoires $X_t$ et $X_{t-k}$ où les dépendances linéaires entre $X_t$ et les variables $X_{t-k'}$ on été enlevées (pour $k'\in[1 .. (k-1)]$)  
\end{frame}

\begin{frame}
  \frametitle{ARMA}
  Apprentissage des paramètres : \\
  \begin{enumerate}
    \item se rapporter à un processus stationnaire : Transformation de box-cox des données, élimination des tendances et saisonnalité
    \item on fixe p et q
      \begin{itemize}
      \item plot des autocorrélations partielles pour p,
      \item plot des autocorrélations  pour q,
      \item la méthode Akaike Information Criterion (AIC) est cependant recommandée
      \end{itemize}
    \item estimation des paramètres $\varphi$ et $\theta$ en utilisant par exemple le maximum de vraisemblance ou le critère des moindres carrés
  \end{enumerate}
\end{frame}

\end{document}
%%% Local Variables:
%%% mode: latex
%%% TeX-master: t
%%% End:
