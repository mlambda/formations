
\begin{frame}
  \frametitle{Votre formation — description}
  Cette formation présente les \textbf{fondamentaux} de l'\textbf{Apprentissage Automatique} à travers des travaux pratiques. \\
\end{frame}

\begin{frame}
  \frametitle{Votre formation — connaissances préalables}
  \begin{itemize}
  \item Connaissances de base en statistiques et algèbre
  \item Connaissances de base en python
  \item Avoir un \textbf{compte Google} afin de pouvoir faire les TPs dans \bluelink{https://colab.research.google.com/notebooks/welcome.ipynb}{Google colaboratory}
  \end{itemize}
\end{frame}

\begin{frame}
  \frametitle{Votre formation — objectifs à atteindre}
  \begin{itemize}
  \item Comprendre les principes de la modélisation statistique
  \item Comprendre les différents types de régressions
  \item Évaluer les performances d'un algorithme prédictif
  \item Sélectionner et classer des données dans de grands volumes de données
  \item Se familiariser avec les librairies scientifiques python (NumPy, seaborn, sikit-learn, ... )
  \end{itemize}
  En bref, se familiariser avec les concepts et techniques de bases du ``Machine Learning''
\end{frame}

\begin{frame}
  \frametitle{Votre formation — programme}
  \begin{itemize}
  \item Introduction à la modélisation
  \item Evaluation de modèles prédictifs
  \item Les algorithmes supervisés/non-supervisés
  \item Projection de données par composantes
  \item Analyse de données textuelles
  \end{itemize}
\end{frame}
