\begin{frame}{Buts}
  \begin{itemize}[<+->]
    \item Mesurer la qualité des prédictions du modèle
    \item Évaluer si un modèle peut remplir un objectif métier
    \item Se protéger des régressions après mise à jour
  \end{itemize}
\end{frame}

\begin{frame}{Évaluation de modèles de régression}
  \begin{itemize}
    \item Erreur absolue moyenne (MAE)
    \item Erreur quadratique moyenne (MSE)
    \item Racine de l'erreur quadratique moyenne (RMSE)
    \item $R^2$-score
  \end{itemize}
\end{frame}

\begin{frame}{Erreur absolue moyenne (MAE)}
  \begin{equation*}
    MAE(y,\hat{y})=\frac{1}{n}\sum^n_{i=1}|y_i-\hat{y_i}|
  \end{equation*}

  \begin{itemize}
    \item Simple à interpréter
    \item Le coût de l'erreur est linéaire
  \end{itemize}
\end{frame}

\begin{frame}{Erreur quadratique moyenne (MSE)}
  \begin{equation*}
    MSE(y,\hat{y})=\frac{1}{n}\sum^n_{i=1}(y_i-\hat{y_i})^2
  \end{equation*}

  \begin{itemize}
    \item Moins simple à interpréter que la MAE
    \item Utile en apprentissage pour les propriétés de convergence
    \item Pénalise les fortes erreurs
  \end{itemize}
\end{frame}

\begin{frame}{Racine de l'erreur quadratique moyenne (RMSE)}
  \begin{equation*}
    RMSE(y,\hat{y})=\sqrt{\frac{1}{n}\sum^n_{i=1}(y_i-\hat{y_i})^2}
  \end{equation*}

  \begin{itemize}
    \item Donne une valeur dans un ordre de grandeur à peu près proche des données à prédire
  \end{itemize}
\end{frame}

\begin{frame}{$R^2$-score}
  \begin{columns}
    \begin{column}{0.5\linewidth}
      Somme du carre des résiduels :
      \begin{equation*}
        RSS(y,\hat{y})=\sum^n_{i=1}(y_i-\hat{y_i})^2
      \end{equation*}
    \end{column}
    \begin{column}{0.5\linewidth}
      Somme totale des carrées :
      \begin{equation*}
        TSS(y)=\sum^n_{i=1}(y_i-\bar{y})^2
      \end{equation*}
    \end{column}
  \end{columns}
  \vfill
  \begin{equation*}
    R^2(y,\hat{y})=1-\frac{RSS(y,\hat{y})}{TSS(y)}
  \end{equation*}

  \begin{itemize}
    \item Compare les erreurs de prédiction à la variance de l'échantillon
    \item Score de 1 : Prédiction parfaite
    \item Score de 0 : Prédit la moyenne 
    \item Score < 0 : Pire qu'une prédiction sur la moyenne
  \end{itemize}
  \vfill
\end{frame}

\begin{frame}{Évaluation de modèles de classification}
  Les deux premières métriques sont les plus utilisées:
  \begin{description}
  \item[Précision]
    \[
    \frac{\text{vrais positifs}}{\text{vrais positifs + faux positifs}}
    \]
  \item[Rappel]
    \[
    \frac{\text{vrais positifs}}{\text{vrais positifs + faux négatifs}}
  \]
  \item[F-mesure] Moyenne harmonique entre précision et rappel (aussi appelée score F1)
  \end{description}
\end{frame}

\begin{frame}{Illustration de la précision et du rappel}
  \V{["img/precisionrecall", "th", 0.7] | image}
\end{frame}

\begin{frame}{Matrice de confusion}
  Montrer clairement quelles sont les erreurs faites par le modèle
  \V{["img/confusion-matrix-nzmog", "th", 0.6] | image}
\end{frame}

\begin{frame}{Espace ROC}
  Mesure la performance du modèle à plusieurs seuils de décision.
  \begin{columns}
    \begin{column}{.48\textwidth}
      \textbf{Abscisse}

      1 - spécificité, taux de faux positifs, ou probabilité de \emph{fausse alerte} ($\frac{\text{FP}}{\text{VN} + \text{FP}}$)
 
      \vspace{1cm}
      \textbf{Ordonnée}
  
      Sensibilité, taux de vrais positifs ou rappel ($\frac{\text{VP}}{\text{VP} + \text{FN}}$)
    \end{column}
    \begin{column}{.52\textwidth}
      \V{["img/roc-space", "tw", 0.8] | image}
    \end{column}
  \end{columns}
\end{frame}

\begin{frame}{AUC --- Aera Under the Curve}
  \begin{columns}
    \begin{column}{0.48\linewidth}
      L'AUC est l'aire sous la courbe ROC.
      \begin{itemize}
        \item Lorsque la reconnaissance est parfaite, l'aire sous la courbe vaut 1.
        \item Lorsque la reconnaissance est aléatoire (sur deux classes), l'aire sous la courbe vaut 0.5.
      \end{itemize}
    \end{column}
    \begin{column}{0.52\linewidth}
      \V{["plt/auc", "tw", 1] | image}
    \end{column}
  \end{columns}
\end{frame}

\begin{frame}{Scores de classement}
  Les scores de classement (\emph{ranking}) fournissent une métrique d'évaluation pour comparer des modèles fournissant des prédictions avec un ensemble de réponses ordonné par leurs vraisemblances.
  \vfill
  \begin{table}
    \centering
    \begin{tabular}{c|c}
      \toprule
      Lettre & Réponses ordonnées\\
      \midrule
      F & 1.M 2.O 3.\textbf{G}\\
      M & 1.\textbf{N} 2.Y 3.X\\
      G & 1.A 2.\textbf{H} 3.E\\
      \bottomrule
    \end{tabular}
    \caption{Trois réponses d'un modèle devant prédire la lettre suivante dans l'alphabet.}
  \end{table}
\end{frame}

\begin{frame}{Rang réciproque moyen (MRR)}
  Le rang réciproque moyen (\emph{Mean Reciprocal Rank}) calcule un score simple proportionné à l'ordre de prédiction de chaque réponse :
  \begin{equation*}
    MRR = \frac{1}{|Q|}\sum_{i=1}^{|Q|}\frac{1}{\text{rank}_i}
  \end{equation*}
\end{frame}

\begin{frame}{Rang réciproque moyen (MRR)}
  \begin{table}
    \centering
    \begin{tabular}{c|c|c|c}
      \toprule
      Lettre & Réponses ordonnées & Rang & Score de rang\\
      \midrule
      F & 1.M 2.O 3.\textbf{G} & 3 & $\frac{1}{3}$\\
      M & 1.\textbf{N} 2.Y 3.X & 1 & $1$\\
      G & 1.A 2.\textbf{H} 3.E & 2 & $\frac{1}{2}$\\
      \bottomrule
    \end{tabular}
    \caption{Trois réponses d'un modèle devant prédire la lettre suivante dans l'alphabet.}
  \end{table}
  Score total moyen $= (\frac{1}{3}+1+\frac{1}{2})/3=\frac{11}{18}$.
\end{frame}

\begin{frame}{Précision pour $k$ (P@k)}
  Lorsqu'il n'existe pas une réponse unique, mais un ensemble de réponses acceptables :
  \begin{equation*}
    P@ k = \frac{\text{Nb valeurs valides sur les k premières réponses}}{k}
  \end{equation*}
  Peut-être moyenné pour donnée un score.
  \begin{table}
    \centering
    \begin{tabular}{c|c|c}
      \toprule
      Lettre & Réponses ordonnées & Score de rang\\
      \midrule
      F & 1.M 2.O 3.\textbf{G} & $0$\\
      M & 1.\textbf{N} 2.Y 3.X & $\frac{1}{3}$\\
      G & 1.\textbf{H} 2.\textbf{H} 3.E & $\frac{2}{3}$\\
      \bottomrule
    \end{tabular}
    \caption{P@3}
  \end{table}
\end{frame}

\begin{frame}{Gain cumulé (CG)}
  Le gain cumulé suppose de calculer un score de pertinence sur chaque réponse donnée $pert$. Ce score peut-être une simple constante lorsque le résultat est valide.

  En considérant les $k$ propositions :
  \begin{equation*}
    CG_k=\sum^k_{i=1} pert_i
  \end{equation*}

  Le gain cumulé ne prend en compte la position d'une réponse en fonction de sa pertinence.
\end{frame}

\begin{frame}{Gain cumulé réduit (DCG)}
  Le gain cumulé réduit pénalise le score si les résultats les plus pertinents sont les moins bien classés.
  \begin{equation*}
    DCG_k=\sum^k_{i=1} \frac{pert_i}{\log_2(i+1)} 
  \end{equation*}
\end{frame}

\begin{frame}{Gain cumulé réduit normalisé (nDCG)}
  Lorsque le nombre de résultats n'est pas fixe il est possible de calculer la version normalisé de la DCG :
  \begin{equation*}
    nDCG_k=\frac{DCG_k}{IDCG_k}
  \end{equation*}
  où $IDCG_k$ correspond au score du meilleur classement possible.
\end{frame}

\begin{frame}{Autres mesures : Perplexité}
  La \textbf{perplexité} mesure à quelle point un modèle est confiant dans la prédiction d'une séquence de taille $n$~:
  \begin{equation*}
    PP(w_1,\dots ,w_n) = \sqrt[n]{\frac{1}{\prod^n_{i=1}p(w_i)}}
  \end{equation*}
  ou sa version $\log$~:
  \begin{equation*}
    log\ PP(w_1,\dots ,w_n) = 2^{-\frac{1}{n}\sum_{i=1}^n \log_2\left(\frac{1}{p(w_i|w_0,\dots,w_{i-1})}\right)}
  \end{equation*}

  \begin{itemize}
    \item Surtout utilisée en prédiction de texte.
  \end{itemize}
\end{frame}

\begin{frame}{Évaluation intrinsèque et extrinsèque}
  \begin{itemize}
    \item \textbf{Évaluation intrinsèque} : Correspond au score de votre modèle sur vos données. 
    \item \textbf{Évaluation extrinsèque} : Évaluation effectuée sur une utilisation avancée du modèle (par un utilisateur du logiciel, par exemple).
  \end{itemize}
\end{frame}

\begin{frame}{A/B~Testing}
  Le A/B Testing permet d'evaluer un nouveau modèle de ML de façon transparente pour le client.

  Étant donné un modèle A classique déjà introduit et un modèle B sensé être plus performant qui devra, à terme, remplacer le modèle A :
  \begin{itemize}
    \item 95 \% d'utilisateurs restent avec le modèle standard A.
    \item 5 \% vont utiliser le nouveau modèle plus performant B.
  \end{itemize}
  Cela permet d'étudier les performances du nouveau modèle en le comparant à l'usage standard de l'ancien modèle.
\end{frame}

\begin{frame}{Multi-Armed Bandit (MAB)}
  Le Multi-Armed Bandit est la variation du A/B Testing qui permet de tester plusieurs modèles et configurations en parallèle.

  \begin{itemize}
    \item Adapte le nombre de personnes testant un modèle en fonction de ses résultats.
    \item Demande d'avoir une base d'utilisateurs important.
  \end{itemize}
\end{frame}

\begin{frame}{Slice-based learning}
  Le (\emph{Slice-based learning}) consiste à utiliser l'information experte de la criticité des données pour améliorer un apprentissage.
  
  \V{["img/NO-LICENCE-slice-learning-slices", "tw", 1] | image}

\end{frame}

\begin{frame}{Slice-based learning}
  Entraînement multi-objectifs via une supervision faible pour améliorer l'apprentissage.
  
  \V{["img/NO-LICENCE-slice-learning-architecture", "tw", 0.7] | image}

\end{frame}

\begin{frame}{Slice-based learning}
  Les sections critiques sont définis grâce à des \og \emph{observateurs}\fg{} simples permettant d'identifier des données critiques.

  Pour l'évaluation, il est ensuite possible de définir simplement les sections critiques dans les données pour définir des tests spécifiques.
\end{frame}

\begin{frame}{Travaux pratiques : A/B~Testing}
  \bluelink{https://colab.research.google.com/drive/1BDBsopKu4CUgYbAmFOw5sUs82JuKgh7p?usp=sharing\#forceEdit=true&sandboxMode=true}{A/B Testing}
\end{frame}




\begin{frame}{Comparaisons de modèles}
Selon la tache considérée, il peut être plus ou moins difficile d'évaluer les résultats fournis par les différentes métriques.

Une méthode simple est de interpréter les résultats est de comparer les métriques sur un autre modèle :
\begin{itemize}
  \item Modèle simple (Baseline)
  \item Performances humaines
  \item Gold standard
\end{itemize}
\end{frame}

\begin{frame}{Baseline}
La baseline consiste à avoir un modèle très simple qui produit des résultats suffisamment pour un coût très réduit :
\begin{itemize}
  \item Modèle aléatoire ou constant : météo $\rightarrow$ prédire le même temps que la veille
  \item Modèle de prédiction simple : réseau de neurones avec quelques couches, naive bayes, etc.
\end{itemize}

\textbf{Points importants}
\begin{itemize}
  \item[\textcolor{green}{+}] Coût de développement et d'utilisation faible. 
  \item[\textcolor{red}{-}] Peu précis. $\approx$ Borne inférieure
\end{itemize}
\end{frame}

\begin{frame}{Performances humaines}
  Un humain est en général capable de produire des résultats très bons sur certaines tâches (reconnaissance d'image, NLP, ...) et peut être utilisé comme modèle d'évaluation.

\textbf{Points importants}
\begin{itemize}
  \item[\textcolor{green}{+}] Très précis sur certaines tâches. Souvent considéré comme un Bayes optimal classifier 
  \item[\textcolor{red}{-}] Très coûteux et long.
  \item Souvent ils s'agit déjà du taggage sur les données utilisées. 
\end{itemize}
\end{frame}

\begin{frame}{Gold standard}
  Le gold standard est, soit le modèle le plus performant sur la tâche à accomplir (modèle \emph{état de l'art}), soit le meilleur modèle théorique possible.

\textbf{Points importants}
\begin{itemize}
  \item[\textcolor{green}{+}] Le meilleur modèle existant. Borne maximale théorique actuelle.
  \item[\textcolor{red}{-}] Difficilement dépassable sans faire de la recherche.
  \item S'il existe autant l'utiliser que d'en refaire un moins performant.
\end{itemize}
\end{frame}
