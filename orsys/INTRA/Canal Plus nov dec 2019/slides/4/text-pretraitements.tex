\documentclass{formation}
\title{Big Data Analytics}
\subtitle{Prétraitements}

\begin{document}

\maketitle

\begin{frame}
  \frametitle{Prétraitements : Collecte}
  Des sources variées :
  \begin{itemize}
  \item Wikipedia
  \item Articles de journaux
  \item Littérature
  \item User Generated Content
    \begin{itemize}
    \item Blogs
    \item Commentaires
    \item Réseaux sociaux
    \end{itemize}
  \end{itemize}
  Une source $\Rightarrow$ un ``web scraper''
\end{frame}

\begin{frame}
  \frametitle{Prétraitements : Tokenisation}
  Séparer une chaine de caractères en token n'est pas trivial : \\
  \newline
  \begin{center}
    Le Dr. Pond èleve des poules. L'éleveur les sur-exploite.
  \end{center}
  $\;$ \\
  (en phrases ou en mots)
\end{frame}

\begin{frame}
  \frametitle{Prétraitements : POS-Tagging}
  Étiquetage Morpho-Syntaxique
  \imgtw[0.8]{pos-tagging}
\end{frame}

\begin{frame}
  \frametitle{Prétraitements : NER}
  Reconnaissance d'Entités Nommées
  \imgtw[0.8]{named-entity-recognition}
\end{frame}

\begin{frame}
  \frametitle{Prétraitements : Parsing tree}
  Arbre Syntaxique
  \imgtw[0.8]{parse-tree}
\end{frame}

\begin{frame}
  \frametitle{Prétraitements : Lemmatisation}
  Exprimer les mots sous leur forme canonique : \\

  \begin{minipage}[l]{0.19\linewidth}
    jouant \\
    ont été jouées \\
    étoiles \\
    claires \\
    noire \\
  \end{minipage}\hfill
  \begin{minipage}[l]{0.80\linewidth}
  $\Rightarrow$ jouer \\
  $\Rightarrow$ jouer \\
  $\Rightarrow$ étoile \\
  $\Rightarrow$ clair \\
  $\Rightarrow$ noir  \\
  \end{minipage}\hfill
  ...
\end{frame}

\begin{frame}
  \frametitle{Outlis : WordNet}
  \imgth[0.7]{wordnet-1}  
  Projet sur le français : WOLF (Wordnet Libre du Français) 
\end{frame}

\begin{frame}
  \frametitle{Outils : DBpedia}
  \imgtw[0.9]{dbpedia}
\end{frame}

\begin{frame}
  \frametitle{Outils : DBpedia}
  \imgtw[0.9]{chatbot}
\end{frame}

\begin{frame}
  \frametitle{Outils : DBpedia}
  \imgtw[0.9]{usine-a-gaz}
\end{frame}

\end{document}
%%% Local Variables:
%%% mode: latex
%%% TeX-master: t
%%% End:
