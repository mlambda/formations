
\begin{frame}{Description}
  L'Intelligence Artificielle a bouleversé de nombreux domaines scientifiques et révolutionné un grand nombre de secteurs économiques.
  Néanmoins, sa présentation dans les grands médias relève souvent du fantasme, très éloignée de ce que sont réellement les domaines du Machine Learning ou du Deep Learning.
  Ce séminaire vous permettra de maîtriser les concepts clé du Deep Learning et de ses différents domaines de spécialisation.
  Vous découvrirez également les principales architectures de réseau existant aujourd'hui.
\end{frame}

\begin{frame}{Prérequis}
  \begin{itemize}
    \item Avoir des bases en programmation
    \item Avoir une bonne maîtrise des outils informatiques et des statistiques
  \end{itemize}
\end{frame}

\begin{frame}{Objectifs pédagogiques}
  \begin{itemize}
    \item Comprendre les clés fondamentales d'une approche Machine ou Deep Learning
    \item Maîtriser les bases théoriques et pratiques d'architecture et de convergence de réseaux de neurones
    \item Connaître les différentes architectures fondamentales existantes et maîtriser leurs implémentations fondamentales
    \item Maîtriser les méthodologies de mise en place de réseaux de neurones, les points forts et les limites de ces outils
  \end{itemize}
\end{frame}

\begin{frame}{Programme}
  \begin{itemize}
    \item Introduction IA, Machine Learning et Deep Learning
    \item Concepts fondamentaux d'un réseau de neurones
    \item Outils usuels Machine Learning et Deep Learning
    \item Convolutional Neural Networks (CNN)
    \item Interpretation de Modèles
    \item Détection d'anomalies dans des images
    \item Recurrent Neural Networks (RNN)
    \item Modèles générationnels : VAE et GAN
  \end{itemize}
\end{frame}
