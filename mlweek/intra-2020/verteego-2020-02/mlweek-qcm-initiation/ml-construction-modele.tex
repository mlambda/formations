\documentclass{formation}
\title{Machine Learning}
\subtitle{Machine Learning}

\begin{document}

\maketitle

\begin{frame}
  \frametitle{Machine Learning}
  \imgtw{data-science}
\end{frame}

\begin{frame}
  \frametitle{Machine Learning}
  \imgth{ia-ml-deep}
\end{frame}

\begin{frame}
  \frametitle{Machine Learning}
  \begin{center}
  Nouvelle manière d'aborder la \textbf{conception logicielle}.
  \end{center}
  \begin{center}
  $\boxed{Programmation\;Implicite \neq Programmation\;Explicite}$
  \end{center}
\end{frame}

\begin{frame}
  \frametitle{Machine Learning}
  \underline{Définition du besoin} :
  \newline
  \newline
  Apprentissage \textbf{supervisé} ou \textbf{non-supervisé} ?
\end{frame}

\begin{frame}
  \frametitle{Machine Learning}
  \textbf{Apprentissage supervisé}
  \newline \newline
  \textbf{Prédire} une valeur numérique ou l'appartenance à une classe\\
  \newline
  Données d'entrainement \textbf{annotées} ! \\
  \newline
  Ex : prédire une note sur un film (Netflix)
\end{frame}

\begin{frame}
  \frametitle{Machine Learning}
  \textbf{Apprentissage non-supervisé}
  \newline \newline
  Faire émerger des profils, des groupes \\
  \newline
  Ex : groupes de clients pour adapter sa stratégie marketing \\
\end{frame}

\begin{frame}
  \frametitle{Machine Learning}
  \imgtw[0.8]{ml-craftmanship}
\end{frame}

\end{document}
%%% Local Variables:
%%% mode: latex
%%% TeX-master: t
%%% End:
