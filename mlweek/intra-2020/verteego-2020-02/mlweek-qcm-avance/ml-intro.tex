\documentclass{formation}
\title{Machine Learning}
\subtitle{Machine Learning}

\begin{document}

\maketitle

\begin{frame}
  \frametitle{Machine Learning}
  \imgtw{data-science}
\end{frame}

\begin{frame}
  \frametitle{Machine Learning}
  \imgth[0.8]{ia-ml-deep}
\end{frame}

\begin{frame}
  \frametitle{Machine Learning}
  \begin{center}
  Nouvelle manière d'aborder la \textbf{conception logicielle}.
  \end{center}
  \begin{center}
  $\boxed{Programmation\;Implicite \neq Programmation\;Explicite}$
  \end{center}
\end{frame}

\begin{frame}
  \frametitle{Machine Learning}
  \imgtw[0.8]{ml-craftmanship}
\end{frame}

\begin{frame}
  \frametitle{Machine Learning}
  \underline{Définition du besoin} :
  \newline
  \newline
  Apprentissage \textbf{supervisé} ou \textbf{non-supervisé} ?
\end{frame}

\begin{frame}
  \frametitle{Machine Learning}
  \textbf{Apprentissage non-supervisé}
  \newline \newline
  Faire émerger des profils, des groupes \\
  \newline
  Ex : groupes de clients pour adapter sa stratégie marketing \\
\end{frame}

\begin{frame}
  \frametitle{Machine Learning}
  \textbf{Apprentissage supervisé}
  \newline \newline
  \textbf{Prédire} une valeur numérique (\textbf{régression}) ou l'appartenance à une classe (\textbf{Classification}) \\
  \newline
  Ex (Régression)     : Prédire le poid d'un individu en fonction de l'âge, la taille et le sexe. \\
  Ex (Classification) : Prédire si une image est un chat ou un chien. \\
\end{frame}

\begin{frame}
  \frametitle{Machine Learning}
  Régression Linéaire : \\
  $Y = a*X + b$
  \imgth[0.7]{regression-line-1}
\end{frame}

\begin{frame}
  \frametitle{Machine Learning}
  Régression Linéaire pour des données en plusieurs dimensions : \\
  $Y = a*X_1 + b*X_2 + c$
  \imgth[0.8]{regression-hyperplan}
\end{frame}

\begin{frame}
  \frametitle{Machine Learning}
  \imgtw[0.8]{variables}
\end{frame}

\begin{frame}
  \frametitle{Machine Learning}
  Régression Logistique (Classification)
  \imgtw[0.65]{mnist}
\end{frame}

\begin{frame}
  \frametitle{Machine Learning}
  Régression Logistique (Classification)
  \imgtw[0.8]{caltech256}
\end{frame}

\begin{frame}
  \frametitle{Machine Learning}
  \textbf{Apprentissage par Renforcement}
  \newline \newline
  Apprendre une \textbf{stratégie} efficace dans un \textbf{univers} où les \textbf{actions} fournissent des \textbf{récompenses} (possiblement négatives)
  \newline
  Ex : Les échecs, le Go , conduire une voiture,...
\end{frame}

\begin{frame}
  \frametitle{Machine Learning}
  \imgtw[0.8]{ml-illustration}
\end{frame}

\end{document}
%%% Local Variables:
%%% mode: latex
%%% TeX-master: t
%%% End:
