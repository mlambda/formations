\documentclass{formation}
\title{Machine Learning}
\subtitle{Données séquentielles - Premières approches }

\begin{document}

\maketitle

\begin{frame}
  \frametitle{Données séquentielles - Premières approches }
  \begin{itemize}
  \item Calcul des différences du premier ou second ordre.
  \item Modèles autoregressifs à moyenne mobile (ARMA, ARIMA, ...)
  \item Modèles de Markov 
  \end{itemize}
\end{frame}

\begin{frame}
  \frametitle{Données séquentielles - Premiers outils}
  Chronogramme : tracé de t $\rightarrow$ X(t) 
  \imgtw[0.9]{time-serie-chronogramme}
\end{frame}

\begin{frame}
  \frametitle{Données séquentielles - Premiers outils}
  Lag-plot : tracé des points (X(t-k),X(t)) \\
  (détection de corrélation temporelles)
  \begin{minipage}[l]{0.499\linewidth}
    \imgtw[0.8]{lag-plot-linear}
    \imgtw[0.8]{lag-plot-sinusoidal}
  \end{minipage}\hfill
  \begin{minipage}[l]{0.499\linewidth}
    \imgtw[0.7]{lag-plot-white-noise}
    \imgtw[0.75]{lag-plot-uniform-noise}
  \end{minipage}\hfill
\end{frame}

\begin{frame}
  \frametitle{Données séquentielles - Premières approches }
  $x_t=g(t)+\phi_t$ \\
  où :
  \begin{itemize}
  \item $g(t)$ est déterministe (tendance globale du signal)
  \item $\phi(t)$ est un bruit stochastique
  \end{itemize}
  Les modèles statistiques ``classiques'' vont alors essayer de décomposer $g(t)$ en plusieurs fonctions dépendant de t ou bien de fonction récurrentes (comme dans ARMA)
\end{frame}

\begin{frame}
  \frametitle{Données séquentielles - Premières approches }
  \imgth[0.9]{time-serie-decomposition}
\end{frame}

\end{document}
%%% Local Variables:
%%% mode: latex
%%% TeX-master: t
%%% End:
