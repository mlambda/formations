\documentclass{formation}
\title{Machine Learning}
\subtitle{Installation de Anaconda}

\begin{document}

\maketitle

\begin{frame}
  \frametitle{Installation de Anaconda}
  Anaconda est une distribution Python faite pour la ``Data Science'' \\
  \imgtw[0.7]{anaconda_logo}
\end{frame}

\begin{frame}
  \frametitle{Installation de Anaconda}
  Ce qui sera installé après ce tutoriel :
  \begin{itemize}
  \item Python
  \item Jupyter Notebook
  \item Des librairies de ``Data Science'' :
    \begin{itemize}
    \item SciPy, Numpy
    \item Pandas, seaborn
    \item scikit-learn, statsmodels
    \item Matplotlib
    \end{itemize}
  \end{itemize}
\end{frame}

\begin{frame}
  \frametitle{Installation de Anaconda}
  \blue{\underline{\href{https://docs.anaconda.com/anaconda/install/}{Installation d'Anaconda}}}
\end{frame}

\begin{frame}
  \frametitle{Installation de Anaconda}
  Une fois installé, lancez le navigateur Anaconda puis cliquez sur ``Jupyter''. \\
  Chargez le fichier ``Anaconda.ipynb'' que vous trouverez dans le dossier ``ressources/''. \\
  Editez et Exécutez les cellules pour prendre en main cet environnement de développement.
\end{frame}

\end{document}
%%% Local Variables:
%%% mode: latex
%%% TeX-master: t
%%% End:
