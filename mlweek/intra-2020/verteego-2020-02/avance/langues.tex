\documentclass{formation}
\title{Machine Learning}
\subtitle{Données à dimension variable : Traitement du langage}

\begin{document}

\maketitle

\begin{frame}
  \frametitle{Traitement du langage : données}
  Classification :
  \begin{itemize}
  \item thème/genre (gutenberg.org : 57k livres)
  \item auteur (gutenberg.org : 57k livres)
  \item sentiment (Kaggle movie review : 222k commentaires rotten tomatoes)
  \item reconnaissance d'entités nommées (Kaggle Annotated Corpus for NER : 1.3M tags)
  \item ...
  \end{itemize}
\end{frame}

\begin{frame}
  \frametitle{Traitement du langage : données}
  Compréhension :
  \begin{itemize}
  \item Question/réponses (SQUAD : 150k questions)
  \item Traduction (europarl : 450k phrases alignées)
  \item ...
  \end{itemize}
\end{frame}

\begin{frame}
  \frametitle{Traitement du langage : word embeddings} 
  mot = indice dans un dictionnaire (dimension > 30000)
  \newline
  mot = vecteur ``sémantique'' (dimension < 1000) 
  \begin{itemize}
  \item word2vec
  \item CBOW/Skip-Gram
  \item Thought vector (pour des phrases ou même des documents entiers)
  \item ...
  \end{itemize}
\end{frame}

\begin{frame}
  \frametitle{Traitement du langage : word embeddings} 
  \imgtw{word2vec}
\end{frame}

\begin{frame}
  \frametitle{Traitement du langage : word embeddings} 
  \imgtw[0.6]{word2vec-analogy}
  \begin{center}
    \href{https://projector.tensorflow.org/}{\blue{Visualisation de l'espace word2vec}}
  \end{center}
\end{frame}

\begin{frame}
  \frametitle{Traitement du langage : Champs d'application}
  \imgtw[1]{nlp-tech-resume-2019}
\end{frame}

\end{document}
%%% Local Variables:
%%% mode: latex
%%% TeX-master: t
%%% End:
