\documentclass{formation}
\title{Les acteurs de l'intelligence artificielle}
\subtitle{dans le monde (et un peu en france)}

\begin{document}

\maketitle

\begin{frame}
  \frametitle{Les acteurs de l'intelligence artificielle}
  \imgtw{trends-ia-tektonik}
\end{frame}

\begin{frame}
  \frametitle{Les acteurs de l'intelligence artificielle}
  \imgtw{ia-landscape-2018}
\end{frame}

\begin{frame}
  \frametitle{Les acteurs de l'intelligence artificielle}
  \imgtw[1.0]{ia-landscape-fr-2018-2}
\end{frame}

\begin{frame}
  \frametitle{Les acteurs de l'intelligence artificielle}
  Top 10 :
  \begin{itemize}
  \item Algolia (moteur de recherche BtoB)
  \item Sophia Genetics (Détecter des Mutations dans l'ADN)
  \item Shift Technology (Détection de fraude à l'assurance)
  \item Navya (Véhicules autonomes)
  \item Prophesee (Traitement d'image)
  \item Vulog (logiciel d'autopartage)
  \item Tinyclues (Marketing predictif)
  \item Linkfluence (Social media intelligence)
  \item Teemo (``Criteo'' pour magasins physiques)
  \item Vekia (Gestion de stocks et logistique)
  \end{itemize}
\end{frame}

\begin{frame}
  \frametitle{Les acteurs de l'intelligence artificielle}
  La stratégie de recherche en IA en France :
  \begin{itemize}
  \item 1.5 Millards d'euros sur 5 ans
  \item 4 Institus interdisciplinaires (santé, environnement, énergie, transport et dévelppement des territoires) dirigés par l'INRIA
  \item Attirer et garder les talents (doubler le nombre de doctorants + recrutement de chercheurs)
  \item Soutien aux projets liés à l'IA (100 Millions d'euro)
  \item Un supercalculateur dédié pour tous les chercheurs
  \item 130 Millions d'euros pour des partenariat public-privé
  \item Coopération avec l'Allemagne, l'Europe et le reste du monde
  \end{itemize}
\end{frame}


\end{document}
%%% Local Variables:
%%% mode: latex
%%% TeX-master: t
%%% End:
