\documentclass{formation}
\title{Machine Learning}
\subtitle{Introduction}

\begin{document}

\maketitle

\begin{frame}
  \frametitle{Machine Learning}
  \begin{center}
  Nouvelle manière d'aborder la \textbf{conception logicielle}.
  \end{center}
  \begin{center}
  $\boxed{Programmation\;Implicite \neq Programmation\;Explicite}$
  \end{center}
\end{frame}

\begin{frame}
  \frametitle{Machine Learning}
  \imgtw[0.8]{ml-craftmanship}
\end{frame}

\begin{frame}
  \frametitle{Machine Learning}
  \underline{Définition du besoin} :
  \newline
  \newline
  Apprentissage \textbf{supervisé} ou \textbf{non-supervisé} ?
\end{frame}

\begin{frame}
  \frametitle{Machine Learning}
  \textbf{Apprentissage supervisé}
  \newline \newline
  \textbf{Prédire} une valeur numérique ou l'appartenance à une classe
  \newline
  Données d'entrainement \textbf{annotées} !
  \newline \newline
  Ex : prédiction CAC40, classification d'image/texte/...
\end{frame}

\begin{frame}
  \frametitle{Machine Learning}
  \textbf{Apprentissage non-supervisé}
  \newline \newline
  Faire émerger des profils, des groupes
  \newline \newline
  Ex : groupes de clients pour adapter sa stratégie marketing
\end{frame}

\begin{frame}
  \frametitle{Machine Learning}
  \begin{minipage}[c]{0.41\linewidth}
    À l'intérieur du \textbf{Modèle}:
    \begin{itemize}
    \item \textbf{Probabilités}
    \item \textbf{Statistiques}
    \item Algèbre linéaire
    \item Théorie de l'Optimisation
    \item Calcul différentiel
    \end{itemize}
  \end{minipage}\hfill
  \begin{minipage}[c]{0.58\linewidth}
    \imgtw[1]{gaussian-full}
  \end{minipage}\hfill
\end{frame}

\begin{frame}
  \frametitle{Machine Learning}
  \begin{minipage}[c]{0.41\linewidth}
    À l'intérieur du \textbf{Modèle}:
    \begin{itemize}
    \item Probabilités
    \item Statistiques
    \item \textbf{Algèbre linéaire}
    \item Théorie de l'Optimisation
    \item Calcul différentiel
    \end{itemize}
  \end{minipage}\hfill
  \begin{minipage}[c]{0.58\linewidth}
    \imgtw[1]{linear-algebra-matrices}
  \end{minipage}\hfill
\end{frame}

\begin{frame}
  \frametitle{Machine Learning}
  \begin{minipage}[c]{0.45\linewidth}
    À l'intérieur du \textbf{Modèle}:
    \begin{itemize}
    \item Probabilités
    \item Statistiques
    \item Algèbre linéaire
    \item \textbf{Théorie de l'Optimisation}
    \item Calcul différentiel
    \end{itemize}
  \end{minipage}\hfill
  \begin{minipage}[c]{0.55\linewidth}
    \imgtw[1]{surface-parametre-erreur1}
  \end{minipage}\hfill
\end{frame}

\begin{frame}
  \frametitle{Machine Learning}
  \begin{minipage}[c]{0.45\linewidth}
    À l'intérieur du \textbf{Modèle}:
    \begin{itemize}
    \item Probabilités
    \item Statistiques
    \item Algèbre linéaire
    \item \textbf{Théorie de l'Optimisation}
    \item Calcul différentiel
    \end{itemize}
  \end{minipage}\hfill
  \begin{minipage}[c]{0.55\linewidth}
    \imgtw[0.70]{curve-fitting-fun}
  \end{minipage}\hfill
\end{frame}

\begin{frame}
  \frametitle{Machine Learning}
  \begin{minipage}[c]{0.41\linewidth}
    À l'intérieur du \textbf{Modèle}:
    \begin{itemize}
    \item Probabilités
    \item Statistiques
    \item Algèbre linéaire
    \item Théorie de l'Optimisation
    \item \textbf{Calcul différentiel}
    \end{itemize}
  \end{minipage}\hfill
  \begin{minipage}[c]{0.58\linewidth}
    \imgtw[0.7]{derive-tangente}
  \end{minipage}\hfill
\end{frame}

\end{document}
%%% Local Variables:
%%% mode: latex
%%% TeX-master: t
%%% End:
