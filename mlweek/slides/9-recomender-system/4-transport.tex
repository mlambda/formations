\documentclass{formation}
\title{Secteurs en mouvement}
\subtitle{Le Transport}

\begin{document}

\maketitle

\begin{frame}
  \frametitle{La voiture autonome}
  \imgtw[0.8]{voiture-autonome-conducteur}
\end{frame}

\begin{frame}
  \frametitle{La voiture autonome}
  \begin{center}
    Atouts : Peu de degrés de liberté
    \newline
    Inconvéniants : Hétérogénéité de l'environnement
  \end{center}
  \imgtw[1]{voiture-autonome-navya}
\end{frame}

\begin{frame}
  \frametitle{La voiture autonome}
  Plusieurs niveaux d'autonomie (Society of Automotive Engineers) :
  \begin{itemize}
  \item[1:] Aide à la conduite (Régulateur de vitesse, freinage d'urgence, ...)
  \item[2:] Automatisation partielle (Régulateur d'embouteillage, assistant de parking, ...)
  \item[3:] Automatisation situationnelle avec surveillance humaine (pilote d'autoroute, parking automatique, ...)
  \item[4:] Automatisation situationnelle sans surveillance humaine
  \item[5:] Autonomie complète
  \end{itemize}
\end{frame}

\begin{frame}
  \frametitle{La voiture autonome}
  \imgtw[0.8]{voiture-autonome}
\end{frame}

\begin{frame}
  \frametitle{La voiture autonome}
  DARPA Grand Challenge (2004)
  \newline
  150 miles dans le désert du Nevada
  \newline
  Aucun gagnant !
  \newline
  5/23 arrivent au bout en 2005
  \imgtw[0.6]{darpa-grand-challenge}
\end{frame}

\begin{frame}
  \frametitle{La voiture autonome}
  DARPA Urban Challenge (2007)
  \newline
  60 miles en milieu urbain + respect du code de la route (sans gps)
  \newline
  6/35 franchissent la ligne d'arrivée
  \imgtw[0.8]{darpa-urban-challenge}
\end{frame}

\begin{frame}
  \frametitle{La voiture autonome}
  Google Car (2010)
  \newline
  Projet visant une voiture autonome de niveau 5
  \imgtw[0.8]{google-car}
\end{frame}

\begin{frame}
  \frametitle{La voiture autonome}
  Tesla Autopilot (2015)
  \newline
  Uniquement avec caméra, pas de LIDAR
  \newline
  \imgtw[0.6]{tesla-autopilot}
\end{frame}

\begin{frame}
  \frametitle{La voiture autonome}
  Waymo One déployé à Phenix (5 décembre 2018)
  \imgtw[0.8]{waymo-one}
\end{frame}

\begin{frame}
  \frametitle{La voiture autonome}
  4 accident mortels implicant des voitures autonomes (3 Tesla, 1 Uber)
  \newline
  Au 7 mai 2016 -2ème accident- Autopilot avait parcourus 210 Millions de kilomètres
  \newline
  1 seul (Uber) ou le pilote automatique est mis en cause
  \imgtw[0.7]{uber-crash-velo}
\end{frame}

\begin{frame}
  \frametitle{La voiture autonome}
  \imgtw[1]{voiture-autonome-acteurs}
\end{frame}

\begin{frame}
  \frametitle{La voiture autonome}
  \imgtw[1]{voiture-autonome-changements}
\end{frame}

\begin{frame}
  \frametitle{La voiture autonome}
  L'IA pour les véhicules, pas uniquement de la conduite automatisée :
  \begin{itemize}
  \item Optimisation des trajets (Uber, Watson on Wheels, Waze)
  \item Maintenance préventive par analyse des vibrations
  \end{itemize}
\end{frame}

\begin{frame}
  \frametitle{Le monde du rail}
  \imgtw[0.8]{train-aiguillage}
\end{frame}

\begin{frame}
  \frametitle{Le monde du rail}
  \begin{itemize}
  \item Prévision de retard et replanification
  \item Conduite
  \item Maintenance préventive par classification d'images (rails, caténaires, isolateurs, materiel roulant, végétation ...)
  \item Relation client
  \end{itemize}
  \imgtw[0.8]{rail-detect-defauts}
\end{frame}

\begin{frame}
  \frametitle{L'Aéronautique}
  \begin{minipage}[c]{0.49\linewidth}
    \begin{itemize}
    \item Production des pièces
    \item Contrôle qualité des pièces
    \item Maintenance préventive
    \item Pilote automatique
    \item Prédiction de retard
    \item Optimisation des flux
    \item Relation Client
    \item Gestion du personnel
    \end{itemize}
  \end{minipage}\hfill
  \begin{minipage}[c]{0.49\linewidth}
    \imgtw[0.8]{avion-intelligent}
  \end{minipage}\hfill
\end{frame}

\end{document}
%%% Local Variables:
%%% mode: latex
%%% TeX-master: t
%%% End:
