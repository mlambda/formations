\documentclass{formation}
\title{Machine Learning}
\subtitle{Statistiques}

\begin{document}

\begin{frame}
  \frametitle{Statistiques}
  Les différents types de variables :
  \imgtw{variable-types}
\end{frame}

\begin{frame}
  \frametitle{Statistiques}
  \textbf{Moyenne}\\
  $\mu(X) = \frac{1}{N}\sum_{N}X_i$ \\ \\
  \textbf{Variance}\\
  $V(X)={\mu \left[\left(X-\mu [X]\right)^{2}\right]}=\mu[X^{2}]-\mu[X]^{2}$ \\ \\
  \textbf{Ecart-type}\\
  $\sigma(X)=\sqrt{V(X)}$
\end{frame}

\begin{frame}
  \frametitle{Statistiques}
  \imgtw[0.8]{bias-variance}
\end{frame}

\begin{frame}
  \frametitle{Statistiques}
  \imgtw{gaussian-full}
\end{frame}

\begin{frame}
  \frametitle{Probabilités}
  \imgtw[0.8]{probabilite-jointe}
\end{frame}


\begin{frame}
  \frametitle{Statistiques}
  On cherche les lois de probabilités qui générent (au mieux) nos données.\\
  4 phases :
  \begin{itemize}
  \item Identifier une question ou un problème
  \item Collecter des données
  \item Analyser les données(moyenne, biais, variance, covariance, intervales de confiance...)
  \item Conclure
  \end{itemize}
\end{frame}

\begin{frame}
  \frametitle{Machine Learning}
  On cherche un modèle qui approxime (au mieux) le comportement d'un ``expert'' sur des données.\\
  \imgtw[0.8]{ml-craftmanship}
\end{frame}

\begin{frame}
  \frametitle{Statistiques \& Machine Learning}
  Les problématiques :
  \begin{itemize}
  \item Les données marginales
  \item Données biaisées
  \item Dépendance statistique $\neq$ causalité 
  \item Pas assez de données ?
  \item ...
  \end{itemize}
\end{frame}

\end{document}
%%% Local Variables:
%%% mode: latex
%%% TeX-master: t
%%% End:
