\documentclass{formation}
\title{Machine Learning}
\subtitle{Deep Q-Networks}

\begin{document}

\maketitle

\begin{frame}
  \frametitle{DQN}
  \begin{itemize}
  \item 49 jeux Atari
  \item 210x160 pixels
  \item 8 à 16 actions
  \end{itemize}
  \imgtw[0.6]{atari-openia-gym-game}
\end{frame}

\begin{frame}
  \frametitle{DQN}
  \begin{itemize}
  \item input 84x84x4
  \item Récompense [-1,0,+1] suivant les variations de score
  \item Même préprocessing pour tous les jeux. (50M frames $\approx$ 38 jours de jeux)
  \item 29/46 jeux : DQN $\geq$ Humain
    \begin{itemize}
    \item dont 22 où l'IA est strictement supérieure
    \item les humains ont 2H d'entrainement par jeu
    \end{itemize}
  \end{itemize}
\end{frame}

\begin{frame}
  \frametitle{DQN}
  \href{https://www.youtube.com/watch?v=lcVg9hVya-c}{\blue{Démo DQN PONG}}
\end{frame}

\end{document}
%%% Local Variables:
%%% mode: latex
%%% TeX-master: t
%%% End:
