\documentclass{formation}
\title{Machine Learning}
\subtitle{Algèbre Linéaire}

\begin{document}

\maketitle


\begin{frame}
  \frametitle{Algèbre Linéaire : les bases}
  Les vecteurs :
  \begin{displaymath}
    v = 
    \begin{pmatrix}
      v_1\\
      v_2\\
      \vdots\\
      v_n
    \end{pmatrix}
    \in \mathbb{R}^n
  \end{displaymath}
\end{frame}

\begin{frame}
  \frametitle{Algèbre Linéaire : les bases}
    \begin{displaymath}
        u + v = 
      \begin{pmatrix}
        u_1 + v_1\\
        u_2 + v_2\\
        \vdots\\
        u_n + v_n
      \end{pmatrix}
    \end{displaymath}
\end{frame}

\begin{frame}
  \frametitle{Algèbre Linéaire : les bases}
    \begin{displaymath}
      \alpha v =
      \begin{pmatrix}
        \alpha v_1\\
        \alpha  v_2\\
        \vdots\\
        \alpha v_n
      \end{pmatrix}
       \qquad (\alpha\in\mathbb{R})
    \end{displaymath}
\end{frame}

\begin{frame}
  \frametitle{Algèbre Linéaire : les bases}
  La norme d'un vecteur :
    \begin{displaymath}
      \parallel v\parallel = \sqrt{v_1^2 + \cdots + v_n^2}
    \end{displaymath}
\end{frame}

\begin{frame}
  \frametitle{Algèbre Linéaire : les bases}
  Le produit scalaire :
    \begin{align*}
      u\cdot v & = u_1\cdot v_1 + \dotsb + u_n \cdot v_n \\[4mm]
      & = \parallel u\parallel \parallel v\parallel \cos\theta
    \end{align*}
\end{frame}

\begin{frame}
  \frametitle{Algèbre Linéaire : les bases}
  Les matrices :
    \begin{align*}
      A & = 
      \begin{bmatrix}
        a_{1,1} & a_{1,2} & a_{1,3} \\
        a_{2,1} & a_{2,2} & a_{2,3} \\
        a_{3,1} & a_{3,2} & a_{3,3}
      \end{bmatrix} 
      =
      \begin{pmatrix}
        a_{1,1} & a_{1,2} & a_{1,3} \\
        a_{2,1} & a_{2,2} & a_{2,3} \\
        a_{3,1} & a_{3,2} & a_{3,3}
      \end{pmatrix} \\[4mm]
      & =
      \begin{Bmatrix}
        a_{1,1} & a_{1,2} & a_{1,3} \\
        a_{2,1} & a_{2,2} & a_{2,3} \\
        a_{3,1} & a_{3,2} & a_{3,3}
      \end{Bmatrix}
      \in \mathbb{R}^{n\times n}
    \end{align*}
\end{frame}

\begin{frame}
  \frametitle{Algèbre Linéaire : les bases}
  Addition :
  \imgtw[0.8]{addition-matrices}
\end{frame}


\begin{frame}
  \frametitle{Algèbre Linéaire : les bases}
  Multiplication :
  \imgtw[0.8]{multiplication-matrices}
  \begin{center}
  \red{\textbf{$\boxed{A*B \neq B*A}$}}
  \end{center}
\end{frame}

\begin{frame}
  \frametitle{Algèbre Linéaire : les bases}
  Transposition :
  \imgtw[0.8]{transposition-matrices}
\end{frame}

\begin{frame}
  \frametitle{Algèbre Linéaire : les bases}
  Inverse : \\
  \[
  A*A^{-1} = A^{-1}*A =
  \begin{pmatrix}
    1 & 0 & \cdots & 0 \\
    0 & \ddots & \ddots & \vdots \\
    \vdots & \ddots & \ddots & 0\\
    0 & \cdots & 0 & 1
  \end{pmatrix}
  \]
\end{frame}

\begin{frame}
  \frametitle{Algèbre Linéaire : les bases}
    $B$ est une base de $V$ ssi (au choix):
    \begin{itemize}
    \item $B$ est générateur de $V$ et de taille minimum
    \item $B$ est composé de vecteurs linéairements indépendants et de taille maximum
    \item $\forall v\in V \;\; \exists! \{{w_i}\}_{i\in\mathbb{K}}$ tels que $v = \sum_{i\in\mathbb{K}} w_i . b_i$ où K est minimal
    \end{itemize}
\end{frame}

\begin{frame}
  \frametitle{Algèbre Linéaire : transformations}
  Une matrice A est une transformation linéaire de $\mathbb{R}^n \rightarrow \mathbb{R}^n$ telle que : 
  \begin{center}
    $\forall x \in \mathbb{R}^n,\;y = A*x$\\
  \end{center}
  \begin{minipage}[r]{0.25\linewidth}
  $n=2$\\
  \\
  $n=2$\\
  \\
  \\
  $n=3$\\
  \\
  \\
  $n=3$\\
  \end{minipage}\hfill
  \begin{minipage}[l]{0.74\linewidth}
  \imgth[0.7]{linear-algebra-matrices}
  \end{minipage}\hfill
\end{frame}

\begin{frame}
  \frametitle{Algèbre Linéaire : transformations}
  Définition des valeurs propres et vecteurs propres:\\
  $Av = \lambda v$ \\
  $Av = \lambda I_k v \;\iff\; (A-\lambda I_k)v = 0$
\end{frame}

\begin{frame}
  \frametitle{Algèbre Linéaire : transformations}
  Les transformations linéaires en vidéo : \\
             {\textbf{video time}}
\end{frame}

\begin{frame}
  \frametitle{Algèbre Linéaire : transformations}
      Certaines matrices sont diagonalisables, ainsi :
      \begin{align*}
        A = Q D Q^{-1} \hspace{1cm} \mbox{ avec } D & =
        \begin{bmatrix}
          \lambda_1 & \cdots & 0 \\
          \vdots & \ddots & 0 \\
          0 & 0 & \lambda_n
        \end{bmatrix} \\[5mm]
        %
        \mbox{et } Q & =
        \begin{bmatrix}
          \mid && \mid \\
          v_1 & \cdots & v_k \\
          \mid && \mid
        \end{bmatrix}
      \end{align*}
\end{frame}

\end{document}
%%% Local Variables:
%%% mode: latex
%%% TeX-master: t
%%% End:
