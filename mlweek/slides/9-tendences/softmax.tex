
\begin{frame}
  \frametitle{Gradient et mise à jour}
  Exemple sur une Sortie Softmax :
  \[
  \forall x_i \in C,\; S(x_i) = \frac{e^{x_i}}{\sum_{x_j \in C}e^{x_j}}
  \]
\end{frame}

\begin{frame}
  \frametitle{Gradient et mise à jour}
  \begin{center}
    \underline{Softmax $\Rightarrow$ loss = cross-entropie + L2(W)}
  \end{center}
  loss :
  \[
  L(y^*,y) = H(y^*,y) + \frac{1}{2}\lambda\sum_{i \in P}w_i^2
  \]
  \[
  H(y^*,y) = -\sum_{(i \in I)}y_i*ln(y_i)
  \]
  gradient :
  \[
  \frac{\partial{L(x_i)}}{\partial{w_i}} = ( S(x_i) - \delta_{i=i^*} )X + \lambda Wi
  \]
\end{frame}

\begin{frame}
  \frametitle{Gradient et mise à jour}
  Learning rate evolutif et adapté :
  \begin{itemize}
  \item SGD : $grad = \alpha*grad_{t-1}(x)+\beta*grad_t \;\;\;$où $\alpha$ et $\beta$ sont positifs et de somme 1.
  \item AdaGrad : gradient adaptatif par pondération des mises à jour
  \item Adam : pondération fine et adaptative des mises à jour de chaque poid
  \end{itemize}
\end{frame}


\begin{frame}
  \frametitle{Gradient et mise à jour}
  Intégrer des contraintes dans le gradient par la loss. On donne une direction différente à l'objectif et donc une surface des paramètres en fonction de l'erreur différente.
  \begin{center}
    $grad = grad_x(x) + grad_{contrainte 1} + grad_{contrainte 2} + ...$
  \end{center}
  \begin{itemize}
  \item ajout de la norme L2 des poids à la loss => contraint les poids à rester faibles
  \item softmax => cross-entropie => sparsification des poids
  \end{itemize}
\end{frame}
