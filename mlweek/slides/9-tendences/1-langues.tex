\documentclass{formation}
\title{Problématiques liées au machine learning}
\subtitle{Données à dimension variable : Traitement du langage}

\begin{document}

\maketitle

\begin{frame}
  \frametitle{Traitement du langage : données}
  Classification :
  \begin{itemize}
  \item thème/genre (gutenberg.org : 57k livres)
  \item auteur (gutenberg.org : 57k livres)
  \item sentiment (Kaggle movie review : 222k commentaires rotten tomatoes)
  \item reconnaissance d'entités nommées (Kaggle Annotated Corpus for NER : 1.3M tags)
  \item ...
  \end{itemize}
\end{frame}

\begin{frame}
  \frametitle{Traitement du langage : données}
  Compréhension :
  \begin{itemize}
  \item Question/réponses (SQUAD : 150k questions)
  \item Traduction (europarl : 450k phrases alignées)
  \item ...
  \end{itemize}
\end{frame}

\begin{frame}
  \frametitle{Traitement du langage : word embeddings} 
  mot = indice dans un dictionnaire (dimension > 30000)
  \newline
  mot = vecteur ``sémantique'' (dimension < 1000) 
  \begin{itemize}
  \item word2vec
  \item CBOW/Skip-Gram
  \item Thought vector (pour des phrases ou même des documents entiers)
  \item ...
  \end{itemize}
\end{frame}

\begin{frame}
  \frametitle{Traitement du langage : word embeddings} 
  \imgtw{word2vec}
\end{frame}

\begin{frame}
  \frametitle{Traitement du langage : word embeddings} 
  \imgtw[0.6]{word2vec-analogy}
  \begin{center}
    \href{https://ml5js.org/docs/word2vec-example}{\blue{Démo dans l'espace word2vec}} \\
    \href{https://projector.tensorflow.org/}{\blue{Visualisation de l'espace word2vec}}
  \end{center}
\end{frame}

\begin{frame}
  \frametitle{Traitement du langage : Modèle à attention} 
  \imgth[0.8]{attention-model}
  \begin{center}
    \href{https://demo.allennlp.org/machine-comprehension}{\blue{Démo modèle à attention}}
  \end{center}
\end{frame}

\begin{frame}
  \frametitle{Traitement du langage : Champs d'application}
  \underline{\textbf{Transcription et synthèse de la parole}}
  \imgtw[0.6]{smart-assistants}
\end{frame}

\begin{frame}
  \frametitle{Traitement du langage : Champs d'application}
  \underline{\textbf{Identification du locuteur}}
  \imgtw[0.6]{id-locuteur}
\end{frame}

\begin{frame}
  \frametitle{Traitement du langage : Champs d'application}
  \underline{\textbf{Chatbot}}
  \imgtw[0.6]{eliza}
\end{frame}

\begin{frame}
  \frametitle{Traitement du langage : Champs d'application}
  \underline{\textbf{Chatbot}}
  \imgtw[0.6]{zo}
\end{frame}

\begin{frame}
  \frametitle{Traitement du langage : Champs d'application}
  \underline{\textbf{Moteur de recherche}}
  \imgtw[0.6]{deep-moteur-recherche}
\end{frame}

\begin{frame}
  \frametitle{Traitement du langage : Champs d'application}
  \underline{\textbf{Extraction de données}}
  \imgtw[0.6]{information-extraction}
  \begin{center}
    \href{https://demo.allennlp.org/named-entity-recognition}{\blue{Démo reconnaissance d'entitées nomées}}
  \end{center}
\end{frame}

\begin{frame}
  \frametitle{Traitement du langage : Champs d'application}
  \underline{\textbf{Analyse de sentiments}}
  \imgtw[0.6]{sentiment-analysis}
  \begin{center}
    \href{https://text-processing.com/demo/sentiment/}{\blue{Démo analyse de sentiments}}
  \end{center}
\end{frame}

\begin{frame}
  \frametitle{Traitement du langage : Champs d'application}
  \underline{\textbf{Résumé}}
  \imgtw[0.8]{auto-summary}
\end{frame}

\begin{frame}
  \frametitle{Traitement du langage : Champs d'application}
  \underline{\textbf{Traduction}}
  \begin{center}
    \imgtw[1.1]{traduction-fr-en-short}
  \end{center}
  \begin{center}
    \href{https://www.deepl.com/fr/translator}{\blue{Démo traduction de DeepL}}
  \end{center}
\end{frame}

\begin{frame}
  \frametitle{Traitement du langage : Champs d'application}
  \imgtw[1]{nlp-tech-resume}
\end{frame}

\end{document}
%%% Local Variables:
%%% mode: latex
%%% TeX-master: t
%%% End:
