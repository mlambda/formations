\documentclass{formation}
\title{Machine Learning}
\subtitle{Données séquentielles - Premières approches }

\begin{document}

\maketitle

\begin{frame}
  \frametitle{Données séquentielles - Premières approches }
  \begin{itemize}
  \item Calcul des différences du premier ou second ordre.
  \item Modèles autoregressifs à moyenne mobile (ARMA)
  \item Modèles de Markov 
  \end{itemize}
\end{frame}

\begin{frame}
  \frametitle{Données séquentielles - Premières approches }
  
  $x_t=g(t)+\phi_t$ \\
  où \\
  $g(t)$ est déterministe (tendance globale du signal) \\
  $\phi(t)$ est un bruit stochastique \\
  \newline
  Les modèles statistiques ``classiques'' vont alors essayer de décomposer $g(t)$ en plusieurs fonctions dépendant de t ou bien de fonction récurrentes (comme dans ARMA)
\end{frame}

\begin{frame}
  \frametitle{Données séquentielles - Premières approches }
  \imgth[0.9]{time-serie-decomposition}
\end{frame}

\begin{frame}
  \frametitle{Données séquentielles - Premières approches }
  Si ces méthodes vous intéressent : \\
  \newline
  \url{http://www.ulb.ac.be/di/map/gbonte/ftp/time_ser.pdf} \\
  \newline
  Un notebook (en anglais) est aussi disponible sur le sujet dans le dossier codelab
\end{frame}

\end{document}
%%% Local Variables:
%%% mode: latex
%%% TeX-master: t
%%% End:
